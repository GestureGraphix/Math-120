\documentclass{report}

\input{preamble}
\input{macros}
\input{letterfonts}

\title{\Huge{Math 120 QR}}
\author{\huge{Alex Hernandez Juarez}}
\date{Fall 2024}

\begin{document}

\maketitle
\newpage% or \cleardoublepage
% \pdfbookmark[<level>]{<title>}{<dest>}
\pdfbookmark[section]{\contentsname}{toc}
\tableofcontents
\pagebreak

\chapter{}
\section{12.1 Notes (Three Dimensional Coodinate Systems)}

\defrose{Distance Formula}{DDefintion: 
	\begin{center}
		\[ |P_{1}P_{2}| = \sqrt{(x_{2} - x_{1})^{2} + (y_{2} - y_{1})^{2} + (z_{2} - z_{1})^{2} }\] 
	\end{center}
}

\defrose{Equation of a sphere}{
	DDefintion: An equation of a sphere with center $C(h,k,l)$, and radius $r$ is 
	\begin{center}
		\[ (x-h)^{2} + (y-k)^{2} + (z-l)^{2}\] 
	\end{center}
	In particular, if the center is the origin $O$, than an equation of the sphere is 
	\begin{center}
		\[ x^{2} + y^{2} + z^{2}\] 
	\end{center}
}
		
\section{12.2 Notes (Vectors)}

\defrose{Vector Addition}{IIf $\textbf{u}$ and $\textbf{v}$ are vectors positioned so the initial
point of $\textbf{v}$ is at the terminal point of $\textbf{u}$, then the $\textbf{sum u + v}$ is the vector from the
initial point of $\textbf{u}$ to the terminal point of $\textbf{v}$.}

\defrose{Scalar Multiplication}{IIf $c$ is a scalar and $\textbf{v}$ is a vector, then the
$\textbf{scalar multiple}$ $c \textbf{v}$ is the vector whose length is $|c|$ times the length of $\textbf{v}$ and whose direction is
the same as $\textbf{v}$ if $c > 0$ and is opposite to $\textbf{v}$ if $c=0$ or $\textbf{v}$ = 0, then c$\textbf{v} = 0$ }


\exrose{}{Given the points $A(x_{1}, y_{1},z_{1})$ and $B(x_{2}, y_{2}, z_{2})$, the vector $\textbf{a}$ with represenation 
	$\ray{AB}$ is: 
	\begin{center}
		\[ a = \langle x_{2} - x_{1}, y_{2}-y_{1}, z_{2} - z_{1} \rangle \] 
	\end{center}
}


\exrose{}{
	If $\textbf{a}  = \langle a_{1}, a_{2}\rangle$ and $\textbf{b} = \langle b_{1}, b_{2} \rangle$, then: 
	\begin{center}
		\[ \textbf{a} + \textbf{b} = \langle a_{1} + b_{1}, a_{2} + b_{2} \rangle \]
		\[ \textbf{a} - \textbf{b} = \langle a_{1} - b_{1}, a_{2} - b_{2} \rangle \]
		\[ c\textbf{a} = \langle ca_{1}, ca_{2} \rangle \]   
	\end{center} 
	
	Similarily, for three demensional vectors, 

	\begin{center}
		\[ \langle a_{1}, a_{2}, a_{3} \rangle + \langle b_{1}, b_{2}, b_{3} \rangle = \langle a_{1} + b_{1}, a_{2} + a_{3} + b_{3} \rangle \] 
		\[ \langle a_{1}, a_{2}, a_{3} \rangle - \langle b_{1}, b_{2}, b_{3} \rangle = \langle a_{1} - b_{1}, a_{2} - a_{3} - b_{3} \rangle \] 
		\[ c \langle a_{1}, a_{2}, a_{3} \rangle =  \langle ca_{1}, ca_{2}, ca_{3} \rangle \] 
	\end{center} 

}

\ntimg{Properties of vectors: If $\textbf{a}$, $\textbf{b}$, and $\textbf{c}$ are vectors in $V_{n}$ and $c$ and $d$ are scalars than 
	\begin{itemize}
		\item $\textbf{a} + \textbf{b} = \textbf{b} + \textbf{a}$
		\item $a + (\textbf{b} + \textbf{c}) = (\textbf{a} + \textbf{b}) + \textbf{c} $
		\item $\textbf{a} + 0 = \textbf{a}$
		\item $ \textbf{a} + \textbf{a} + \textbf{-a} = 0 $
		\item $c(\textbf{a} + \textbf{b}) = c\textbf{a} + c\textbf{b}$ 
		\item $(c + d)a = c \textbf{a} + d \textbf{a} $
		\item $(cd) \textbf{a} = c(d\textbf{a})$
		\item $l\textbf{a} = \textbf{a} $
	\end{itemize}
}

\section{12.3 Notes (Dot Product)}

\defrose{Dot Product}{
	IIf $\textbf{a} = \langle a_{1}, a_{2}, a_{3} \rangle $ and $\textbf{b} = \langle b_{1}, b_{2}, 
	b_{3} \rangle $, then the $\textbf{dot  product}$ of $\mathbf{a}$ and $\mathbf{b}$ is the number $\textbf{a} \cdot \textbf{b}$ given by 
	\begin{center}
		\[ \textbf{a} \cdot \textbf{b} = a_{1}b_{1} + a_{2}b_{2} + a_{3}b_{3} \] 
	\end{center} 
	Properties of the Dot Product: If $\mathbf{a}, \mathbf{b},$ and $\mathbf{c}$ are vectors in $V_3$ and $c$ is a scalar, then 
	\begin{enumerate}
		\item $\mathbf{a} \cdot \mathbf{a} = |\mathbf{a}|^2$
		\item $\mathbf{a} \cdot \mathbf{b} = \mathbf{b} \cdot \mathbf{a}$
		\item $\mathbf{a} \cdot (\mathbf{b} + \mathbf{c}) = \mathbf{a} \cdot \mathbf{b} + \mathbf{a} \cdot \mathbf{c}$
		\item $(c\mathbf{a}) \cdot \mathbf{b} = c(\mathbf{a} \cdot \mathbf{b}) = \mathbf{a} \cdot (c\mathbf{b})$
		\item $\mathbf{0} \cdot \mathbf{a} = 0$
	\end{enumerate} 
}

\defrose{Geometric Definition of the Dot Product}{IIf $\theta$ is the angle between vectors $\mathbf{a}$ and $\mathbf{b}$, then 
	\begin{center}
		\[ \mathbf{a} \cdot \mathbf{b} = |\mathbf{a}| |\mathbf{b}| \cos(\theta) \] 
	\end{center}
	Proof: 
	\begin{center}
		\begin{tikzpicture}[line cap=round,line join=round,>=triangle 45, scale=1.5]
			% Axes
			\draw[->] (0,0,0) -- (2,0,0) node[anchor=north east]{$x$};
			\draw[->] (0,0,0) -- (0,2,0) node[anchor=north west]{$y$};
			\draw[->] (0,0,0) -- (0,0,2) node[anchor=south]{$z$};
			
			% Vectors a, b, and a-b
			\draw[->, thick, blue] (0,0,0) -- (2,0,0) node[midway, below]{$\mathbf{a}$} node[anchor=west]{$A$};
			\draw[->, thick, blue] (0,0,0) -- (1,1,0) node[midway, left]{$\mathbf{b}$} node[anchor=east]{$B$};
			\draw[->, thick, blue] (1,1,0) -- (2,0,0) node[midway, above]{$\mathbf{a-b}$};
			
			% Angle label theta
			\draw (0.5,0,0) arc[start angle=0,end angle=45,radius=0.5] node[midway, right]{$\theta$};
			
			% Origin O
			\node[anchor=east] at (0,0,0) {$O$};
			
		\end{tikzpicture}
		\[ |AB|^{2} = |OA|^{2} + |OB|^{2} - 2 |OA| |OB| \cos \theta \] 
	\end{center}

	Corollary: If $\theta$ is the angle between nonzero vectors $\mathbf{a}$ and $\mathbf{b}$, then 
	\begin{center}
		\[ \cos(\theta)  = \frac{\mathbf{a} \cdot \mathbf{b} }{|\mathbf{a}| |\mathbf{b}| } \] 
	\end{center}
}

\ntimg{
	Two vectors $\mathbf{a}$ and $\mathbf{b}$ are orthogonal if an only if $\mathbf{a} \cdot \mathbf{b} = 0$
}

\ex{Direction Angles and Cosines}{
	The \textbf{direction angles} of a nonzero vector $\mathbf{a}$ are the angles $\alpha$, $\beta$, and $\gamma$ (in the interval $[0, \pi]$) that $\mathbf{a}$ makes with the positive $x$-, $y$-, and $z$-axes, respectively .

	The cosines of these direction angles, $\cos \alpha$, $\cos \beta$, and $\cos \gamma$, are called the \textbf{direction cosines} of the vector $\mathbf{a}$. Using Corollary 6 with $\mathbf{b}$ replaced by $\mathbf{i}$, we obtain:

	\[ \cos \alpha = \frac{\mathbf{a} \cdot \mathbf{i}}{|\mathbf{a}||\mathbf{i}|} = \frac{a_1}{|\mathbf{a}|} \tag{1} \]

	Similarly, we also have:
	\[ \cos \beta = \frac{a_2}{|\mathbf{a}|} \quad \text{and} \quad \cos \gamma = \frac{a_3}{|\mathbf{a}|} \tag{2} \]
	By squaring the expressions in Equations 8 and 9 and adding, we see that:
	\[ \cos^2 \alpha + \cos^2 \beta + \cos^2 \gamma = 1 \tag{3} \]
	We can also use Equations 8 and 9 to write:
	\[ \mathbf{a} = \langle a_1, a_2, a_3 \rangle = \langle |\mathbf{a}| \cos \alpha, |\mathbf{a}| \cos \beta, |\mathbf{a}| \cos \gamma \rangle
	= |\mathbf{a}| \langle \cos \alpha, \cos \beta, \cos \gamma \rangle \]
	Therefore,
	\[ \frac{1}{|\mathbf{a}|} \mathbf{a} = \langle \cos \alpha, \cos \beta, \cos \gamma \rangle \tag{4} \]
	which says that the direction cosines of $\mathbf{a}$ are the components of the unit vector in the direction of $\mathbf{a}$.

}

\defrose{Projections}{
	TThe \textbf{scalar projection} of $\mathbf{b}$ onto $\mathbf{a}$ (also called the \textbf{component of $\mathbf{b}$ along $\mathbf{a}$}) is defined to be the signed magnitude of the vector projection, which is the number $|\mathbf{b}| \cos \theta$, where $\theta$ is the angle between $\mathbf{a}$ and $\mathbf{b}$.
	This is denoted by $\text{comp}_{\mathbf{a}} \mathbf{b}$. Observe that it is negative if $\pi/2 < \theta \leq \pi$. The equation
	\[ 	\mathbf{a} \cdot \mathbf{b} = |\mathbf{a}| |\mathbf{b}| \cos \theta = |\mathbf{a}| (|\mathbf{b}| \cos \theta) \]
	shows that the dot product of $\mathbf{a}$ and $\mathbf{b}$ can be interpreted as the length of $\mathbf{a}$ times the scalar projection of $\mathbf{b}$ onto $\mathbf{a}$. Since
	\[ 	|\mathbf{b}| \cos \theta = \frac{\mathbf{a} \cdot \mathbf{b}}{|\mathbf{a}|} = \frac{\mathbf{a}}{|\mathbf{a}|} \cdot \mathbf{b} \]
	the component of $\mathbf{b}$ along $\mathbf{a}$ can be computed by taking the dot product of $\mathbf{b}$ with the unit vector in the direction of $\mathbf{a}$. We summarize these ideas as follows.
	\\
	\textbf{Scalar projection of $\mathbf{b}$ onto $\mathbf{a}$:} \quad $\text{comp}_{\mathbf{a}} \mathbf{b} = \frac{\mathbf{a} \cdot \mathbf{b}}{|\mathbf{a}|}$


	\textbf{Vector projection of $\mathbf{b}$ onto $\mathbf{a}$:} \quad $\text{proj}_{\mathbf{a}} \mathbf{b} = \left( \frac{\mathbf{a} \cdot \mathbf{b}}{|\mathbf{a}|^2} \right) \mathbf{a} = \frac{\mathbf{a} \cdot \mathbf{b}}{|\mathbf{a}|^2} \mathbf{a}$
}


\section{12.4 Notes (Cross Product)}

\defrose{Cross Product}{
	GGiven two nonzero vectors $\mathbf{a} = \langle a_1, a_2, a_3 \rangle$ and $\mathbf{b} = \langle b_1, b_2, b_3 \rangle$, suppose that a nonzero vector $\mathbf{c} = \langle c_1, c_2, c_3 \rangle$ is perpendicular to both $\mathbf{a}$ and $\mathbf{b}$. Then $\mathbf{a} \cdot \mathbf{c} = 0$ and $\mathbf{b} \cdot \mathbf{c} = 0$, and so:

	\begin{align}
	a_1c_1 + a_2c_2 + a_3c_3 &= 0 \tag{1} \\
	b_1c_1 + b_2c_2 + b_3c_3 &= 0 \tag{2}
	\end{align}

	To eliminate $c_3$, we multiply (1) by $b_3$ and (2) by $a_3$ and subtract:

	\[ (a_1b_3 - a_3b_1)c_1 + (a_2b_3 - a_3b_2)c_2 = 0 \tag{3} \]

	Equation (3) has the form $pc_1 + qc_2 = 0$, for which an obvious solution is $c_1 = q$ and $c_2 = -p$. So, a solution of (3) is:

	\begin{align*}
		c_1 &= a_2b_3 - a_3b_2 \\
		c_2 &= a_3b_1 - a_1b_3
	\end{align*}

	Substituting these values into (1) and (2), we then get:

	\[ 	c_3 = a_1b_2 - a_2b_1 	\]

	This means that a vector perpendicular to both $\mathbf{a}$ and $\mathbf{b}$ is:

	\[ 	\langle c_1, c_2, c_3 \rangle = \langle a_2b_3 - a_3b_2, a_3b_1 - a_1b_3, a_1b_2 - a_2b_1 \rangle \]

	The resulting vector is called the \textbf{cross product} of $\mathbf{a}$ and $\mathbf{b}$ and is denoted by $\mathbf{a} \times \mathbf{b}$.

}

\defrose{Cross Product of two vectors}{
	IIf $\mathbf{a} = \langle a_{1}, a_{2}, a_{3} \rangle $ and $\mathbf{b} = \langle b_{1}, b_{2}, b_{3} \rangle $ 
	then the \textbf{cross product} of \textbf{a} and \textbf{b} is: 
	\begin{center}
		\[ \mathbf{a} \times \mathbf{b} = \langle a_{2}b_{3} - a_{3}b_{2}, a_{3}b_{1} - a_{1}b_{3}, a_{1}b_{2} - a_{2}b_{1} \rangle \] 
	\end{center}
}

\ntimg{Determinant of order 2: 
	\begin{center}
		\[ \begin{vmatrix}
		a & b \\
		c & d
		\end{vmatrix}
		= ad - bc \]
	\end{center}
}

\ntimg{Determinant of order 3: 
	\[ 
	\begin{vmatrix}
	a_1 & a_2 & a_3 \\
	b_1 & b_2 & b_3 \\
	c_1 & c_2 & c_3
	\end{vmatrix}
	= a_1
	\begin{vmatrix}
	b_2 & b_3 \\
	c_2 & c_3
	\end{vmatrix}
	- a_2
	\begin{vmatrix}
	b_1 & b_3 \\
	c_1 & c_3
	\end{vmatrix}
	+ a_3
	\begin{vmatrix}
	b_1 & b_2 \\
	c_1 & c_2
	\end{vmatrix}
	\]
}

\defrose{Second definition of cross product}{
	AArithmetic Definition: 
	\begin{center}
		\[ a \times b = \left[ \begin{matrix}
			i & j & k  \\
			a_{1} & a_{2} & a_{3} \\
			b_{1} & b_{2} & b_{3} \\
		\end{matrix}\right] = |a||b| \sin(\theta) \] 
		\[ \left[ \begin{matrix}
			a_{2} & a_{3} \\
			b_{2} & b_{3}  \\
		\end{matrix}\right] i - \left[ \begin{matrix}
			a_{1} & a_{3} \\
			b_{1} & b_{3}  \\
		\end{matrix}\right] j + \left[ \begin{matrix}
			a_{1} & a_{2} \\
			b_{1} & b_{2} k  \\
		\end{matrix}\right]\] 
		\[ = (a_{2}b_{3} - a_{3}b_{2}) i - (a_{1}b_{3} - a_{3}b_{1}) j + (a_{1} b_{2} - a_{2}b_{1}) k \]\
	\end{center}

	The vector \textbf{a} $\times$ \textbf{b} is orthogonal to both \textbf{a} and \textbf{b} 
}

\exrose{Proof that \textbf{a} $\times$ \textbf{b} is orthogonal to both \textbf{a}}{
	\begin{align*}
		(\mathbf{a} \times \mathbf{b}) \cdot \mathbf{a} &= 
		\begin{vmatrix}
		a_2 & a_3 \\
		b_2 & b_3
		\end{vmatrix} a_1 - 
		\begin{vmatrix}
		a_1 & a_3 \\
		b_1 & b_3
		\end{vmatrix} a_2 + 
		\begin{vmatrix}
		a_1 & a_2 \\
		b_1 & b_2
		\end{vmatrix} a_3 \\
		&= a_1(a_2b_3 - a_3b_2) - a_2(a_1b_3 - a_3b_1) + a_3(a_1b_2 - a_2b_1) \\
		&= a_1a_2b_3 - a_1a_3b_2 - a_2a_1b_3 + a_2a_3b_1 + a_3a_1b_2 - a_3a_2b_1 \\
		&= 0
	\end{align*}
}

\defrose{sin definition of cross product}{
	IIf $\theta$ is the angle between \textbf{a} and \textbf{b} (so $0 \leq \theta \leq \pi$), then 
	the length of the cross product \textbf{a} $\times$ \textbf{b} is given by: 
	\[ |\mathbf{a} \times \mathbf{b} | = |\mathbf{a}| |\mathbf{b}| \sin(\theta) \]
	Proof: 
	\begin{align*}
		|\mathbf{a} \times \mathbf{b}|^2 &= (a_2b_3 - a_3b_2)^2 + (a_3b_1 - a_1b_3)^2 + (a_1b_2 - a_2b_1)^2 \\
		\\
		&= a_2^2b_3^2 - 2a_2a_3b_2b_3 + a_3^2b_2^2 + a_3^2b_1^2 - 2a_1a_3b_1b_3 + a_1^2b_3^2 + a_1^2b_2^2 - 2a_1a_2b_1b_2 + a_2^2b_1^2 \\
		\\
		&= (a_1^2 + a_2^2 + a_3^2)(b_1^2 + b_2^2 + b_3^2) - (a_1b_1 + a_2b_2 + a_3b_3)^2 \\
		\\
		&= |\mathbf{a}|^2 |\mathbf{b}|^2 - (\mathbf{a} \cdot \mathbf{b})^2 \\
		\\
		&= |\mathbf{a}|^2 |\mathbf{b}|^2 - |\mathbf{a}|^2 |\mathbf{b}|^2 \cos^2 \theta \quad \text{(by Theorem 12.3.3)} \\
		\\
		&= |\mathbf{a}|^2 |\mathbf{b}|^2 (1 - \cos^2 \theta) \\
		\\
		&= |\mathbf{a}|^2 |\mathbf{b}|^2 \sin^2 \theta
	\end{align*}

	Taking square roots and observing that \(\sqrt{\sin^2 \theta} = \sin \theta\) because \(\sin \theta \geq 0\) when \(0 \leq \theta \leq \pi\), we have
	\[ |\mathbf{a} \times \mathbf{b}| = |\mathbf{a}| |\mathbf{b}| \sin \theta \]
}

\ntimg{
	Two nonzero vectors \textbf{a} and \textbf{b} are parallel if and only if 
	\[ \mathbf{a} \times \mathbf{b} = 0 \] 
}

\exrose{Geometric interpretation of $|\mathbf{a} \times \mathbf{b}| = |\mathbf{a}| |\mathbf{b}| \sin \theta$}{
	If \textbf{a} and \textbf{b} are represented by directed line segments with the same inital point, then they determine 
	a parallelogram with base $|\mathbf{a}|$, altitude $\mathbf{b}\sin(\theta)$ and area 
	\[ A = |\mathbf{a}|(|\mathbf{b}| \sin \theta ) = |\mathbf{a} \times \mathbf{b}|  \] 
	Thus we have the following way of interpreting the magnitude of a cross product: \\
	The length of the cross product of \textbf{a} $\times$ \textbf{b} is equal to the area of the 
	parallelogram determined by \textbf{a} and \textbf{b} 
}

\ntimg{
	If we apply the following theorem:

	The vector \(\mathbf{a} \times \mathbf{b}\) is orthogonal to both \(\mathbf{a}\) and \(\mathbf{b}\), and 
	\[ 	|\mathbf{a} \times \mathbf{b}| = |\mathbf{a}| |\mathbf{b}| \sin \theta \]
	to the standard basis vectors \(\mathbf{i}, \mathbf{j}, \mathbf{k}\) using \(\theta = \frac{\pi}{2}\), we obtain

	\[
	\begin{aligned}
		\mathbf{i} \times \mathbf{j} &= \mathbf{k} \quad &\mathbf{j} \times \mathbf{k} &= \mathbf{i} \quad &\mathbf{k} \times \mathbf{i} &= \mathbf{j} \\
		\mathbf{j} \times \mathbf{i} &= -\mathbf{k} \quad &\mathbf{k} \times \mathbf{j} &= -\mathbf{i} \quad &\mathbf{i} \times \mathbf{k} &= -\mathbf{j}
	\end{aligned}
	\]
}


\ntimg{
	If \(\mathbf{a}\), \(\mathbf{b}\), and \(\mathbf{c}\) are vectors and \(c\) is a scalar, then
	\begin{enumerate}
		\item \(\mathbf{a} \times \mathbf{b} = -\mathbf{b} \times \mathbf{a}\)
		\item \((c\mathbf{a}) \times \mathbf{b} = c(\mathbf{a} \times \mathbf{b}) = \mathbf{a} \times (c\mathbf{b})\)
		\item \(\mathbf{a} \times (\mathbf{b} + \mathbf{c}) = \mathbf{a} \times \mathbf{b} + \mathbf{a} \times \mathbf{c}\)
		\item \((\mathbf{a} + \mathbf{b}) \times \mathbf{c} = \mathbf{a} \times \mathbf{c} + \mathbf{b} \times \mathbf{c}\)
		\item \(\mathbf{a} \cdot (\mathbf{b} \times \mathbf{c}) = (\mathbf{a} \times \mathbf{b}) \cdot \mathbf{c}\)
		\item \(\mathbf{a} \times (\mathbf{b} \times \mathbf{c}) = (\mathbf{a} \cdot \mathbf{c})\mathbf{b} - (\mathbf{a} \cdot \mathbf{b})\mathbf{c}\)
	\end{enumerate}
}

\exrose{Proof of property 5 of cross products}{
	If \(\mathbf{a} = \langle a_1, a_2, a_3 \rangle\), \(\mathbf{b} = \langle b_1, b_2, b_3 \rangle\), and \(\mathbf{c} = \langle c_1, c_2, c_3 \rangle\), then: 
	\begin{align*}
		\mathbf{a} \cdot (\mathbf{b} \times \mathbf{c}) &= a_1(b_2c_3 - b_3c_2) + a_2(b_3c_1 - b_1c_3) + a_3(b_1c_2 - b_2c_1)\\ 
		\\
		&= a_1b_2c_3 - a_1b_3c_2 + a_2b_3c_1 - a_2b_1c_3 + a_3b_1c_2 - a_3b_2c_1 \\
		\\
		&= (a_2b_3 - a_3b_2)c_1 + (a_3b_1 - a_1b_3)c_2 + (a_1b_2 - a_2b_1)c_3 \\
		\\
		&= (\mathbf{a} \times \mathbf{b}) \cdot \mathbf{c} \\
	\end{align*}
}

\defrose{Triple Products}{ 
	TThe product \(\mathbf{a} \cdot (\mathbf{b} \times \mathbf{c})\) that occurs in Property 5 is called the \textit{scalar triple product} of the vectors \(\mathbf{a}\), \(\mathbf{b}\), and \(\mathbf{c}\). 

	\[ 	\mathbf{a} \cdot (\mathbf{b} \times \mathbf{c}) = 
	\begin{vmatrix}
	a_1 & a_2 & a_3 \\
	b_1 & b_2 & b_3 \\
	c_1 & c_2 & c_3
	\end{vmatrix} \]
	
	The geometric significance of the scalar triple product can be seen by considering the parallelepiped determined by the vectors \(\mathbf{a}\), \(\mathbf{b}\), and \(\mathbf{c}\). The area of the base parallelogram is \(A = |\mathbf{b} \times \mathbf{c}|\). If \(\theta\) is the angle between \(\mathbf{a}\) and \(\mathbf{b} \times \mathbf{c}\), then the height \(h\) of the parallelepiped is \(h = |\mathbf{a}| |\cos \theta|\). (We must use \(|\cos \theta|\) instead of \(\cos \theta\) in case \(\theta > \pi / 2\).) Therefore, the volume of the parallelepiped is
	
	\[ 	V = A h = |\mathbf{b} \times \mathbf{c}| |\mathbf{a}| |\cos \theta| = |\mathbf{a} \cdot (\mathbf{b} \times \mathbf{c})| \]
	
	Thus, we have proved the following formula: 
	The volume of the parallelepiped determined by the vectors \textbf{a}, \textbf{b}, and \textbf{c} is the magnitude of their scalar triple product:
	\[ V = |\mathbf{a} \cdot (\mathbf{b} \times \mathbf{c})| \] 
 }
 
 \section{12.5 Notes (Equations of Lines and Planes)}

\defrose{Hi}{
	Likewise, a line \(L\) in three-dimensional space is determined when we know a point
	\(P_0(x_0, y_0, z_0)\) on \(L\) and a direction for \(L\), which is conveniently 
	described by a vector \(\mathbf{v}\) parallel to the line. Let \(P(x, y, z)\) be an
	arbitrary point on \(L\) and let \(\mathbf{r_0}\) and \(\mathbf{r}\) be the position
	vectors of \(P_0\) and \(P\) (that is, they have representations
	\(\overrightarrow{OP_0}\) and \(\overrightarrow{OP}\)). If \(\mathbf{a}\) is the
	vector with representation \(\overrightarrow{P_0P}\), as in Figure 1, then the
	Triangle Law for vector addition gives

	\[
	\mathbf{r} = \mathbf{r_0} + \mathbf{a}.
	\]

	\insertpng[0.25]{lines.png}
	\[ r = r_{0} + t \mathbf{v}\] 
}


\end{document}
