\documentclass[9pt]{article}
\usepackage{amsmath, amssymb, graphicx, geometry, multicol, tcolorbox, titlesec, xcolor, titling}
\geometry{a4paper, margin=0.1in}
\setlength{\parskip}{0pt}
\setlength{\parindent}{0pt}
\linespread{0.3}
\titlespacing{\section}{0pt}{0pt}{0pt}
\setlength{\columnsep}{0.1cm}

% Brighter pastel colors
\definecolor{brightblue}{HTML}{B3D9FF}
\definecolor{brightgreen}{HTML}{C2FABC}
\definecolor{brightpink}{HTML}{FFD3D6}
\definecolor{brightyellow}{HTML}{FFFCAB}

\tcbset{
    boxrule=0.1mm,
    colback=white,
    fonttitle=\bfseries\footnotesize,
    sharp corners,
    coltitle=black,
    top=0pt,
    bottom=0pt,
    left=0.3mm,
    right=0.3mm,
    boxsep=0.2mm,
    arc=0.3mm
}

\setlength{\droptitle}{-70pt}

\title{\footnotesize \textbf{Comprehensive Review for Exams 1, 2, and Final}}
\date{}

\begin{document}
\maketitle
\vspace{-15pt}

\begin{multicols}{2}
\footnotesize

\noindent \textbf{Review Exam 1 Concepts:}\\
Foundations of multivariable calculus: vectors, geometry in space, and basic derivatives. Mastering these fundamentals helps you visualize problems and provides the tools to solve more advanced topics later.

\begin{tcolorbox}[title=, colframe=brightblue]
\textbf{1. Vectors}\\
\textit{Concept:} Vectors represent quantities with magnitude and direction. They enable a geometric interpretation of problems in higher dimensions.

\begin{itemize}
    \item \textbf{Magnitude:} For $\mathbf{a}=\langle a_1,a_2,a_3\rangle$, $\|\mathbf{a}\|=\sqrt{a_1^2+a_2^2+a_3^2}$.
    \item \textbf{Operations:} If $\mathbf{a},\mathbf{b}\in\mathbb{R}^3$ and $c\in\mathbb{R}$, then:
    \[
    \mathbf{a}+\mathbf{b}=\langle a_1+b_1,a_2+b_2,a_3+b_3\rangle,\quad c\mathbf{a}=\langle ca_1,ca_2,ca_3\rangle.
    \]
    \item \textbf{Position Vectors:} For points $A=(x_1,y_1,z_1)$ and $B=(x_2,y_2,z_2)$:
    \[
    \overrightarrow{AB}=\langle x_2 - x_1,\, y_2 - y_1,\, z_2 - z_1\rangle.
    \]
\end{itemize}

\textbf{Tips:}
\begin{itemize}
    \item Visualize vectors as arrows in space.
    \item Use vector subtraction to find directions between points.
\end{itemize}
\end{tcolorbox}

\begin{tcolorbox}[title=, colframe=brightgreen]
\textbf{2. Dot Product}\\
\textit{Concept:} The dot product measures how two vectors align with each other. It relates closely to angles and projections.

\[
\mathbf{a}\cdot \mathbf{b}=\|\mathbf{a}\|\|\mathbf{b}\|\cos\theta.
\]

\textbf{Projection:}
\[
\text{proj}_{\mathbf{b}}(\mathbf{a})=\left(\frac{\mathbf{a}\cdot\mathbf{b}}{\|\mathbf{b}\|^2}\right)\mathbf{b}.
\]

\textbf{Tips:}
\begin{itemize}
    \item If $\mathbf{a}\cdot\mathbf{b}=0$, the vectors are perpendicular.
    \item Use projections to find components of forces, velocities, etc. in a given direction.
\end{itemize}
\end{tcolorbox}

\begin{tcolorbox}[title=, colframe=brightpink]
\textbf{3. Cross Product}\\
\textit{Concept:} The cross product yields a vector perpendicular to both $\mathbf{a}$ and $\mathbf{b}$. It’s useful for finding normals and computing areas.

\[
\mathbf{a}\times \mathbf{b} = \begin{vmatrix}
\mathbf{i} & \mathbf{j} & \mathbf{k}\\
a_1 & a_2 & a_3 \\
b_1 & b_2 & b_3
\end{vmatrix}.
\]

\textbf{Tips:}
\begin{itemize}
    \item Use right-hand rule to determine direction.
    \item Norm $\|\mathbf{a}\times \mathbf{b}\|$ gives area of parallelogram formed by $\mathbf{a}$ and $\mathbf{b}$.
\end{itemize}
\end{tcolorbox}

\begin{tcolorbox}[title=, colframe=brightyellow]
\textbf{4. Planes}\\
\textit{Concept:} A plane is defined by a point and a normal vector. The normal vector determines the plane’s orientation.

\[
ax+by+cz=d.
\]

\textbf{Tips:}
\begin{itemize}
    \item To find a plane given three points, first find two direction vectors and then their cross product to get the normal.
    \item Check if a point lies on a plane by plugging coordinates into the equation.
\end{itemize}
\end{tcolorbox}

\begin{tcolorbox}[title=, colframe=brightblue]
\textbf{5. Distances}\\
\textit{Concept:} Distance formulas are essential for optimization and geometric interpretations.

\textbf{From point to plane:}
\[
\text{distance} = \frac{|ax_0+by_0+cz_0 - d|}{\sqrt{a^2 + b^2 + c^2}}.
\]

\textbf{From point to line:}
\[
\text{distance}=\frac{\|\overrightarrow{P_0A}\times \mathbf{v}\|}{\|\mathbf{v}\|}.
\]

\textbf{Tips:}
\begin{itemize}
    \item Always identify the normal vector for point-to-plane distance.
    \item For point-to-line distance, use cross products to avoid messy algebra.
\end{itemize}
\end{tcolorbox}

\begin{tcolorbox}[title=, colframe=brightgreen]
\textbf{6. Derivative of Vector Functions}\\
\textit{Concept:} The derivative represents instantaneous change. For $\mathbf{r}(t)=\langle x(t),y(t),z(t)\rangle$:
\[
\mathbf{r}'(t)=\langle x'(t),y'(t),z'(t)\rangle.
\]

\textbf{Tips:}
\begin{itemize}
    \item Interpret $\mathbf{r}'(t)$ as velocity if $\mathbf{r}(t)$ is a position.
    \item Higher derivatives can represent acceleration, etc.
\end{itemize}
\end{tcolorbox}

\begin{tcolorbox}[title=, colframe=brightpink]
\textbf{7. Tangent Line}\\
\textit{Concept:} The tangent line to a curve at $t=t_0$ is a linear approximation:
\[
\mathbf{r}(t)=\mathbf{r}(t_0)+\mathbf{r}'(t_0)(t-t_0).
\]

\textbf{Tips:}
\begin{itemize}
    \item Use tangent lines as first-order approximations for curves.
    \item Helpful for local linearization and quick estimates.
\end{itemize}
\end{tcolorbox}

\begin{tcolorbox}[title=, colframe=brightyellow]
\textbf{8. Integrals of Vector Functions}\\
\textit{Concept:} Integrating vector functions finds accumulated displacement, areas, or other geometric quantities.

\[
\int_a^b \mathbf{r}'(t)\,dt=\mathbf{r}(b)-\mathbf{r}(a).
\]

\textbf{Tips:}
\begin{itemize}
    \item Integrals of velocity give displacement.
    \item Consider each component integral separately.
\end{itemize}
\end{tcolorbox}

\begin{tcolorbox}[title=, colframe=brightblue]
\textbf{9. Functions of Several Variables}\\
\textit{Concept:} $f(x,y,z)$ defines surfaces and level sets. Visualizing these helps understand contour maps and 3D geometry.

\textbf{Tips:}
\begin{itemize}
    \item Level surfaces $f(x,y,z)=c$ show 3D shapes.
    \item Identify maxima/minima by examining level sets closely.
\end{itemize}
\end{tcolorbox}

\begin{tcolorbox}[title=, colframe=brightgreen]
\textbf{10. Implicit Differentiation}\\
\textit{Concept:} For implicit relations $F(x,y,z)=0$:
\[
\frac{\partial z}{\partial x} = -\frac{F_x}{F_z}, \quad \frac{\partial z}{\partial y} = -\frac{F_y}{F_z}.
\]

\textbf{Tips:}
\begin{itemize}
    \item Keep track of partial derivatives carefully.
    \item Treat all variables as functions of each other when not given explicitly.
\end{itemize}
\end{tcolorbox}

\end{multicols}

\noindent Exam 2 expands on these fundamentals: partial derivatives, finding extrema, integrals over domains, and line integrals of vector fields. Mastering these will enable more complex problem-solving.

\begin{multicols}{2}
\footnotesize

\noindent \textbf{Review Exam 2 Topics:}

\begin{tcolorbox}[title=, colframe=brightpink]
\textbf{1. Directional Derivatives}\\
\textit{Concept:} Rate of change of $f$ in direction $\mathbf{u}$:
\[
D_{\mathbf{u}} f(x_0,y_0)=\nabla f(x_0,y_0)\cdot \mathbf{u}.
\]

\textbf{Tips:}
\begin{itemize}
    \item Max directional derivative occurs in direction of $\nabla f$.
    \item Normalize direction vectors to ensure consistent magnitude.
\end{itemize}
\end{tcolorbox}

\begin{tcolorbox}[title=, colframe=brightyellow]
\textbf{2. Tangent Plane for $z=f(x,y)$}\\
\textit{Concept:} Local linear approximation:
\[
z-z_0=f_x(x_0,y_0)(x-x_0)+f_y(x_0,y_0)(y-y_0).
\]

\textbf{Tips:}
\begin{itemize}
    \item Use tangent planes to approximate function values near a point.
    \item The gradient at the point gives the orientation of this plane.
\end{itemize}
\end{tcolorbox}

\begin{tcolorbox}[title=, colframe=brightblue]
\textbf{3. Critical Points \& 2nd-Derivative Test}\\
\textit{Concept:} At critical points:
\[
f_x=0,\quad f_y=0.
\]
Use the Hessian determinant $D=f_{xx}f_{yy}-f_{xy}^2$ to classify.

\textbf{Tips:}
\begin{itemize}
    \item $D>0$ and $f_{xx}>0$: local min; $D>0$ and $f_{xx}<0$: local max; $D<0$: saddle point.
    \item Always check boundary conditions if the domain is restricted.
\end{itemize}
\end{tcolorbox}

\begin{tcolorbox}[title=, colframe=brightgreen]
\textbf{4. Lagrange Multipliers}\\
\textit{Concept:} Optimize $f(x,y,z)$ subject to $g(x,y,z)=c$:
\[
\nabla f=\lambda \nabla g.
\]

\textbf{Tips:}
\begin{itemize}
    \item Set up the system of equations and include the constraint.
    \item Geometrically, gradients align at extrema under constraints.
\end{itemize}
\end{tcolorbox}

\begin{tcolorbox}[title=, colframe=brightpink]
\textbf{5. Double Integrals}\\
\textit{Concept:} Compute volumes or mass:
\[
\iint_{R} f(x,y)\,dA.
\]

\textbf{Tips:}
\begin{itemize}
    \item Switch to polar coordinates for circular regions: $dA=r\,dr\,d\theta$.
    \item Carefully determine integration bounds by sketching regions.
\end{itemize}
\end{tcolorbox}

\begin{tcolorbox}[title=, colframe=brightyellow]
\textbf{6. Line Integrals}\\
\textit{Concept:} Integrate along a path $C$:
\[
\int_C \mathbf{F}\cdot d\mathbf{r}=\int_a^b \mathbf{F}(\mathbf{r}(t))\cdot \mathbf{r}'(t)\,dt.
\]

\textbf{Tips:}
\begin{itemize}
    \item Parameterize the curve $C$ first.
    \item For conservative fields, use the Fundamental Theorem of Line Integrals.
\end{itemize}
\end{tcolorbox}

\begin{tcolorbox}[title=, colframe=brightblue]
\textbf{7. Fundamental Theorem of Line Integrals}\\
\[
\int_C \nabla f \cdot d\mathbf{r}=f(\text{end})-f(\text{start}).
\]

\textbf{Tips:}
\begin{itemize}
    \item This drastically simplifies evaluation for conservative fields.
    \item Always check if $\mathbf{F}=\nabla f$ exists.
\end{itemize}
\end{tcolorbox}

\begin{tcolorbox}[title=, colframe=brightgreen]
\textbf{8. Vector Fields}\\
\textit{Concept:} $\mathbf{F}(x,y,z)=\langle P,Q,R\rangle$ assigns a vector to each point.

\textbf{Tips:}
\begin{itemize}
    \item Identify if fields are conservative by checking $\nabla \times \mathbf{F}$.
    \item Think physically: $\mathbf{F}$ could represent fluid flow or forces.
\end{itemize}
\end{tcolorbox}

\begin{tcolorbox}[title=, colframe=brightpink]
\textbf{9. Green's Theorem}\\
\[
\oint_{C} (P\,dx + Q\,dy) = \iint_{D}\left(\frac{\partial Q}{\partial x} - \frac{\partial P}{\partial y}\right)dA.
\]

\textbf{Tips:}
\begin{itemize}
    \item Use Green’s to convert complicated line integrals into double integrals.
    \item Ensure $C$ is positively oriented (counterclockwise).
\end{itemize}
\end{tcolorbox}

\begin{tcolorbox}[title=, colframe=brightyellow]
\textbf{10. Conservative Vector Fields}\\
\textit{Concept:} If $\mathbf{F}=\nabla f$, then line integrals depend only on endpoints.

\textbf{Tips:}
\begin{itemize}
    \item If $\nabla \times \mathbf{F}=0$ on a simply connected domain, $\mathbf{F}$ is conservative.
    \item Find $f$ by integrating $P$, $Q$, or $R$ and matching terms.
\end{itemize}
\end{tcolorbox}

\end{multicols}

\noindent \textbf{Review for the Final Exam:}\\
Now we extend to curls, divergence, and big theorems like Stokes’ and the Divergence Theorem. These connect line, surface, and volume integrals, giving powerful tools to solve complex problems.

\begin{multicols}{2}
\footnotesize

\begin{tcolorbox}[title=, colframe=brightblue]
\textbf{1. Curl of a Vector Field}\\
\[
\nabla \times \mathbf{F}=\left(\frac{\partial R}{\partial y}-\frac{\partial Q}{\partial z}\right)\mathbf{i}+\left(\frac{\partial P}{\partial z}-\frac{\partial R}{\partial x}\right)\mathbf{j}+\left(\frac{\partial Q}{\partial x}-\frac{\partial P}{\partial y}\right)\mathbf{k}.
\]

\textbf{Tips:}
\begin{itemize}
    \item Curl measures rotational tendency. 
    \item If curl is zero, consider potential functions.
\end{itemize}
\end{tcolorbox}

\begin{tcolorbox}[title=, colframe=brightgreen]
\textbf{2. Divergence of a Vector Field}\\
\[
\nabla \cdot \mathbf{F}= \frac{\partial P}{\partial x}+\frac{\partial Q}{\partial y}+\frac{\partial R}{\partial z}.
\]

\textbf{Tips:}
\begin{itemize}
    \item Positive divergence suggests a source; negative suggests a sink.
    \item Crucial in applying the Divergence Theorem.
\end{itemize}
\end{tcolorbox}

\begin{tcolorbox}[title=, colframe=brightpink]
\textbf{3. Parametric Planes}\\
\[
\mathbf{r}(u,v)=\mathbf{r}_0 + u\mathbf{r}_u + v\mathbf{r}_v.
\]

\textbf{Tips:}
\begin{itemize}
    \item Identify direction vectors from given points or known directions.
    \item Useful for constructing surfaces or parameterizing patches.
\end{itemize}
\end{tcolorbox}

\begin{tcolorbox}[title=, colframe=brightyellow]
\textbf{4. Parametric Surfaces}\\
\[
\mathbf{r}(u,v)=\langle x(u,v),y(u,v),z(u,v)\rangle.
\]

\textbf{Tips:}
\begin{itemize}
    \item Choose parameters that simplify the shape (e.g., spherical for spheres).
    \item Ensures easier integration over complex surfaces.
\end{itemize}
\end{tcolorbox}

\begin{tcolorbox}[title=, colframe=brightblue]
\textbf{5. Tangent Planes to Surfaces}\\
If $\mathbf{r}(u,v)$ describes a surface:
\[
\mathbf{r}_u=\frac{\partial \mathbf{r}}{\partial u},\quad \mathbf{r}_v=\frac{\partial \mathbf{r}}{\partial v}.
\]
$\mathbf{r}_u$ and $\mathbf{r}_v$ span the tangent plane.

\textbf{Tips:}
\begin{itemize}
    \item Evaluate partials at the point of interest.
    \item Use the tangent plane to approximate surface behavior locally.
\end{itemize}
\end{tcolorbox}

\begin{tcolorbox}[title=, colframe=brightgreen]
\textbf{6. Surface Integrals}\\
For a scalar function on a surface $S$:
\[
\iint_S f(x,y,z)\,dS.
\]
If $\mathbf{r}(u,v)$ parameterizes $S$, then $dS=\|\mathbf{r}_u\times\mathbf{r}_v\|du\,dv$.

\textbf{Tips:}
\begin{itemize}
    \item Always compute $\mathbf{r}_u\times \mathbf{r}_v$ first.
    \item Choose parameterizations that simplify this cross product.
\end{itemize}
\end{tcolorbox}

\begin{tcolorbox}[title=, colframe=brightpink]
\textbf{7. Surface Orientation}\\
Orientation matters for flux integrals. Usually choose outward or upward normals depending on context.

\textbf{Tips:}
\begin{itemize}
    \item Consistent orientation is key in applying Stokes’ or Divergence Theorems.
    \item If orientation isn’t specified, pick the most natural one (e.g., outward normal).
\end{itemize}
\end{tcolorbox}

\begin{tcolorbox}[title=, colframe=brightyellow]
\textbf{8. Flux Integral}\\
\[
\iint_S \mathbf{F}\cdot\mathbf{n}\,dS.
\]
Parameterizing $S$: 
\[
\mathbf{n}\,dS=(\mathbf{r}_u\times\mathbf{r}_v)\,du\,dv.
\]

\textbf{Tips:}
\begin{itemize}
    \item Check if $\mathbf{F}$ is simpler in another coordinate system.
    \item Sometimes applying the Divergence Theorem is easier than direct flux computation.
\end{itemize}
\end{tcolorbox}

\begin{tcolorbox}[title=, colframe=brightblue]
\textbf{9. Stokes' Theorem}\\
\[
\oint_C \mathbf{F}\cdot d\mathbf{r}=\iint_S (\nabla \times \mathbf{F})\cdot \mathbf{n}\,dS.
\]

\textbf{Tips:}
\begin{itemize}
    \item Convert difficult line integrals into (possibly simpler) surface integrals.
    \item Check if $C$ is the boundary of a nicely parameterized surface.
\end{itemize}
\end{tcolorbox}

\begin{tcolorbox}[title=, colframe=brightgreen]
\textbf{10. Triple Integrals}\\
\[
\iiint_W f(x,y,z)\,dV.
\]

\textbf{Tips:}
\begin{itemize}
    \item Use rectangular, cylindrical, or spherical coordinates as appropriate.
    \item Identify bounds from geometric descriptions.
\end{itemize}
\end{tcolorbox}

\begin{tcolorbox}[title=, colframe=brightpink]
\textbf{11. Cylindrical Coordinates}\\
\[
x=r\cos\theta,\, y=r\sin\theta,\, z=z,\, dV=r\,dr\,d\theta\,dz.
\]

\textbf{Tips:}
\begin{itemize}
    \item Ideal for cylinders, cones, and other rotationally symmetric objects.
    \item Align axis of symmetry with the $z$-axis.
\end{itemize}
\end{tcolorbox}

\begin{tcolorbox}[title=, colframe=brightyellow]
\textbf{12. Spherical Coordinates}\\
\[
x=\rho\sin\phi\cos\theta,\, y=\rho\sin\phi\sin\theta,\, z=\rho\cos\phi,\, dV=\rho^2\sin\phi\,d\rho\,d\phi\,d\theta.
\]

\textbf{Tips:}
\begin{itemize}
    \item Perfect for spheres, partial spheres, and radial symmetry.
    \item Identify which surfaces are spheres or spherical shells.
\end{itemize}
\end{tcolorbox}

\begin{tcolorbox}[title=, colframe=brightblue]
\textbf{13. Divergence Theorem}\\
\[
\iint_{S}\mathbf{F}\cdot\mathbf{n}\,dS=\iiint_{V}(\nabla \cdot \mathbf{F})\,dV.
\]

\textbf{Tips:}
\begin{itemize}
    \item If flux integral is complicated, try switching to a volume integral of divergence.
    \item Ensure $S$ is the closed boundary of $V$.
\end{itemize}
\end{tcolorbox}

\begin{tcolorbox}[title=, colframe=brightgreen]
\textbf{14. Remarks}\\
\textit{Concept:} Integration theorems and coordinate transformations are problem-solving shortcuts.

\textbf{Tips:}
\begin{itemize}
    \item Always consider symmetry and appropriate coordinate systems.
    \item Check conditions for theorems before applying them (e.g., vector field smoothness, domain type).
\end{itemize}
\end{tcolorbox}

\end{multicols}

\noindent By now, you've seen how all pieces connect: from vector basics to powerful integral theorems. Approaching problems systematically, choosing the right method, and visualizing scenarios will guide you to success.

\begin{multicols}{2}
\footnotesize

\noindent \textbf{Additional Strategies and Tips:}

\begin{tcolorbox}[title=, colframe=brightpink]
\textbf{Problem-Solving Processes:}
\begin{enumerate}
    \item \textbf{Identify What’s Asked:} Are you finding maxima, computing a line integral, or evaluating flux?
    \item \textbf{Check for Simplifications:} Is the vector field conservative? Can you apply Green’s/Stokes’/Divergence Theorem?
    \item \textbf{Pick Coordinates Wisely:} If symmetry is present, use cylindrical/spherical coordinates.
    \item \textbf{Relate Back to Basics:} Use gradients, curls, and divergences to transform integrals.
    \item \textbf{Verify Results:} Check dimensions, units, and boundary conditions for reasonableness.
\end{enumerate}
\end{tcolorbox}

\begin{tcolorbox}[title=, colframe=brightyellow]
\textbf{General Tips:}
\begin{itemize}
    \item \textbf{Draw Diagrams:} Visual aids clarify boundaries and orientations.
    \item \textbf{Use Gradients:} For surfaces defined implicitly, $\nabla F$ gives a normal.
    \item \textbf{Check for Curl/Divergence:} If $\nabla\times\mathbf{F}=0$, then $\mathbf{F}$ may be $\nabla f$. If $\nabla\cdot\mathbf{F}=0$, certain flux integrals simplify.
    \item \textbf{Practice with Examples:} Work through representative problems to build intuition.
\end{itemize}
\end{tcolorbox}

\end{multicols}

\end{document}
