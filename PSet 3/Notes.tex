\documentclass{report}

\input{preamble}
\input{macros}
\input{letterfonts}

\title{\Huge{Math 120}}
\author{\huge{PSet 2}}
\date{Sep 12 2024}

\begin{document}

\maketitle
\newpage% or \cleardoublepage
% \pdfbookmark[<level>]{<title>}{<dest>}
\pdfbookmark[section]{\contentsname}{toc}
\tableofcontents
\pagebreak

\chapter{}
\section{PSet 2}

\qs{}{
	\insertpng[0.5]{Prob1.png}
	\begin{enumerate}
		\item[(a)] $\vec{r}(t) = \langle \cos t, \sin t, t \rangle$
		\item[(b)] $\vec{r}(t) = t \langle \cos t, \sin t, t \rangle$
		\item[(c)] $\vec{r}(t) = \langle \cos t, \sin t, t^3 \rangle$
		\item[(d)] $\vec{r}(t) = \langle \cos(t^3), \sin(t^3), t^3 \rangle$
		\item[(e)] $\vec{r}(u) = \langle \cos u, \sin u, 1 + \sin(4u) \rangle$
		\item[(f)] $\vec{r}(u) = \langle \cos u, \sin u, 1 + 4\sin(u) \rangle$
		\item[(g)] $\vec{r}(t) = \langle 2\cos t, 1 + 4 \cos t, 3\cos t \rangle$
	\end{enumerate}
}

\sol{
}



\qs{}{
	2. Find a vector function that represents the curve of intersection of the plane $z = -2$ and the sphere $x^2 + (y-1)^2 + (z+1)^2 = 9$.
}
\sol{
	\[ x^{2} + (y-1)^{2} + ((-2) + 1)^{2} = 9 \] 
	\[ x^{2} + (y-1)^{2} = 8 \]
	\[ r = 2 \sqrt{2} \]
	\[ x(t) = 2\sqrt{2} \cos (t) \]
	\[ y - 1 = 2\sqrt{2} \sin(t) \Rightarrow y = 2\sqrt{t} \sin(t) + 1 \]
	\[ \vec{r}(t) = \langle 2\sqrt{2} \cos(t), 2\sqrt{2} \sin(t) + 1, -2 \rangle \] 
}
\qs{}{
	Consider the vector-valued function $\vec{r}_1(t) = \langle 2\sin t, -3\cos t, 0 \rangle$, $0 \leq t \leq 2\pi$.
	\begin{enumerate}
		\item[(a)] Sketch the plane curve given by $\vec{r}_1(t)$.
		\item[(b)] Compute and draw on your sketch from part (a) the position vector $\vec{r}_1 \left( \frac{2\pi}{3} \right)$ and the tangent vector $\vec{r}_1' \left( \frac{2\pi}{3} \right)$.
		\item[(c)] The vector-valued function $\vec{r}_2(t) = \langle 2\cos(3t), -3\sin(3t) \rangle$ parameterizes the same curve. Find the smallest $t^* > 0$ such that $\vec{r}_2(t^*) = \vec{r}_1 \left( \frac{2\pi}{3} \right)$, and compute $\vec{r}_2'(t^*)$. Explain how and why $\vec{r}_2'(t^*)$ differs from the tangent vector $\vec{r}_1' \left( \frac{2\pi}{3} \right)$ you computed in part (b). 
	\end{enumerate}
}

\sol{

}

\qs{}{
	Find parametric equations for the tangent line to the curve parameterized by 
	\[
	x = 2t + 1, \quad y = e^{t^2 - 4}, \quad z = \ln(1 + t^2)
	\]
	at the point $(5, 1, \ln 5)$.
}

\sol{
	\[ x(t) = 2(t) + 1 \quad y(t) = e^{t^{2} - 4} \quad z(t) = \ln(1 + t)^{2} \] 
	\[ x'(t) = 2 \quad y'(t) = 2te^{t^{2}} \quad z'(t) = \frac{2t}{1+t^{2}} \]
	\[ 5 = 2t + 1 \Rightarrow 4 = 2t \Rightarrow t = 2\]  
	\[ x'(2) = 2 \quad y'(2) =  4e^{4t} \quad z'(t) = \frac{4}{5} \] 
}

\newpage 

\qs{}{
	\begin{enumerate}
		\item[(a)] Evaluate the integral $\int \left( \tan t \, \hat{i} + \sin^2 t \, \hat{j} + \sec^2 t \, \tan t \, \hat{k} \right) \, dt$.
		\item[(b)] Suppose a particle is at the point $(-2, 1, 4)$ at time $t = 0$, and moves according to the velocity function $\vec{v}(t) = \tan t \, \hat{i} + \sin^2 t \, \hat{j} + \sec^2 t \, \tan t \, \hat{k}$. Find the particle's position at time $t = \frac{\pi}{4}$.
	\end{enumerate}
}

\sol{

}

\qs{}{
	Consider the curve parameterized by $\vec{r}(t) = \langle e^{2t}, e^{-2t}, \sqrt{8t} \rangle$, $0 \leq t \leq 1$.
	\begin{enumerate}
		\item[(a)] Sketch the projections of $\vec{r}(t)$ in the $xy$-, $zx$-, and $yz$-planes.
		\item[(b)] Find the length of the curve. \textit{Hint:} To integrate, you will need to write $\left( \frac{dx}{dt} \right)^2 + \left( \frac{dy}{dt} \right)^2 + \left( \frac{dz}{dt} \right)^2$ as a perfect square.
	\end{enumerate}
}

\sol{ \\
	a) 
	\[ x(t) = e^{2t} \quad y(t) = e^{-2t} \]
	\[ x \cdot y = e^{2t} \cdot e^{-2y} = 1\]
	\[ x(t) = e^{2t} \quad z(t) = \sqrt{8}t \]
	b)
	\[ L = \int_{0}^{1} ||\vec{r}(t)|| dt \]
	\[ \vec{r}(t) = \langle e^{2t}, e^{-2t}, \sqrt{8}t \rangle \]
	\[ \vec{r}'(t) = \left\langle \frac{d}{dt} \left(e^{2t} \right), \frac{d}{dt} \left(e^{-2t} \right) \frac{d}{dt} \left( \sqrt{8}t \right) \right\rangle \]   
	\[ \vec{r}'(t) = \left\langle 2e^{2t} , -2e^{-2t}, \sqrt{8} \right\rangle \]   
	\[ L = |\vec{r}'(t)| = \int_{0}^{1} \sqrt{\left(2e^{2t} \right)^{2} + \left(2e^{-2t} \right)^{2} + \left(\sqrt{8}\right)^{2}}\] 
	\[ \left(2e^{2t} \right)^{2} + \left(2e^{-2t} \right)^{2} + \left(\sqrt{8}\right)^{2} = \left(2e^{2t} \right)^{2} + \left(2e^{-2t} \right)^{2} + 8 \] 
	\[ \left(2e^{2t} \right)^{2} + \left(2e^{-2t} \right)^{2} + 8 = \left(2e^{2t} + 2e^{-2t} \right)^{2}  \] 
	\[ L = \int_{0}^{1} \sqrt{\left(2e^{2t} + 2e^{-2t} \right)^{2}} \Rightarrow \int_{0}^{1} \left(2e^{2t} + 2e^{-2t} \right) \] 
	\[ 2e^{2t} + 2e^{-2t} = 4\cosh(2t) \]
	\[ L = \int_{0}^{1} 4\cosh(2t) dt \Rightarrow 2\sinh(2t)\big|_{0}^{1} \rightarrow 2\sinh(2(1)) - 2\sinh(2(0))\] 
	\[ L =  2\sinh(2) - 2\sinh(0) \]  
}


\qs{}{
	Let $C$ be the curve of intersection of the cylinder $x^2 + y^2 = 4$ and the plane $2x + y + z = 4$.
	\begin{enumerate}
		\item[(a)] Find a parameterization of $C$.
		\item[(b)] Write down an integral for the length of $C$.
		\item[(c)] Find the length accurate to five decimal places by using Desmos: \url{https://www.desmos.com/calculator}. (Click on the keyboard icon, then “functions”, then “Misc”, to find the integral symbol.)
	\end{enumerate}}

\sol{
	\\
	a) 
	\[ x^{2} + y^{2} = 4 \quad \text{r} = 2\] 
	\[ x(t) = 2\cos(t) \quad y(t) = 2\sin(t)\] 
	\[ 2 (2cos(t)) + 2 \sin(t) + z = 4 \Rightarrow z = 4 - 4 \cos(t) - 2 \sin(t) \]
	\[ x^{2} + y^{2} = 4 \quad \text{r} = 2 \quad z(t) = 4 - 4\cos(t) - 2 \sin(t) \]
	\[ L = \int_{a}^{b} \sqrt{\left( \frac{d}{dt} x(t) \right)^{2} + \left( \frac{d}{dt} y(t) \right)^{2} + \left( \frac{d}{dt} z(t) \right)^{2} }\]  
	\[ \vec{r}'(t) = \langle \rangle \] 

}


\qs{}{
	Find the velocity and position vectors of a particle that has acceleration given by
	\[
	\vec{a}(t) = 2\hat{i} + 6t\hat{j} + 12t^2 \hat{k},
	\]
	and initial velocity and position given by
	\[
	\vec{v}(0) = \hat{i} \quad \text{and} \quad \vec{r}(0) = \hat{j} - \hat{k}.
	\]
}

\sol{ 
       
}

\qs{}{
	Consider the function $f(x, y) = \frac{\sqrt{y} - 3x}{\ln(4 - x^2 - y^2)}$.
	\begin{enumerate}
		\item[(a)] Find and sketch the domain of $f$.
		\item[(b)] On your sketch from part (a), mark where $f(x, y) = 0$, and indicate the region(s) where $f(x, y)$ is positive and negative.
	\end{enumerate}
}

\sol{
 
}

\newpage 
\qs{}{
	Here are several surfaces. \\
	Match each function with its graph. Justify your answers.
	\begin{enumerate}
		\item[(a)] $f(x, y) = x^2$
		\item[(b)] $f(x, y) = \sqrt{x^2 + y^2}$
		\item[(c)] $f(x, y) = e^{x^2 + y^2} - 1$
		\item[(d)] $f(x, y) = y \sin x$
		\item[(e)] $f(x, y) = \sin(x + y)$
		\item[(f)] $f(x, y) = \sin\left(\sqrt{x^2 + y^2}\right)$
	\end{enumerate}
}

\sol{
}

\qs{}{
	Draw a contour map of the function $f(x, y) = x^2 e^{-y}$ showing several level curves.
}

\newpage 

\qs{}{
	Match the function with its graph (labeled A-F below) and with its contour map (labeled I-VI). Give reasons for your choices.
	\begin{enumerate}
		\item[(a)] $z = e^x \cos y$
		\item[(b)] $z = \sin x - \sin y$
		\item[(c)] $z = \frac{x - y}{1 + x^2 + y^2}$
	\end{enumerate}
	\insertpng[0.6]{q12.png}
}

\end{document}
