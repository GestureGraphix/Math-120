\documentclass[9pt]{article}
\usepackage{amsmath, amssymb, graphicx, geometry, multicol, tcolorbox, titlesec, xcolor, titling}
\geometry{a4paper, margin=0.1in} % Reduce margin to 0.1 inches
\setlength{\parskip}{0pt}
\setlength{\parindent}{0pt}
\linespread{0.7} % Reduce line spacing further
\titlespacing{\section}{0pt}{1pt}{1pt}
\setlength{\columnsep}{0.2cm} % Reduce column separation

\definecolor{lightblue}{HTML}{D3E6FE}
\definecolor{lightgreen}{HTML}{E2F0D9}
\definecolor{lightpink}{HTML}{FDE7E8}
\definecolor{lightyellow}{HTML}{FFF9DB}

% Reduce padding inside tcolorbox
\tcbset{
    boxrule=0.3mm, % Box border thickness
    colback=white,
    fonttitle=\bfseries\footnotesize,
    sharp corners,
    coltitle=black,
    top=0pt,
    bottom=0pt,
    left=0.3mm,
    right=0.3mm,
    boxsep=0.2mm, % Space between box content and border
    arc=0.3mm % Rounded corners
}

\setlength{\droptitle}{-70pt} % Remove space above the title

\title{\footnotesize \textbf{Math 120 Cheat Sheet}} % Reduce title size
\date{} % Disable date

\begin{document}
\maketitle
\vspace{-15pt} % Remove space below the title
\begin{multicols}{2}
\footnotesize % Reduce font size for the content

% Directional Derivatives and Gradient Vector
\begin{tcolorbox}[title=\textbf{Directional Derivatives and Gradient Vector}, colframe=lightblue]
\textbf{Directional Derivative:} 	
\[ D_{\mathbf{u}} f(x, y) = \nabla f(x, y) \cdot \mathbf{u} \]
where \( \mathbf{u} \) is a unit vector. \\
\textbf{Gradient Vector:}
\[ \nabla f(x, y) = \langle f_x, f_y \rangle \]
\textbf{Properties:}
- \( \nabla f \) points in the direction of max increase of \( f \). \\
- \( \nabla f \) is perpendicular to level curves of \( f \). \\
\textbf{Max Rate of Change:}
\[ \text{Max Rate} = |\nabla f(x, y)| \]
\end{tcolorbox}

% Maximum and Minimum Values
\begin{tcolorbox}[title=\textbf{Maximum and Minimum Values}, colframe=lightgreen]
\textbf{Second Derivative Test:} \\
Compute \( D = f_{xx} f_{yy} - (f_{xy})^2 \). \\
- If \( D > 0 \) and \( f_{xx} > 0 \), local min at \( (a, b) \). \\
- If \( D > 0 \) and \( f_{xx} < 0 \), local max at \( (a, b) \). \\
- If \( D < 0 \), saddle point at \( (a, b) \). \\
- If \( D = 0 \), test inconclusive. \\
\textbf{Critical Points:} Solve \( f_x = 0 \), \( f_y = 0 \).
\end{tcolorbox}

% Lagrange Multipliers
\begin{tcolorbox}[title=\textbf{Lagrange Multipliers}, colframe=lightpink]
\textbf{Purpose:} Find extrema of \( f(x, y) \) subject to \( g(x, y) = 0 \). \\
\textbf{Method:}
1. \( \nabla f = \lambda \nabla g \) \\
   \( f_x = \lambda g_x, \quad f_y = \lambda g_y \) \\
2. Include \( g(x, y) = 0 \). \\
3. Solve for \( x, y, \lambda \). \\
4. Evaluate \( f \) at solutions.
\end{tcolorbox}

% Double Integrals over Rectangles
\begin{tcolorbox}[title=\textbf{Double Integrals over Rectangles}, colframe=lightyellow]
\textbf{Definition:}
\[ \iint_R f(x, y) \, dA = \int_a^b \int_c^d f(x, y) \, dy \, dx \]
where \( R = [a, b] \times [c, d] \). \\
\textbf{Fubini's Theorem:} If \( f \) is continuous:
\[ \iint_R f(x, y) \, dA = \int_c^d \int_a^b f(x, y) \, dx \, dy \]
\end{tcolorbox}

% Average Value of a Function
\begin{tcolorbox}[title=\textbf{Average Value of a Function}, colframe=lightpink]
\textbf{Average Value over \( R \):}
\[ f_{\text{avg}} = \frac{1}{(b - a)(d - c)} \iint_R f(x, y) \, dA \]
\end{tcolorbox}

% Double Integrals over General Regions
\begin{tcolorbox}[title=\textbf{Double Integrals over General Regions}, colframe=lightblue]
\textbf{Type I Region (Vertical):}
\[ D = \{ (x, y) \mid a \leq x \leq b,\ g_1(x) \leq y \leq g_2(x) \} \]
\[ \iint_D f \, dA = \int_a^b \int_{g_1(x)}^{g_2(x)} f \, dy \, dx \]
\textbf{Type II Region (Horizontal):}
\[ D = \{ (x, y) \mid c \leq y \leq d,\ h_1(y) \leq x \leq h_2(y) \} \]
\[ \iint_D f \, dA = \int_c^d \int_{h_1(y)}^{h_2(y)} f \, dx \, dy \]
\end{tcolorbox}

% Double Integrals in Polar Coordinates
\begin{tcolorbox}[title=\textbf{Double Integrals in Polar Coordinates}, colframe=lightgreen]
\textbf{When to Convert:}
- Circular regions or integrands with \( x^2 + y^2 \). \\
- When \( f(x, y) \) is easier to integrate in polar form. \\
\textbf{Transformation:}
\[ x = r \cos \theta,\ y = r \sin \theta \]
\[ dA = r \, dr \, d\theta \]
\textbf{Integral:}
\[ \iint_D f(x, y) \, dA = \int_{\theta_1}^{\theta_2} \int_{r_1(\theta)}^{r_2(\theta)} f(r \cos \theta, r \sin \theta) r \, dr \, d\theta \]
\textbf{Tips:}
- Adjust limits of \( r \) and \( \theta \) to match \( D \). \\
- Common for circles, sectors, annuli.
\end{tcolorbox}

% Vector Fields
\begin{tcolorbox}[title=\textbf{Vector Fields}, colframe=lightpink]
\textbf{Definition:}
\( \mathbf{F}(x, y) = P(x, y) \mathbf{i} + Q(x, y) \mathbf{j} \) \\
\textbf{Gradient Field:} \( \mathbf{F} = \nabla f \) \\
\textbf{Conservative Field:} \( \mathbf{F} = \nabla f \). \\
\textbf{Curl in \( \mathbb{R}^2 \):}
\[ \text{curl } \mathbf{F} = Q_x - P_y \]
\end{tcolorbox}

% Line Integrals
\begin{tcolorbox}[title=\textbf{Line Integrals}, colframe=lightyellow]
\textbf{When to Use:}
- To compute work done by a force field along a path. \\
- To integrate a scalar function over a curve (mass, length). \\
\textbf{Types of Line Integrals:}
- \textbf{Scalar Line Integral} (with respect to arc length): \( \int_C f \, ds \) \\
- \textbf{Vector Line Integral} (work): \( \int_C \mathbf{F} \cdot d\mathbf{r} \) \\
\textbf{How to Compute:}
1. Parameterize \( C \) by \( \mathbf{r}(t) \), \( t \in [a, b] \). \\
2. Compute \( \mathbf{r}'(t) \) and \( |\mathbf{r}'(t)| \) if necessary. \\
3. Substitute into the integral:
   - Scalar: \( \int_a^b f(\mathbf{r}(t)) |\mathbf{r}'(t)| dt \) \\
   - Vector: \( \int_a^b \mathbf{F}(\mathbf{r}(t)) \cdot \mathbf{r}'(t) dt \) \\
\textbf{When to Convert to Polar Coordinates:}
- When \( C \) is a circle or curve naturally described in polar coordinates. \\
- When integrand involves \( x^2 + y^2 \) or trigonometric functions. \\
\textbf{Converting to Polar Coordinates:}
- Use \( x = r \cos \theta \), \( y = r \sin \theta \). \\
- Express \( \mathbf{F} \) and \( d\mathbf{r} \) in terms of \( r \) and \( \theta \). \\
\textbf{Tips:}
- Choose the simplest parameterization possible. \\
- For circles: \( x = a \cos t,\ y = a \sin t \), \( t \in [0, 2\pi] \). \\
- For straight lines, use linear parameterizations. \\
\textbf{Applications:}
- Calculating work, circulation, or flux. \\
- Finding mass of a wire with variable density.
\end{tcolorbox}

% Fundamental Theorem for Line Integrals
\begin{tcolorbox}[title=\textbf{Fundamental Theorem for Line Integrals}, colframe=lightblue]
If \( \mathbf{F} = \nabla f \), then:
\[ \int_C \mathbf{F} \cdot d\mathbf{r} = f(B) - f(A) \]
\textbf{Conservative Field Test:}
- If \( P_y = Q_x \), then \( \mathbf{F} \) is conservative.
\end{tcolorbox}

% Green's Theorem
\begin{tcolorbox}[title=\textbf{Green's Theorem}, colframe=lightgreen]
\textbf{When to Use:}
- To convert a difficult line integral into a double integral (or vice versa). \\
- When dealing with circulation or flux over a closed curve \( C \) in the plane. \\
- \( C \) must be a positively oriented (counter-clockwise) simple closed curve. \\
\textbf{Statement:}
\[ \oint_C P\,dx + Q\,dy = \iint_D \left( \frac{\partial Q}{\partial x} - \frac{\partial P}{\partial y} \right) dA \]
\textbf{Applications:}
- Calculating area: \( \text{Area} = \frac{1}{2} \oint_C x\, dy - y\, dx \) \\
- Computing work done by a force field around a closed path. \\
\textbf{How to Apply:}
1. Verify conditions (closed curve, positive orientation). \\
2. Identify \( P(x, y) \) and \( Q(x, y) \). \\
3. Compute \( Q_x - P_y \). \\
4. Evaluate \( \iint_D (Q_x - P_y) dA \). \\
\textbf{Tips:}
- Simplify the integrand before integrating. \\
- Choose the order of integration based on \( D \). \\
- For circular regions, consider polar coordinates.
\end{tcolorbox}

% Example Problem: Evaluating a Line Integral Using Green's Theorem
\begin{tcolorbox}[title=\textbf{Example: Evaluating a Line Integral Using Green's Theorem}, colframe=lightpink]
\textbf{Problem:} Let \( \vec{F}(x, y) = \left( x^2 + y^2,\ \frac{1}{3} x^3 + 2xy + x \right) \). Compute \( \int_C \vec{F} \cdot d\vec{r} \) along the semicircle \( C \) defined by \( x^2 + y^2 = 16 \) for \( y \geq 0 \).

\textbf{Solution:}
1. \textbf{Close the Curve:}
   - Add the interval from \( x = 4 \) to \( x = -4 \) along \( y = 0 \) to form a closed curve \( C' \).
2. \textbf{Apply Green's Theorem:}
   \[ \int_{C'} \vec{F} \cdot d\vec{r} = \iint_D \left( \frac{\partial Q}{\partial x} - \frac{\partial P}{\partial y} \right) dx dy \]
   - Compute \( \frac{\partial Q}{\partial x} = x^2 + 2y + 1 \), \( \frac{\partial P}{\partial y} = 2y \).
   - Integrand: \( x^2 + 1 \).
3. \textbf{Compute the Double Integral:}
   \[ \iint_D (x^2 + 1)\, dx dy = \int_{0}^{\pi} \int_{0}^{4} (r^2 \cos^2 \theta + 1)\, r dr d\theta \]
   - Evaluate:
     \[ \int_{0}^{\pi} \left( \frac{64}{4} \cos^2 \theta + 8 \right) d\theta = 40\pi \]
4. \textbf{Compute Integral over Straight Segment:}
   - Parametrize: \( x = t,\ y = 0,\ t \in [4, -4] \).
   - \( \vec{F} \cdot d\vec{r} = x^2 dx \).
   - Integral: \( \int_{4}^{-4} x^2 dx = -\frac{128}{3} \).
5. \textbf{Find Integral along \( C \):}
   \[ \int_C \vec{F} \cdot d\vec{r} = \int_{C'} \vec{F} \cdot d\vec{r} - \left( -\frac{128}{3} \right) = 40\pi + \frac{128}{3} \]

\textbf{Answer:} \( \int_C \vec{F} \cdot d\vec{r} = 40\pi + \dfrac{128}{3} \)
\end{tcolorbox}

% Trigonometric Identities
\begin{tcolorbox}[title=\textbf{Trigonometric Identities}, colframe=lightpink]
\textbf{Pythagorean:}
\[ \sin^2 \theta + \cos^2 \theta = 1 \]
\[ 1 + \tan^2 \theta = \sec^2 \theta \]
\textbf{Double Angle:}
\[ \sin 2\theta = 2 \sin \theta \cos \theta \]
\[ \cos 2\theta = \cos^2 \theta - \sin^2 \theta \]
\textbf{Sum and Difference:}
\[ \sin (A \pm B) = \sin A \cos B \pm \cos A \sin B \]
\[ \cos (A \pm B) = \cos A \cos B \mp \sin A \sin B \]
\end{tcolorbox}

% Common Derivatives and Integrals
\begin{tcolorbox}[title=\textbf{Common Derivatives and Integrals}, colframe=lightyellow]
\textbf{Derivatives:}
\[ \frac{d}{dx} e^{ax} = a e^{ax} \]
\[ \frac{d}{dx} \ln x = \frac{1}{x} \]
\[ \frac{d}{dx} \sqrt{x} = \frac{1}{2 \sqrt{x}} \]
\[ \frac{d}{dx} \sin ax = a \cos ax \]
\[ \frac{d}{dx} \cos ax = -a \sin ax \]
\[ \frac{d}{dx} \tan ax = a \sec^2 ax \]
\textbf{Integrals:}
\[ \int e^{ax} dx = \frac{1}{a} e^{ax} + C \]
\[ \int \frac{1}{x} dx = \ln |x| + C \]
\[ \int \sqrt{x}\, dx = \frac{2}{3} x^{3/2} + C \]
\[ \int \sin ax\, dx = -\frac{1}{a} \cos ax + C \]
\[ \int \cos ax\, dx = \frac{1}{a} \sin ax + C \]
\[ \int \sec^2 ax\, dx = \frac{1}{a} \tan ax + C \]
\textbf{Techniques:}
- \textbf{Substitution:} Let \( u = g(x) \). \\
- \textbf{Integration by Parts:} \( \int u\, dv = uv - \int v\, du \).
\end{tcolorbox}

% Jacobian Determinant
\begin{tcolorbox}[title=\textbf{Jacobian Determinant}, colframe=lightgreen]
Transformation from \( (x, y) \) to \( (u, v) \):
\[ J = \left| \frac{\partial(x, y)}{\partial(u, v)} \right| = 
\begin{vmatrix}
x_u & x_v \\
y_u & y_v
\end{vmatrix}
= x_u y_v - x_v y_u \]
\textbf{Use in Integration:}
\[ \iint_D f(x, y)\, dA = \iint_{D'} f(x(u, v), y(u, v)) |J|\, du\, dv \]
\end{tcolorbox}

% Conservative Vector Fields
\begin{tcolorbox}[title=\textbf{Conservative Vector Fields}, colframe=lightblue]
\textbf{Tests:}
- If \( P_y = Q_x \), \( \mathbf{F} \) is conservative. \\
\textbf{Finding Potential \( f \):}
1. Integrate \( P \) w.r.t \( x \) to get \( f \). \\
2. Differentiate \( f \) w.r.t \( y \), compare with \( Q \). \\
3. Adjust \( f \) as needed.
\end{tcolorbox}

% Coordinate Transformations
\begin{tcolorbox}[title=\textbf{Coordinate Transformations}, colframe=lightgreen]
\textbf{Polar to Cartesian:}
\[ x = r \cos \theta,\ y = r \sin \theta \]
\textbf{Cartesian to Polar:}
\[ r = \sqrt{x^2 + y^2},\ \theta = \arctan \left( \frac{y}{x} \right) \]
\textbf{Cylindrical:}
\[ x = r \cos \theta,\ y = r \sin \theta,\ z = z \]
\textbf{Spherical:}
\[ x = \rho \sin \phi \cos \theta \]
\[ y = \rho \sin \phi \sin \theta \]
\[ z = \rho \cos \phi \]
\end{tcolorbox}

% Derivative Rules
\begin{tcolorbox}[title=\textbf{Derivative Rules}, colframe=lightblue]
\[ \frac{d}{dx} c = 0 \]
\[ \frac{d}{dx} x^n = nx^{n-1} \]
\[ \frac{d}{dx} [cf(x)] = c f'(x) \]
\[ \frac{d}{dx} [f \pm g] = f' \pm g' \]
\[ \frac{d}{dx} [fg] = f'g + fg' \]
\[ \frac{d}{dx} \left( \frac{f}{g} \right) = \frac{f'g - fg'}{g^2} \]
\[ \frac{d}{dx} f(g(x)) = f'(g(x)) g'(x) \]
\end{tcolorbox}

% Example Problem: Computing Area Between Circles
\begin{tcolorbox}[title=\textbf{Example: Computing Area Between Circles}, colframe=lightpink]
    \textbf{Problem:} Compute the area of the region \( R \) with \( y \geq 0 \) outside \( C_2 \) and inside \( C_1 \), where:
    \[ C_1: (x - 1)^2 + y^2 = 1, \quad C_2: x^2 + y^2 = 2 \]
    
    \textbf{Solution Steps:}
    1. \textbf{Express the curves in polar coordinates:}
       - For \( C_1 \):
         \[ (x - 1)^2 + y^2 = 1 \]
         Substitute \( x = r \cos \theta \), \( y = r \sin \theta \):
         \[ (r \cos \theta - 1)^2 + (r \sin \theta)^2 = 1 \]
         Simplify:
         \[ r^2 - 2 r \cos \theta = 0 \]
         So \( r = 0 \) or \( r = 2 \cos \theta \).
         Since \( r = 0 \) is trivial, \( C_1 \) corresponds to \( r = 2 \cos \theta \).
    
       - For \( C_2 \):
         \[ x^2 + y^2 = 2 \]
         In polar coordinates:
         \[ r^2 = 2 \]
         So \( r = \sqrt{2} \).
    
    2. \textbf{Determine the limits of integration:}
       - Find the angle \( \theta \) where the curves intersect:
         \[ r = \sqrt{2} = 2 \cos \theta \]
         \[ \cos \theta = \frac{\sqrt{2}}{2} \]
         \[ \theta = \frac{\pi}{4} \]
       - Therefore, \( \theta \) ranges from \( 0 \) to \( \frac{\pi}{4} \).
    
    3. \textbf{Set up the double integral in polar coordinates:}
       \[ A = \int_{\theta = 0}^{\frac{\pi}{4}} \int_{r = \sqrt{2}}^{r = 2 \cos \theta} r \, dr \, d\theta \]
    
    4. \textbf{Compute the integral:}
       - Integrate with respect to \( r \):
         \[ \int_{r = \sqrt{2}}^{r = 2 \cos \theta} r \, dr = \left[ \frac{1}{2} r^2 \right]_{r = \sqrt{2}}^{r = 2 \cos \theta} = \frac{1}{2} \left( (2 \cos \theta)^2 - (\sqrt{2})^2 \right) = \frac{1}{2} \left( 4 \cos^2 \theta - 2 \right) \]
       - Integrate with respect to \( \theta \):
         \[ A = \int_{0}^{\frac{\pi}{4}} \left( 2 \cos^2 \theta - 1 \right) d\theta \]
    
    5. \textbf{Simplify and evaluate the integral:}
       - Use the identity \( \cos 2\theta = 2 \cos^2 \theta - 1 \):
         \[ 2 \cos^2 \theta - 1 = \cos 2\theta \]
       - Therefore:
         \[ A = \int_{0}^{\frac{\pi}{4}} \cos 2\theta \, d\theta = \left[ \frac{1}{2} \sin 2\theta \right]_{0}^{\frac{\pi}{4}} = \frac{1}{2} \left( \sin \frac{\pi}{2} - \sin 0 \right) = \frac{1}{2} (1 - 0) = \frac{1}{2} \]
    
    6. \textbf{Final Answer:}
       - The area \( A = \frac{1}{2} \) square units.
\end{tcolorbox} 


% Example Problem: Evaluating a Line Integral
\begin{tcolorbox}[title=\textbf{Example: Evaluating a Line Integral}, colframe=lightpink]
    \textbf{Problem:} Evaluate the line integral \( \int_C \vec{F} \cdot d\vec{r} \) where \( \vec{F} = \langle 4xy^2 + 9x^2, 3e^y + 4x^2 y \rangle \) and \( C \) is the part of the parabola \( 4y = x^2 \) from \( (2, 1) \) to \( (-2, 1) \).
    
    \textbf{Solution:}
    1. \textbf{Verify if the Vector Field is Conservative:}
       - Let \( P = 4xy^2 + 9x^2 \) and \( Q = 3e^y + 4x^2 y \).
       - Compute \( \frac{\partial P}{\partial y} = 8xy \) and \( \frac{\partial Q}{\partial x} = 8xy \).
       - Since \( \frac{\partial P}{\partial y} = \frac{\partial Q}{\partial x} \), \( \vec{F} \) is conservative.
    
    2. \textbf{Find the Potential Function \( f(x, y) \):}
       - \( f_x = 4xy^2 + 9x^2 \implies f(x, y) = \int (4xy^2 + 9x^2) \, dx = 2x^2 y^2 + 3x^3 + g(y) \).
       - Differentiate with respect to \( y \): \( f_y = 4x^2 y + g'(y) \).
       - Set equal to \( Q \): \( 4x^2 y + g'(y) = 3e^y + 4x^2 y \implies g'(y) = 3e^y \).
       - Integrate \( g'(y) \): \( g(y) = 3e^y \).
       - Potential function: \( f(x, y) = 2x^2 y^2 + 3x^3 + 3e^y \).
    
    3. \textbf{Apply the Fundamental Theorem for Line Integrals:}
       \[ \int_C \vec{F} \cdot d\vec{r} = f(-2, 1) - f(2, 1). \]
       - Compute \( f(2, 1) = 2(2)^2(1)^2 + 3(2)^3 + 3e^1 = 8 + 24 + 3e \).
       - Compute \( f(-2, 1) = 2(-2)^2(1)^2 + 3(-2)^3 + 3e^1 = 8 - 24 + 3e \).
       - Result: \[ \int_C \vec{F} \cdot d\vec{r} = (8 - 24 + 3e) - (8 + 24 + 3e) = -48. \]
    
    4. \textbf{Path Independence Verification:}
       - Choose the line segment \( C' \) from \( (2, 1) \) to \( (-2, 1) \) and parametrize by \( \vec{r}(t) = \langle -t, 1 \rangle \) with \( -2 \leq t \leq 2 \).
       - \( d\vec{r} = \langle -1, 0 \rangle dt \) and \( \vec{F}(\vec{r}(t)) = \langle 4t + 9t^2, 3e + 4t^2 \rangle \).
       - \( \vec{F} \cdot d\vec{r} = -4t - 9t^2 \).
    
    5. \textbf{Evaluate the Integral Directly:}
       \[ \int_{-2}^{2} (-4t - 9t^2) \, dt = \left[ -2t^2 - 3t^3 \right]_{-2}^{2} = -48. \]
    
    \textbf{Answer:} \( \int_C \vec{F} \cdot d\vec{r} = -48 \).
\end{tcolorbox}

\end{multicols}
\end{document}
