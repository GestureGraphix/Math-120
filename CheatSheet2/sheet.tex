\documentclass[2pt]{article}
\usepackage{amsmath, amssymb, graphicx, geometry, multicol, tcolorbox, titlesec, parskip, xcolor}
\geometry{a4paper, margin=0.3in} % Reduce margin to 0.3 inches
\setlength{\parskip}{0pt}
\setlength{\parindent}{0pt}
\linespread{0.85} % Reduce line spacing further
\titlespacing{\section}{0pt}{1pt}{1pt}
\definecolor{lightblue}{HTML}{D3E6FE}
\definecolor{lightgreen}{HTML}{E2F0D9}
\definecolor{lightpink}{HTML}{FDE7E8}
\definecolor{lightyellow}{HTML}{FFF9DB}

% Reduce padding inside tcolorbox
\tcbset{
    boxrule=0.5mm, % Box border thickness
    colback=white,
    fonttitle=\bfseries,
    sharp corners,
    coltitle=black,
    top=0pt,
    bottom=0pt,
    left=1mm,
    right=1mm,
    boxsep=0.5mm, % Space between box content and border
    arc=1mm % Rounded corners
}

\title{\small \textbf{Math 120 Cheat Sheet}} % Reduce title size
\date{} % Disable date

\begin{document}
\maketitle
\vspace{-10pt} % Adjust the value as needed to remove space below the title
\begin{multicols}{2}

% Directional Derivatives and Gradient Vector
\begin{tcolorbox}[title=\textbf{Directional Derivatives and Gradient Vector}, colframe=lightblue]
    \textbf{Directional Derivative:} 	
    \[ D_{\mathbf{u}} f(x, y) = \nabla f(x, y) \cdot \mathbf{u} \]
    where \( \mathbf{u} \) is a unit vector. \\
    \textbf{Gradient Vector:}
    \[ \nabla f(x, y) = \left\langle f_x(x, y), \, f_y(x, y) \right\rangle \]
    \textbf{Properties:}
    - \( \nabla f \) points in the direction of maximum increase of \( f \). \\
    - \( \nabla f \) is perpendicular to level curves (surfaces) of \( f \). \\
    \textbf{Maximum Rate of Change:}
    \[ \text{Max Rate} = |\nabla f(x, y)| \]
\end{tcolorbox}

% Maximum and Minimum Values
\begin{tcolorbox}[title=\textbf{Maximum and Minimum Values}, colframe=lightgreen]
    \textbf{Second Derivative Test:} \\
    Compute \( D = f_{xx}(a,b) f_{yy}(a,b) - [f_{xy}(a,b)]^2 \). \\
    - If \( D > 0 \) and \( f_{xx}(a,b) > 0 \), then local minimum at \( (a, b) \). \\
    - If \( D > 0 \) and \( f_{xx}(a,b) < 0 \), then local maximum at \( (a, b) \). \\
    - If \( D < 0 \), saddle point at \( (a, b) \). \\
    - If \( D = 0 \), test is inconclusive. \\
    \textbf{Critical Points:} Solve \( f_x = 0 \) and \( f_y = 0 \).
\end{tcolorbox}

% Lagrange Multipliers
\begin{tcolorbox}[title=\textbf{Lagrange Multipliers}, colframe=lightpink]
    To find extrema of \( f(x, y, z) \) subject to constraint \( g(x, y, z) = 0 \):
    \[ \nabla f = \lambda \nabla g \]
    Solve:
    \[ f_x = \lambda g_x, \quad f_y = \lambda g_y, \quad f_z = \lambda g_z \]
    \[ g(x, y, z) = 0 \]
    \textbf{For Two Variables:}
    \[ f_x = \lambda g_x, \quad f_y = \lambda g_y, \quad g(x, y) = 0 \]
\end{tcolorbox}

% Double Integrals over Rectangles
\begin{tcolorbox}[title=\textbf{Double Integrals over Rectangles}, colframe=lightyellow]
    \textbf{Definition:}
    \[ \iint_R f(x, y) \, dA = \int_a^b \int_c^d f(x, y) \, dy \, dx \]
    where \( R = [a, b] \times [c, d] \). \\
    \textbf{Fubini's Theorem:} If \( f \) is continuous on \( R \):
    \[ \iint_R f(x, y) \, dA = \int_c^d \int_a^b f(x, y) \, dx \, dy \]
\end{tcolorbox}

% Double Integrals over General Regions
\begin{tcolorbox}[title=\textbf{Double Integrals over General Regions}, colframe=lightblue]
    \textbf{Type I Region (Vertical):}
    \[ D = \{ (x, y) \mid a \leq x \leq b, \ g_1(x) \leq y \leq g_2(x) \} \]
    \[ \iint_D f(x, y) \, dA = \int_a^b \int_{g_1(x)}^{g_2(x)} f(x, y) \, dy \, dx \]
    \textbf{Type II Region (Horizontal):}
    \[ D = \{ (x, y) \mid c \leq y \leq d, \ h_1(y) \leq x \leq h_2(y) \} \]
    \[ \iint_D f(x, y) \, dA = \int_c^d \int_{h_1(y)}^{h_2(y)} f(x, y) \, dx \, dy \]
\end{tcolorbox}

% Double Integrals in Polar Coordinates
\begin{tcolorbox}[title=\textbf{Double Integrals in Polar Coordinates}, colframe=lightgreen]
    \textbf{Transformation:}
    \[ x = r \cos \theta, \quad y = r \sin \theta \]
    \textbf{Jacobian:}
    \[ dA = r \, dr \, d\theta \]
    \textbf{Integral:}
    \[ \iint_D f(x, y) \, dA = \int_{\theta_1}^{\theta_2} \int_{r_1(\theta)}^{r_2(\theta)} f(r \cos \theta, r \sin \theta) \, r \, dr \, d\theta \]
\end{tcolorbox}

% Vector Fields
\begin{tcolorbox}[title=\textbf{Vector Fields}, colframe=lightpink]
    \textbf{Definition:} \\
    In \( \mathbb{R}^2 \):
    \[ \mathbf{F}(x, y) = P(x, y) \, \mathbf{i} + Q(x, y) \, \mathbf{j} \]
    \textbf{Gradient Field:} \( \mathbf{F} = \nabla f \) \\
    \textbf{Conservative Field:} If \( \mathbf{F} = \nabla f \), then \( \mathbf{F} \) is conservative. \\
    \textbf{Curl in \( \mathbb{R}^2 \):}
    \[ \text{curl } \mathbf{F} = \frac{\partial Q}{\partial x} - \frac{\partial P}{\partial y} \]
\end{tcolorbox}

% Line Integrals
\begin{tcolorbox}[title=\textbf{Line Integrals}, colframe=lightyellow]
    \textbf{Scalar Function:}
    \[ \int_C f(x, y) \, ds = \int_a^b f(x(t), y(t)) \, |\mathbf{r}'(t)| \, dt \]
    \textbf{Vector Field:}
    \[ \int_C \mathbf{F} \cdot d\mathbf{r} = \int_C P \, dx + Q \, dy \]
    \[ = \int_a^b \left[ P(x(t), y(t)) x'(t) + Q(x(t), y(t)) y'(t) \right] dt \]
\end{tcolorbox}

% Fundamental Theorem for Line Integrals
\begin{tcolorbox}[title=\textbf{Fundamental Theorem for Line Integrals}, colframe=lightblue]
    If \( \mathbf{F} = \nabla f \) is conservative, then for any curve \( C \) from \( A \) to \( B \):
    \[ \int_C \mathbf{F} \cdot d\mathbf{r} = f(B) - f(A) \]
    \textbf{Conservative Field Test:}
    - In \( \mathbb{R}^2 \), if \( \frac{\partial P}{\partial y} = \frac{\partial Q}{\partial x} \) in a simply connected domain, then \( \mathbf{F} \) is conservative.
\end{tcolorbox}

% Green's Theorem
\begin{tcolorbox}[title=\textbf{Green's Theorem}, colframe=lightgreen]
    For a positively oriented, piecewise smooth, simple closed curve \( C \) enclosing region \( D \):
    \[ \oint_C P \, dx + Q \, dy = \iint_D \left( \frac{\partial Q}{\partial x} - \frac{\partial P}{\partial y} \right) \, dA \]
    \textbf{Area Using Green's Theorem:}
    \[ \text{Area} = \frac{1}{2} \oint_C x \, dy - y \, dx \]
\end{tcolorbox}

% Trigonometric Identities
\begin{tcolorbox}[title=\textbf{Trigonometric Identities}, colframe=lightpink]
    \textbf{Pythagorean Identities:}
    \[ \sin^2 \theta + \cos^2 \theta = 1 \]
    \[ 1 + \tan^2 \theta = \sec^2 \theta \]
    \[ 1 + \cot^2 \theta = \csc^2 \theta \]
    \textbf{Double Angle Formulas:}
    \[ \sin 2\theta = 2 \sin \theta \cos \theta \]
    \[ \cos 2\theta = \cos^2 \theta - \sin^2 \theta = 2 \cos^2 \theta - 1 = 1 - 2 \sin^2 \theta \]
    \[ \tan 2\theta = \frac{2 \tan \theta}{1 - \tan^2 \theta} \]
    \textbf{Sum and Difference Formulas:}
    \[ \sin (A \pm B) = \sin A \cos B \pm \cos A \sin B \]
    \[ \cos (A \pm B) = \cos A \cos B \mp \sin A \sin B \]
    \[ \tan (A \pm B) = \frac{\tan A \pm \tan B}{1 \mp \tan A \tan B} \]
    \textbf{Half-Angle Formulas:}
    \[ \sin^2 \theta = \frac{1 - \cos 2\theta}{2} \]
    \[ \cos^2 \theta = \frac{1 + \cos 2\theta}{2} \]
    \textbf{Product to Sum:}
    \[ \sin A \sin B = \frac{1}{2} [ \cos(A - B) - \cos(A + B) ] \]
    \[ \cos A \cos B = \frac{1}{2} [ \cos(A - B) + \cos(A + B) ] \]
    \[ \sin A \cos B = \frac{1}{2} [ \sin(A + B) + \sin(A - B) ] \]
\end{tcolorbox}

% Common Derivatives and Integrals
\begin{tcolorbox}[title=\textbf{Common Derivatives and Integrals}, colframe=lightyellow]
    \textbf{Derivatives:}
    \[ \frac{d}{dx} [e^{ax}] = a e^{ax} \]
    \[ \frac{d}{dx} [\ln x] = \frac{1}{x} \]
    \[ \frac{d}{dx} [\sin ax] = a \cos ax \]
    \[ \frac{d}{dx} [\cos ax] = -a \sin ax \]
    \[ \frac{d}{dx} [\tan ax] = a \sec^2 ax \]
    \textbf{Integrals:}
    \[ \int e^{ax} \, dx = \frac{1}{a} e^{ax} + C \]
    \[ \int \frac{1}{x} \, dx = \ln |x| + C \]
    \[ \int \sin ax \, dx = -\frac{1}{a} \cos ax + C \]
    \[ \int \cos ax \, dx = \frac{1}{a} \sin ax + C \]
    \[ \int \sec^2 ax \, dx = \frac{1}{a} \tan ax + C \]
    \textbf{Integration Techniques:}
    - \textbf{Substitution:} Let \( u = g(x) \), then \( \int f(g(x)) g'(x) \, dx = \int f(u) \, du \). \\
    - \textbf{Integration by Parts:} \( \int u \, dv = uv - \int v \, du \).
\end{tcolorbox}

% Jacobian Determinant
\begin{tcolorbox}[title=\textbf{Jacobian Determinant}, colframe=lightgreen]
    For a transformation from \( (x, y) \) to \( (u, v) \):
    \[ J = \left| \frac{\partial(x, y)}{\partial(u, v)} \right| = 
    \begin{vmatrix}
    \dfrac{\partial x}{\partial u} & \dfrac{\partial x}{\partial v} \\[2ex]
    \dfrac{\partial y}{\partial u} & \dfrac{\partial y}{\partial v}
    \end{vmatrix}
    = \frac{\partial x}{\partial u} \frac{\partial y}{\partial v} - \frac{\partial x}{\partial v} \frac{\partial y}{\partial u} \]
    \textbf{Use in Integration:}
    \[ \iint_D f(x, y) \, dA = \iint_{D'} f(x(u, v), y(u, v)) \left| J \right| \, du \, dv \]
\end{tcolorbox}

% Properties of Conservative Vector Fields
\begin{tcolorbox}[title=\textbf{Conservative Vector Fields}, colframe=lightblue]
    \textbf{Tests for Conservativeness in \( \mathbb{R}^2 \):}
    - \( \frac{\partial P}{\partial y} = \frac{\partial Q}{\partial x} \) \\
    \textbf{Finding Potential Function \( f \):}
    - Integrate \( P \) with respect to \( x \), then \( Q \) with respect to \( y \), and combine results.
    \textbf{Zero Curl Condition:}
    - For \( \mathbf{F} = P \mathbf{i} + Q \mathbf{j} + R \mathbf{k} \), if \( \nabla \times \mathbf{F} = \mathbf{0} \), then \( \mathbf{F} \) is conservative (in simply connected domains).
\end{tcolorbox}

% Common Coordinate Transformations
\begin{tcolorbox}[title=\textbf{Common Coordinate Transformations}, colframe=lightgreen]
    \textbf{Polar to Cartesian:}
    \[ x = r \cos \theta, \quad y = r \sin \theta \]
    \textbf{Cartesian to Polar:}
    \[ r = \sqrt{x^2 + y^2}, \quad \theta = \arctan \left( \frac{y}{x} \right) \]
    \textbf{Cylindrical Coordinates:}
    \[ x = r \cos \theta, \quad y = r \sin \theta, \quad z = z \]
    \textbf{Spherical Coordinates:}
    \[ x = \rho \sin \phi \cos \theta \]
    \[ y = \rho \sin \phi \sin \theta \]
    \[ z = \rho \cos \phi \]
\end{tcolorbox}

\begin{tcolorbox}[title=\textbf{Derivative Rules}, colframe=lightblue]
    \[ \frac{d}{dx} [c] = 0 \]
    \[ \frac{d}{dx} [x^n] = nx^{n-1} \]
    \[ \frac{d}{dx} [c \cdot f(x)] = c \cdot f'(x) \]
    \[ \frac{d}{dx} [f(x) + g(x)] = f'(x) + g'(x) \]
    \[ \frac{d}{dx} [f(x) - g(x)] = f'(x) - g'(x) \]
    \[ \frac{d}{dx} [f(x) \cdot g(x)] = f'(x) g(x) + f(x) g'(x) \]
    \[ \frac{d}{dx} \left[ \frac{f(x)}{g(x)} \right] = \frac{f'(x) g(x) - f(x) g'(x)}{[g(x)]^2} \]
    \[ \frac{d}{dx} [f(g(x))] = f'(g(x)) \cdot g'(x) \]
\end{tcolorbox}

\end{multicols}
\end{document}
