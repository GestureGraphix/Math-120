\documentclass{report}

\input{preamble}
\input{macros}
\input{letterfonts}

\title{\Huge{Math 120}}
\author{\huge{PSet 9}}
\date{Nov 5 2024}

\begin{document}

\maketitle
\newpage% or \cleardoublepage
% \pdfbookmark[<level>]{<title>}{<dest>}
\pdfbookmark[section]{\contentsname}{toc}
\tableofcontents
\pagebreak

\chapter{}
\section{PSet 9}

\qs{}{
    The vector field \( \vec{F} \) is shown below in the \( xy \)-plane and looks the same in all other horizontal planes. (In other words, \( \vec{F} \) is independent of \( z \) and its \( z \)-component is 0.)
    \begin{enumerate}
        \item[(a)] Is \( \operatorname{div} \vec{F} \) positive, negative, or zero? Explain.
        \item[(b)] Determine whether \( \operatorname{curl} \vec{F} = \vec{0} \). If not, in which direction does \( \operatorname{curl} \vec{F} \) point?
    \end{enumerate}
}

\sol{
    \begin{enumerate}
        \item[(a)] The divergence of \( \vec{F} \) is negative because the vectors converge toward the origin, indicating a net inward flux. This suggests material is "flowing in" rather than spreading out.
        
        \item[(b)] The curl of \( \vec{F} \) is zero, as there is no rotational pattern; the vectors are purely radial and do not exhibit any circular motion.
    \end{enumerate}
}

\qs{}{
    Find the curl and divergence of the given vector field.
    \begin{enumerate}
        \item \( \vec{F}(x, y, z) = \langle x^2 yz, xy^2 z, xy z^2 \rangle \)
    \item \( \vec{F}(x, y, z) = e^{xy} \sin z \, \hat{\jmath} + y \arctan(x/z) \, \hat{k} \)
    \end{enumerate}
}

\sol{
    \[ P = x^{2}yx \quad Q = xy^{2}z \quad R = xyz^{2} \] 
    \[ \frac{\partial P}{\partial x} + \frac{\partial Q}{\partial y} + \frac{\partial R}{\partial z} \] 
    \[ \text{div } \vec{F} = 2xyz + 2xyz + 2xyz = 6xyz \]
    \[
    \begin{aligned}
    \frac{\partial P}{\partial x} &= 2 x y z, \\
    \frac{\partial Q}{\partial y} &= 2 x y z, \\
    \frac{\partial R}{\partial z} &= 2 x y z.
    \end{aligned}
    \]
    \[
    \operatorname{div} \vec{F} = \frac{\partial P}{\partial x} + \frac{\partial Q}{\partial y} + \frac{\partial R}{\partial z} = 2 x y z + 2 x y z + 2 x y z = 6 x y z.
    \]

\[
\left( \operatorname{curl} \vec{F} \right)_x = \frac{\partial R}{\partial y} - \frac{\partial Q}{\partial z} = x z^2 - x y^2.
\]
\[
\left( \operatorname{curl} \vec{F} \right)_y = \frac{\partial P}{\partial z} - \frac{\partial R}{\partial x} = x^2 y - y z^2.
\]
\[
\left( \operatorname{curl} \vec{F} \right)_z = \frac{\partial Q}{\partial x} - \frac{\partial P}{\partial y} = y^2 z - x^2 z.
\]
\[
\operatorname{curl} \vec{F} = \left( x z^2 - x y^2 \right) \hat{\imath} + \left( x^2 y - y z^2 \right) \hat{\jmath} + \left( y^2 z - x^2 z \right) \hat{k}.
\]
b)
\[
\vec{F}(x, y, z) = e^{x y} \sin z \, \hat{\jmath} + y \arctan\left( \frac{x}{z} \right) \hat{k}
\]

\[
\begin{aligned}
P &= 0, \\
Q &= e^{x y} \sin z, \\
R &= y \arctan\left( \dfrac{x}{z} \right).
\end{aligned}
\]
\[
\begin{aligned}
\frac{\partial P}{\partial x} &= 0, \\
\frac{\partial Q}{\partial y} &= x e^{x y} \sin z, \\
\frac{\partial R}{\partial z} &= -\frac{x y}{x^2 + z^2}.
\end{aligned}
\]
\[
\operatorname{div} \vec{F} = \frac{\partial P}{\partial x} + \frac{\partial Q}{\partial y} + \frac{\partial R}{\partial z} = 0 + x e^{x y} \sin z - \frac{x y}{x^2 + z^2}.
\]
\[
\begin{aligned}
\left( \operatorname{curl} \vec{F} \right)_x &= \frac{\partial R}{\partial y} - \frac{\partial Q}{\partial z} \\
&= \arctan\left( \frac{x}{z} \right) - e^{x y} \cos z.
\end{aligned}
\]
\[
\begin{aligned}
\left( \operatorname{curl} \vec{F} \right)_y &= \frac{\partial P}{\partial z} - \frac{\partial R}{\partial x} \\
&= 0 - \left( y \cdot \frac{z}{x^2 + z^2} \right) \\
&= -\frac{y z}{x^2 + z^2}.
\end{aligned}
\]
\[
\begin{aligned}
\left( \operatorname{curl} \vec{F} \right)_z &= \frac{\partial Q}{\partial x} - \frac{\partial P}{\partial y} \\
&= y e^{x y} \sin z - 0 \\
&= y e^{x y} \sin z.
\end{aligned}
\]
\[
\operatorname{curl} \vec{F} = \left( \arctan\left( \frac{x}{z} \right) - e^{x y} \cos z \right) \hat{\imath} - \left( \frac{y z}{x^2 + z^2} \right) \hat{\jmath} + \left( y e^{x y} \sin z \right) \hat{k}.
\]

    
}

\qs{}{
    Consider the surface given by \( \sin(x - y) - x - y + z = 0 \). Find a parametrization of the part of the surface that has \( |x| \leq 3 \) and \( |y| \leq 3 \). Be sure to include the bounds for your parameters.
    \begin{enumerate}
        \item[(a)] Look at the parametrization that you found in part (a). It should be of the form \( \vec{r}(u, v) = \langle u, v, f(u, v) \rangle \). Explain why it is not possible to find a parametrization of the surface \( x^2 - y + z^2 = 1 \) that is of the form \( \vec{r}(u, v) = \langle u, v, f(u, v) \rangle \).
        \item[(b)] Find a parametrization of \( x^2 - y + z^2 = 1 \) that is of the form \( \vec{r}(u, v) = \langle u, f(u, v), v \rangle \).
        \item[(c)] Consider the surface \( x^2 - y^2 + \frac{z^2}{4} = 1 \). Explain why it is not possible to find a parametrization of this surface that is of the form \( \vec{r}(u, v) = \langle u, v, f(u, v) \rangle \), or of the form \( \vec{r}(u, v) = \langle f(u, v), u, v \rangle \).
        \item[(d)] Find a parametrization of the surface \( x^2 - y^2 + \frac{z^2}{4} = 1 \) of the form
        \[
        \vec{r}(u, v) = \langle f(v) \cos u, v, g(v) \sin u \rangle,
        \]
        where \( 0 \leq u \leq 2\pi \) and \( -\infty < v < \infty \).
    \end{enumerate}
}

\qs{}{
    Identify and sketch the surface with the given parameterization.
    \begin{enumerate}
        \item[(a)] \( \vec{r}(u, v) = (2 \sin u) \, \hat{\jmath} + (3 \cos u) \, \hat{\imath} + v \, \hat{k}, \quad 0 \leq u \leq \pi, \quad -2 \leq v \leq 2 \)
        \item[(b)] \( \vec{r}(u, v) = \langle u \sin v, u^2, u \cos v \rangle, \quad 0 \leq u \leq 3, \quad 0 \leq v \leq 2\pi \)
    \end{enumerate}
}

\qs{}{
    Consider the surface \( S \) described by \( x - 4y^2 - z^2 + 3 = 0 \).
    \begin{enumerate}
        \item[(a)] Find a parametrization of \( S \) of the form \( \vec{r}_1(u, v) = \langle f(u, v), u, v \rangle \). Give the domain for your parametrization.
        \item[(b)] Find a parametrization of \( S \) of the form \( \vec{r}_2(u, v) = \langle v, f(v) \cos u, g(v) \sin u \rangle \). Give the domain for your parametrization.
        \item[(c)] How must we restrict the parameters \( (u, v) \) in part (a) if we only want the part of \( S \) that lies in front of the \( yz \)-plane, i.e., where \( x \geq 0 \)?
        \item[(d)] How must we restrict the parameters \( (u, v) \) in part (b) if we only want the part of \( S \) that lies in front of the \( yz \)-plane?
    \end{enumerate}
}

\qs{}{
    Find parametric equations for each of the following surfaces.
    \begin{enumerate}
        \item[(a)] The part of the plane \( z = x + 3 \) that lies inside the cylinder \( x^2 + y^2 = 1 \).
        \item[(b)] The surface obtained by rotating the curve \( x = 4y^2 - y^4 \), \( -2 \leq y \leq 2 \) about the \( y \)-axis.
        \item[(c)] The ellipsoid \( \frac{x^2}{4} + 4y^2 + \frac{z^2}{9} = 1 \).
    \end{enumerate}
}

\qs{}{
    Find the tangent plane to the parametric surface \( \vec{r}(u, v) = \langle u \sin v, u^2, u \cos v \rangle \) at the point where \( u = 1 \) and \( v = \frac{\pi}{3} \). Write the plane both in the vector form \( \vec{r}(u, v) = \vec{r}_0 + u \vec{a} + v \vec{b} \) and in the form \( ax + by + cz = d \).
}
\end{document}