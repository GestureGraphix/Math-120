\documentclass{report}

\input{preamble}
\input{macros}
\input{letterfonts}

\title{\Huge{Math 120}}
\author{\huge{Final Review}}
\date{Nov 20 2024}

\begin{document}

\maketitle
\newpage% or \cleardoublepage
% \pdfbookmark[<level>]{<title>}{<dest>}
\pdfbookmark[section]{\contentsname}{toc}
\tableofcontents
\pagebreak

\chapter*{12 Vectors and the Geometry of Space}
\addcontentsline{toc}{chapter}{Vectors and the Geometry of Space }

\section*{12.1 Three-Dimensional Coordinate Systems}
\addcontentsline{toc}{section}{Three-Dimensional Coordinate Systems}

\defrose{hi}{Hi}

\section*{12.2}

\section*{12.3}

\section*{12.4}

\section*{12.5}

\chapter*{13}

\section*{13.1}

\section*{13.2}

\section*{13.3}

\section*{13.4}

\chapter*{14}

\section*{14.1}

\section*{14.3}

\section*{14.5}

\section*{14.6}

\section*{14.7}

\section*{14.8}

\chapter*{15}

\section*{15.1}

\section*{15.2}

\section*{15.3}

\section*{15.6}

\section*{15.7}

\section*{15.8}

\chapter*{16 Vector Calculus}
\addcontentsline{toc}{chapter}{Vector Calculus}

\section*{16.1}

\section*{16.2}

\section*{16.3}

\section*{16.4}

\section*{16.5 Curl and Divergence}
\addcontentsline{toc}{section}{Curl and Divergence}

\section*{16.6 Parametric Surfaces and Their Areas}
\addcontentsline{toc}{section}{Parametric Surfaces and Their Areas}

\subsection*{Parametric Surfaces}

\defrose{Parametric Surface}{
    AA \textit{parametric surface} is a surface in three-dimensional space $\mathbb{R}^3$ defined by a vector-valued function $\mathbf{r}(u, v)$, which depends on two parameters $u$ and $v$. The function is expressed as:
    \[ \mathbf{r}(u, v) = x(u, v) \, \mathbf{i} + y(u, v) \, \mathbf{j} + z(u, v) \, \mathbf{k}, \]
    where $x(u, v)$, $y(u, v)$, and $z(u, v)$ are the component functions of $\mathbf{r}$, representing the $x$-, $y$-, and $z$-coordinates of the surface, respectively. These functions are defined over a region $D$ in the $uv$-plane. The set of all points $(x, y, z) \in \mathbb{R}^3$ that satisfy:
    \[ x = x(u, v), \quad y = y(u, v), \quad z = z(u, v), \]
    as $(u, v)$ varies over $D$, forms the parametric surface $S$.

}

\subsection*{Parametric Equations}

\defrose{Parametric Equations}{
    FFor a parametric surface the \textit{parametric equations} are equations that describe the coordinates $(x, y, z)$ of points on the surface as functions of two independent parameters $u$ and $v$. For a parametric surface $S$, these equations are given by:
    \[ x = x(u, v), \quad y = y(u, v), \quad z = z(u, v), \]
    where $x(u, v)$, $y(u, v)$, and $z(u, v)$ are the component functions of a vector-valued function $\mathbf{r}(u, v)$. These equations define the spatial coordinates of the surface for every pair of parameters $(u, v)$ in a specified domain $D$ in the $uv$-plane.
}

\subsection*{Grid Curves}

\defrose{Grid Curves}{
    OOn a parametric surface $s$ \textit{grid curves} are families of curves defined by the vector function $\mathbf{r}(u, v)$. They are obtained by fixing one parameter and varying the other:

    1. \textbf{Curves with $u = u_0$:} When $u$ is held constant, the parametric surface reduces to a curve:
    \[
    \mathbf{r}(u_0, v) = \langle x(u_0, v), y(u_0, v), z(u_0, v) \rangle,
    \]
    which traces a curve $C_1$ on the surface as $v$ varies.

    2. \textbf{Curves with $v = v_0$:} When $v$ is held constant, the parametric surface reduces to a curve:
    \[
    \mathbf{r}(u, v_0) = \langle x(u, v_0), y(u, v_0), z(u, v_0) \rangle,
    \]
    which traces a curve $C_2$ on the surface as $u$ varies.

    These two families of curves correspond to horizontal and vertical lines in the $uv$-plane and form a grid-like structure when plotted on the surface.

}

\subsection*{Spherical Coordinates}

\subsection*{Surfaces of Revolution}

\defrose{Surfaces of Revolution}{
    AA \textbf{surface of revolution} is generated by rotating a curve \( C \), defined parametrically or as a function, about a fixed axis in three-dimensional space. The parametric equations of the surface can be expressed as:

    \[
    \begin{aligned}
    x &= u, \\
    y &= r(u) \cos \theta, \\
    z &= r(u) \sin \theta,
    \end{aligned}
    \]

    where:
    \begin{itemize}
        \item \( u \) is a parameter describing the curve \( C \),
        \item \( r(u) \) is the radial distance of the curve from the axis of rotation,
        \item \( \theta \in [0, 2\pi] \) is the angle of rotation.
    \end{itemize}

    The domain of the parameters \( u \) and \( \theta \) depends on the curve and the extent of rotation.


}

\subsection*{Tangent Planes}

\defrose{Tangent Planes}{
    T
    The \textbf{tangent plane} to a parametric surface \( S \) at a point \( P_0(u_0, v_0) \) is the plane that best approximates \( S \) near \( P_0 \). 
    
    If \( S \) is defined by a vector-valued function:
    \[ \mathbf{r}(u, v) = x(u, v) \mathbf{i} + y(u, v) \mathbf{j} + z(u, v) \mathbf{k}, \]
    then the tangent plane at \( P_0 \) is determined by the two tangent vectors at \( P_0 \):
    \[ \mathbf{r}_u = \frac{\partial \mathbf{r}}{\partial u} = \frac{\partial x}{\partial u} \mathbf{i} + \frac{\partial y}{\partial u} \mathbf{j} + \frac{\partial z}{\partial u} \mathbf{k}, \]
    \[ \mathbf{r}_v = \frac{\partial \mathbf{r}}{\partial v} = \frac{\partial x}{\partial v} \mathbf{i} + \frac{\partial y}{\partial v} \mathbf{j} + \frac{\partial z}{\partial v} \mathbf{k}.  \]
    
    The tangent plane at \( P_0 \) is spanned by \(\mathbf{r}_u\) and \(\mathbf{r}_v\). A normal vector to the plane is given by:
    \[ \mathbf{n} = \mathbf{r}_u \times \mathbf{r}_v. \]
    
    The equation of the tangent plane can be expressed in the point-normal form:
    \[ \mathbf{n} \cdot (\mathbf{r} - \mathbf{r}(u_0, v_0)) = 0, \]
    where \(\mathbf{r}(u_0, v_0)\) is the position vector of \( P_0 \).
    
    For the tangent plane to exist, the cross product \(\mathbf{r}_u \times \mathbf{r}_v\) must be nonzero, ensuring that \( S \) is smooth at \( P_0 \).
    
}
\subsection*{Surface Area for a Parametric Surface}

\defrose{Surface Area for a Parametric Surface}{
    TThe \textbf{surface area} of a smooth parametric surface \( S \), defined by the vector-valued function:
    \[ \mathbf{r}(u, v) = x(u, v) \mathbf{i} + y(u, v) \mathbf{j} + z(u, v) \mathbf{k}, \quad (u, v) \in D, \]
    where \( D \) is the parameter domain, is given by the integral:
    \[ A(S) = \iint_D \left| \mathbf{r}_u \times \mathbf{r}_v \right| \, dA, \]
    where:
    \[ \mathbf{r}_u = \frac{\partial x}{\partial u} \mathbf{i} + \frac{\partial y}{\partial u} \mathbf{j} + \frac{\partial z}{\partial u} \mathbf{k}, \quad
    \mathbf{r}_v = \frac{\partial x}{\partial v} \mathbf{i} + \frac{\partial y}{\partial v} \mathbf{j} + \frac{\partial z}{\partial v} \mathbf{k}. \]
    
    The cross product \(\mathbf{r}_u \times \mathbf{r}_v\) represents a vector orthogonal to the tangent plane at each point on the surface, and its magnitude \(\left| \mathbf{r}_u \times \mathbf{r}_v \right|\) gives the infinitesimal area of a parallelogram spanned by the tangent vectors \(\mathbf{r}_u\) and \(\mathbf{r}_v\). Integrating this quantity over the parameter domain \( D \) yields the total surface area of \( S \).
        
}

\subsection*{Surface Area of the Graph of a Function}

\defrose{Surface Area of the Graph of a Function}{
    TThe \textbf{surface area} of the graph of a function \( z = f(x, y) \), where \( f(x, y) \) has continuous partial derivatives, over a region \( D \) in the \( xy \)-plane is given by:
    \[
    A(S) = \iint_D \sqrt{1 + \left( \frac{\partial f}{\partial x} \right)^2 + \left( \frac{\partial f}{\partial y} \right)^2} \, dA.
    \]
    
    \subsection*{Explanation}
    \begin{itemize}
        \item The parametric representation of the surface is:
        \[ \mathbf{r}(x, y) = x \mathbf{i} + y \mathbf{j} + f(x, y) \mathbf{k}. \]
        \item The tangent vectors are:
        \[ \mathbf{r}_x = \mathbf{i} + \frac{\partial f}{\partial x} \mathbf{k}, \quad \mathbf{r}_y = \mathbf{j} + \frac{\partial f}{\partial y} \mathbf{k}. \]
        \item The magnitude of the cross product of the tangent vectors is:
        \[ \left| \mathbf{r}_x \times \mathbf{r}_y \right| = \sqrt{1 + \left( \frac{\partial f}{\partial x} \right)^2 + \left( \frac{\partial f}{\partial y} \right)^2}. \]
    \end{itemize}
    
    Integrating this quantity over the region \( D \) in the \( xy \)-plane gives the total surface area of the graph of \( f(x, y) \).    
}


\section*{16.7 Surface Integral}
\addcontentsline{toc}{section}{Surface Integral}

\subsection*{Parametric Surfaces}

\defrose{Surface Integral for Parametric Surfaces}{
    TThe \textbf{surface integral} of a scalar function \( f(x, y, z) \) over a parametric surface \( S \), defined by the vector equation:
    \[ \mathbf{r}(u, v) = x(u, v) \mathbf{i} + y(u, v) \mathbf{j} + z(u, v) \mathbf{k}, \quad (u, v) \in D, \]
    is given by:
    \[ \iint_S f(x, y, z) \, dS = \iint_D f(\mathbf{r}(u, v)) \left| \mathbf{r}_u \times \mathbf{r}_v \right| \, dA. \]
    
    \subsection*{Explanation}
    \begin{itemize}
        \item The parameter domain \( D \) is divided into subrectangles with dimensions \( \Delta u \) and \( \Delta v \), and each corresponding surface patch is approximated as a parallelogram in the tangent plane.
        \item The area of a surface patch is approximated as:
        \[ \Delta S_{ij} \approx \left| \mathbf{r}_u \times \mathbf{r}_v \right| \Delta u \Delta v.\]
        \item The surface integral is defined as the limit of a Riemann sum:
        \[ \iint_S f(x, y, z) \, dS = \lim_{m, n \to \infty} \sum_{i=1}^m \sum_{j=1}^n f(P_{ij}^*) \Delta S_{ij}. \]
    \end{itemize}
    
    \subsection*{Key Components}
    \begin{itemize}
        \item \(\mathbf{r}_u\) and \(\mathbf{r}_v\) are the partial derivatives of \(\mathbf{r}(u, v)\) with respect to \( u \) and \( v \), respectively:
        \[ \mathbf{r}_u = \frac{\partial \mathbf{r}}{\partial u}, \quad \mathbf{r}_v = \frac{\partial \mathbf{r}}{\partial v}. \]
        \item \(\mathbf{r}_u \times \mathbf{r}_v\) gives a vector normal to the surface at each point, and \(\left| \mathbf{r}_u \times \mathbf{r}_v \right|\) represents the infinitesimal surface area element.
    \end{itemize}
    
    This integral evaluates the contribution of \( f(x, y, z) \) across the entire surface \( S \).
        
}

\subsection*{Graphs of Functions}
\defrose{Surafe Integrals for Graphs of Functions}{
    TThe \textbf{surface integral} of a scalar function \( f(x, y, z) \) over the graph of a function \( z = g(x, y) \), where \( g(x, y) \) has continuous partial derivatives, is given by:
    \[
    \iint_S f(x, y, z) \, dS = \iint_D f(x, y, g(x, y)) \sqrt{\left(\frac{\partial g}{\partial x}\right)^2 + \left(\frac{\partial g}{\partial y}\right)^2 + 1} \, dA.
    \]
    
    \subsection*{Explanation}
    \begin{itemize}
        \item The graph of the function \( z = g(x, y) \) can be regarded as a parametric surface with:
        \[
        x = x, \quad y = y, \quad z = g(x, y).
        \]
        \item The tangent vectors to this surface are:
        \[
        \mathbf{r}_x = \mathbf{i} + \frac{\partial g}{\partial x} \mathbf{k}, \quad \mathbf{r}_y = \mathbf{j} + \frac{\partial g}{\partial y} \mathbf{k}.
        \]
        \item The cross product of the tangent vectors is:
        \[
        \mathbf{r}_x \times \mathbf{r}_y = -\frac{\partial g}{\partial x} \mathbf{j} - \frac{\partial g}{\partial y} \mathbf{i} + \mathbf{k}.
        \]
        \item The magnitude of the cross product is:
        \[
        \left| \mathbf{r}_x \times \mathbf{r}_y \right| = \sqrt{\left(\frac{\partial g}{\partial x}\right)^2 + \left(\frac{\partial g}{\partial y}\right)^2 + 1}.
        \]
    \end{itemize}
    
    By integrating this quantity over the region \( D \) in the \( xy \)-plane, we account for the contributions of \( f(x, y, z) \) over the entire surface \( S \).    
}

\subsection*{Oriented Surfaces}

\defrose{Oriented Surfaces}{
    AAn \textbf{oriented surface} is an orientable (two-sided) surface \( S \) where it is possible to define a continuous, unit normal vector \( \mathbf{n} \) at every point \( (x, y, z) \) on the surface, except possibly at boundary points.

    \subsection*{Key Properties}
    \begin{itemize}
        \item \textbf{Two Possible Orientations}: For any orientable surface, there are two choices for the unit normal vector:
        \begin{itemize}
            \item \( \mathbf{n}_1 \), the chosen unit normal vector.
            \item \( \mathbf{n}_2 = -\mathbf{n}_1 \), the opposite orientation.
        \end{itemize}
        \item A surface is called \textbf{orientable} if it is possible to assign \( \mathbf{n} \) continuously over the entire surface \( S \).
        \item A classic example of a non-orientable surface is the Möbius strip, which has only one side and no consistent orientation.
    \end{itemize}

    \subsection*{Explanation}
    An oriented surface requires the existence of a consistent way to assign a "positive" or "negative" side across all points on the surface. The orientation is provided by the chosen direction of the normal vector \( \mathbf{n} \), which varies smoothly across the surface.
}

\subsection*{Surface Integrals of Vector Fields; Flux}
\defrose{Flux}{
    TThe \textbf{surface integral of a vector field} (also called the \textbf{flux}) over an oriented surface \( S \) with a unit normal vector \( \mathbf{n} \) is defined as:
    \[
    \iint_S \mathbf{F} \cdot d\mathbf{S} = \iint_S \mathbf{F} \cdot \mathbf{n} \, dS,
    \]
    where:
    \begin{itemize}
        \item \( \mathbf{F} \) is a continuous vector field defined on \( S \),
        \item \( \mathbf{n} \) is the unit normal vector to \( S \),
        \item \( dS \) represents the infinitesimal surface area element.
    \end{itemize}

    \subsection*{Physical Interpretation}
    This integral measures the total "flow" of \( \mathbf{F} \) through the surface \( S \) in the direction of the normal vector \( \mathbf{n} \). For example:
    \begin{itemize}
        \item If \( \mathbf{F} = \rho \mathbf{v} \), where \( \rho(x, y, z) \) is the density of a fluid and \( \mathbf{v}(x, y, z) \) is the velocity field, the flux represents the mass flow rate of the fluid through \( S \).
    \end{itemize}

    \subsection*{Parametric Form}
    If the surface \( S \) is parameterized by \( \mathbf{r}(u, v) \), with tangent vectors:
    \[
    \mathbf{r}_u = \frac{\partial \mathbf{r}}{\partial u}, \quad \mathbf{r}_v = \frac{\partial \mathbf{r}}{\partial v},
    \]
    then the flux integral can be written as:
    \[
    \iint_S \mathbf{F} \cdot d\mathbf{S} = \iint_D \mathbf{F}(\mathbf{r}(u, v)) \cdot (\mathbf{r}_u \times \mathbf{r}_v) \, dA,
    \]
    where \( D \) is the parameter domain.

    \subsection*{Notes}
    \begin{itemize}
        \item The orientation of \( S \) determines the sign of the flux.
        \item If the opposite orientation is used, the flux is multiplied by \( -1 \).
    \end{itemize}
}

\section*{16.8 Stokes' Theorem} 
\addcontentsline{toc}{section}{Stokes' Theorem}


\section*{16.9}




\end{document}