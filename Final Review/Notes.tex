\documentclass{report}

\input{preamble}
\input{macros}
\input{letterfonts}

\title{\Huge{Math 120}}
\author{\huge{Final Review}}
\date{Nov 20 2024}

\begin{document}

\maketitle
\newpage% or \cleardoublepage
% \pdfbookmark[<level>]{<title>}{<dest>}
\pdfbookmark[section]{\contentsname}{toc}
\tableofcontents
\pagebreak

\chapter*{12 Vectors and the Geometry of Space}
\addcontentsline{toc}{chapter}{12 Vectors and the Geometry of Space }


\section*{12.1 Three-Dimensional Coordinate Systems}
\addcontentsline{toc}{section}{12.1 Three-Dimensional Coordinate Systems}

\subsection*{3D Space}

\defrose{Three Dimensional Rectangular Coordinate System}{
    TThe carstesian product $ \mathbb{R} \times \mathbb{R} \times \mathbb{R} = \left[ (x,y,z) \, | \, x,y,z \in \mathbb{R} \right]$. is the set of all real 
    numbers and is denoted by $ \mathbb{R}$
}

\addcontentsline{toc}{subsection}{3D Space}

\subsection*{Surfaces and Solids}
\addcontentsline{toc}{subsection}{Surfaces and Solids}

\ntimg{
    In three-dimensional analytic, an equation in $x,y$ and, $z$ represents a surface in 
    $\mathbb{R}$  
}

\subsection*{Distance and Spheres}
\addcontentsline{toc}{subsection}{Distances and Spheres}

\defrose{Distance Formula in Three Dimensions}{
    TThe distance $| P_{1} P_{2}|$ between the points $P_{1}(x_{1},y_{1},z_{1})$ and $P_{2}(x_{1}, y_{2}, z_{2})$ is 
    \[ |P_{1}P_{2}| = \sqrt{\left(x_{2} - x_{1}\right)^{2} + \left(y_{2} - y_{1}\right)^{2} + \left(z_{2} - z_{1}\right)^{2}}\] 
}

\section*{12.2 Vectors}
\addcontentsline{toc}{section}{12.2 Vectors}

\defrose{Vectors}{
    AA \textit{vector} indicates a quantity that has both magnitude and direction
}

\subsection*{Geometric Description of Vectors}
\addcontentsline{toc}{subsection}{Geometric Description of Vectors}

\defrose{Vector Addition}{
    FFor vectors \textbf{Vector Addition} means the sum of two vectors $\mathbf{u} + \mathbf{v}$ is the vector from the initial point of $\mathbf{u}$ to the terminal point of $\mathbf{v}$, with $\mathbf{v}$ placed so its tail starts at the tip of $\mathbf{u}$..
}

\defrose{Triangle Law}{
    FFor vectors that \textbf{Triangle Law} refers to the sum of two vectors representing the third side of a triangle by placing the second vector's tail at the first vector's tip.
}

\defrose{Parallelogram Law}{
    FFor vectors \textbf{Parallelogram Law} refers to the sum of two vectors is representing the diagonal of a parallelogram formed when both vectors originate from the same point.
}

\defrose{Scalar Multiplication}{
    FFor vectors \textbf{Scalar Multiplication} means that if $c$ is a scalar and $\mathbf{v}$ is a vector, the scalar multiple $c\mathbf{v}$ is a vector whose length is $|c|$ times the length of $\mathbf{v}$. Its direction matches $\mathbf{v}$ if $c > 0$, is opposite to $\mathbf{v}$ if $c < 0$, and is $\mathbf{0}$ if $c = 0$ or $\mathbf{v} = \mathbf{0}$.
}

\subsection*{Components of Vectors}
\addcontentsline{toc}{subsection}{Components of Vectors}

\defrose{Vector Representation}{
    GGiven points $A(x_1, y_1, z_1)$ and $B(x_2, y_2, z_2)$, the vector $\mathbf{a}$ representing the directed segment $\overrightarrow{AB}$ is 
    \[ \mathbf{a} = \langle x_2 - x_1, y_2 - y_1, z_2 - z_1 \rangle. \]
}

\defrose{Magnitude of a Vector}{
    TThe length of a vector $\mathbf{a}$ is 
    \[ |\mathbf{a}| = \sqrt{a_1^2 + a_2^2} \quad \text{(2D)}, \quad |\mathbf{a}| = \sqrt{a_1^2 + a_2^2 + a_3^2} \quad \text{(3D)}. \]
}
\ntimg{
    \textbf{Vector Addition:} Add corresponding components:
    \[\mathbf{a} + \mathbf{b} = \langle a_1 + b_1, a_2 + b_2 \rangle \quad \text{(2D)}, \quad \mathbf{a} + \mathbf{b} = \langle a_1 + b_1, a_2 + b_2, a_3 + b_3 \rangle \quad \text{(3D)}.\]

    \textbf{Vector Subtraction:} Subtract corresponding components:
    \[ \mathbf{a} - \mathbf{b} = \langle a_1 - b_1, a_2 - b_2 \rangle \quad \text{(2D)}, \quad \mathbf{a} - \mathbf{b} = \langle a_1 - b_1, a_2 - b_2, a_3 - b_3 \rangle \quad \text{(3D)}. \]

    \textbf{Scalar Multiplication:} Multiply each component by the scalar:
    \[ c\mathbf{a} = \langle c a_1, c a_2 \rangle \quad \text{(2D)}, \quad c\mathbf{a} = \langle c a_1, c a_2, c a_3 \rangle \quad \text{(3D)}. \]
}

\defrose{Unit Vector}{
    AA vector whose magnitude is 1
}

\newpage

\section*{12.3 The Dot Product}
\addcontentsline{toc}{section}{12.3 The dot product}

\subsection*{Dot Product of Two Vectors}
\addcontentsline{toc}{subsection}{Dot Product of Two Vectors}

\defrose{Dot Product}{
    TThe \textit{dot product} of two vectors $\mathbf{a} = \langle a_1, a_2, a_3 \rangle$ and $\mathbf{b} = \langle b_1, b_2, b_3 \rangle$ is a scalar given by:
    \[ \mathbf{a} \cdot \mathbf{b} = a_1b_1 + a_2b_2 + a_3b_3. \]
}

\thm{Angle Between Vectors}{
    The dot product of two vectors $\mathbf{a}$ and $\mathbf{b}$ relates their magnitudes and the cosine of the angle $\theta$ between them:
    \[ \mathbf{a} \cdot \mathbf{b} = |\mathbf{a}| |\mathbf{b}| \cos\theta. \]
    This formula shows how the dot product measures both the lengths of the vectors and their alignment in space.
}

\cor{Angle Between Vectors}{
    If $\theta$ is the angle between the nonzero vectors \textbf{a} annd \textbf{b}, then
    \[ \cos \theta = \frac{\textbf{a} \cdot \textbf{b}}{|\textbf{a}||\textbf{b}|} \] 
}

\ntimg{
    Two vectors \textbf{a} and \textbf{b} are orthogonal if and only if  $\textbf{a} \cdot \textbf{b} = 0 $
}

\subsection*{Direction Angles and Direction Cosines}
\addcontentsline{toc}{subsection}{Direction Angles and Direction Cosines}

\defrose{Direction Cosines}{
    TThe direction angles ($\alpha, \beta, \gamma$) of a nonzero vector $\mathbf{a}$ are the angles it makes with the positive $x$-, $y$-, and $z$-axes, respectively. The cosines of these angles ($\cos\alpha, \cos\beta, \cos\gamma$) are called the direction cosines, and they satisfy:
    \[ \cos^2\alpha + \cos^2\beta + \cos^2\gamma = 1. \]
}

\ntimg{
    The vector $\mathbf{a}$ can be expressed in terms of its magnitude and direction cosines as:
    \[ \mathbf{a} = |\mathbf{a}| \langle \cos\alpha, \cos\beta, \cos\gamma \rangle, \]
    and the unit vector in the direction of $\mathbf{a}$ is:
    \[ \frac{1}{|\mathbf{a}|} \mathbf{a} = \langle \cos\alpha, \cos\beta, \cos\gamma \rangle. \]
}

\subsection*{Projections}
\addcontentsline{toc}{subsection}{Projections}

\defrose{Projections}{
    TThe \textbf{scalar projection} of $\mathbf{b}$ onto $\mathbf{a}$, denoted $\text{comp}_{\mathbf{a}} \mathbf{b}$, represents the magnitude of $\mathbf{b}$ in the direction of $\mathbf{a}$. It is the length of the "shadow" of $\mathbf{b}$ onto $\mathbf{a}$ and is given by:
    \[ \text{comp}_{\mathbf{a}} \mathbf{b} = \frac{\mathbf{a} \cdot \mathbf{b}}{|\mathbf{a}|}. \]

    The \textbf{Vector Projection:} The vector projection of $\mathbf{b}$ onto $\mathbf{a}$, denoted $\text{proj}_{\mathbf{a}} \mathbf{b}$, represents the vector component of $\mathbf{b}$ in the direction of $\mathbf{a}$. It is the scaled vector of $\mathbf{a}$ that aligns with the projection, given by:
    \[ \text{proj}_{\mathbf{a}} \mathbf{b} = \frac{\mathbf{a} \cdot \mathbf{b}}{|\mathbf{a}|^2} \mathbf{a}. \]
}

\section*{12.4 The Cross Product}
\addcontentsline{toc}{section}{12.4 The Cross Product}

\subsection*{The Cross Product of Two Vectors}
\addcontentsline{toc}{subsection}{The Cross Product of Two Vectors}

\defrose{Cross Product}{
    TThe \textbf{cross product} of two vectors \(\vec{a} = \langle a_1, a_2, a_3 \rangle\) and \(\vec{b} = \langle b_1, b_2, b_3 \rangle\) is a vector \(\vec{a} \times \vec{b}\) defined as:
    \[ \vec{a} \times \vec{b} = \langle a_2b_3 - a_3b_2, \, a_3b_1 - a_1b_3, \, a_1b_2 - a_2b_1 \rangle \]

    The cross product is a vector:
    \begin{itemize}
        \item \textbf{Perpendicular} to both \(\vec{a}\) and \(\vec{b}\).
        \item \textbf{Magnitude} equal to \(\|\vec{a}\| \|\vec{b}\| \sin \theta\), where \(\theta\) is the angle between \(\vec{a}\) and \(\vec{b}\).
        \item \textbf{Direction} determined by the right-hand rule.
    \end{itemize}
}

\subsection*{Properties of the Cross Product}
\addcontentsline{toc}{subsection}{Properties of the Cross Product}

\thm{Resulting Vector}{
    The vector \textbf{a} $\times$ \textbf{b} is orthogonal to both \textbf{a} and \textbf{b} 
}

\thm{Geometric Description of the Length of the Cross Product}{
    If $\theta$ is the angle between \textbf{a} and \textbf{b} (s0 $ 0 \leq \theta \leq \pi$), then the 
    length of the cross product $\textbf{a} \times \textbf{b}$ is given by 
    \[ |\textbf{a} \times \textbf{b} | = |\textbf{a}| |texbt{b}| \sin \theta \] 
}

\cor{Parallellity using Cross Product}{
    Two nonzero vectors \textbf{a} and \textbf{b} are parallel if and only if 
    \[ \textbf{a} \times \textbf{b} = 0 \] 
}

\ntimg{
    The length of the cross product \textbf{a} $\times$ \textbf{b} is equal to the area of the parallelogram determined
    by \textbf{a} and \textbf{b}  
}

\subsection*{Triple Products}
\addcontentsline{toc}{subsection}{Triple Products}  

\defrose{Triple Product}{
    TThe \textbf{scalar triple product} of three vectors \(\mathbf{a}\), \(\mathbf{b}\), and \(\mathbf{c}\) is given by:
    \[ \mathbf{a} \cdot (\mathbf{b} \times \mathbf{c}), \]
    which can also be expressed as the determinant of the matrix formed by placing the vectors as rows (or columns):
    \[ \mathbf{a} \cdot (\mathbf{b} \times \mathbf{c}) = 
    \begin{vmatrix}
    a_1 & a_2 & a_3 \\
    b_1 & b_2 & b_3 \\
    c_1 & c_2 & c_3
    \end{vmatrix}. \]
        
    \begin{itemize}
        \item The magnitude of the scalar triple product, \(|\mathbf{a} \cdot (\mathbf{b} \times \mathbf{c})|\), represents the \textbf{volume} of the parallelepiped formed by the vectors \(\mathbf{a}\), \(\mathbf{b}\), and \(\mathbf{c}\).
        \item If \(\mathbf{a} \cdot (\mathbf{b} \times \mathbf{c}) = 0\), then the vectors \(\mathbf{a}\), \(\mathbf{b}\), and \(\mathbf{c}\) are \textbf{coplanar}, meaning they lie in the same plane.
    \end{itemize}
}

\section*{12.5 Equations of Lines and Planes}
\addcontentsline{toc}{section}{12.5 Equations of Lines and Planes}

\subsection*{Lines}
\addcontentsline{toc}{subsection}{Lines}

\defrose{Line in 3D Space}{
    AA line in three-dimensional space is defined by:
    \begin{itemize}
        \item A point \(\mathbf{r}_0 = \langle x_0, y_0, z_0 \rangle\),
        \item A direction vector \(\mathbf{v} = \langle a, b, c \rangle\).
    \end{itemize}
    
    Its vector equation is:
    \[ \mathbf{r} = \mathbf{r}_0 + t\mathbf{v}, \]
    where \(t\) is a scalar parameter.
    
    In parametric form, the equation becomes:
    \[ x = x_0 + ta, \quad y = y_0 + tb, \quad z = z_0 + tc. \]
}

\subsection*{Planes}
\addcontentsline{toc}{subsection}{Planes}

\defrose{Plane in 3D Space}{
    AA plane in three-dimensional space is defined by:
    \begin{itemize}
        \item A point \(\mathbf{r}_0 = \langle x_0, y_0, z_0 \rangle\) on the plane,
        \item A normal vector \(\mathbf{n} = \langle a, b, c \rangle\), orthogonal to the plane.
    \end{itemize}

    The vector equation of the plane is:
    \[
    \mathbf{n} \cdot (\mathbf{r} - \mathbf{r}_0) = 0,
    \]
    where \(\mathbf{r} = \langle x, y, z \rangle\) is the position vector of any point on the plane.

    The scalar equation of the plane is:
    \[
    a(x - x_0) + b(y - y_0) + c(z - z_0) = 0,
    \]
    where \((x_0, y_0, z_0)\) is a point on the plane, and \(\mathbf{n} = \langle a, b, c \rangle\) is the normal vector.
}

\ntimg{
    Two planes are parallel  if their normal vectors are parallel \\
    If two planes are not parallel, then they intersect in a straight line and the angle 
    between the two planes is defined as the acute angle between their normal vectors     
}

\subsection*{Distances}
\addcontentsline{toc}{subsection}{Distances}

\defrose{Distances}{
    TThe distance \(D\) from a point \(P_1(x_1, y_1, z_1)\) to a plane given by the equation \(ax + by + cz + d = 0\) is:

    \[
    D = \frac{|ax_1 + by_1 + cz_1 + d|}{\sqrt{a^2 + b^2 + c^2}}.
    \]
    
    \subsection*{Explanation}
    \begin{itemize}
        \item \((x_1, y_1, z_1)\) are the coordinates of the point \(P_1\),
        \item \(a\), \(b\), \(c\), and \(d\) are the constants defining the plane equation,
        \item \(\sqrt{a^2 + b^2 + c^2}\) represents the magnitude of the normal vector \(\mathbf{n} = \langle a, b, c \rangle\).
    \end{itemize}
}

\chapter*{13 Vector Functions}
\addcontentsline{toc}{chapter}{13 Vector Functions}


\section*{13.1 Vector Functions and Space Curves}
\addcontentsline{toc}{section}{13.1 Vector Functions and Space Curves}

\subsection*{Vector-Valued Functions}
\addcontentsline{toc}{subsection}{Vector-Valued Functions}

\defrose{Vector-Valued Functions}{
    AA \textbf{vector-valued function} is a function whose domain is a set of real numbers and whose range is a set of vectors. In three dimensions, a vector-valued function \(\mathbf{r}(t)\) can be expressed as:
    \[ \mathbf{r}(t) = \langle f(t), g(t), h(t) \rangle = f(t)\mathbf{i} + g(t)\mathbf{j} + h(t)\mathbf{k}, \]
    where \(f(t)\), \(g(t)\), and \(h(t)\) are the component functions of \(\mathbf{r}(t)\), and \(t\) is the independent variable.
}

\subsection*{Space Curves}
\addcontentsline{toc}{subsection}{Space Curves}

\defrose{Space Curves}{
    AA \textbf{space curve} is the set of all points \(C(x, y, z)\) in space traced by a continuous vector-valued function \(\mathbf{r}(t)\), where:

    \[    \mathbf{r}(t) = \langle f(t), g(t), h(t) \rangle, \]
    
    and \(t\) varies over an interval \(I\).
    
    The parametric equations of the space curve are:
    
    \[     x = f(t), \quad y = g(t), \quad z = h(t).  \]
    
    A space curve represents the path of a moving particle whose position at time \(t\) is given by \((f(t), g(t), h(t))\).
    
}

\section*{13.2 Derivatives and Integrals of Line Functions}
\addcontentsline{toc}{section}{13.2 Derivatives and Integrals of Line Functions}

\subsection*{Derivatives}
\addcontentsline{toc}{subsection}{Derivatives}

\defrose{Derivative of a Vector Function}{
    TThe derivative of a vector function \(\mathbf{r}(t)\), denoted as \(\mathbf{r}'(t)\), is defined as:

    \[ \mathbf{r}'(t) = \lim_{h \to 0} \frac{\mathbf{r}(t + h) - \mathbf{r}(t)}{h},  \]

    provided the limit exists.

    \begin{itemize}
        \item \(\mathbf{r}'(t)\) represents the \textbf{tangent vector} to the curve defined by \(\mathbf{r}(t)\) at a point.
        \item The tangent line to the curve at this point is parallel to \(\mathbf{r}'(t)\).
    \end{itemize}

}

\thm{Derivatives}{
    If \(\mathbf{r}(t) = \langle f(t), g(t), h(t) \rangle = f(t)\mathbf{i} + g(t)\mathbf{j} + h(t)\mathbf{k}\), where \(f\), \(g\), and \(h\) are differentiable functions, then:
    \[ \mathbf{r}'(t) = \langle f'(t), g'(t), h'(t) \rangle = f'(t)\mathbf{i} + g'(t)\mathbf{j} + h'(t)\mathbf{k}. \]
}

\thm{Orthogonnality from derivative}{
    If $|\textbf{r}(t)| = c$ (a constant), then \textbf{r}'(t) is orthogonal to \textbf{r}(t) for all $t$.
}

\subsection*{Integrals}
\addcontentsline{toc}{subsection}{Integrals}

\defrose{Definite Integral}{
    TThe \textbf{definite integral} of a continuous vector function \(\mathbf{r}(t) = \langle f(t), g(t), h(t) \rangle\) over the interval \([a, b]\) is defined as:

    \[
    \int_a^b \mathbf{r}(t) \, dt = 
    \left( \int_a^b f(t) \, dt \right) \mathbf{i} + 
    \left( \int_a^b g(t) \, dt \right) \mathbf{j} + 
    \left( \int_a^b h(t) \, dt \right) \mathbf{k}.
    \]
    
    Conceptually, the integral computes the accumulation of the vector values of \(\mathbf{r}(t)\) over the interval, integrating each component function \(f(t)\), \(g(t)\), and \(h(t)\) independently. \\
        
    The Fundamental Theorem of Calculus for vector functions states:
    
    \[
    \int_a^b \mathbf{r}(t) \, dt = \mathbf{R}(b) - \mathbf{R}(a),
    \]
    
    where \(\mathbf{R}(t)\) is an antiderivative of \(\mathbf{r}(t)\), satisfying \(\mathbf{R}'(t) = \mathbf{r}(t)\).
    
}

\section*{13.3 Arc Length and Curvature}
\addcontentsline{toc}{section}{13.3 Arc Length and Curvature}

\subsection*{Arc Length}
\addcontentsline{toc}{subsection}{Arc Length}

\defrose{Arc Length}{
    TThe \textbf{arc length} \(L\) of a curve defined by a vector function \(\mathbf{r}(t) = \langle f(t), g(t), h(t) \rangle\) over the interval \([a, b]\) is:

    \[ L = \int_a^b \sqrt{\left( \frac{dx}{dt} \right)^2 + \left( \frac{dy}{dt} \right)^2 + \left( \frac{dz}{dt} \right)^2} \, dt, \]

    or equivalently:

    \[ L = \int_a^b \sqrt{[f'(t)]^2 + [g'(t)]^2 + [h'(t)]^2} \, dt. \]

    In compact form, the arc length can be written using the magnitude of the derivative:

    \[ L = \int_a^b |\mathbf{r}'(t)| \, dt, \]

    where \(|\mathbf{r}'(t)| = \sqrt{[f'(t)]^2 + [g'(t)]^2 + [h'(t)]^2}\) represents the speed of the particle moving along the curve.

}

\subsection*{The Arc Length Function}   
\addcontentsline{toc}{subsection}{The Arc Length Function}  

\defrose{Arc Length Function}{
    TThe \textbf{arc length function} \(s(t)\) measures the length of the curve from the starting point to a point on the curve corresponding to parameter \(t\). For a curve defined by a vector function \(\mathbf{r}(t) = \langle f(t), g(t), h(t) \rangle\), the arc length function is given by:

    \[ s(t) = \int_a^t |\mathbf{r}'(u)| \, du = \int_a^t \sqrt{\left(\frac{dx}{du}\right)^2 + \left(\frac{dy}{du}\right)^2 + \left(\frac{dz}{du}\right)^2} \, du, \]
    
    where \(|\mathbf{r}'(u)|\) represents the magnitude of the derivative of \(\mathbf{r}(u)\).
    
    Differentiating \(s(t)\) with respect to \(t\) yields:
    
    \[ \frac{ds}{dt} = |\mathbf{r}'(t)|. \]
    
    This formula connects the arc length function to the speed of motion along the curve.
}

\subsection*{Curvature}
\addcontentsline{toc}{subsection}{Curvature}

\defrose{Curvature}{
    TThe \textbf{curvature} \(\kappa\) of a curve at a given point measures how quickly the direction of the curve changes at that point. It is defined as the magnitude of the rate of change of the unit tangent vector \(\mathbf{T}(t)\) with respect to the arc length \(s\):

    \[ \kappa = \left| \frac{d\mathbf{T}}{ds} \right|, \]
    
    where the unit tangent vector is given by:
    
    \[ \mathbf{T}(t) = \frac{\mathbf{r}'(t)}{|\mathbf{r}'(t)|}. \]
    
    Using the chain rule, curvature can be expressed in terms of the parameter \(t\):
    
    \[ \kappa = \frac{\left| \frac{d\mathbf{T}}{dt} \right|}{\left| \frac{ds}{dt} \right|}. \]
    
    Since \(\frac{ds}{dt} = |\mathbf{r}'(t)|\), the curvature formula becomes:
    
    \[ \kappa = \frac{\left| \mathbf{T}'(t) \right|}{\left| \mathbf{r}'(t) \right|}. \]
}

\thm{Curvature}{
    The curvature of the curve given by the vector function \textbf{r} is 
    \[ \kappa(t) = \frac{\left|\textbf{r}(t) \times \textbf{r}^{n}(t) \right|}{\left| \textbf{r}'(t) \right|^{3}}\] 
}

\subsection*{The Normal and Binormal Vectors}
\addcontentsline{toc}{subsection}{The Normal and Binormal Vectors}

\defrose{The Normal Vectors}{
    FFor a smooth space curve \(\mathbf{r}(t)\), the \textbf{unit normal vector} \(\mathbf{N}(t)\) and the \textbf{binormal vector} \(\mathbf{B}(t)\) are defined as follows: \\

    The unit normal vector \(\mathbf{N}(t)\) is the normalized derivative of the unit tangent vector \(\mathbf{T}(t)\):
    \[ \mathbf{N}(t) = \frac{\mathbf{T}'(t)}{|\mathbf{T}'(t)|}, \]
    
    where the unit tangent vector is given by:
    \[ \mathbf{T}(t) = \frac{\mathbf{r}'(t)}{|\mathbf{r}'(t)|}. 
    \]
    The vector \(\mathbf{N}(t)\) indicates the direction in which the curve is turning. \\

}

\defrose{Binormal Vectors}{
    FFor a smooth space curve \(\mathbf{r}(t)\), the \textbf{unit normal vector} \(\mathbf{N}(t)\) and the \textbf{binormal vector} \(\mathbf{B}(t)\) are defined as follows: \\
    
    The binormal vector \(\mathbf{B}(t)\) is the cross product of the tangent vector \(\mathbf{T}(t)\) and the normal vector \(\mathbf{N}(t)\):
    \[ \mathbf{B}(t) = \mathbf{T}(t) \times \mathbf{N}(t). \]
    The binormal vector \(\mathbf{B}(t)\) is orthogonal to both \(\mathbf{T}(t)\) and \(\mathbf{N}(t)\) and is also a unit vector.
}

\defrose{Normal Plane}{
    TThe \textbf{normal plane} of a curve \(C\) at a point \(P\) is the plane determined by the normal vector \(\mathbf{N}(t)\) and the binormal vector \(\mathbf{B}(t)\). It is orthogonal to the tangent vector \(\mathbf{T}(t)\) and consists of all lines passing through \(P\) that are orthogonal to \(\mathbf{T}(t)\).}

\defrose{Osculating Planes}{
    TThe \textbf{osculating plane} of a curve \(C\) at a point \(P\) is the plane determined by the tangent vector \(\mathbf{T}(t)\) and the normal vector \(\mathbf{N}(t)\). It is the plane that comes closest to containing the part of the curve near \(P\).
}

\defrose{Circle of Curvature}{ 
    TThe \textbf{circle of curvature}, or \textbf{osculating circle}, of \(C\) at \(P\) is the circle that lies in the osculating plane, passes through \(P\), and has a radius of \(\frac{1}{\kappa}\), where \(\kappa\) is the curvature of the curve at \(P\). The center of the circle is located a distance of \(\frac{1}{\kappa}\) along the direction of the normal vector \(\mathbf{N}(t)\).
    The circle of curvature best describes how the curve \(C\) behaves near \(P\), sharing the same tangent, normal, and curvature at \(P\).
}

\newpage

\section*{13.4 Motion in Space: Veolcity and Acceleration}
\addcontentsline{toc}{section}{13.4 Motion in Space: Velocity and Acceleration}

\subsection*{Velolcity, Speed and Acceleration}
\addcontentsline{toc}{subsection}{Velocity, Speed, and Acceleration}    

\defrose{Velocity and Speed}{
    TThe \textbf{velocity vector} \(\mathbf{v}(t)\) of a particle moving through space is defined as the derivative of its position vector \(\mathbf{r}(t)\) with respect to time:
    \[ \mathbf{v}(t) = \lim_{h \to 0} \frac{\mathbf{r}(t + h) - \mathbf{r}(t)}{h} = \mathbf{r}'(t). \]
    It represents the direction and rate of change of the particle's position and points along the tangent to the particle's path. \\

    The \textbf{speed} of the particle at time \(t\) is the magnitude of the velocity vector:
    \[ |\mathbf{v}(t)| = |\mathbf{r}'(t)| = \frac{ds}{dt}. \]
    It represents the rate of change of distance with respect to time.
}

\defrose{Acceleration Vector}{
    TThe \textbf{acceleration vector} \(\mathbf{a}(t)\) is defined as the derivative of the velocity vector \(\mathbf{v}(t)\) with respect to time:
    \[ \mathbf{a}(t) = \mathbf{v}'(t) = \mathbf{r}''(t).  \]
    It represents the rate of change of the velocity of the particle.

}

\chapter*{14 Partial Derivatives}
\addcontentsline{toc}{chapter}{14 Partial Derivatives}

\section*{14.1 Functions of Several Variables}
\addcontentsline{toc}{section}{Functions of Several Variables}

\subsection*{Functions of Two Variables}
\addcontentsline{toc}{subsection}{Functions of Two Variables}

\defrose{Functions of Two Variables}{
    AA \textbf{function of two variables} \(f(x, y)\) is a rule that assigns a unique real number \(f(x, y)\) to each ordered pair of real numbers \((x, y)\) in a domain \(D\).

    \begin{itemize}
        \item The \textbf{domain} of \(f\) is the set \(D\) of all pairs \((x, y)\) for which \(f(x, y)\) is defined.
        \item The \textbf{range} of \(f\) is the set of all values \(f(x, y)\) takes on.
    \end{itemize}
    
    The variables \(x\) and \(y\) are called \textbf{independent variables}, and \(z = f(x, y)\) is the \textbf{dependent variable}.
}

\section*{14.3 Parital Derivatives}
\addcontentsline{toc}{section}{Partial Derivatives}

\section*{14.5 The Chain Rule}
\addcontentsline{toc}{section}{The Chain Rule}

\section*{14.6 Directional Derivatives and their Gradient Vectors}
\addcontentsline{toc}{section}{Directional Derivatives and their Gradient Vectors}

\section*{14.7 Maximum and Minimum Values}
\addcontentsline{toc}{section}{Maximum and Minimum Values}

\section*{14.8 Lagrange Multipliers}
\addcontentsline{toc}{section}{Lagrange Multipliers}


\chapter*{15 Multiple Integrals}
\addcontentsline{toc}{chapter}{Multiple Integrals}

\section*{15.1 Double Integrals over Rectangles}
\addcontentsline{toc}{section}{Double Integrals over Rectangles}

\section*{15.2 Double Integrals over General Regions}
\addcontentsline{toc}{section}{Double Integrals over Geenral Regions}

\section*{15.3 Double Integrals in Polar Coordinates}
\addcontentsline{toc}{section}{Double Integrals in Polar Coordinates}

\section*{15.6 Triple Integrals}
\addcontentsline{toc}{section}{Triple integrals}

\section*{15.7 Triple Integrals in Cylindrical Coordinates}
\addcontentsline{toc}{section}{Triple Integral in Cylindrical Coordinates}

\section*{15.8 Triple Integrals in Spherical Coordinates}
\addcontentsline{toc}{section}{Triple Integrals in Spherical Coordinates}

\chapter*{16 Vector Calculus}
\addcontentsline{toc}{chapter}{Vector Calculus}

\section*{16.1 Vector Fields}
\addcontentsline{toc}{section}{Vector Fields}

\section*{16.2 Line Integrals}
\addcontentsline{toc}{section}{Line Integrals}

\section*{16.3 The fundamental Theorem for Line Integrals}
\addcontentsline{toc}{section}{The fundamental Theorem for Line Integrals}

\section*{16.4 Green's Theorem}
\addcontentsline{toc}{section}{Stokes' Theorem}

\section*{16.5 Curl and Divergence}
\addcontentsline{toc}{section}{Curl and Divergence}

\subsection*{Curl}

\defrose{Curl}{
    TThe \textbf{curl} of a vector field $\mathbf{F}$ in $\mathbb{R}^3$, where $\mathbf{F} = P\mathbf{i} + Q\mathbf{j} + R\mathbf{k}$, is a vector field that measures the rotational tendency or circulation of $\mathbf{F}$ at a point. Mathematically, it is given by:
    \[
    \text{curl} \, \mathbf{F} = \nabla \times \mathbf{F}
    \]
    
    This can be computed using the cross product of the del operator ($\nabla = \mathbf{i} \frac{\partial}{\partial x} + \mathbf{j} \frac{\partial}{\partial y} + \mathbf{k} \frac{\partial}{\partial z}$) and the vector field $\mathbf{F}$. The result is:
    \[
    \text{curl} \, \mathbf{F} = \begin{vmatrix}
    \mathbf{i} & \mathbf{j} & \mathbf{k} \\
    \frac{\partial}{\partial x} & \frac{\partial}{\partial y} & \frac{\partial}{\partial z} \\
    P & Q & R
    \end{vmatrix}
    \]
    
    Explicitly, the components are:
    \[
    \text{curl} \, \mathbf{F} = \left( \frac{\partial R}{\partial y} - \frac{\partial Q}{\partial z} \right) \mathbf{i} + \left( \frac{\partial P}{\partial z} - \frac{\partial R}{\partial x} \right) \mathbf{j} + \left( \frac{\partial Q}{\partial x} - \frac{\partial P}{\partial y} \right) \mathbf{k}
    \]
    
    This definition captures the curl as the formal representation of the infinitesimal rotation of $\mathbf{F}$ in three-dimensional space.
}

\ntimg{
    The curl of the gradient of a scalar function \( f \) with continuous second-order partial derivatives is always zero:
    \[ \text{curl}(\nabla f) = 0 \]
}

\ntimg{
    If a vector field \( \mathbf{F} \) is defined on all of \( \mathbb{R}^3 \), has continuous partial derivatives, and satisfies \( \text{curl} \, \mathbf{F} = 0 \), then \( \mathbf{F} \) is a conservative vector field.
}

\ntimg{
    The term \textbf{curl} is associated with rotations in a vector field. Specifically:

    \begin{enumerate}
        \item \textbf{Curl and Rotation}: The curl vector indicates the axis and direction of rotation of nearby particles, following the right-hand rule. The magnitude of the curl measures the speed of this rotation.
        
        \item \textbf{Irrotational Flow}: If \( \text{curl} \, \mathbf{F} = 0 \) at a point \( P \), the vector field is called \textbf{irrotational} at \( P \). Here, particles move with the flow but do not rotate about their own axis.
        
        \item \textbf{Rotational Flow}: If \( \text{curl} \, \mathbf{F} \neq 0 \), the particles rotate about their axis. The direction of the curl vector points along the axis of rotation.
        
        \item \textbf{Illustration with Fluid Flow}:
        \begin{itemize}
            \item In a fluid velocity field, a paddle wheel at a point \( P_1 \) where \( \text{curl} \, \mathbf{F} \neq 0 \) would rotate about its axis, indicating rotational flow.
            \item At a point \( P_2 \) where \( \text{curl} \, \mathbf{F} = 0 \), the paddle wheel moves with the flow but does not rotate, indicating irrotational flow.
        \end{itemize}
    \end{enumerate}
}
\defrose{Divergence}{
    TThe \textbf{divergence} of a vector field \( \mathbf{F} \) in \( \mathbb{R}^3 \), where \( \mathbf{F} = P\mathbf{i} + Q\mathbf{j} + R\mathbf{k} \), is a scalar function that measures the rate at which the vector field spreads out (or converges) at a given point. It is defined as:
    \[
    \text{div} \, \mathbf{F} = \frac{\partial P}{\partial x} + \frac{\partial Q}{\partial y} + \frac{\partial R}{\partial z}
    \]
    
    Alternatively, using the del operator (\( \nabla = \mathbf{i} \frac{\partial}{\partial x} + \mathbf{j} \frac{\partial}{\partial y} + \mathbf{k} \frac{\partial}{\partial z} \)), the divergence can be expressed as the dot product of \( \nabla \) and \( \mathbf{F} \):
    \[
    \text{div} \, \mathbf{F} = \nabla \cdot \mathbf{F}
    \]
    
    \subsection*{Key Property}
    For a vector field \( \mathbf{F} \) in \( \mathbb{R}^3 \) with continuous second-order partial derivatives, the divergence of the curl of \( \mathbf{F} \) is always zero:
    \[
    \text{div} (\text{curl} \, \mathbf{F}) = 0
    \]
    
    This captures the idea that the curl of a vector field does not "spread out" or "converge" in space.    
}

\ntimg{
    \begin{enumerate}
        \item \textbf{Divergence in Fluid Flow}:
        \begin{itemize}
            \item If $\mathbf{F}(x, y, z)$ represents the velocity of a fluid or gas, $\text{div} \, \mathbf{F}(x, y, z)$ measures the net rate of change of the mass of fluid (or gas) flowing from the point $(x, y, z)$ per unit volume.
            \item It indicates the tendency of the fluid to spread out (diverge) from a point.
        \end{itemize}

        \item \textbf{Incompressible Flow}:
        \begin{itemize}
            \item If $\text{div} \, \mathbf{F} = 0$, the fluid is said to be \textbf{incompressible}, meaning there is no net outflow or inflow at that point.
        \end{itemize}

        \item \textbf{Illustration}:
        \begin{itemize}
            \item \textbf{Case 1 ($\text{div} \, \mathbf{F} \neq 0$)}: 
            \begin{itemize}
                \item At a point $P_1$, if $\text{div} \, \mathbf{F} < 0$, the flow is inward (net inflow).
                \item At a point $P_2$, if $\text{div} \, \mathbf{F} > 0$, the flow is outward (net outflow).
            \end{itemize}
            \item \textbf{Case 2 ($\text{div} \, \mathbf{F} = 0$)}: 
            \begin{itemize}
                \item The flow is balanced (no net divergence or convergence).
            \end{itemize}
        \end{itemize}
    \end{enumerate}
}

\subsection*{Vector Forms of Green's Theorem}

\defrose{Vector Forms of Green's Theorem}{
    Green's Theorem relates a line integral around a simple closed curve \( C \) to a double integral over the plane region \( D \) enclosed by \( C \). It provides two key vector forms:
    
    \subsection*{1. Curl Form (Tangential Component):}
    The line integral of the tangential component of \( \mathbf{F} \) along \( C \) is equal to the double integral of the vertical component of \( \text{curl} \, \mathbf{F} \) over \( D \):
    \[
    \oint_C \mathbf{F} \cdot d\mathbf{r} = \oint_C \mathbf{F} \cdot \mathbf{T} \, ds = \iint_D (\text{curl} \, \mathbf{F}) \cdot \mathbf{k} \, dA
    \]
    where 
    \[
    (\text{curl} \, \mathbf{F}) \cdot \mathbf{k} = \frac{\partial Q}{\partial x} - \frac{\partial P}{\partial y}.
    \]
    
    \subsection*{2. Divergence Form (Normal Component):}
    The line integral of the normal component of \( \mathbf{F} \) along \( C \) is equal to the double integral of the divergence of \( \mathbf{F} \) over \( D \):
    \[
    \oint_C \mathbf{F} \cdot \mathbf{n} \, ds = \iint_D \text{div} \, \mathbf{F} \, dA
    \]
    where 
    \[
    \text{div} \, \mathbf{F} = \frac{\partial P}{\partial x} + \frac{\partial Q}{\partial y}.
    \]
    
    \subsection*{Geometric Interpretation:}
    \begin{itemize}
        \item The curl form expresses the relationship between the circulation of \( \mathbf{F} \) along the boundary \( C \) and the "rotational tendency" of \( \mathbf{F} \) over the region \( D \).
        \item The divergence form relates the flux of \( \mathbf{F} \) across \( C \) to the net outflow (or inflow) of \( \mathbf{F} \) over \( D \).
    \end{itemize}
    
    These two forms of Green's Theorem are fundamental in understanding the relationship between local properties of vector fields and their global integrals.
}

\section*{16.6 Parametric Surfaces and Their Areas}
\addcontentsline{toc}{section}{Parametric Surfaces and Their Areas}

\subsection*{Parametric Surfaces}

\defrose{Parametric Surface}{
    AA \textit{parametric surface} is a surface in three-dimensional space $\mathbb{R}^3$ defined by a vector-valued function $\mathbf{r}(u, v)$, which depends on two parameters $u$ and $v$. The function is expressed as:
    \[ \mathbf{r}(u, v) = x(u, v) \, \mathbf{i} + y(u, v) \, \mathbf{j} + z(u, v) \, \mathbf{k}, \]
    where $x(u, v)$, $y(u, v)$, and $z(u, v)$ are the component functions of $\mathbf{r}$, representing the $x$-, $y$-, and $z$-coordinates of the surface, respectively. These functions are defined over a region $D$ in the $uv$-plane. The set of all points $(x, y, z) \in \mathbb{R}^3$ that satisfy:
    \[ x = x(u, v), \quad y = y(u, v), \quad z = z(u, v), \]
    as $(u, v)$ varies over $D$, forms the parametric surface $S$.

}

\subsection*{Parametric Equations}

\defrose{Parametric Equations}{
    FFor a parametric surface the \textit{parametric equations} are equations that describe the coordinates $(x, y, z)$ of points on the surface as functions of two independent parameters $u$ and $v$. For a parametric surface $S$, these equations are given by:
    \[ x = x(u, v), \quad y = y(u, v), \quad z = z(u, v), \]
    where $x(u, v)$, $y(u, v)$, and $z(u, v)$ are the component functions of a vector-valued function $\mathbf{r}(u, v)$. These equations define the spatial coordinates of the surface for every pair of parameters $(u, v)$ in a specified domain $D$ in the $uv$-plane.
}

\subsection*{Grid Curves}

\defrose{Grid Curves}{
    OOn a parametric surface $s$ \textit{grid curves} are families of curves defined by the vector function $\mathbf{r}(u, v)$. They are obtained by fixing one parameter and varying the other:

    1. \textbf{Curves with $u = u_0$:} When $u$ is held constant, the parametric surface reduces to a curve:
    \[
    \mathbf{r}(u_0, v) = \langle x(u_0, v), y(u_0, v), z(u_0, v) \rangle,
    \]
    which traces a curve $C_1$ on the surface as $v$ varies.

    2. \textbf{Curves with $v = v_0$:} When $v$ is held constant, the parametric surface reduces to a curve:
    \[
    \mathbf{r}(u, v_0) = \langle x(u, v_0), y(u, v_0), z(u, v_0) \rangle,
    \]
    which traces a curve $C_2$ on the surface as $u$ varies.

    These two families of curves correspond to horizontal and vertical lines in the $uv$-plane and form a grid-like structure when plotted on the surface.

}

\subsection*{Spherical Coordinates}

\subsection*{Surfaces of Revolution}

\defrose{Surfaces of Revolution}{
    AA \textbf{surface of revolution} is generated by rotating a curve \( C \), defined parametrically or as a function, about a fixed axis in three-dimensional space. The parametric equations of the surface can be expressed as:

    \[
    \begin{aligned}
    x &= u, \\
    y &= r(u) \cos \theta, \\
    z &= r(u) \sin \theta,
    \end{aligned}
    \]

    where:
    \begin{itemize}
        \item \( u \) is a parameter describing the curve \( C \),
        \item \( r(u) \) is the radial distance of the curve from the axis of rotation,
        \item \( \theta \in [0, 2\pi] \) is the angle of rotation.
    \end{itemize}

    The domain of the parameters \( u \) and \( \theta \) depends on the curve and the extent of rotation.


}

\subsection*{Tangent Planes}

\defrose{Tangent Planes}{
    T
    The \textbf{tangent plane} to a parametric surface \( S \) at a point \( P_0(u_0, v_0) \) is the plane that best approximates \( S \) near \( P_0 \). 
    
    If \( S \) is defined by a vector-valued function:
    \[ \mathbf{r}(u, v) = x(u, v) \mathbf{i} + y(u, v) \mathbf{j} + z(u, v) \mathbf{k}, \]
    then the tangent plane at \( P_0 \) is determined by the two tangent vectors at \( P_0 \):
    \[ \mathbf{r}_u = \frac{\partial \mathbf{r}}{\partial u} = \frac{\partial x}{\partial u} \mathbf{i} + \frac{\partial y}{\partial u} \mathbf{j} + \frac{\partial z}{\partial u} \mathbf{k}, \]
    \[ \mathbf{r}_v = \frac{\partial \mathbf{r}}{\partial v} = \frac{\partial x}{\partial v} \mathbf{i} + \frac{\partial y}{\partial v} \mathbf{j} + \frac{\partial z}{\partial v} \mathbf{k}.  \]
    
    The tangent plane at \( P_0 \) is spanned by \(\mathbf{r}_u\) and \(\mathbf{r}_v\). A normal vector to the plane is given by:
    \[ \mathbf{n} = \mathbf{r}_u \times \mathbf{r}_v. \]
    
    The equation of the tangent plane can be expressed in the point-normal form:
    \[ \mathbf{n} \cdot (\mathbf{r} - \mathbf{r}(u_0, v_0)) = 0, \]
    where \(\mathbf{r}(u_0, v_0)\) is the position vector of \( P_0 \).
    
    For the tangent plane to exist, the cross product \(\mathbf{r}_u \times \mathbf{r}_v\) must be nonzero, ensuring that \( S \) is smooth at \( P_0 \).
    
}
\subsection*{Surface Area for a Parametric Surface}

\defrose{Surface Area for a Parametric Surface}{
    TThe \textbf{surface area} of a smooth parametric surface \( S \), defined by the vector-valued function:
    \[ \mathbf{r}(u, v) = x(u, v) \mathbf{i} + y(u, v) \mathbf{j} + z(u, v) \mathbf{k}, \quad (u, v) \in D, \]
    where \( D \) is the parameter domain, is given by the integral:
    \[ A(S) = \iint_D \left| \mathbf{r}_u \times \mathbf{r}_v \right| \, dA, \]
    where:
    \[ \mathbf{r}_u = \frac{\partial x}{\partial u} \mathbf{i} + \frac{\partial y}{\partial u} \mathbf{j} + \frac{\partial z}{\partial u} \mathbf{k}, \quad
    \mathbf{r}_v = \frac{\partial x}{\partial v} \mathbf{i} + \frac{\partial y}{\partial v} \mathbf{j} + \frac{\partial z}{\partial v} \mathbf{k}. \]
    
    The cross product \(\mathbf{r}_u \times \mathbf{r}_v\) represents a vector orthogonal to the tangent plane at each point on the surface, and its magnitude \(\left| \mathbf{r}_u \times \mathbf{r}_v \right|\) gives the infinitesimal area of a parallelogram spanned by the tangent vectors \(\mathbf{r}_u\) and \(\mathbf{r}_v\). Integrating this quantity over the parameter domain \( D \) yields the total surface area of \( S \).
        
}

\subsection*{Surface Area of the Graph of a Function}

\defrose{Surface Area of the Graph of a Function}{
    TThe \textbf{surface area} of the graph of a function \( z = f(x, y) \), where \( f(x, y) \) has continuous partial derivatives, over a region \( D \) in the \( xy \)-plane is given by:
    \[
    A(S) = \iint_D \sqrt{1 + \left( \frac{\partial f}{\partial x} \right)^2 + \left( \frac{\partial f}{\partial y} \right)^2} \, dA.
    \]
    
    \subsection*{Explanation}
    \begin{itemize}
        \item The parametric representation of the surface is:
        \[ \mathbf{r}(x, y) = x \mathbf{i} + y \mathbf{j} + f(x, y) \mathbf{k}. \]
        \item The tangent vectors are:
        \[ \mathbf{r}_x = \mathbf{i} + \frac{\partial f}{\partial x} \mathbf{k}, \quad \mathbf{r}_y = \mathbf{j} + \frac{\partial f}{\partial y} \mathbf{k}. \]
        \item The magnitude of the cross product of the tangent vectors is:
        \[ \left| \mathbf{r}_x \times \mathbf{r}_y \right| = \sqrt{1 + \left( \frac{\partial f}{\partial x} \right)^2 + \left( \frac{\partial f}{\partial y} \right)^2}. \]
    \end{itemize}
    
    Integrating this quantity over the region \( D \) in the \( xy \)-plane gives the total surface area of the graph of \( f(x, y) \).    
}


\section*{16.7 Surface Integral}
\addcontentsline{toc}{section}{Surface Integral}

\subsection*{Parametric Surfaces}

\defrose{Surface Integral for Parametric Surfaces}{
    TThe \textbf{surface integral} of a scalar function \( f(x, y, z) \) over a parametric surface \( S \), defined by the vector equation:
    \[ \mathbf{r}(u, v) = x(u, v) \mathbf{i} + y(u, v) \mathbf{j} + z(u, v) \mathbf{k}, \quad (u, v) \in D, \]
    is given by:
    \[ \iint_S f(x, y, z) \, dS = \iint_D f(\mathbf{r}(u, v)) \left| \mathbf{r}_u \times \mathbf{r}_v \right| \, dA. \]
    
    \subsection*{Explanation}
    \begin{itemize}
        \item The parameter domain \( D \) is divided into subrectangles with dimensions \( \Delta u \) and \( \Delta v \), and each corresponding surface patch is approximated as a parallelogram in the tangent plane.
        \item The area of a surface patch is approximated as:
        \[ \Delta S_{ij} \approx \left| \mathbf{r}_u \times \mathbf{r}_v \right| \Delta u \Delta v.\]
        \item The surface integral is defined as the limit of a Riemann sum:
        \[ \iint_S f(x, y, z) \, dS = \lim_{m, n \to \infty} \sum_{i=1}^m \sum_{j=1}^n f(P_{ij}^*) \Delta S_{ij}. \]
    \end{itemize}
    
    \subsection*{Key Components}
    \begin{itemize}
        \item \(\mathbf{r}_u\) and \(\mathbf{r}_v\) are the partial derivatives of \(\mathbf{r}(u, v)\) with respect to \( u \) and \( v \), respectively:
        \[ \mathbf{r}_u = \frac{\partial \mathbf{r}}{\partial u}, \quad \mathbf{r}_v = \frac{\partial \mathbf{r}}{\partial v}. \]
        \item \(\mathbf{r}_u \times \mathbf{r}_v\) gives a vector normal to the surface at each point, and \(\left| \mathbf{r}_u \times \mathbf{r}_v \right|\) represents the infinitesimal surface area element.
    \end{itemize}
    
    This integral evaluates the contribution of \( f(x, y, z) \) across the entire surface \( S \).
        
}

\subsection*{Graphs of Functions}
\defrose{Surafe Integrals for Graphs of Functions}{
    TThe \textbf{surface integral} of a scalar function \( f(x, y, z) \) over the graph of a function \( z = g(x, y) \), where \( g(x, y) \) has continuous partial derivatives, is given by:
    \[
    \iint_S f(x, y, z) \, dS = \iint_D f(x, y, g(x, y)) \sqrt{\left(\frac{\partial g}{\partial x}\right)^2 + \left(\frac{\partial g}{\partial y}\right)^2 + 1} \, dA.
    \]
    
    \subsection*{Explanation}
    \begin{itemize}
        \item The graph of the function \( z = g(x, y) \) can be regarded as a parametric surface with:
        \[
        x = x, \quad y = y, \quad z = g(x, y).
        \]
        \item The tangent vectors to this surface are:
        \[
        \mathbf{r}_x = \mathbf{i} + \frac{\partial g}{\partial x} \mathbf{k}, \quad \mathbf{r}_y = \mathbf{j} + \frac{\partial g}{\partial y} \mathbf{k}.
        \]
        \item The cross product of the tangent vectors is:
        \[
        \mathbf{r}_x \times \mathbf{r}_y = -\frac{\partial g}{\partial x} \mathbf{j} - \frac{\partial g}{\partial y} \mathbf{i} + \mathbf{k}.
        \]
        \item The magnitude of the cross product is:
        \[
        \left| \mathbf{r}_x \times \mathbf{r}_y \right| = \sqrt{\left(\frac{\partial g}{\partial x}\right)^2 + \left(\frac{\partial g}{\partial y}\right)^2 + 1}.
        \]
    \end{itemize}
    
    By integrating this quantity over the region \( D \) in the \( xy \)-plane, we account for the contributions of \( f(x, y, z) \) over the entire surface \( S \).    
}

\subsection*{Oriented Surfaces}

\defrose{Oriented Surfaces}{
    AAn \textbf{oriented surface} is an orientable (two-sided) surface \( S \) where it is possible to define a continuous, unit normal vector \( \mathbf{n} \) at every point \( (x, y, z) \) on the surface, except possibly at boundary points.

    \subsection*{Key Properties}
    \begin{itemize}
        \item \textbf{Two Possible Orientations}: For any orientable surface, there are two choices for the unit normal vector:
        \begin{itemize}
            \item \( \mathbf{n}_1 \), the chosen unit normal vector.
            \item \( \mathbf{n}_2 = -\mathbf{n}_1 \), the opposite orientation.
        \end{itemize}
        \item A surface is called \textbf{orientable} if it is possible to assign \( \mathbf{n} \) continuously over the entire surface \( S \).
        \item A classic example of a non-orientable surface is the Möbius strip, which has only one side and no consistent orientation.
    \end{itemize}

    \subsection*{Explanation}
    An oriented surface requires the existence of a consistent way to assign a "positive" or "negative" side across all points on the surface. The orientation is provided by the chosen direction of the normal vector \( \mathbf{n} \), which varies smoothly across the surface.
}

\subsection*{Surface Integrals of Vector Fields; Flux}
\defrose{Flux}{
    The \textbf{surface integral of a vector field} (also called the \textbf{flux}) over an oriented surface \( S \) with a unit normal vector \( \mathbf{n} \) is defined as:
    \[
    \iint_S \mathbf{F} \cdot d\mathbf{S} = \iint_S \mathbf{F} \cdot \mathbf{n} \, dS,
    \]
    where:
    \begin{itemize}
        \item \( \mathbf{F} \) is a continuous vector field defined on \( S \),
        \item \( \mathbf{n} \) is the unit normal vector to \( S \),
        \item \( dS \) represents the infinitesimal surface area element.
    \end{itemize}
    
    \subsection*{Special Case: Surface Defined by \( z = g(x, y) \)}
    If the surface \( S \) is defined by the graph \( z = g(x, y) \), and \( \mathbf{F}(x, y, z) = P \mathbf{i} + Q \mathbf{j} + R \mathbf{k} \), the surface integral can be expressed as:
    \[
    \iint_S \mathbf{F} \cdot d\mathbf{S} = \iint_D \left( -P \frac{\partial g}{\partial x} - Q \frac{\partial g}{\partial y} + R \right) \, dA,
    \]
    where \( D \) is the projection of the surface onto the \( xy \)-plane.
    
    This formula assumes the upward orientation of \( S \). For a downward orientation, the integral is multiplied by \( -1 \).
    
    \subsection*{Parametric Form}
    If the surface \( S \) is parameterized by \( \mathbf{r}(u, v) \), with tangent vectors:
    \[
    \mathbf{r}_u = \frac{\partial \mathbf{r}}{\partial u}, \quad \mathbf{r}_v = \frac{\partial \mathbf{r}}{\partial v},
    \]
    then the flux integral can be written as:
    \[
    \iint_S \mathbf{F} \cdot d\mathbf{S} = \iint_D \mathbf{F}(\mathbf{r}(u, v)) \cdot (\mathbf{r}_u \times \mathbf{r}_v) \, dA,
    \]
    where \( D \) is the parameter domain.
    
    \subsection*{Physical Interpretation}
    The flux integral measures the total flow of the vector field \( \mathbf{F} \) across the surface \( S \), representing quantities like mass flow rate, electric flux, or fluid flow through \( S \).
}    

\section*{16.8 Stokes' Theorem} 

\defrose{Stokes' Theorem}{
    SStokes' Theorem relates the surface integral of the curl of a vector field over an oriented surface \( S \) to the line integral of the vector field along the boundary curve \( C \) of \( S \). Mathematically, it is expressed as:

    \[
    \iint_S (\nabla \times \mathbf{F}) \cdot d\mathbf{S} = \oint_C \mathbf{F} \cdot d\mathbf{r},
    \]

    where:
    \begin{itemize}
        \item \( S \) is a piecewise-smooth, oriented surface with unit normal vector \( \mathbf{n} \),
        \item \( C \) is the positively oriented, closed boundary curve of \( S \),
        \item \( \mathbf{F} \) is a vector field with continuous partial derivatives,
        \item \( \nabla \times \mathbf{F} \) is the curl of \( \mathbf{F} \),
        \item \( d\mathbf{S} = \mathbf{n} \, dS \) is the oriented surface element,
        \item \( d\mathbf{r} \) is the infinitesimal vector along \( C \).
    \end{itemize}

    \subsection*{Key Notes}
    \begin{itemize}
        \item \textbf{Positive Orientation:} The orientation of \( C \) is determined by the right-hand rule: when you walk along \( C \) with your head pointing in the direction of \( \mathbf{n} \), the surface \( S \) remains on your left.
        \item \textbf{Special Case:} If \( S \) lies flat in the \( xy \)-plane and \( \mathbf{n} = \mathbf{k} \), Stokes' Theorem reduces to Green's Theorem:
        \[
        \oint_C \mathbf{F} \cdot d\mathbf{r} = \iint_D (\nabla \times \mathbf{F}) \cdot \mathbf{k} \, dA.
        \]
    \end{itemize}

    Stokes' Theorem provides a fundamental relationship between the circulation of \( \mathbf{F} \) along \( C \) and the total rotational effects (curl) of \( \mathbf{F} \) over the surface \( S \).
}

\ntimg{
    Stokes' Theorem allows us to compute a surface integral simply by knowing the values of \( \mathbf{F} \) on the boundary curve \( C \). This means that if we have another oriented surface with the same boundary curve \( C \), then we get exactly the same value for the surface integral. In general, if \( S_1 \) and \( S_2 \) are oriented surfaces with the same oriented boundary curve \( C \) and both satisfy the hypotheses of Stokes' Theorem, then:
    \[
    \int_{S_1} (\nabla \times \mathbf{F}) \cdot d\mathbf{S} = \int_C \mathbf{F} \cdot d\mathbf{r} = \int_{S_2} (\nabla \times \mathbf{F}) \cdot d\mathbf{S}.
    \]

    This fact is useful when it is difficult to integrate over one surface but easy to integrate over the other.
}
\addcontentsline{toc}{section}{Stokes' Theorem}


\section*{16.9 The Divergence Theorem} 
\addcontentsline{toc}{section}{The Divergence Theorem}







\end{document}