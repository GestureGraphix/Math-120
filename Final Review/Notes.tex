\documentclass{report}

\input{preamble}
\input{macros}
\input{letterfonts}

\title{\Huge{Math 120}}
\author{\huge{Final Review}}
\date{Nov 20 2024}

\begin{document}

\maketitle
\newpage% or \cleardoublepage
% \pdfbookmark[<level>]{<title>}{<dest>}
\pdfbookmark[section]{\contentsname}{toc}
\tableofcontents
\pagebreak

\chapter*{12 Vectors and the Geometry of Space}
\addcontentsline{toc}{chapter}{12 Vectors and the Geometry of Space }


\section*{12.1 Three-Dimensional Coordinate Systems}
\addcontentsline{toc}{section}{12.1 Three-Dimensional Coordinate Systems}

\subsection*{3D Space}

\defrose{Three Dimensional Rectangular Coordinate System}{
    TThe carstesian product $ \mathbb{R} \times \mathbb{R} \times \mathbb{R} = \left[ (x,y,z) \, | \, x,y,z \in \mathbb{R} \right]$. is the set of all real 
    numbers and is denoted by $ \mathbb{R}$
}

\addcontentsline{toc}{subsection}{3D Space}

\subsection*{Surfaces and Solids}
\addcontentsline{toc}{subsection}{Surfaces and Solids}

\ntimg{
    In three-dimensional analytic, an equation in $x,y$ and, $z$ represents a surface in 
    $\mathbb{R}$  
}

\subsection*{Distance and Spheres}
\addcontentsline{toc}{subsection}{Distances and Spheres}

\defrose{Distance Formula in Three Dimensions}{
    TThe distance $| P_{1} P_{2}|$ between the points $P_{1}(x_{1},y_{1},z_{1})$ and $P_{2}(x_{1}, y_{2}, z_{2})$ is 
    \[ |P_{1}P_{2}| = \sqrt{\left(x_{2} - x_{1}\right)^{2} + \left(y_{2} - y_{1}\right)^{2} + \left(z_{2} - z_{1}\right)^{2}}\] 
}

\section*{12.2 Vectors}
\addcontentsline{toc}{section}{12.2 Vectors}

\defrose{Vectors}{
    AA \textit{vector} indicates a quantity that has both magnitude and direction
}

\subsection*{Geometric Description of Vectors}
\addcontentsline{toc}{subsection}{Geometric Description of Vectors}

\defrose{Vector Addition}{
    FFor vectors \textbf{Vector Addition} means the sum of two vectors $\mathbf{u} + \mathbf{v}$ is the vector from the initial point of $\mathbf{u}$ to the terminal point of $\mathbf{v}$, with $\mathbf{v}$ placed so its tail starts at the tip of $\mathbf{u}$..
}

\defrose{Triangle Law}{
    FFor vectors that \textbf{Triangle Law} refers to the sum of two vectors representing the third side of a triangle by placing the second vector's tail at the first vector's tip.
}

\defrose{Parallelogram Law}{
    FFor vectors \textbf{Parallelogram Law} refers to the sum of two vectors is representing the diagonal of a parallelogram formed when both vectors originate from the same point.
}

\defrose{Scalar Multiplication}{
    FFor vectors \textbf{Scalar Multiplication} means that if $c$ is a scalar and $\mathbf{v}$ is a vector, the scalar multiple $c\mathbf{v}$ is a vector whose length is $|c|$ times the length of $\mathbf{v}$. Its direction matches $\mathbf{v}$ if $c > 0$, is opposite to $\mathbf{v}$ if $c < 0$, and is $\mathbf{0}$ if $c = 0$ or $\mathbf{v} = \mathbf{0}$.
}

\subsection*{Components of Vectors}
\addcontentsline{toc}{subsection}{Components of Vectors}

\defrose{Vector Representation}{
    GGiven points $A(x_1, y_1, z_1)$ and $B(x_2, y_2, z_2)$, the vector $\mathbf{a}$ representing the directed segment $\overrightarrow{AB}$ is 
    \[ \mathbf{a} = \langle x_2 - x_1, y_2 - y_1, z_2 - z_1 \rangle. \]
}

\defrose{Magnitude of a Vector}{
    TThe length of a vector $\mathbf{a}$ is 
    \[ |\mathbf{a}| = \sqrt{a_1^2 + a_2^2} \quad \text{(2D)}, \quad |\mathbf{a}| = \sqrt{a_1^2 + a_2^2 + a_3^2} \quad \text{(3D)}. \]
}
\ntimg{
    \textbf{Vector Addition:} Add corresponding components:
    \[\mathbf{a} + \mathbf{b} = \langle a_1 + b_1, a_2 + b_2 \rangle \quad \text{(2D)}, \quad \mathbf{a} + \mathbf{b} = \langle a_1 + b_1, a_2 + b_2, a_3 + b_3 \rangle \quad \text{(3D)}.\]

    \textbf{Vector Subtraction:} Subtract corresponding components:
    \[ \mathbf{a} - \mathbf{b} = \langle a_1 - b_1, a_2 - b_2 \rangle \quad \text{(2D)}, \quad \mathbf{a} - \mathbf{b} = \langle a_1 - b_1, a_2 - b_2, a_3 - b_3 \rangle \quad \text{(3D)}. \]

    \textbf{Scalar Multiplication:} Multiply each component by the scalar:
    \[ c\mathbf{a} = \langle c a_1, c a_2 \rangle \quad \text{(2D)}, \quad c\mathbf{a} = \langle c a_1, c a_2, c a_3 \rangle \quad \text{(3D)}. \]
}

\defrose{Unit Vector}{
    AA vector whose magnitude is 1
}

\newpage

\section*{12.3 The Dot Product}
\addcontentsline{toc}{section}{12.3 The dot product}

\subsection*{Dot Product of Two Vectors}
\addcontentsline{toc}{subsection}{Dot Product of Two Vectors}

\defrose{Dot Product}{
    TThe \textit{dot product} of two vectors $\mathbf{a} = \langle a_1, a_2, a_3 \rangle$ and $\mathbf{b} = \langle b_1, b_2, b_3 \rangle$ is a scalar given by:
    \[ \mathbf{a} \cdot \mathbf{b} = a_1b_1 + a_2b_2 + a_3b_3. \]
}

\thm{Angle Between Vectors}{
    The dot product of two vectors $\mathbf{a}$ and $\mathbf{b}$ relates their magnitudes and the cosine of the angle $\theta$ between them:
    \[ \mathbf{a} \cdot \mathbf{b} = |\mathbf{a}| |\mathbf{b}| \cos\theta. \]
    This formula shows how the dot product measures both the lengths of the vectors and their alignment in space.
}

\cor{Angle Between Vectors}{
    If $\theta$ is the angle between the nonzero vectors \textbf{a} annd \textbf{b}, then
    \[ \cos \theta = \frac{\textbf{a} \cdot \textbf{b}}{|\textbf{a}||\textbf{b}|} \] 
}

\ntimg{
    Two vectors \textbf{a} and \textbf{b} are orthogonal if and only if  $\textbf{a} \cdot \textbf{b} = 0 $
}

\subsection*{Direction Angles and Direction Cosines}
\addcontentsline{toc}{subsection}{Direction Angles and Direction Cosines}

\defrose{Direction Cosines}{
    TThe direction angles ($\alpha, \beta, \gamma$) of a nonzero vector $\mathbf{a}$ are the angles it makes with the positive $x$-, $y$-, and $z$-axes, respectively. The cosines of these angles ($\cos\alpha, \cos\beta, \cos\gamma$) are called the direction cosines, and they satisfy:
    \[ \cos^2\alpha + \cos^2\beta + \cos^2\gamma = 1. \]
}

\ntimg{
    The vector $\mathbf{a}$ can be expressed in terms of its magnitude and direction cosines as:
    \[ \mathbf{a} = |\mathbf{a}| \langle \cos\alpha, \cos\beta, \cos\gamma \rangle, \]
    and the unit vector in the direction of $\mathbf{a}$ is:
    \[ \frac{1}{|\mathbf{a}|} \mathbf{a} = \langle \cos\alpha, \cos\beta, \cos\gamma \rangle. \]
}

\subsection*{Projections}
\addcontentsline{toc}{subsection}{Projections}

\defrose{Projections}{
    TThe \textbf{scalar projection} of $\mathbf{b}$ onto $\mathbf{a}$, denoted $\text{comp}_{\mathbf{a}} \mathbf{b}$, represents the magnitude of $\mathbf{b}$ in the direction of $\mathbf{a}$. It is the length of the "shadow" of $\mathbf{b}$ onto $\mathbf{a}$ and is given by:
    \[ \text{comp}_{\mathbf{a}} \mathbf{b} = \frac{\mathbf{a} \cdot \mathbf{b}}{|\mathbf{a}|}. \]

    The \textbf{Vector Projection:} The vector projection of $\mathbf{b}$ onto $\mathbf{a}$, denoted $\text{proj}_{\mathbf{a}} \mathbf{b}$, represents the vector component of $\mathbf{b}$ in the direction of $\mathbf{a}$. It is the scaled vector of $\mathbf{a}$ that aligns with the projection, given by:
    \[ \text{proj}_{\mathbf{a}} \mathbf{b} = \frac{\mathbf{a} \cdot \mathbf{b}}{|\mathbf{a}|^2} \mathbf{a}. \]
}

\section*{12.4 The Cross Product}
\addcontentsline{toc}{section}{12.4 The Cross Product}

\subsection*{The Cross Product of Two Vectors}
\addcontentsline{toc}{subsection}{The Cross Product of Two Vectors}

\defrose{Cross Product}{
    TThe \textbf{cross product} of two vectors \(\vec{a} = \langle a_1, a_2, a_3 \rangle\) and \(\vec{b} = \langle b_1, b_2, b_3 \rangle\) is a vector \(\vec{a} \times \vec{b}\) defined as:
    \[ \vec{a} \times \vec{b} = \langle a_2b_3 - a_3b_2, \, a_3b_1 - a_1b_3, \, a_1b_2 - a_2b_1 \rangle \]

    The cross product is a vector:
    \begin{itemize}
        \item \textbf{Perpendicular} to both \(\vec{a}\) and \(\vec{b}\).
        \item \textbf{Magnitude} equal to \(\|\vec{a}\| \|\vec{b}\| \sin \theta\), where \(\theta\) is the angle between \(\vec{a}\) and \(\vec{b}\).
        \item \textbf{Direction} determined by the right-hand rule.
    \end{itemize}
}

\subsection*{Properties of the Cross Product}
\addcontentsline{toc}{subsection}{Properties of the Cross Product}

\thm{Resulting Vector}{
    The vector \textbf{a} $\times$ \textbf{b} is orthogonal to both \textbf{a} and \textbf{b} 
}

\thm{Geometric Description of the Length of the Cross Product}{
    If $\theta$ is the angle between \textbf{a} and \textbf{b} (s0 $ 0 \leq \theta \leq \pi$), then the 
    length of the cross product $\textbf{a} \times \textbf{b}$ is given by 
    \[ |\textbf{a} \times \textbf{b} | = |\textbf{a}| |texbt{b}| \sin \theta \] 
}

\cor{Parallellity using Cross Product}{
    Two nonzero vectors \textbf{a} and \textbf{b} are parallel if and only if 
    \[ \textbf{a} \times \textbf{b} = 0 \] 
}

\ntimg{
    The length of the cross product \textbf{a} $\times$ \textbf{b} is equal to the area of the parallelogram determined
    by \textbf{a} and \textbf{b}  
}

\subsection*{Triple Products}
\addcontentsline{toc}{subsection}{Triple Products}  

\defrose{Triple Product}{
    TThe \textbf{scalar triple product} of three vectors \(\mathbf{a}\), \(\mathbf{b}\), and \(\mathbf{c}\) is given by:
    \[ \mathbf{a} \cdot (\mathbf{b} \times \mathbf{c}), \]
    which can also be expressed as the determinant of the matrix formed by placing the vectors as rows (or columns):
    \[ \mathbf{a} \cdot (\mathbf{b} \times \mathbf{c}) = 
    \begin{vmatrix}
    a_1 & a_2 & a_3 \\
    b_1 & b_2 & b_3 \\
    c_1 & c_2 & c_3
    \end{vmatrix}. \]
        
    \begin{itemize}
        \item The magnitude of the scalar triple product, \(|\mathbf{a} \cdot (\mathbf{b} \times \mathbf{c})|\), represents the \textbf{volume} of the parallelepiped formed by the vectors \(\mathbf{a}\), \(\mathbf{b}\), and \(\mathbf{c}\).
        \item If \(\mathbf{a} \cdot (\mathbf{b} \times \mathbf{c}) = 0\), then the vectors \(\mathbf{a}\), \(\mathbf{b}\), and \(\mathbf{c}\) are \textbf{coplanar}, meaning they lie in the same plane.
    \end{itemize}
}

\section*{12.5 Equations of Lines and Planes}
\addcontentsline{toc}{section}{12.5 Equations of Lines and Planes}

\subsection*{Lines}
\addcontentsline{toc}{subsection}{Lines}

\defrose{Line in 3D Space}{
    AA line in three-dimensional space is defined by:
    \begin{itemize}
        \item A point \(\mathbf{r}_0 = \langle x_0, y_0, z_0 \rangle\),
        \item A direction vector \(\mathbf{v} = \langle a, b, c \rangle\).
    \end{itemize}
    
    Its vector equation is:
    \[ \mathbf{r} = \mathbf{r}_0 + t\mathbf{v}, \]
    where \(t\) is a scalar parameter.
    
    In parametric form, the equation becomes:
    \[ x = x_0 + ta, \quad y = y_0 + tb, \quad z = z_0 + tc. \]
}

\subsection*{Planes}
\addcontentsline{toc}{subsection}{Planes}

\defrose{Plane in 3D Space}{
    AA plane in three-dimensional space is defined by:
    \begin{itemize}
        \item A point \(\mathbf{r}_0 = \langle x_0, y_0, z_0 \rangle\) on the plane,
        \item A normal vector \(\mathbf{n} = \langle a, b, c \rangle\), orthogonal to the plane.
    \end{itemize}

    The vector equation of the plane is:
    \[
    \mathbf{n} \cdot (\mathbf{r} - \mathbf{r}_0) = 0,
    \]
    where \(\mathbf{r} = \langle x, y, z \rangle\) is the position vector of any point on the plane.

    The scalar equation of the plane is:
    \[
    a(x - x_0) + b(y - y_0) + c(z - z_0) = 0,
    \]
    where \((x_0, y_0, z_0)\) is a point on the plane, and \(\mathbf{n} = \langle a, b, c \rangle\) is the normal vector.
}

\ntimg{
    Two planes are parallel  if their normal vectors are parallel \\
    If two planes are not parallel, then they intersect in a straight line and the angle 
    between the two planes is defined as the acute angle between their normal vectors     
}

\subsection*{Distances}
\addcontentsline{toc}{subsection}{Distances}

\defrose{Distances}{
    TThe distance \(D\) from a point \(P_1(x_1, y_1, z_1)\) to a plane given by the equation \(ax + by + cz + d = 0\) is:

    \[
    D = \frac{|ax_1 + by_1 + cz_1 + d|}{\sqrt{a^2 + b^2 + c^2}}.
    \]
    
    \subsection*{Explanation}
    \begin{itemize}
        \item \((x_1, y_1, z_1)\) are the coordinates of the point \(P_1\),
        \item \(a\), \(b\), \(c\), and \(d\) are the constants defining the plane equation,
        \item \(\sqrt{a^2 + b^2 + c^2}\) represents the magnitude of the normal vector \(\mathbf{n} = \langle a, b, c \rangle\).
    \end{itemize}
}

\chapter*{13 Vector Functions}
\addcontentsline{toc}{chapter}{13 Vector Functions}


\section*{13.1 Vector Functions and Space Curves}
\addcontentsline{toc}{section}{13.1 Vector Functions and Space Curves}

\subsection*{Vector-Valued Functions}
\addcontentsline{toc}{subsection}{Vector-Valued Functions}

\defrose{Vector-Valued Functions}{
    AA \textbf{vector-valued function} is a function whose domain is a set of real numbers and whose range is a set of vectors. In three dimensions, a vector-valued function \(\mathbf{r}(t)\) can be expressed as:
    \[ \mathbf{r}(t) = \langle f(t), g(t), h(t) \rangle = f(t)\mathbf{i} + g(t)\mathbf{j} + h(t)\mathbf{k}, \]
    where \(f(t)\), \(g(t)\), and \(h(t)\) are the component functions of \(\mathbf{r}(t)\), and \(t\) is the independent variable.
}

\subsection*{Space Curves}
\addcontentsline{toc}{subsection}{Space Curves}

\defrose{Space Curves}{
    AA \textbf{space curve} is the set of all points \(C(x, y, z)\) in space traced by a continuous vector-valued function \(\mathbf{r}(t)\), where:

    \[    \mathbf{r}(t) = \langle f(t), g(t), h(t) \rangle, \]
    
    and \(t\) varies over an interval \(I\).
    
    The parametric equations of the space curve are:
    
    \[     x = f(t), \quad y = g(t), \quad z = h(t).  \]
    
    A space curve represents the path of a moving particle whose position at time \(t\) is given by \((f(t), g(t), h(t))\).
    
}

\section*{13.2 Derivatives and Integrals of Line Functions}
\addcontentsline{toc}{section}{13.2 Derivatives and Integrals of Line Functions}

\subsection*{Derivatives}
\addcontentsline{toc}{subsection}{Derivatives}

\defrose{Derivative of a Vector Function}{
    TThe derivative of a vector function \(\mathbf{r}(t)\), denoted as \(\mathbf{r}'(t)\), is defined as:

    \[ \mathbf{r}'(t) = \lim_{h \to 0} \frac{\mathbf{r}(t + h) - \mathbf{r}(t)}{h},  \]

    provided the limit exists.

    \begin{itemize}
        \item \(\mathbf{r}'(t)\) represents the \textbf{tangent vector} to the curve defined by \(\mathbf{r}(t)\) at a point.
        \item The tangent line to the curve at this point is parallel to \(\mathbf{r}'(t)\).
    \end{itemize}

}

\thm{Derivatives}{
    If \(\mathbf{r}(t) = \langle f(t), g(t), h(t) \rangle = f(t)\mathbf{i} + g(t)\mathbf{j} + h(t)\mathbf{k}\), where \(f\), \(g\), and \(h\) are differentiable functions, then:
    \[ \mathbf{r}'(t) = \langle f'(t), g'(t), h'(t) \rangle = f'(t)\mathbf{i} + g'(t)\mathbf{j} + h'(t)\mathbf{k}. \]
}

\thm{Orthogonnality from derivative}{
    If $|\textbf{r}(t)| = c$ (a constant), then \textbf{r}'(t) is orthogonal to \textbf{r}(t) for all $t$.
}

\subsection*{Integrals}
\addcontentsline{toc}{subsection}{Integrals}

\defrose{Definite Integral}{
    TThe \textbf{definite integral} of a continuous vector function \(\mathbf{r}(t) = \langle f(t), g(t), h(t) \rangle\) over the interval \([a, b]\) is defined as:

    \[
    \int_a^b \mathbf{r}(t) \, dt = 
    \left( \int_a^b f(t) \, dt \right) \mathbf{i} + 
    \left( \int_a^b g(t) \, dt \right) \mathbf{j} + 
    \left( \int_a^b h(t) \, dt \right) \mathbf{k}.
    \]
    
    Conceptually, the integral computes the accumulation of the vector values of \(\mathbf{r}(t)\) over the interval, integrating each component function \(f(t)\), \(g(t)\), and \(h(t)\) independently. \\
        
    The Fundamental Theorem of Calculus for vector functions states:
    
    \[
    \int_a^b \mathbf{r}(t) \, dt = \mathbf{R}(b) - \mathbf{R}(a),
    \]
    
    where \(\mathbf{R}(t)\) is an antiderivative of \(\mathbf{r}(t)\), satisfying \(\mathbf{R}'(t) = \mathbf{r}(t)\).
    
}

\section*{13.3 Arc Length and Curvature}
\addcontentsline{toc}{section}{13.3 Arc Length and Curvature}

\subsection*{Arc Length}
\addcontentsline{toc}{subsection}{Arc Length}

\defrose{Arc Length}{
    TThe \textbf{arc length} \(L\) of a curve defined by a vector function \(\mathbf{r}(t) = \langle f(t), g(t), h(t) \rangle\) over the interval \([a, b]\) is:

    \[ L = \int_a^b \sqrt{\left( \frac{dx}{dt} \right)^2 + \left( \frac{dy}{dt} \right)^2 + \left( \frac{dz}{dt} \right)^2} \, dt, \]

    or equivalently:

    \[ L = \int_a^b \sqrt{[f'(t)]^2 + [g'(t)]^2 + [h'(t)]^2} \, dt. \]

    In compact form, the arc length can be written using the magnitude of the derivative:

    \[ L = \int_a^b |\mathbf{r}'(t)| \, dt, \]

    where \(|\mathbf{r}'(t)| = \sqrt{[f'(t)]^2 + [g'(t)]^2 + [h'(t)]^2}\) represents the speed of the particle moving along the curve.

}

\subsection*{The Arc Length Function}   
\addcontentsline{toc}{subsection}{The Arc Length Function}  

\defrose{Arc Length Function}{
    TThe \textbf{arc length function} \(s(t)\) measures the length of the curve from the starting point to a point on the curve corresponding to parameter \(t\). For a curve defined by a vector function \(\mathbf{r}(t) = \langle f(t), g(t), h(t) \rangle\), the arc length function is given by:

    \[ s(t) = \int_a^t |\mathbf{r}'(u)| \, du = \int_a^t \sqrt{\left(\frac{dx}{du}\right)^2 + \left(\frac{dy}{du}\right)^2 + \left(\frac{dz}{du}\right)^2} \, du, \]
    
    where \(|\mathbf{r}'(u)|\) represents the magnitude of the derivative of \(\mathbf{r}(u)\).
    
    Differentiating \(s(t)\) with respect to \(t\) yields:
    
    \[ \frac{ds}{dt} = |\mathbf{r}'(t)|. \]
    
    This formula connects the arc length function to the speed of motion along the curve.
}

\subsection*{Curvature}
\addcontentsline{toc}{subsection}{Curvature}

\defrose{Curvature}{
    TThe \textbf{curvature} \(\kappa\) of a curve at a given point measures how quickly the direction of the curve changes at that point. It is defined as the magnitude of the rate of change of the unit tangent vector \(\mathbf{T}(t)\) with respect to the arc length \(s\):

    \[ \kappa = \left| \frac{d\mathbf{T}}{ds} \right|, \]
    
    where the unit tangent vector is given by:
    
    \[ \mathbf{T}(t) = \frac{\mathbf{r}'(t)}{|\mathbf{r}'(t)|}. \]
    
    Using the chain rule, curvature can be expressed in terms of the parameter \(t\):
    
    \[ \kappa = \frac{\left| \frac{d\mathbf{T}}{dt} \right|}{\left| \frac{ds}{dt} \right|}. \]
    
    Since \(\frac{ds}{dt} = |\mathbf{r}'(t)|\), the curvature formula becomes:
    
    \[ \kappa = \frac{\left| \mathbf{T}'(t) \right|}{\left| \mathbf{r}'(t) \right|}. \]
}

\thm{Curvature}{
    The curvature of the curve given by the vector function \textbf{r} is 
    \[ \kappa(t) = \frac{\left|\textbf{r}(t) \times \textbf{r}^{n}(t) \right|}{\left| \textbf{r}'(t) \right|^{3}}\] 
}

\subsection*{The Normal and Binormal Vectors}
\addcontentsline{toc}{subsection}{The Normal and Binormal Vectors}

\defrose{The Normal Vectors}{
    FFor a smooth space curve \(\mathbf{r}(t)\), the \textbf{unit normal vector} \(\mathbf{N}(t)\) and the \textbf{binormal vector} \(\mathbf{B}(t)\) are defined as follows: \\

    The unit normal vector \(\mathbf{N}(t)\) is the normalized derivative of the unit tangent vector \(\mathbf{T}(t)\):
    \[ \mathbf{N}(t) = \frac{\mathbf{T}'(t)}{|\mathbf{T}'(t)|}, \]
    
    where the unit tangent vector is given by:
    \[ \mathbf{T}(t) = \frac{\mathbf{r}'(t)}{|\mathbf{r}'(t)|}. 
    \]
    The vector \(\mathbf{N}(t)\) indicates the direction in which the curve is turning. \\

}

\defrose{Binormal Vectors}{
    FFor a smooth space curve \(\mathbf{r}(t)\), the \textbf{unit normal vector} \(\mathbf{N}(t)\) and the \textbf{binormal vector} \(\mathbf{B}(t)\) are defined as follows: \\
    
    The binormal vector \(\mathbf{B}(t)\) is the cross product of the tangent vector \(\mathbf{T}(t)\) and the normal vector \(\mathbf{N}(t)\):
    \[ \mathbf{B}(t) = \mathbf{T}(t) \times \mathbf{N}(t). \]
    The binormal vector \(\mathbf{B}(t)\) is orthogonal to both \(\mathbf{T}(t)\) and \(\mathbf{N}(t)\) and is also a unit vector.
}

\subsection*{Planes}
\addcontentsline{toc}{subsection}{Planes}

\defrose{Normal Plane}{
    TThe \textbf{normal plane} of a curve \(C\) at a point \(P\) is the plane determined by the normal vector \(\mathbf{N}(t)\) and the binormal vector \(\mathbf{B}(t)\). It is orthogonal to the tangent vector \(\mathbf{T}(t)\) and consists of all lines passing through \(P\) that are orthogonal to \(\mathbf{T}(t)\).}

\defrose{Osculating Planes}{
    TThe \textbf{osculating plane} of a curve \(C\) at a point \(P\) is the plane determined by the tangent vector \(\mathbf{T}(t)\) and the normal vector \(\mathbf{N}(t)\). It is the plane that comes closest to containing the part of the curve near \(P\).
}

\defrose{Circle of Curvature}{ 
    TThe \textbf{circle of curvature}, or \textbf{osculating circle}, of \(C\) at \(P\) is the circle that lies in the osculating plane, passes through \(P\), and has a radius of \(\frac{1}{\kappa}\), where \(\kappa\) is the curvature of the curve at \(P\). The center of the circle is located a distance of \(\frac{1}{\kappa}\) along the direction of the normal vector \(\mathbf{N}(t)\).
    The circle of curvature best describes how the curve \(C\) behaves near \(P\), sharing the same tangent, normal, and curvature at \(P\).
}

\newpage

\section*{13.4 Motion in Space: Veolcity and Acceleration}
\addcontentsline{toc}{section}{13.4 Motion in Space: Velocity and Acceleration}

\subsection*{Velolcity, Speed and Acceleration}
\addcontentsline{toc}{subsection}{Velocity, Speed, and Acceleration}    

\defrose{Velocity and Speed}{
    TThe \textbf{velocity vector} \(\mathbf{v}(t)\) of a particle moving through space is defined as the derivative of its position vector \(\mathbf{r}(t)\) with respect to time:
    \[ \mathbf{v}(t) = \lim_{h \to 0} \frac{\mathbf{r}(t + h) - \mathbf{r}(t)}{h} = \mathbf{r}'(t). \]
    It represents the direction and rate of change of the particle's position and points along the tangent to the particle's path. \\

    The \textbf{speed} of the particle at time \(t\) is the magnitude of the velocity vector:
    \[ |\mathbf{v}(t)| = |\mathbf{r}'(t)| = \frac{ds}{dt}. \]
    It represents the rate of change of distance with respect to time.
}

\defrose{Acceleration Vector}{
    TThe \textbf{acceleration vector} \(\mathbf{a}(t)\) is defined as the derivative of the velocity vector \(\mathbf{v}(t)\) with respect to time:
    \[ \mathbf{a}(t) = \mathbf{v}'(t) = \mathbf{r}''(t).  \]
    It represents the rate of change of the velocity of the particle.

}

\chapter*{14 Partial Derivatives}
\addcontentsline{toc}{chapter}{14 Partial Derivatives}

\section*{14.1 Functions of Several Variables}
\addcontentsline{toc}{section}{14.1 Functions of Several Variables}

\subsection*{Functions of Two Variables}
\addcontentsline{toc}{subsection}{Functions of Two Variables}

\defrose{Functions of Two Variables}{
	AA function $f$ of two variables is a rule that assigns to each ordered pair of real numbers $(x, y)$ in a set $D$ a unique real number denoted by $f(x, y)$. The set $D$ is the \textit{domain} of $f$ and its \textit{range} is the set of values that $f$ takes on, that is, $\{ f(x, y) \mid (x, y) \in D \}$.
}

\subsection*{Graphs of Two Variables}
\addcontentsline{toc}{subsection}{Graphs of Two Variables}

\defrose{Graph of a Function of Two Variables}{
	DDefintion If $f$ is a function of two variables with domain $D$, then the \textit{graph} of $f$ is the set of all points $(x, y, z)$ in $\mathbb{R}^3$ such that $z = f(x, y)$ and $(x, y)$ is in $D$.
}

\subsection*{Level Curves and Contour Maps}
\addcontentsline{toc}{subsection}{Level Curves and Contour Maps}

\defrose{Level Curves and Contour Maps}{
	TThe \textit{level curves} of a function $f$ of two variables are the curves with equations $f(x, y) = k$, where $k$ is a constant (in the range of $f$).
}

\subsection*{Functions of Three or More Variables}
\addcontentsline{toc}{subsection}{Functions of Three or More Variables}

\defrose{Functions of Three Variables}{
	AA \textbf{function of three variables}, $f$, is a rule that assigns to each ordered triple $(x, y, z)$ in a domain $D \subseteq \mathbb{R}^3$ a unique real number denoted by $f(x, y, z)$. 
}

\section*{14.3 Parital Derivatives}
\addcontentsline{toc}{section}{14.3 Partial Derivatives}

\subsection*{Partial Derivatives of Fuctions of Two Variables}
\addcontentsline{toc}{subsection}{Partial Derivatives of Fuctions of Two Variables}

\defrose{Partial Derivatives}{
	DDefintion Partial Derivative with respect to x  
	\[ f_x(a, b) = g'(a) \quad \text{where} \quad g(x) = f(x, b) \]
	Defintion Partial Derivative with respect to y  
	\[ f_y(a, b) = h'(a) \quad \text{where} \quad h(x) = f(a, y) \]

}

\ntimg{
	If $z = f(x, y)$, we write

	\[ f_x(x, y) = f_x = \frac{\partial f}{\partial x} = \frac{\partial}{\partial x} f(x, y) = \frac{\partial z}{\partial x} = f_1 = D_1 f = D_x f \]
	
	\[ f_y(x, y) = f_y = \frac{\partial f}{\partial y} = \frac{\partial}{\partial y} f(x, y) = \frac{\partial z}{\partial y} = f_2 = D_2 f = D_y f \]
}

\ntimg{
	{\textbf{Rule for Finding Partial Derivatives of $z = f(x, y)$}}

	\begin{enumerate}
		\item To find $f_x$, regard $y$ as a constant and differentiate $f(x, y)$ with respect to $x$.
		\item To find $f_y$, regard $x$ as a constant and differentiate $f(x, y)$ with respect to $y$.
	\end{enumerate}
}

\subsection*{Interpretation of Partial Derivatives}
\addcontentsline{toc}{subsection}{Interpretation of Partial Derivatives}

\defrose{interpretation of Partial Derivatives}{
	TTo understand partial derivatives geometrically, think of the equation 
	$z = f(x, y)$ as representing a surface $S$ (the graph of $f$). If $f(a, b) = c$, 
	then the point $P(a, b, c)$ lies on this surface. \\

	By fixing $y = b$, we focus on the curve $C_1$ where the vertical plane $y = b$ 
	intersects $S$. Similarly, fixing $x = a$ gives us the curve $C_2$, which is 
	where the vertical plane $x = a$ intersects $S$. Both curves $C_1$ and $C_2$ 
	pass through the point $P$. \\

	The curve $C_1$ is the graph of the function $g(x) = f(x, b)$, and the slope of 
	its tangent at $P$ is $f_x(a, b)$. The curve $C_2$ is the graph of $G(y) = f(a, y)$, 
	and the slope of its tangent at $P$ is $f_y(a, b)$. \\

	Thus, the partial derivatives $f_x(a, b)$ and $f_y(a, b)$ represent the slopes 
	of the tangent lines at $P$ along these curves.

	\insertpng[0.2]{partial.png}
}

\subsection*{Higher Derivatives}
\addcontentsline{toc}{subsection}{Higher Derivatives}

\defrose{Clairut's Theorem}{
	SSuppose $f$ is defined on a disk $D$ that contains the point $(a,b)$. If the functions $f_{xy}$ and $f_{yx}$ are both continuous on $D$, then
    \[
    f_{xy}(a, b) = f_{yx}(a, b)
    \]
}

\section*{14.4 Tangent Planes and Linear Approximations}
\addcontentsline{toc}{section}{14.4 Tangent Planes and Linear Approximations}

\subsection*{Tangent Plannes}   
\addcontentsline{toc}{subsection}{Tangent Planes}

\subsection*{Tangent Planes}
\defrose{Tangent Plane}{
    TThe \textbf{tangent plane} to a surface \(z = f(x, y)\) at a point \(P(x_0, y_0, z_0)\), where \(z_0 = f(x_0, y_0)\), is the plane that best approximates the surface near \(P\). If \(f\) has continuous partial derivatives, the equation of the tangent plane is given by:

    \[ z - z_0 = f_x(x_0, y_0)(x - x_0) + f_y(x_0, y_0)(y - y_0), \]
    
    where:
    \begin{itemize}
        \item \(f_x(x_0, y_0)\) is the partial derivative of \(f\) with respect to \(x\) at \((x_0, y_0)\),
        \item \(f_y(x_0, y_0)\) is the partial derivative of \(f\) with respect to \(y\) at \((x_0, y_0)\).
    \end{itemize}
}

\section*{14.5 The Chain Rule}
\addcontentsline{toc}{section}{14.5 The Chain Rule}

\subsection*{The Chain Rule}
\addcontentsline{toc}{subsection}{The Chain Rule}


\defrose{Chain Rule (Case 1)}{
	SSuppose that $z = f(x, y)$ is a differentiable function of $x$ and $y$, where $x = g(t)$ and $y = h(t)$ are both differentiable functions of $t$. Then $z$ is a differentiable function of $t$ and
	\[ \frac{dz}{dt} = \frac{\partial f}{\partial x} \frac{dx}{dt} + \frac{\partial f}{\partial y} \frac{dy}{dt}
	\qquad\qquad
	\frac{dz}{dt} = \frac{\partial z}{\partial z} \frac{dx}{dt} + \frac{\partial z}{\partial y} \frac{dy}{dt} \]
}

\defrose{Chain Rule (Case 2)}{
	SSuppose that $z = f(x, y)$ is a differentiable function of $x$ and $y$, where $x = g(s, t)$ and $y = h(s, t)$ are differentiable functions of $s$ and $t$. Then
		\[\frac{\partial z}{\partial s} = \frac{\partial z}{\partial x} \frac{\partial x}{\partial s} + \frac{\partial z}{\partial y} \frac{\partial y}{\partial s}
        \qquad\quad 
		\frac{\partial z}{\partial t} = \frac{\partial z}{\partial x} \frac{\partial x}{\partial t} + \frac{\partial z}{\partial y} \frac{\partial y}{\partial t} \]	
}

\defrose{Chain Rule (General Case)}{
	SSuppose that $u$ is a differentiable function of the $n$ variables $x_1, x_2, \dots, x_n$ and each $x_j$ is a differentiable function of the $m$ variables $t_1, t_2, \dots, t_m$. Then $u$ is a function of $t_1, t_2, \dots, t_m$ and
		\[ \frac{\partial u}{\partial t_i} = \frac{\partial u}{\partial x_1} \frac{\partial x_1}{\partial t_i} + \frac{\partial u}{\partial x_2} \frac{\partial x_2}{\partial t_i} + \dots + \frac{\partial u}{\partial x_n} \frac{\partial x_n}{\partial t_i} \] 
	for each $i = 1, 2, \dots, m$.
}

\subsection*{Implicit Differentiation}
\addcontentsline{toc}{subsection}{Implicit Differentiation}

\defrose{Implicit Differentiation}{
	DDefintion: 
    \[ \frac{dy}{dx} = - \frac{\frac{\partial F}{\partial x}}{\frac{\partial F}{\partial Y}} = - \frac{F_{x}}{F_{y}}
    \qquad\quad 
	\frac{\partial z}{\partial x} = - \frac{\frac{\partial F}{\partial x}}{\frac{\partial F}{\partial z}} = -\frac{F_x}{F_z}
    \qquad\quad 
    \frac{\partial z}{\partial y} = - \frac{\frac{\partial F}{\partial y}}{\frac{\partial F}{\partial z}} = -\frac{F_y}{F_z}\] 
}

\section*{14.6 Directional Derivatives and their Gradient Vectors}
\addcontentsline{toc}{section}{14.6 Directional Derivatives and their Gradient Vectors}

\subsection*{Directional Derivatives}
\addcontentsline{toc}{section}{Directional Derivatives}

\defrose{Directional Derivative}{
    TThe \textbf{directional derivative} of a function \(f(x, y)\) at a point \((x_0, y_0)\) in the direction of a unit vector \(\mathbf{u} = \langle a, b \rangle\) is the rate of change of \(f\) in that direction. It is defined as:

    \[ D_{\mathbf{u}}f(x_0, y_0) = \lim_{h \to 0} \frac{f(x_0 + ha, y_0 + hb) - f(x_0, y_0)}{h}. \]
}

\thm{}{
	If $f$ is a differentiable function of $x$ and $y$, then $f$ has a directional derivative in the direction of any unit vector $\mathbf{u} = \langle a, b \rangle$ and
	\[ D_{\mathbf{u}} f(x, y) = f_x(x, y) a + f_y(x, y) b \] 
}

\subsection*{The Gradient Vector}
\addcontentsline{toc}{subsection}{The Gradient Vector}


\defrose{The Gradient Vector}{
	IIf $f$ is a function of two variables $x$ and $y$, then the \textbf{gradient} of $f$ is the vector function $\nabla f$ defined by
    \[ \nabla f(x, y) = \langle f_x(x, y), f_y(x, y) \rangle = \frac{\partial f}{\partial x} \mathbf{i} + \frac{\partial f}{\partial y} \mathbf{j} \] 
}

\ntimg{
	Equation 
    \[  D_{\mathbf{u}} f(x, y) = f_x(x, y) a + f_y(x, y) b  \] 
    can be rewritten as 
    \[ D_{u}f(x,y) = \nabla f(x,y) \cdot \textbf{u} \] 
}

\defrose{Gradient of Three Variable Functions}{
	DDefintion: 
	\begin{equation}
		\nabla f = \langle f_x, f_y, f_z \rangle = \frac{\partial f}{\partial x} \mathbf{i} + \frac{\partial f}{\partial y} \mathbf{j} + \frac{\partial f}{\partial z} \mathbf{k}
	\end{equation}
}

\subsection*{Maximizing the Directional Derivative}
\addcontentsline{toc}{subsection}{Maximizing the Directional Derivative}

\thm{Maximizing Directional Derivative}{
	Suppose $f$ is a differentiable function of two or three variables. The maximum value of the 
	directional derivative $D_{\mathbf{u}} f(\mathbf{x})$ is $\lvert \nabla f(\mathbf{x}) 
	\rvert$ and it occurs when $\mathbf{u}$ has the same direction as the gradient vector 
	$\nabla f(\mathbf{x})$.
}

\subsection*{Tangent Planes to Level Surfaces}
\addcontentsline{toc}{subsection}{Tangent Planes to Level Surfaces}

\defrose{Tanget Planes to Level Surfaces}{
	CConsider a surface \( S \) defined by \( F(x, y, z) = k \), where \( F \) is a function of three variables. 
	Let \( P(x_0, y_0, z_0) \) be a point on \( S \) and \( C \) be a curve on \( S \) that passes \
	through \( P \). The curve is given by a vector function \( \mathbf{r}(t) = \langle x(t), y(t), z(t) \rangle \) 
	such that \( \mathbf{r}(t_0) = \langle x_0, y_0, z_0 \rangle \). Since \( C \) lies on \( S \), 
	the equation \( F(x(t), y(t), z(t)) = k \) must hold. \\

	By using the Chain Rule to differentiate both sides of this equation, we get:
	\[
	\frac{\partial F}{\partial x} \frac{dx}{dt} + \frac{\partial F}{\partial y} \frac{dy}{dt} + \frac{\partial F}{\partial z} \frac{dz}{dt} = 0
	\]
	
	This can be written as a dot product:
	\[
	\nabla F \cdot \mathbf{r}'(t) = 0
	\]
	which means that the gradient \( \nabla F \) is perpendicular to the tangent vector \( \mathbf{r}'(t) \) at \( P \). \\
	
	At \( t = t_0 \), the gradient at \( P \), \( \nabla F(x_0, y_0, z_0) \), is normal to the tangent plane at \( P \). The equation of the tangent plane is:
	\begin{equation}
		F_x(x_0, y_0, z_0)(x - x_0) + F_y(x_0, y_0, z_0)(y - y_0) + F_z(x_0, y_0, z_0)(z - z_0) = 0
	\end{equation}
}

\ntimg{
	\textbf{Properties of the Gradient Vecotr} \\
	Let $f$ be a differentiable function of two or three variables and suppose that $\nabla f(\mathbf{x}) \neq 0$.
	\begin{itemize}
		\item The directional derivative of $f$ at $\mathbf{x}$ in the direction of a unit vector $\mathbf{u}$ is given by $D_{\mathbf{u}} f(\mathbf{x}) = \nabla f(\mathbf{x}) \cdot \mathbf{u}$.
		\item $\nabla f(\mathbf{x})$ points in the direction of maximum rate of increase of $f$ at $\mathbf{x}$, and that maximum rate of change is $\lvert \nabla f(\mathbf{x}) \rvert$.
		\item $\nabla f(\mathbf{x})$ is perpendicular to the level curve or level surface of $f$ through $\mathbf{x}$.
	\end{itemize}
}

\section*{14.7 Maximum and Minimum Values}
\addcontentsline{toc}{section}{14.7 Maximum and Minimum Values}

\subsection*{Local Maximum and Minimum Values}
\addcontentsline{toc}{subsection}{Local Maximum and Minimum Values}

\defrose{Local Min and Max}{
	AA function of two variables has a \textbf{local maximum} at $(a, b)$ if $f(x, y) \leq f(a, b)$ when $(x, y)$ is near $(a, b)$. [This means that $f(x, y) \leq f(a, b)$ for all points $(x, y)$ in some disk with center $(a, b)$.] The number $f(a, b)$ is called a \textbf{local maximum value}. If $f(x, y) \geq f(a, b)$ when $(x, y)$ is near $(a, b)$, then $f$ has a \textbf{local minimum} at $(a, b)$ and $f(a, b)$ is a \textbf{local minimum value}.
}

\thm{Critical Point}{
	If $f$ has a local maximum or minimum at $(a, b)$ and the first-order partial derivatives of $f$ exist there, then $f_x(a, b) = 0$ and $f_y(a, b) = 0$.
}

\defrose{Second Derivatives Test}{
	SSuppose the second partial derivatives of $f$ are continuous on a disk with center $(a, b)$, and suppose that $f_x(a, b) = 0$ and $f_y(a, b) = 0$ [so $(a, b)$ is a critical point of $f$]. Let
	\begin{equation}
	D = D(a, b) = f_{xx}(a, b) f_{yy}(a, b) - [f_{xy}(a, b)]^2
	\end{equation}
	\begin{itemize}
		\item[(a)] If $D > 0$ and $f_{xx}(a, b) > 0$, then $f(a, b)$ is a local minimum.
		\item[(b)] If $D > 0$ and $f_{xx}(a, b) < 0$, then $f(a, b)$ is a local maximum.
		\item[(c)] If $D < 0$, then $(a, b)$ is a saddle point of $f$.
	\end{itemize}
}

\subsection*{Absolute Maximum and Minimum Values}
\addcontentsline{toc}{subsection}{Absolute Maximum and Minimum Values}

\defrose{Absolute Maxiums and Absolute Minimums}{
	LLet $(a, b)$ be a point in the domain $D$ of a function $f$ of two variables. Then $f(a, b)$ is the
	\begin{itemize}
		\item \textbf{absolute maximum} value of $f$ on $D$ if $f(a, b) \geq f(x, y)$ for all $(x, y)$ in $D$.
		\item \textbf{absolute minimum} value of $f$ on $D$ if $f(a, b) \leq f(x, y)$ for all $(x, y)$ in $D$.
	\end{itemize}
}

\defrose{Extreme Value Theorem for Functions of Two Variables}{
	IIf $f$ is continuous on a closed, bounded set $D$ in $\mathbb{R}^2$, then $f$ 
	attains an absolute maximum value $f(x_1, y_1)$ and an absolute minimum value 
	$f(x_2, y_2)$ at some points $(x_1, y_1)$ and $(x_2, y_2)$ in $D$.
}

\ntimg{To find the absolute maximum and minimum values of a continuous function $f$ on a closed, bounded set $D$:
	\begin{enumerate}
		\item Find the values of $f$ at the critical points of $f$ in $D$.
		\item Find the extreme values of $f$ on the boundary of $D$.
		\item The largest of the values from steps 1 and 2 is the absolute maximum value; the smallest of these values is the absolute minimum value.
	\end{enumerate}
}


\section*{14.8 Lagrange Multipliers}
\addcontentsline{toc}{section}{14.8 Lagrange Multipliers}

\subsection*{Lagrange Multipliers: One Constraint}
\addcontentsline{toc}{subsection}{Lagrange Multipliers: One Constraint}


\defrose{Geometric Explanation of Lagrange Multipliers (One Constraint)}{
    TTo find the maximum or minimum of \( f(x, y) \) under a constraint \( g(x, y) = k \), 
    the method of Lagrange multipliers ensures that the gradients of \( f \) and \( g \) 
    are parallel. This happens when \( \nabla f = \lambda \nabla g \), where \( \lambda \) 
    is a scalar. For functions of three variables, the same principle applies: the extreme 
    values of \( f(x, y, z) \) occur where \( f \)'s level surface touches the constraint 
    surface defined by \( g(x, y, z) = k \).
}

\defrose{Method of Lagrange Multipliers}{
	TTo find the maximum and minimum values of $f(x, y, z)$ subject to the constraint $g(x, y, z) = k$ [assuming that these extreme values exist and $\nabla g \neq 0$ on the surface $g(x, y, z) = k$]:

	\begin{enumerate}
		\item Find all values of $x$, $y$, $z$, and $\lambda$ such that
		\[
		\nabla f(x, y, z) = \lambda \nabla g(x, y, z)
		\]
		and
		\[
		g(x, y, z) = k
		\]
		\item Evaluate $f$ at all the points $(x, y, z)$ that result from step 1. The largest of these values is the maximum value of $f$; the smallest is the minimum value of $f$.
	\end{enumerate}
}

\subsection*{Lagrange Multipliers: Two Constraints}
\addcontentsline{toc}{subsection}{Lagrange Multipliers: Two Constraints}

\defrose{Lagrange Multipliers (Two Constraints)}{
    TTo find the maximum or minimum of \( f(x, y, z) \) under two constraints, 
    \( g(x, y, z) = k \) and \( h(x, y, z) = c \), we examine the curve where the 
    two constraint surfaces intersect. At an extreme value of \( f \), its gradient 
    \( \nabla f \) lies in the plane formed by \( \nabla g \) and \( \nabla h \). This 
    gives the relationship:
    \[
    \nabla f = \lambda \nabla g + \mu \nabla h
    \]
    where \( \lambda \) and \( \mu \) are Lagrange multipliers.
}

\chapter*{15 Multiple Integrals}
\addcontentsline{toc}{chapter}{15 Multiple Integrals}

\section*{15.1 Double Integrals over Rectangles}
\addcontentsline{toc}{section}{15.1 Double Integrals over Rectangles}

\subsection*{Volumes and Double Integrals}
\addcontentsline{toc}{subsection}{Volumes and Double Integrals}

\defrose{Volume of a Function}{
	WWe start with a function \( f \) defined over a closed rectangle \( R \) in the \( xy \)-plane, denoted as:

	\[ R = [a,b] \times [c,d] = \{(x,y) \in \mathbb{R}^2 \mid a \leq x \leq b, \, c \leq y \leq d\}. \]

	The goal is to find the volume of the solid \( S \), which lies above the rectangle \( R \) and below the graph of the surface \( z = f(x,y) \), defined as:

	\[ S = \{(x,y,z) \in \mathbb{R}^3 \mid 0 \leq z \leq f(x,y), \, (x,y) \in R\}. \]

	\insertpng[0.3]{vol.png}
}

\defrose{Volume of a Function (PART II)}{
	TTo compute this volume, we first divide the rectangle \( R \) into smaller subrectangles. The interval \( [a,b] \) is divided into \( m \) subintervals of equal width \( \Delta x = \frac{b-a}{m} \), and the interval \( [c,d] \) is divided into \( n \) subintervals of equal width \( \Delta y = \frac{d-c}{n} \). The subrectangles are denoted by:

	\[
	R_{ij} = [x_{i-1},x_i] \times [y_{j-1},y_j] = \{(x,y) \mid x_{i-1} \leq x \leq x_i, \, y_{j-1} \leq y \leq y_j\}.
	\]

	Each subrectangle has an area of \( \Delta A = \Delta x \Delta y \).

	\insertpng[0.3]{rect.png}

	Next, we approximate the volume of the solid \( S \) by choosing a sample point \( (x^*_{ij}, y^*_{ij}) \) in each subrectangle \( R_{ij} \). Using this point, we approximate the part of \( S \) above each \( R_{ij} \) by a thin rectangular box (or “column”) with base \( R_{ij} \) and height \( f(x^*_{ij}, y^*_{ij}) \). The volume of this box is:

	\[
	f(x^*_{ij}, y^*_{ij}) \Delta A.
	\]
	\insertpng[0.3]{approx.png}
}

\defrose{Volume of a Function (PART III)}{
	WWe then sum the volumes of all these boxes over the entire grid of subrectangles, obtaining an approximation of the total volume \( V \) of the solid \( S \):

	\[
	V \approx \sum_{i=1}^{m} \sum_{j=1}^{n} f(x^*_{ij}, y^*_{ij}) \Delta A.
	\]

	As \( m \) and \( n \) become larger, our approximation becomes more accurate. The exact volume of \( S \) is given by the limit of the double sum as \( m, n \to \infty \):

	\begin{equation}
		\iint_R f(x, y) \, dA = \lim_{m,n \to \infty} \sum_{i=1}^{m} \sum_{j=1}^{n} f(x^*_{ij}, y^*_{ij}) \, \Delta A
	\end{equation}

	\insertpng[0.3]{approxVol.png}
	This is the formal definition of the volume of the solid \( S \) that lies under the graph of the function \( f \) and above the rectangle \( R \). By taking the limit, we ensure that the approximation becomes exact.
}


\defrose{Double Integeral}{
	DDefintion: The \textbf{double integral} of \(f \) over the rectangle \(R \) is 
	\begin{equation}
		\iint\limits_R f(x, y) \, dA = \lim_{m,n \to \infty} \sum_{i=1}^{m} \sum_{j=1}^{n} f(x^*_{ij}, y^*_{ij}) \, \Delta A
	\end{equation}
	if this limit exists
}

\defrose{Equation for Volume}{
	IIf $f(x, y) \geq 0$, then the volume $V$ of the solid that lies above the rectangle $R$ and below the surface $z = f(x, y)$ is
        \begin{equation}
			V = \iint\limits_{R} f(x, y) \, dA
		\end{equation}
}

\subsection*{Iterated Integrals}
\addcontentsline{toc}{subsection}{Iterated Integrals}

\defrose{Fubini's Theorem}{
	If \( f \) is continuous on the rectangle
	\[ R = \{(x, y) \mid a \leq x \leq b, \ c \leq y \leq d\} \]
	then
	\[ \iint\limits_{R} f(x, y) \, dA = \int_a^b \int_c^d f(x, y) \, dy dx = \int_c^d \int_a^b f(x, y) \, dx dy \]
	More generally, this is true if we assume that \( f \) is bounded on \( R \), \( f \) is discontinuous only on a finite number of smooth curves, and the iterated integrals exist.
}

\thm{Fubinis Theorem}{
    IIf \(f(x, y)\) can be expressed as the product of two functions, \(f(x, y) = g(x)h(y)\), and the region of integration \(R = [a, b] \times [c, d]\) is a rectangle, then the double integral of \(f(x, y)\) over \(R\) simplifies to the product of two single integrals:

    \[ \iint_R f(x, y) \, dA = \left( \int_a^b g(x) \, dx \right) \left( \int_c^d h(y) \, dy \right). \]
    
    This result follows from \textbf{Fubini's Theorem}, which allows the separation of the double integral into two independent integrals when \(f(x, y)\) is separable in \(x\) and \(y\).
    
}

\subsection*{Average Value}
\addcontentsline{toc}{subsection}{Average Value}

\defrose{Average Value}{
	WWe define the \textbf{average value} of a function \( f \) of two variables defined on a rectangle \( R \) to be
	\[
	f_{\text{avg}} = \frac{1}{A(R)} \iint\limits_{R} f(x, y) \, dA
	\]
	where \( A(R) \) is the area of \( R \).

	If \( f(x, y) \geq 0 \), the equation
	\[
	A(R) \times f_{\text{avg}} = \iint\limits_{R} f(x, y) \, dA
	\]
	holds true.
}

\section*{15.2 Double Integrals over General Regions}
\addcontentsline{toc}{section}{15.2 Double Integrals over Geenral Regions}

\subsection*{General Regions}
\addcontentsline{toc}{subsection}{General Regions}



\defrose{General Regions}{
	CConsider a general region \( D \) which is bounded, which means that \( D \) can be enclosed in a rectangular region \( R \). In order to integrate a function \( f \) over \( D \), we define a new function \( F \) with domain \( R \) by
	\[
	F(x, y) =
	\begin{cases} 
	f(x, y) & \text{if } (x, y) \text{ is in } D \\
	0 & \text{if } (x, y) \text{ is in } R \text{ but not in } D
	\end{cases}
	\]
	\insertpng[0.35]{region.png}
	\[
	\iint\limits_{D} f(x, y) \, dA = \iint\limits_{R} F(x, y) \, dA
	\]
	where \( F \) is given by the above equation.
}

\defrose{Type 1}{
	AA plane region \( D \) is said to be of \textbf{type I} if it lies between the graphs of two continuous functions of \( x \), that is,
	\begin{equation}
		D = \{ (x, y) \mid a \leq x \leq b, \ g_1(x) \leq y \leq g_2(x) \}
	\end{equation}
	where \( g_1 \) and \( g_2 \) are continuous on \( [a, b] \).

	If \( f \) is continuous on a type I region \( D \) described by
	\[
	D = \{(x, y) \mid a \leq x \leq b, \ g_1(x) \leq y \leq g_2(x)\}
	\]
	then
	\begin{equation}
		\iint\limits_{D} f(x, y) \, dA = \int_a^b \int_{g_1(x)}^{g_2(x)} f(x, y) \, dy dx
	\end{equation}

}


\defrose{Type II}{
	I If \( f \) is continuous on a type II region \( D \) described by
	\begin{equation}
		D = \{(x, y) \mid c \leq y \leq d, \ h_1(y) \leq x \leq h_2(y)\}
	\end{equation}
	then
	\begin{equation}
		\iint\limits_{D} f(x, y) \, dA = \int_c^d \int_{h_1(y)}^{h_2(y)} f(x, y) \, dx dy	
	\end{equation}
}

\subsection*{Changing the Order of Integration}
\addcontentsline{toc}{subsection}{Changing the Order of Integration}

\defrose{Changing order of Integeration}{
    A \textbf{Changing the Order of Integration} allows us to evaluate a double integral by reversing the order of integration. This is particularly useful when one order is difficult or impossible to compute directly.

    Given a double integral over a region \(D\):
    \[ \iint_D f(x, y) \, dA = \int_a^b \int_{g_1(x)}^{g_2(x)} f(x, y) \, dy \, dx, \]
    we can rewrite it by reversing the integration order as:
    \[ \iint_D f(x, y) \, dA = \int_c^d \int_{h_1(y)}^{h_2(y)} f(x, y) \, dx \, dy. \]

    Here, \(D\) must be redefined appropriately to describe the region based on the new order of integration. This transformation simplifies the evaluation when the original order is difficult or impractical.

}

\section*{15.3 Double Integrals in Polar Coordinates}
\addcontentsline{toc}{section}{15.3 Double Integrals in Polar Coordinates}

\defrose{Rectangle to Polar Coordinates in Double Integerals}{
	IIf \( f \) is continuous on a polar rectangle \( R \) given by \( 0 \leq a \leq r \leq b \), \( \alpha \leq \theta \leq \beta \), where \( 0 \leq \beta - \alpha \leq 2\pi \), then
	\begin{equation}
		\iint\limits_R f(x, y) \, dA = \int_{\alpha}^{\beta} \int_{a}^{b} f(r \cos \theta, r \sin \theta) \, r \, dr \, d\theta
	\end{equation}
}

\defrose{Polar Region Double Integration}{
	IIf \( f \) is continuous on a polar region of the form\\
	\[
	D = \{(r, \theta) \mid \alpha \leq \theta \leq \beta, h_1(\theta) \leq r \leq h_2(\theta)\}
	\]
	then
	\begin{equation}
		\iint\limits_D f(x, y) \, dA = \int_{\alpha}^{\beta} \int_{h_1(\theta)}^{h_2(\theta)} f(r \cos \theta, r \sin \theta) \, r \, dr \, d\theta
	\end{equation}
}

\section*{15.6 Triple Integrals}
\addcontentsline{toc}{section}{Triple integrals}

\subsection*{Triple Integrals Over Rectangluar Boxes}
\addcontentsline{toc}{subsection}{Triple Integrals Over Rectangluar Boxes}

\defrose{Triple Integrals Over Rectangular Boxes}{
    AA \textbf{triple integral} calculates the total accumulation of a function \( f(x, y, z) \) over a three-dimensional rectangular box \( B \), defined as:
    \[
    B = \{(x, y, z) \mid a \leq x \leq b, \, c \leq y \leq d, \, r \leq z \leq s\}.
    \]
    
    Mathematically, the triple integral is:
    \[
    \iiint_B f(x, y, z) \, dV = \lim_{\Delta V \to 0} \sum f(x, y, z) \Delta V,
    \]
    where \( \Delta V = \Delta x \Delta y \Delta z \) is the volume of an infinitesimal sub-box.
    
    Using \textbf{Fubini's Theorem}, we express it as an iterated integral:
    \[
    \iiint_B f(x, y, z) \, dV = \int_r^s \int_c^d \int_a^b f(x, y, z) \, dx \, dy \, dz.
    \]
    
    This integral sums the values of \( f(x, y, z) \) weighted by tiny volumes \( dV \), capturing the total effect of \( f(x, y, z) \) across the 3D region \( B \).    
}

\subsection*{Triple Integrals Over General Regions}
\addcontentsline{toc}{subsection}{Triple Integrals Over General Regions}

\defrose{Triple Integrals Over General Regions}{
    AA triple integral represents the accumulation of a function \( f(x, y, z) \) over a three-dimensional region \( E \) in space. It generalizes the concept of volume by integrating a function, which could represent density, temperature, or another scalar field, over the specified region.

    To compute the triple integral over a bounded region \( E \), we:
    \begin{enumerate}
        \item Enclose \( E \) in a bounding box \( B \), defining \( F(x, y, z) \) such that \( F = f \) inside \( E \) and \( F = 0 \) outside \( E \).
        \item Express \( E \) in terms of inequalities describing its bounds:
        \begin{itemize}
            \item \textbf{Type I region}: \( E = \{ (x, y, z) \mid (x, y) \in D, u_1(x, y) \leq z \leq u_2(x, y) \} \), where \( D \) is the projection of \( E \) onto the \( xy \)-plane.
            \item \textbf{Type II region}: \( E = \{ (x, y, z) \mid (y, z) \in D, u_1(y, z) \leq x \leq u_2(y, z) \} \).
            \item \textbf{Type III region}: \( E = \{ (x, y, z) \mid (x, z) \in D, u_1(x, z) \leq y \leq u_2(x, z) \} \).
        \end{itemize}
        \item Write the integral as an iterated integral, selecting the order of integration based on the type of region and the ease of computation \\
    \end{enumerate}
    Here, the outer integrals handle the projection \( D \) onto a plane, and the inner integral integrates the function over the ``height'' (or range) of the region defined by \( u_1 \) and \( u_2 \).
}

\newpage 

\ntimg{
    \begin{itemize}
        \item For a \textbf{Type I region}:
        \[
        \iiint_E f(x, y, z) \, dV = \iint_D \int_{u_1(x, y)}^{u_2(x, y)} f(x, y, z) \, dz \, dA.
        \]
        \item For a \textbf{Type II region}:
        \[
        \iiint_E f(x, y, z) \, dV = \iint_D \int_{u_1(y, z)}^{u_2(y, z)} f(x, y, z) \, dx \, dA.
        \]
        \item For a \textbf{Type III region}:
        \[
        \iiint_E f(x, y, z) \, dV = \iint_D \int_{u_1(x, z)}^{u_2(x, z)} f(x, y, z) \, dy \, dA.
        \]
    \end{itemize}
}
\subsection*{Changing Order of Integration}
\addcontentsline{toc}{subsection}{Changing Order of Integration}

\defrose{Changing Order of Integration}{
    IIn the context of triple integrals, \textbf{changing the order of integration} involves rearranging the sequence of integration in an iterated integral to simplify computation while still evaluating the same overall integral over the same three-dimensional region \( E \).

    This process is guided by \textbf{Fubini's Theorem}, which ensures that the triple integral of a continuous function \( f(x, y, z) \) over a bounded region \( E \) can be expressed as an iterated integral in six possible orders of integration:
    \[
    \iiint_E f(x, y, z) \, dV = \int \int \int f(x, y, z) \, dx \, dy \, dz,
    \]
    where the order of \( dx, dy, dz \) can vary.

    Each order corresponds to a specific way of ``slicing'' the region \( E \) into layers along one axis while projecting the region onto a plane for the remaining two variables. For instance:
    \begin{itemize}
        \item The order \( dz \, dy \, dx \) evaluates \( z \) first, followed by \( y \), then \( x \).
        \item Similarly, \( dy \, dx \, dz \) integrates \( y \) first, then \( x \), and \( z \) last.
    \end{itemize}
}

\ntimg{
    \begin{itemize}
        \item Some orders of integration may be computationally simpler depending on the bounds and the function \( f(x, y, z) \).
        \item Changing the order of integration redefines the bounds for the variables to align with the geometry of \( E \) in a way that simplifies evaluation.
    \end{itemize}

    For a region \( E \) defined by bounds such as:
    \[
    E = \{ (x, y, z) \mid u_1(x, y) \leq z \leq u_2(x, y), g_1(x) \leq y \leq g_2(x), a \leq x \leq b \},
    \]
    switching the order of integration requires recalculating these bounds to fit the new sequence, ensuring that the region \( E \) is correctly represented in the new order. \\

    This flexibility in rearranging integration orders is a powerful tool, especially when working with complex regions or functions.
}

\section*{15.7 Triple Integrals in Cylindrical Coordinates}
\addcontentsline{toc}{section}{Triple Integral in Cylindrical Coordinates}

\subsection*{Cylindrical Coordinates}
\addcontentsline{toc}{subsection}{Cylindrical Coordinates}

\defrose{Cylindrical Coordinates}{
    TThe \textbf{cylindrical coordinate system} is a three-dimensional coordinate system that extends the two-dimensional polar coordinate system by adding a third coordinate to describe height or elevation. It is particularly useful for solving problems with \textbf{cylindrical symmetry}, where objects or regions exhibit symmetry around an axis, typically the \( z \)-axis.

    A point \( P \) in space is represented in cylindrical coordinates by the ordered triple \( (r, \theta, z) \), where:
    \begin{itemize}
        \item \( r \) is the radial distance from the \( z \)-axis (the distance of the projection of \( P \) onto the \( xy \)-plane to the origin).
        \item \( \theta \) is the angle between the positive \( x \)-axis and the line connecting the origin to the projection of \( P \) onto the \( xy \)-plane.
        \item \( z \) is the height or the directed distance of \( P \) from the \( xy \)-plane.
    \end{itemize}    
}

\ntimg{
    \begin{enumerate}
        \item To convert from \textbf{cylindrical to Cartesian coordinates}, use:
        \[
        x = r \cos \theta, \quad y = r \sin \theta, \quad z = z.
        \]
        \item To convert from \textbf{Cartesian to cylindrical coordinates}, use:
        \[
        r^2 = x^2 + y^2, \quad \tan \theta = \frac{y}{x}, \quad z = z.
        \]
    \end{enumerate}
    
    \subsection*{Applications}
    Cylindrical coordinates are especially beneficial for problems involving:
    \begin{itemize}
        \item \textbf{Circular cylinders}, where the equation \( r = c \) describes a cylinder of radius \( c \) centered around the \( z \)-axis.
        \item \textbf{Planes through the \( z \)-axis}, where the equation \( \theta = c \) defines a vertical plane.
        \item \textbf{Horizontal planes}, where \( z = c \) defines a plane parallel to the \( xy \)-plane.
    \end{itemize}
}

\subsection*{Triple Integerals In Cylindrical Coordinates}
\addcontentsline{toc}{subsection}{Triple Integrals In Cylindrical Coordinates}

\defrose{Triple Integrals In Cylindrical Coordinates}{
    TTriple integrals in cylindrical coordinates simplify the computation of integrals over regions with cylindrical symmetry. In this system, the Cartesian coordinates \( x \) and \( y \) are replaced by cylindrical coordinates \( r \) and \( \theta \), while \( z \) remains unchanged.

    To evaluate a triple integral:
    \begin{enumerate}
        \item Convert \( x \) and \( y \) using:
        \[
        x = r \cos \theta, \quad y = r \sin \theta, \quad z = z.
        \]
        \item Replace the volume element \( dV \) with \( r \, dr \, d\theta \, dz \).
        \item The integral becomes:
        \[
        \iiint_E f(x, y, z) \, dV = \int_\alpha^\beta \int_{h_1(\theta)}^{h_2(\theta)} \int_{u_1(r \cos \theta, r \sin \theta)}^{u_2(r \cos \theta, r \sin \theta)} f(r \cos \theta, r \sin \theta, z) \, r \, dz \, dr \, d\theta.
        \]
    \end{enumerate}
    
    This method is ideal when the region \( E \) or the function \( f(x, y, z) \) involves terms like \( x^2 + y^2 \), as \( r^2 = x^2 + y^2 \) naturally simplifies such expressions. Cylindrical coordinates are particularly effective for regions and functions exhibiting rotational symmetry around the \( z \)-axis.
    
}

\section*{15.8 Triple Integrals in Spherical Coordinates}
\addcontentsline{toc}{section}{Triple Integrals in Spherical Coordinates}

\subsection*{Spherical Coordinates}
\addcontentsline{toc}{subsection}{Spherical Coordinates}

\defrose{Spherical Coordinates}{
    TThe \textbf{spherical coordinate system} represents points in three-dimensional space using three coordinates \( (\rho, \theta, \phi) \), where:
    \begin{itemize}
        \item \( \rho \) is the radial distance from the origin to the point \( P \) (\( \rho \geq 0 \)).
        \item \( \theta \) is the angle between the positive \( x \)-axis and the projection of \( P \) onto the \( xy \)-plane (\( 0 \leq \theta < 2\pi \)).
        \item \( \phi \) is the angle between the positive \( z \)-axis and the line segment \( OP \) (\( 0 \leq \phi \leq \pi \)).
    \end{itemize}
    
    \begin{enumerate}
        \item To convert from \textbf{spherical to Cartesian coordinates}:
        \[
        x = \rho \sin \phi \cos \theta, \quad y = \rho \sin \phi \sin \theta, \quad z = \rho \cos \phi.
        \]
        \item To convert from \textbf{Cartesian to spherical coordinates}:
        \[
        \rho^2 = x^2 + y^2 + z^2, \quad \tan \theta = \frac{y}{x}, \quad \phi = \arccos\left(\frac{z}{\rho}\right).
        \]
    \end{enumerate}
}

\ntimg{
    \subsection*{Applications}
    The spherical coordinate system is particularly useful in problems with radial symmetry about a point, such as:
    \begin{itemize}
        \item \textbf{Spheres}, where the equation \( \rho = c \) defines a sphere of radius \( c \) centered at the origin.
        \item \textbf{Vertical half-planes}, defined by \( \theta = c \).
        \item \textbf{Cones}, defined by \( \phi = c \).
    \end{itemize}
}

\subsection*{Triple Integerals in Spherical Coordinates}
\addcontentsline{toc}{subsection}{Triple Integrals in Spherical Coordinates}

\defrose{Triple Integrals in Spherical Coordinates}{
    TTriple integrals in spherical coordinates are used to compute the integral of a function \( f(x, y, z) \) over a three-dimensional region \( E \) that exhibits spherical symmetry. In this system, the Cartesian coordinates \( (x, y, z) \) are replaced by the spherical coordinates \( (\rho, \theta, \phi) \), where:
    \begin{itemize}
        \item \( \rho \) is the radial distance from the origin to the point (\( \rho \geq 0 \)).
        \item \( \theta \) is the angle in the \( xy \)-plane, measured from the positive \( x \)-axis (\( 0 \leq \theta < 2\pi \)).
        \item \( \phi \) is the angle between the positive \( z \)-axis and the line segment connecting the origin to the point (\( 0 \leq \phi \leq \pi \)).
    \end{itemize}
    
    \subsection*{Conversion and Formula}
    To evaluate a triple integral:
    \begin{enumerate}
        \item Convert the Cartesian coordinates:
        \[
        x = \rho \sin \phi \cos \theta, \quad y = \rho \sin \phi \sin \theta, \quad z = \rho \cos \phi.
        \]
        \item Replace the volume element \( dV \) with \( \rho^2 \sin \phi \, d\rho \, d\theta \, d\phi \).
        \item Use the following formula for integration:
        \[
        \iiint_E f(x, y, z) \, dV = \int_c^d \int_\alpha^\beta \int_a^b f(\rho \sin \phi \cos \theta, \rho \sin \phi \sin \theta, \rho \cos \phi) \rho^2 \sin \phi \, d\rho \, d\theta \, d\phi,
        \]
        where \( E \) is a spherical wedge defined by:
        \[
        E = \{ (\rho, \theta, \phi) \mid a \leq \rho \leq b, \, \alpha \leq \theta \leq \beta, \, c \leq \phi \leq d \}.
        \]
    \end{enumerate}
}

\ntimg{
    \subsection*{Applications}
    This method is particularly advantageous when the region \( E \) or the function \( f(x, y, z) \) involves spherical symmetry, such as:
    \begin{itemize}
        \item Spheres or spherical shells.
        \item Cones, which are naturally described by constant \( \phi \).
        \item Regions with radial symmetry about the origin.
    \end{itemize}
    
    By leveraging the spherical coordinate system, triple integrals over symmetric regions become computationally efficient and geometrically intuitive.
}


\chapter*{16 Vector Calculus}
\addcontentsline{toc}{chapter}{Vector Calculus}

\section*{16.1 Vector Fields}
\addcontentsline{toc}{section}{16.1 Vector Fields}

\subsection*{Vector Fields in $\mathbb{R}^{2} \text{ and } \mathbb{R}^{3}$}
\addcontentsline{toc}{subsection}{Vector Fields in $\mathbb{R}^{2} \text{ and } \mathbb{R}^{3}$}

\defrose{Vector Field}{
	LLet \( D \) be a set in \( \mathbb{R}^2 \) (a plane region). A \textbf{vector field on \( \mathbb{R}^2 \)} is a function \( \mathbf{F} \) that assigns to each point \( (x, y) \) in \( D \) a two-dimensional vector \( \mathbf{F}(x, y) \).
}

\defrose{3 Dimensional Vector Fields}{
	LLet \( E \) be a subset of \( \mathbb{R}^3 \). A \textbf{vector field on \( \mathbb{R}^3 \)} is a function \( \mathbf{F} \) that assigns to each point \( (x, y, z) \) in \( E \) a three-dimensional vector \( \mathbf{F}(x, y, z) \).
}

\subsection*{Gradient Fields}
\addcontentsline{toc}{subsection}{Gradient Fields}

\defrose{Gradient Fields}{
	IIf \( f \) is a scalar function of two variables, its gradient \( \nabla f \) (also called grad \( f \)) is defined by
	\[
	\nabla f(x, y) = f_x(x, y) \mathbf{i} + f_y(x, y) \mathbf{j}
	\]
	Therefore, \( \nabla f \) is a vector field on \( \mathbb{R}^2 \) and is called a \textbf{gradient vector field}. Similarly, if \( f \) is a scalar function of three variables, its gradient is a vector field on \( \mathbb{R}^3 \), given by
	\[
	\nabla f(x, y, z) = f_x(x, y, z) \mathbf{i} + f_y(x, y, z) \mathbf{j} + f_z(x, y, z) \mathbf{k}
	\]
}

\section*{16.2 Line Integrals}
\addcontentsline{toc}{section}{16.2 Line Integrals}


\subsection*{Line Integrals in the Planes}
\addcontentsline{toc}{subsection}{Line Integrals in the Plane}

\defrose{Line Integerals}{
	IIf \( f \) is defined on a smooth curve \( C \) given by 
	\[ x = x(t) \quad y = y(t) \quad a \leq t \leq b \] 
	then the \textbf{line integral} of \( f \) along \( C \) is
	\begin{equation}
		\int_C f(x, y) \, ds = \lim_{n \to \infty} \sum_{i=1}^{n} f(x_i^*, y_i^*) \Delta s_i
	\end{equation}
	if this limit exists.
}

\defrose{Evaluation of Line Integerals}{
	TThe length of the curve \( C \) is
	\[
	L = \int_a^b \sqrt{\left( \frac{dx}{dt} \right)^2 + \left( \frac{dy}{dt} \right)^2} \, dt
	\]

	If \( f \) is continuous, the following formula can be used to evaluate the line integral of \( f \) along \( C \):
	\begin{equation}
		\int_C f(x, y) \, ds = \int_a^b f(x(t), y(t)) \sqrt{\left( \frac{dx}{dt} \right)^2 + \left( \frac{dy}{dt} \right)^2} \, dt
	\end{equation}
}

\subsection*{Line Integrals with Respect to $x$ or $y$}
\addcontentsline{toc}{subsection}{Line Integrals with Respect to $x$ or $y$}

\defrose{Line Integrals with Respect to x or y}{
	TTwo other types of line integrals are obtained by replacing \( \Delta s_i \) by either \( \Delta x_i = x_i - x_{i-1} \) or \( \Delta y_i = y_i - y_{i-1} \). They are called the \textbf{line integrals of \( f \) along \( C \) with respect to \( x \) and \( y \)}:
	\begin{equation}
		\int_C f(x, y) \, dx = \lim_{n \to \infty} \sum_{i=1}^{n} f(x_i^*, y_i^*) \Delta x_i
	\end{equation}

	\begin{equation}
		\int_C f(x, y) \, dy = \lim_{n \to \infty} \sum_{i=1}^{n} f(x_i^*, y_i^*) \Delta y_i
	\end{equation}	

	These line integrals can also be evaluated in terms of a parameter \( t \), where \( x = x(t) \), \( y = y(t) \), and:
	
	\begin{equation}
		\int_C f(x, y) \, dx = \int_a^b f(x(t), y(t)) x'(t) \, dt
	\end{equation}

	\begin{equation}
		\int_C f(x, y) \, dy = \int_a^b f(x(t), y(t)) y'(t) \, dt
	\end{equation}
}

\subsection*{Line Integrals in Space}
\addcontentsline{toc}{subsection}{Line Integrals in Space}

\defrose{Line Integerals in Space}{
	EEquation: 
	\begin{equation}
		\int_C f(x, y, z) \, ds = \int_a^b f(x(t), y(t), z(t)) \sqrt{\left( \frac{dx}{dt} \right)^2 + \left( \frac{dy}{dt} \right)^2 + \left( \frac{dz}{dt} \right)^2} \, dt
	\end{equation}
}

\subsection*{Line Integrals of Vector Fields; Work}

\defrose{Line Integral and Work on a Vector Field}{
    TThe \textbf{work} \( W \) done by a continuous vector field \( \mathbf{F} \) along a smooth curve \( C \) is defined as the line integral of \( \mathbf{F} \) along \( C \). If 
    \[
    \mathbf{F}(x, y, z) = P(x, y, z)\mathbf{i} + Q(x, y, z)\mathbf{j} + R(x, y, z)\mathbf{k},
    \]
    and \( C \) is parameterized by a vector function 
    \[
    \mathbf{r}(t) = x(t)\mathbf{i} + y(t)\mathbf{j} + z(t)\mathbf{k}, \quad t \in [a, b],
    \]
    then the work is given by:
    \[
    W = \int_C \mathbf{F} \cdot d\mathbf{r} = \int_a^b \mathbf{F}(\mathbf{r}(t)) \cdot \mathbf{r}'(t) \, dt.
    \]
    
    Equivalently, this can be expressed in terms of the components of \( \mathbf{F} \) and the derivatives of \( \mathbf{r}(t) \) as:
    \[
    W = \int_a^b \left[ P(x(t), y(t), z(t)) \frac{dx}{dt} + Q(x(t), y(t), z(t)) \frac{dy}{dt} + R(x(t), y(t), z(t)) \frac{dz}{dt} \right] dt.
    \]
    
    Or, in differential form:
    \[
    W = \int_C P \, dx + Q \, dy + R \, dz.
    \]
    
}

\section*{16.3 The fundamental Theorem for Line Integrals}
\addcontentsline{toc}{section}{16.3 The fundamental Theorem for Line Integrals}

\subsection*{The Fundamental Theorem for Line Integrals}
\addcontentsline{toc}{subsection}{The Fundamental Theorem for Line Integrals}

\defrose{Fundamental Theorem for Line Integrals}{
    TThe Fundamental Theorem for Line Integrals states that if a vector field is conservative (i.e., it is the gradient of some scalar function \( f \)), the line integral of the vector field over a smooth curve \( C \) depends only on the values of \( f \) at the endpoints of the curve. This means the integral is path-independent.

    \vspace{1em}
    
    \textbf{Mathematical Definition:}  
    Let \( \mathbf{F} = \nabla f \) (a conservative vector field where \( f \) is the potential function), and let \( C \) be a smooth curve parameterized by \( \mathbf{r}(t) \) for \( t \in [a, b] \). Then:
    \[
    \int_C \mathbf{F} \cdot d\mathbf{r} = f(\mathbf{r}(b)) - f(\mathbf{r}(a)),
    \]
    where:
    \begin{itemize}
        \item \( \mathbf{r}(a) \) and \( \mathbf{r}(b) \) are the initial and terminal points of the curve \( C \),
        \item \( \nabla f \) is the gradient of \( f \), which represents the vector field \( \mathbf{F} \).
    \end{itemize}    
}

\subsection*{Independence of path}
\addcontentsline{toc}{subsection}{Independence of Path}

\defrose{Path Independence}{
    TThe \textbf{Independence of Path} property states that for a conservative vector field \( \mathbf{F} \), the line integral between two points depends only on the endpoints, not on the path taken. This implies that the line integral over any closed path is zero. \\


    Additionally, if the line integral of \( \mathbf{F} \) over all closed paths in a domain \( D \) is zero, then \( \mathbf{F} \) is a conservative vector field, and there exists a potential function \( f \) such that \( \mathbf{F} = \nabla f \).

    \vspace{1em}

    \textbf{Mathematical Definition:}

    1. \textbf{Independence of Path:}  
    If \( \mathbf{F} \) is a continuous vector field in a domain \( D \), then \( \int_C \mathbf{F} \cdot d\mathbf{r} \) is independent of the path \( C \) if and only if:
    \[
    \int_C \mathbf{F} \cdot d\mathbf{r} = 0
    \]
    for every closed path \( C \) in \( D \). \\

    2. \textbf{Theorem:}  
    Suppose \( \mathbf{F} \) is a vector field that is continuous on an open connected region \( D \). If \( \int_C \mathbf{F} \cdot d\mathbf{r} \) is independent of path in \( D \), then \( \mathbf{F} \) is a conservative vector field on \( D \), meaning there exists a potential function \( f \) such that:
    \[
    \nabla f = \mathbf{F}.
    \]
}

\subsection*{Conservative Vector Fields and Potential Functions}
\addcontentsline{toc}{subsection}{Conservative Vector Fields and Potential Functions}

\defrose{Conservative Vector Fields}{
    AA vector field \( \mathbf{F} \) is conservative if it can be expressed as the gradient of a scalar potential function \( f \), i.e., \( \mathbf{F} = \nabla f \). This implies that the curl of \( \mathbf{F} \) is zero, and the field is path-independent.  \\

    For conservative fields in two dimensions, specific conditions involving the partial derivatives of the components \( P \) and \( Q \) can help verify conservativeness.

    \vspace{1em}

    \textbf{Mathematical Definitions:}

    1. If \( \mathbf{F}(x, y) = P(x, y) \, \mathbf{i} + Q(x, y) \, \mathbf{j} \) is a conservative vector field, where \( P \) and \( Q \) have continuous first-order partial derivatives throughout a domain \( D \), then:
    \[
    \frac{\partial P}{\partial y} = \frac{\partial Q}{\partial x}.
    \]

    2. Let \( \mathbf{F}(x, y) = P(x, y) \, \mathbf{i} + Q(x, y) \, \mathbf{j} \) be a vector field on an open simply-connected region \( D \). If \( P \) and \( Q \) have continuous first-order partial derivatives and:
    \[
    \frac{\partial P}{\partial y} = \frac{\partial Q}{\partial x},
    \]
    throughout \( D \), then \( \mathbf{F} \) is a conservative vector field on \( D \).
}

\exrose{Generalized Steps to Solve for Potential Functions}{
    \begin{enumerate}
        \item \textbf{Check Conservativity}: Verify if the vector field is conservative:
        \[
        \nabla \times \mathbf{F} = 0 \quad \text{(in } \mathbb{R}^3 \text{)}, \quad 
        \frac{\partial P}{\partial y} = \frac{\partial Q}{\partial x} \quad \text{(in } \mathbb{R}^2 \text{)}.
        \]
    
        \item \textbf{Set Gradient Relations}: Assume \( \mathbf{F} = \nabla f \), which gives:
        \[
        f_x = P, \quad f_y = Q, \quad f_z = R.
        \]
    
        \item \textbf{Integrate First Component}: Solve for \( f \) using \( f_x = P \):
        \[
        f(x, y, z) = \int P \, dx + g(y, z).
        \]
    
        \item \textbf{Differentiate and Compare}: Use \( f_y = Q \) to determine \( g(y, z) \):
        \[
        f_y = \frac{\partial}{\partial y}\left(\int P \, dx + g(y, z)\right),
        \]
        and compare with \( Q \).
    
        \item \textbf{Repeat for Remaining Component}: Use \( f_z = R \) to find any remaining terms in \( g(y, z) \):
        \[
        f_z = \frac{\partial}{\partial z} f(x, y, z).
        \]
    
        \item \textbf{Combine Results}: Combine all terms to construct \( f(x, y, z) \), including constants of integration.
    \end{enumerate}}

\section*{16.4 Green's Theorem}
\addcontentsline{toc}{section}{16.4 Green's Theorem}

\subsection*{Green's Theorem}
\addcontentsline{toc}{subsection}{Green's Theorem}

\defrose{Green's Theorem}{
    A \textbf{Definition:} Green's Theorem states that the line integral of a vector field \( \mathbf{F} = \langle P, Q \rangle \) around a simple, closed curve \( C \) is equal to the double integral of the curl of \( \mathbf{F} \) over the region \( D \) enclosed by \( C \).

    \[ \oint_C P \, dx + Q \, dy = \iint_D \left( \frac{\partial Q}{\partial x} - \frac{\partial P}{\partial y} \right) \, dA. \]
    
    \textbf{Concept:} Green's Theorem relates the circulation of a vector field along a curve (the boundary \( C \)) to the cumulative curl (rotation) of the field within the enclosed region \( D \). It connects local properties (curl) to global properties (line integral).
}

\subsection*{Finding Areas with Green's Theorem}
\addcontentsline{toc}{subsection}{Finding Areas with Green's Theorem}

\defrose{Finding Area using Green's Theorem}{
    A 
    \textbf{Conceptual Explanation:}  
    Green's Theorem can be used to compute the area \( A \) of a region \( D \) enclosed by a positively oriented, simple closed curve \( C \). By choosing \( P(x, y) \) and \( Q(x, y) \) such that the curl condition 
    \[
    \frac{\partial Q}{\partial x} - \frac{\partial P}{\partial y} = 1
    \]
    is satisfied, the area can be calculated directly using a line integral.
    
    \textbf{Mathematical Formulation:}  
    From Green's Theorem:
    \[
    A = \iint_D 1 \, dA = \oint_C \left( P \, dx + Q \, dy \right),
    \]
    where \( P(x, y) \) and \( Q(x, y) \) satisfy:
    \[
    \frac{\partial Q}{\partial x} - \frac{\partial P}{\partial y} = 1.
    \]

    \textbf{Key Insight:}  
    The line integral simplifies the area computation by converting the problem of finding the area \( D \) into evaluating the boundary integral around \( C \).    
}

\exrose{Special Cases}{
    \textbf{Special Cases for \( P(x, y) \) and \( Q(x, y) \):}
    \begin{itemize}
        \item \( P(x, y) = 0 \), \( Q(x, y) = x \): 
        \[
        A = \oint_C x \, dy
        \]
        \item \( P(x, y) = -y \), \( Q(x, y) = 0 \): 
        \[
        A = -\oint_C y \, dx
        \]
        \item \( P(x, y) = -\frac{1}{2}y \), \( Q(x, y) = \frac{1}{2}x \): 
        \[
        A = \frac{1}{2} \oint_C \left( x \, dy - y \, dx \right).
        \]
    \end{itemize}
    
    \textbf{Example:}  
    For an ellipse \( \frac{x^2}{a^2} + \frac{y^2}{b^2} = 1 \), parametrized by:
    \[
    x = a \cos t, \quad y = b \sin t, \quad t \in [0, 2\pi],
    \]
    the area can be computed as:
    \[
    A = \frac{1}{2} \oint_C \left( x \, dy - y \, dx \right).
    \]
    
    Substitute \( x = a \cos t \) and \( y = b \sin t \):
    \[
    A = \frac{1}{2} \int_0^{2\pi} \left[ (a \cos t)(b \cos t) - (b \sin t)(-a \sin t) \right] dt.
    \]
    
    Simplify:
    \[
    A = \frac{ab}{2} \int_0^{2\pi} dt = \pi ab.
    \]
    
}

\subsection*{Extended Version of Green's Theorem}
\addcontentsline{toc}{subsection}{Extended Version of Green's Theorem}

\defrose{Extended Version of Green's Theorem}{
    GGreen's Theorem can be extended to apply to regions that are unions of multiple simple regions or regions with holes. For such cases:
    \begin{itemize}
        \item The region \( D \) is divided into subregions \( D_1, D_2, \dots \), each with simple boundaries.
        \item The boundaries of the subregions are oriented consistently such that shared edges cancel when summing the line integrals, preserving the integrity of the theorem.
    \end{itemize}
    
    \textbf{Mathematical Formulation:}  
    For a region \( D = D_1 \cup D_2 \) (union of two subregions):
    \[ \oint_{C_1 \cup C_3} P \, dx + Q \, dy = \iint_{D_1} \left( \frac{\partial Q}{\partial x} - \frac{\partial P}{\partial y} \right) \, dA, \]
    \[ \oint_{C_2 \cup (-C_3)} P \, dx + Q \, dy = \iint_{D_2} \left( \frac{\partial Q}{\partial x} - \frac{\partial P}{\partial y} \right) \, dA. \]
    
    Adding these results cancels the shared boundary integrals:
    \[ \oint_{C_1 \cup C_2} P \, dx + Q \, dy = \iint_{D} \left( \frac{\partial Q}{\partial x} - \frac{\partial P}{\partial y} \right) \, dA, \]
    where \( C = C_1 \cup C_2 \) is the outer boundary of \( D \). \\
    
    For regions with holes, the boundary \( C \) includes both the outer boundary \( C_1 \) (counterclockwise) and the inner boundary \( C_2 \) (clockwise):
    \[ \iint_{D} \left( \frac{\partial Q}{\partial x} - \frac{\partial P}{\partial y} \right) \, dA = \oint_{C_1} P \, dx + Q \, dy + \oint_{C_2} P \, dx + Q \, dy. \]
}

\section*{16.5 Curl and Divergence}
\addcontentsline{toc}{section}{Curl and Divergence}

\subsection*{Curl}

\defrose{Curl}{
    TThe \textbf{curl} of a vector field $\mathbf{F}$ in $\mathbb{R}^3$, where $\mathbf{F} = P\mathbf{i} + Q\mathbf{j} + R\mathbf{k}$, is a vector field that measures the rotational tendency or circulation of $\mathbf{F}$ at a point. Mathematically, it is given by:
    \[
    \text{curl} \, \mathbf{F} = \nabla \times \mathbf{F}
    \]
    
    This can be computed using the cross product of the del operator ($\nabla = \mathbf{i} \frac{\partial}{\partial x} + \mathbf{j} \frac{\partial}{\partial y} + \mathbf{k} \frac{\partial}{\partial z}$) and the vector field $\mathbf{F}$. The result is:
    \[
    \text{curl} \, \mathbf{F} = \begin{vmatrix}
    \mathbf{i} & \mathbf{j} & \mathbf{k} \\
    \frac{\partial}{\partial x} & \frac{\partial}{\partial y} & \frac{\partial}{\partial z} \\
    P & Q & R
    \end{vmatrix}
    \]
    
    Explicitly, the components are:
    \[
    \text{curl} \, \mathbf{F} = \left( \frac{\partial R}{\partial y} - \frac{\partial Q}{\partial z} \right) \mathbf{i} + \left( \frac{\partial P}{\partial z} - \frac{\partial R}{\partial x} \right) \mathbf{j} + \left( \frac{\partial Q}{\partial x} - \frac{\partial P}{\partial y} \right) \mathbf{k}
    \]
    
    This definition captures the curl as the formal representation of the infinitesimal rotation of $\mathbf{F}$ in three-dimensional space.
}

\ntimg{
    The curl of the gradient of a scalar function \( f \) with continuous second-order partial derivatives is always zero:
    \[ \text{curl}(\nabla f) = 0 \]
}

\ntimg{
    If a vector field \( \mathbf{F} \) is defined on all of \( \mathbb{R}^3 \), has continuous partial derivatives, and satisfies \( \text{curl} \, \mathbf{F} = 0 \), then \( \mathbf{F} \) is a conservative vector field.
}

\ntimg{
    The term \textbf{curl} is associated with rotations in a vector field. Specifically:

    \begin{enumerate}
        \item \textbf{Curl and Rotation}: The curl vector indicates the axis and direction of rotation of nearby particles, following the right-hand rule. The magnitude of the curl measures the speed of this rotation.
        
        \item \textbf{Irrotational Flow}: If \( \text{curl} \, \mathbf{F} = 0 \) at a point \( P \), the vector field is called \textbf{irrotational} at \( P \). Here, particles move with the flow but do not rotate about their own axis.
        
        \item \textbf{Rotational Flow}: If \( \text{curl} \, \mathbf{F} \neq 0 \), the particles rotate about their axis. The direction of the curl vector points along the axis of rotation.
        
        \item \textbf{Illustration with Fluid Flow}:
        \begin{itemize}
            \item In a fluid velocity field, a paddle wheel at a point \( P_1 \) where \( \text{curl} \, \mathbf{F} \neq 0 \) would rotate about its axis, indicating rotational flow.
            \item At a point \( P_2 \) where \( \text{curl} \, \mathbf{F} = 0 \), the paddle wheel moves with the flow but does not rotate, indicating irrotational flow.
        \end{itemize}
    \end{enumerate}
}
\defrose{Divergence}{
    TThe \textbf{divergence} of a vector field \( \mathbf{F} \) in \( \mathbb{R}^3 \), where \( \mathbf{F} = P\mathbf{i} + Q\mathbf{j} + R\mathbf{k} \), is a scalar function that measures the rate at which the vector field spreads out (or converges) at a given point. It is defined as:
    \[
    \text{div} \, \mathbf{F} = \frac{\partial P}{\partial x} + \frac{\partial Q}{\partial y} + \frac{\partial R}{\partial z}
    \]
    
    Alternatively, using the del operator (\( \nabla = \mathbf{i} \frac{\partial}{\partial x} + \mathbf{j} \frac{\partial}{\partial y} + \mathbf{k} \frac{\partial}{\partial z} \)), the divergence can be expressed as the dot product of \( \nabla \) and \( \mathbf{F} \):
    \[
    \text{div} \, \mathbf{F} = \nabla \cdot \mathbf{F}
    \]
    
    \subsection*{Key Property}
    For a vector field \( \mathbf{F} \) in \( \mathbb{R}^3 \) with continuous second-order partial derivatives, the divergence of the curl of \( \mathbf{F} \) is always zero:
    \[
    \text{div} (\text{curl} \, \mathbf{F}) = 0
    \]
    
    This captures the idea that the curl of a vector field does not "spread out" or "converge" in space.    
}

\ntimg{
    \begin{enumerate}
        \item \textbf{Divergence in Fluid Flow}:
        \begin{itemize}
            \item If $\mathbf{F}(x, y, z)$ represents the velocity of a fluid or gas, $\text{div} \, \mathbf{F}(x, y, z)$ measures the net rate of change of the mass of fluid (or gas) flowing from the point $(x, y, z)$ per unit volume.
            \item It indicates the tendency of the fluid to spread out (diverge) from a point.
        \end{itemize}

        \item \textbf{Incompressible Flow}:
        \begin{itemize}
            \item If $\text{div} \, \mathbf{F} = 0$, the fluid is said to be \textbf{incompressible}, meaning there is no net outflow or inflow at that point.
        \end{itemize}

        \item \textbf{Illustration}:
        \begin{itemize}
            \item \textbf{Case 1 ($\text{div} \, \mathbf{F} \neq 0$)}: 
            \begin{itemize}
                \item At a point $P_1$, if $\text{div} \, \mathbf{F} < 0$, the flow is inward (net inflow).
                \item At a point $P_2$, if $\text{div} \, \mathbf{F} > 0$, the flow is outward (net outflow).
            \end{itemize}
            \item \textbf{Case 2 ($\text{div} \, \mathbf{F} = 0$)}: 
            \begin{itemize}
                \item The flow is balanced (no net divergence or convergence).
            \end{itemize}
        \end{itemize}
    \end{enumerate}
}

\subsection*{Vector Forms of Green's Theorem}

\defrose{Vector Forms of Green's Theorem}{
    Green's Theorem relates a line integral around a simple closed curve \( C \) to a double integral over the plane region \( D \) enclosed by \( C \). It provides two key vector forms:
    
    \subsection*{1. Curl Form (Tangential Component):}
    The line integral of the tangential component of \( \mathbf{F} \) along \( C \) is equal to the double integral of the vertical component of \( \text{curl} \, \mathbf{F} \) over \( D \):
    \[
    \oint_C \mathbf{F} \cdot d\mathbf{r} = \oint_C \mathbf{F} \cdot \mathbf{T} \, ds = \iint_D (\text{curl} \, \mathbf{F}) \cdot \mathbf{k} \, dA
    \]
    where 
    \[
    (\text{curl} \, \mathbf{F}) \cdot \mathbf{k} = \frac{\partial Q}{\partial x} - \frac{\partial P}{\partial y}.
    \]
    
    \subsection*{2. Divergence Form (Normal Component):}
    The line integral of the normal component of \( \mathbf{F} \) along \( C \) is equal to the double integral of the divergence of \( \mathbf{F} \) over \( D \):
    \[
    \oint_C \mathbf{F} \cdot \mathbf{n} \, ds = \iint_D \text{div} \, \mathbf{F} \, dA
    \]
    where 
    \[
    \text{div} \, \mathbf{F} = \frac{\partial P}{\partial x} + \frac{\partial Q}{\partial y}.
    \]
    
    \subsection*{Geometric Interpretation:}
    \begin{itemize}
        \item The curl form expresses the relationship between the circulation of \( \mathbf{F} \) along the boundary \( C \) and the "rotational tendency" of \( \mathbf{F} \) over the region \( D \).
        \item The divergence form relates the flux of \( \mathbf{F} \) across \( C \) to the net outflow (or inflow) of \( \mathbf{F} \) over \( D \).
    \end{itemize}
    
    These two forms of Green's Theorem are fundamental in understanding the relationship between local properties of vector fields and their global integrals.
}

\section*{16.6 Parametric Surfaces and Their Areas}
\addcontentsline{toc}{section}{Parametric Surfaces and Their Areas}

\subsection*{Parametric Surfaces}

\defrose{Parametric Surface}{
    AA \textit{parametric surface} is a surface in three-dimensional space $\mathbb{R}^3$ defined by a vector-valued function $\mathbf{r}(u, v)$, which depends on two parameters $u$ and $v$. The function is expressed as:
    \[ \mathbf{r}(u, v) = x(u, v) \, \mathbf{i} + y(u, v) \, \mathbf{j} + z(u, v) \, \mathbf{k}, \]
    where $x(u, v)$, $y(u, v)$, and $z(u, v)$ are the component functions of $\mathbf{r}$, representing the $x$-, $y$-, and $z$-coordinates of the surface, respectively. These functions are defined over a region $D$ in the $uv$-plane. The set of all points $(x, y, z) \in \mathbb{R}^3$ that satisfy:
    \[ x = x(u, v), \quad y = y(u, v), \quad z = z(u, v), \]
    as $(u, v)$ varies over $D$, forms the parametric surface $S$.

}

\subsection*{Parametric Equations}

\defrose{Parametric Equations}{
    FFor a parametric surface the \textit{parametric equations} are equations that describe the coordinates $(x, y, z)$ of points on the surface as functions of two independent parameters $u$ and $v$. For a parametric surface $S$, these equations are given by:
    \[ x = x(u, v), \quad y = y(u, v), \quad z = z(u, v), \]
    where $x(u, v)$, $y(u, v)$, and $z(u, v)$ are the component functions of a vector-valued function $\mathbf{r}(u, v)$. These equations define the spatial coordinates of the surface for every pair of parameters $(u, v)$ in a specified domain $D$ in the $uv$-plane.
}

\subsection*{Grid Curves}

\defrose{Grid Curves}{
    OOn a parametric surface $s$ \textit{grid curves} are families of curves defined by the vector function $\mathbf{r}(u, v)$. They are obtained by fixing one parameter and varying the other:

    1. \textbf{Curves with $u = u_0$:} When $u$ is held constant, the parametric surface reduces to a curve:
    \[
    \mathbf{r}(u_0, v) = \langle x(u_0, v), y(u_0, v), z(u_0, v) \rangle,
    \]
    which traces a curve $C_1$ on the surface as $v$ varies.

    2. \textbf{Curves with $v = v_0$:} When $v$ is held constant, the parametric surface reduces to a curve:
    \[
    \mathbf{r}(u, v_0) = \langle x(u, v_0), y(u, v_0), z(u, v_0) \rangle,
    \]
    which traces a curve $C_2$ on the surface as $u$ varies.

    These two families of curves correspond to horizontal and vertical lines in the $uv$-plane and form a grid-like structure when plotted on the surface.

}

\subsection*{Spherical Coordinates}

\subsection*{Surfaces of Revolution}

\defrose{Surfaces of Revolution}{
    AA \textbf{surface of revolution} is generated by rotating a curve \( C \), defined parametrically or as a function, about a fixed axis in three-dimensional space. The parametric equations of the surface can be expressed as:

    \[
    \begin{aligned}
    x &= u, \\
    y &= r(u) \cos \theta, \\
    z &= r(u) \sin \theta,
    \end{aligned}
    \]

    where:
    \begin{itemize}
        \item \( u \) is a parameter describing the curve \( C \),
        \item \( r(u) \) is the radial distance of the curve from the axis of rotation,
        \item \( \theta \in [0, 2\pi] \) is the angle of rotation.
    \end{itemize}

    The domain of the parameters \( u \) and \( \theta \) depends on the curve and the extent of rotation.


}

\subsection*{Tangent Planes}

\defrose{Tangent Planes}{
    T
    The \textbf{tangent plane} to a parametric surface \( S \) at a point \( P_0(u_0, v_0) \) is the plane that best approximates \( S \) near \( P_0 \). 
    
    If \( S \) is defined by a vector-valued function:
    \[ \mathbf{r}(u, v) = x(u, v) \mathbf{i} + y(u, v) \mathbf{j} + z(u, v) \mathbf{k}, \]
    then the tangent plane at \( P_0 \) is determined by the two tangent vectors at \( P_0 \):
    \[ \mathbf{r}_u = \frac{\partial \mathbf{r}}{\partial u} = \frac{\partial x}{\partial u} \mathbf{i} + \frac{\partial y}{\partial u} \mathbf{j} + \frac{\partial z}{\partial u} \mathbf{k}, \]
    \[ \mathbf{r}_v = \frac{\partial \mathbf{r}}{\partial v} = \frac{\partial x}{\partial v} \mathbf{i} + \frac{\partial y}{\partial v} \mathbf{j} + \frac{\partial z}{\partial v} \mathbf{k}.  \]
    
    The tangent plane at \( P_0 \) is spanned by \(\mathbf{r}_u\) and \(\mathbf{r}_v\). A normal vector to the plane is given by:
    \[ \mathbf{n} = \mathbf{r}_u \times \mathbf{r}_v. \]
    
    The equation of the tangent plane can be expressed in the point-normal form:
    \[ \mathbf{n} \cdot (\mathbf{r} - \mathbf{r}(u_0, v_0)) = 0, \]
    where \(\mathbf{r}(u_0, v_0)\) is the position vector of \( P_0 \).
    
    For the tangent plane to exist, the cross product \(\mathbf{r}_u \times \mathbf{r}_v\) must be nonzero, ensuring that \( S \) is smooth at \( P_0 \).
    
}
\subsection*{Surface Area for a Parametric Surface}

\defrose{Surface Area for a Parametric Surface}{
    TThe \textbf{surface area} of a smooth parametric surface \( S \), defined by the vector-valued function:
    \[ \mathbf{r}(u, v) = x(u, v) \mathbf{i} + y(u, v) \mathbf{j} + z(u, v) \mathbf{k}, \quad (u, v) \in D, \]
    where \( D \) is the parameter domain, is given by the integral:
    \[ A(S) = \iint_D \left| \mathbf{r}_u \times \mathbf{r}_v \right| \, dA, \]
    where:
    \[ \mathbf{r}_u = \frac{\partial x}{\partial u} \mathbf{i} + \frac{\partial y}{\partial u} \mathbf{j} + \frac{\partial z}{\partial u} \mathbf{k}, \quad
    \mathbf{r}_v = \frac{\partial x}{\partial v} \mathbf{i} + \frac{\partial y}{\partial v} \mathbf{j} + \frac{\partial z}{\partial v} \mathbf{k}. \]
    
    The cross product \(\mathbf{r}_u \times \mathbf{r}_v\) represents a vector orthogonal to the tangent plane at each point on the surface, and its magnitude \(\left| \mathbf{r}_u \times \mathbf{r}_v \right|\) gives the infinitesimal area of a parallelogram spanned by the tangent vectors \(\mathbf{r}_u\) and \(\mathbf{r}_v\). Integrating this quantity over the parameter domain \( D \) yields the total surface area of \( S \).
        
}

\subsection*{Surface Area of the Graph of a Function}

\defrose{Surface Area of the Graph of a Function}{
    TThe \textbf{surface area} of the graph of a function \( z = f(x, y) \), where \( f(x, y) \) has continuous partial derivatives, over a region \( D \) in the \( xy \)-plane is given by:
    \[
    A(S) = \iint_D \sqrt{1 + \left( \frac{\partial f}{\partial x} \right)^2 + \left( \frac{\partial f}{\partial y} \right)^2} \, dA.
    \]
    
    \subsection*{Explanation}
    \begin{itemize}
        \item The parametric representation of the surface is:
        \[ \mathbf{r}(x, y) = x \mathbf{i} + y \mathbf{j} + f(x, y) \mathbf{k}. \]
        \item The tangent vectors are:
        \[ \mathbf{r}_x = \mathbf{i} + \frac{\partial f}{\partial x} \mathbf{k}, \quad \mathbf{r}_y = \mathbf{j} + \frac{\partial f}{\partial y} \mathbf{k}. \]
        \item The magnitude of the cross product of the tangent vectors is:
        \[ \left| \mathbf{r}_x \times \mathbf{r}_y \right| = \sqrt{1 + \left( \frac{\partial f}{\partial x} \right)^2 + \left( \frac{\partial f}{\partial y} \right)^2}. \]
    \end{itemize}
    
    Integrating this quantity over the region \( D \) in the \( xy \)-plane gives the total surface area of the graph of \( f(x, y) \).    
}


\section*{16.7 Surface Integral}
\addcontentsline{toc}{section}{Surface Integral}

\subsection*{Parametric Surfaces}

\defrose{Surface Integral for Parametric Surfaces}{
    TThe \textbf{surface integral} of a scalar function \( f(x, y, z) \) over a parametric surface \( S \), defined by the vector equation:
    \[ \mathbf{r}(u, v) = x(u, v) \mathbf{i} + y(u, v) \mathbf{j} + z(u, v) \mathbf{k}, \quad (u, v) \in D, \]
    is given by:
    \[ \iint_S f(x, y, z) \, dS = \iint_D f(\mathbf{r}(u, v)) \left| \mathbf{r}_u \times \mathbf{r}_v \right| \, dA. \]
    
    \subsection*{Explanation}
    \begin{itemize}
        \item The parameter domain \( D \) is divided into subrectangles with dimensions \( \Delta u \) and \( \Delta v \), and each corresponding surface patch is approximated as a parallelogram in the tangent plane.
        \item The area of a surface patch is approximated as:
        \[ \Delta S_{ij} \approx \left| \mathbf{r}_u \times \mathbf{r}_v \right| \Delta u \Delta v.\]
        \item The surface integral is defined as the limit of a Riemann sum:
        \[ \iint_S f(x, y, z) \, dS = \lim_{m, n \to \infty} \sum_{i=1}^m \sum_{j=1}^n f(P_{ij}^*) \Delta S_{ij}. \]
    \end{itemize}
    
    \subsection*{Key Components}
    \begin{itemize}
        \item \(\mathbf{r}_u\) and \(\mathbf{r}_v\) are the partial derivatives of \(\mathbf{r}(u, v)\) with respect to \( u \) and \( v \), respectively:
        \[ \mathbf{r}_u = \frac{\partial \mathbf{r}}{\partial u}, \quad \mathbf{r}_v = \frac{\partial \mathbf{r}}{\partial v}. \]
        \item \(\mathbf{r}_u \times \mathbf{r}_v\) gives a vector normal to the surface at each point, and \(\left| \mathbf{r}_u \times \mathbf{r}_v \right|\) represents the infinitesimal surface area element.
    \end{itemize}
    
    This integral evaluates the contribution of \( f(x, y, z) \) across the entire surface \( S \).
        
}

\subsection*{Graphs of Functions}
\defrose{Surafe Integrals for Graphs of Functions}{
    TThe \textbf{surface integral} of a scalar function \( f(x, y, z) \) over the graph of a function \( z = g(x, y) \), where \( g(x, y) \) has continuous partial derivatives, is given by:
    \[
    \iint_S f(x, y, z) \, dS = \iint_D f(x, y, g(x, y)) \sqrt{\left(\frac{\partial g}{\partial x}\right)^2 + \left(\frac{\partial g}{\partial y}\right)^2 + 1} \, dA.
    \]
    
    \subsection*{Explanation}
    \begin{itemize}
        \item The graph of the function \( z = g(x, y) \) can be regarded as a parametric surface with:
        \[
        x = x, \quad y = y, \quad z = g(x, y).
        \]
        \item The tangent vectors to this surface are:
        \[
        \mathbf{r}_x = \mathbf{i} + \frac{\partial g}{\partial x} \mathbf{k}, \quad \mathbf{r}_y = \mathbf{j} + \frac{\partial g}{\partial y} \mathbf{k}.
        \]
        \item The cross product of the tangent vectors is:
        \[
        \mathbf{r}_x \times \mathbf{r}_y = -\frac{\partial g}{\partial x} \mathbf{j} - \frac{\partial g}{\partial y} \mathbf{i} + \mathbf{k}.
        \]
        \item The magnitude of the cross product is:
        \[
        \left| \mathbf{r}_x \times \mathbf{r}_y \right| = \sqrt{\left(\frac{\partial g}{\partial x}\right)^2 + \left(\frac{\partial g}{\partial y}\right)^2 + 1}.
        \]
    \end{itemize}
    
    By integrating this quantity over the region \( D \) in the \( xy \)-plane, we account for the contributions of \( f(x, y, z) \) over the entire surface \( S \).    
}

\subsection*{Oriented Surfaces}

\defrose{Oriented Surfaces}{
    AAn \textbf{oriented surface} is an orientable (two-sided) surface \( S \) where it is possible to define a continuous, unit normal vector \( \mathbf{n} \) at every point \( (x, y, z) \) on the surface, except possibly at boundary points.

    \subsection*{Key Properties}
    \begin{itemize}
        \item \textbf{Two Possible Orientations}: For any orientable surface, there are two choices for the unit normal vector:
        \begin{itemize}
            \item \( \mathbf{n}_1 \), the chosen unit normal vector.
            \item \( \mathbf{n}_2 = -\mathbf{n}_1 \), the opposite orientation.
        \end{itemize}
        \item A surface is called \textbf{orientable} if it is possible to assign \( \mathbf{n} \) continuously over the entire surface \( S \).
        \item A classic example of a non-orientable surface is the Möbius strip, which has only one side and no consistent orientation.
    \end{itemize}

    \subsection*{Explanation}
    An oriented surface requires the existence of a consistent way to assign a "positive" or "negative" side across all points on the surface. The orientation is provided by the chosen direction of the normal vector \( \mathbf{n} \), which varies smoothly across the surface.
}

\subsection*{Surface Integrals of Vector Fields; Flux}
\defrose{Flux}{
    The \textbf{surface integral of a vector field} (also called the \textbf{flux}) over an oriented surface \( S \) with a unit normal vector \( \mathbf{n} \) is defined as:
    \[
    \iint_S \mathbf{F} \cdot d\mathbf{S} = \iint_S \mathbf{F} \cdot \mathbf{n} \, dS,
    \]
    where:
    \begin{itemize}
        \item \( \mathbf{F} \) is a continuous vector field defined on \( S \),
        \item \( \mathbf{n} \) is the unit normal vector to \( S \),
        \item \( dS \) represents the infinitesimal surface area element.
    \end{itemize}
    
    \subsection*{Special Case: Surface Defined by \( z = g(x, y) \)}
    If the surface \( S \) is defined by the graph \( z = g(x, y) \), and \( \mathbf{F}(x, y, z) = P \mathbf{i} + Q \mathbf{j} + R \mathbf{k} \), the surface integral can be expressed as:
    \[
    \iint_S \mathbf{F} \cdot d\mathbf{S} = \iint_D \left( -P \frac{\partial g}{\partial x} - Q \frac{\partial g}{\partial y} + R \right) \, dA,
    \]
    where \( D \) is the projection of the surface onto the \( xy \)-plane.
    
    This formula assumes the upward orientation of \( S \). For a downward orientation, the integral is multiplied by \( -1 \).
    
    \subsection*{Parametric Form}
    If the surface \( S \) is parameterized by \( \mathbf{r}(u, v) \), with tangent vectors:
    \[
    \mathbf{r}_u = \frac{\partial \mathbf{r}}{\partial u}, \quad \mathbf{r}_v = \frac{\partial \mathbf{r}}{\partial v},
    \]
    then the flux integral can be written as:
    \[
    \iint_S \mathbf{F} \cdot d\mathbf{S} = \iint_D \mathbf{F}(\mathbf{r}(u, v)) \cdot (\mathbf{r}_u \times \mathbf{r}_v) \, dA,
    \]
    where \( D \) is the parameter domain.
    
    \subsection*{Physical Interpretation}
    The flux integral measures the total flow of the vector field \( \mathbf{F} \) across the surface \( S \), representing quantities like mass flow rate, electric flux, or fluid flow through \( S \).
}    

\section*{16.8 Stokes' Theorem} 

\defrose{Stokes' Theorem}{
    SStokes' Theorem relates the surface integral of the curl of a vector field over an oriented surface \( S \) to the line integral of the vector field along the boundary curve \( C \) of \( S \). Mathematically, it is expressed as:

    \[
    \iint_S (\nabla \times \mathbf{F}) \cdot d\mathbf{S} = \oint_C \mathbf{F} \cdot d\mathbf{r},
    \]

    where:
    \begin{itemize}
        \item \( S \) is a piecewise-smooth, oriented surface with unit normal vector \( \mathbf{n} \),
        \item \( C \) is the positively oriented, closed boundary curve of \( S \),
        \item \( \mathbf{F} \) is a vector field with continuous partial derivatives,
        \item \( \nabla \times \mathbf{F} \) is the curl of \( \mathbf{F} \),
        \item \( d\mathbf{S} = \mathbf{n} \, dS \) is the oriented surface element,
        \item \( d\mathbf{r} \) is the infinitesimal vector along \( C \).
    \end{itemize}

    \subsection*{Key Notes}
    \begin{itemize}
        \item \textbf{Positive Orientation:} The orientation of \( C \) is determined by the right-hand rule: when you walk along \( C \) with your head pointing in the direction of \( \mathbf{n} \), the surface \( S \) remains on your left.
        \item \textbf{Special Case:} If \( S \) lies flat in the \( xy \)-plane and \( \mathbf{n} = \mathbf{k} \), Stokes' Theorem reduces to Green's Theorem:
        \[
        \oint_C \mathbf{F} \cdot d\mathbf{r} = \iint_D (\nabla \times \mathbf{F}) \cdot \mathbf{k} \, dA.
        \]
    \end{itemize}

    Stokes' Theorem provides a fundamental relationship between the circulation of \( \mathbf{F} \) along \( C \) and the total rotational effects (curl) of \( \mathbf{F} \) over the surface \( S \).
}

\ntimg{
    Stokes' Theorem allows us to compute a surface integral simply by knowing the values of \( \mathbf{F} \) on the boundary curve \( C \). This means that if we have another oriented surface with the same boundary curve \( C \), then we get exactly the same value for the surface integral. In general, if \( S_1 \) and \( S_2 \) are oriented surfaces with the same oriented boundary curve \( C \) and both satisfy the hypotheses of Stokes' Theorem, then:
    \[
    \int_{S_1} (\nabla \times \mathbf{F}) \cdot d\mathbf{S} = \int_C \mathbf{F} \cdot d\mathbf{r} = \int_{S_2} (\nabla \times \mathbf{F}) \cdot d\mathbf{S}.
    \]

    This fact is useful when it is difficult to integrate over one surface but easy to integrate over the other.
}
\addcontentsline{toc}{section}{Stokes' Theorem}


\section*{16.9 The Divergence Theorem} 
\addcontentsline{toc}{section}{The Divergence Theorem}







\end{document}