\documentclass{report}

\input{preamble}
\input{macros}
\input{letterfonts}

\title{\Huge{Math 120}}
\author{\huge{PSet 4}}
\date{Sep 22 2024}

\begin{document}

\maketitle
\newpage% or \cleardoublepage
% \pdfbookmark[<level>]{<title>}{<dest>}
\pdfbookmark[section]{\contentsname}{toc}
\tableofcontents
\pagebreak

\chapter{}
\section{PSet 4}

\qs{}{
	By finding both $f_{xy} = (f_x)_y$ and $f_{yx} = (f_y)_x$, verify that Clairaut's Theorem holds for the function $f(x,y) = y \arctan(xy)$.
}

\sol{
}

\qs{}{
	Level curves are shown below for a function $f$ 
	\insertpng[0.3]{Prob2.png}
	\begin{enumerate}
		\item[(a)] Determine the signs of $f_x, f_y, f_{xx}, f_{yy}$ and $f_{xy}$ at the point $P$. Explain your reasoning. \textit{You should assume that the undrawn level curves are nicely and evenly distributed between the ones drawn.}
		
		\item[(b)] Mark a point on the contour plot where $f_x = 0$. (You can either mark the point on a screenshot and insert the picture in your homework file, or just make a rough copy of the contour plot by hand.)
	\end{enumerate}
}

\qs{}{
	Use the Chain Rule to find the indicated partial derivatives.
    \begin{enumerate}
        \item[(a)] Compute $\frac{dz}{dt}$ if $z = \tan(y/x)$, $x = e^t$, and $y = 1 - e^{-t}$.
        
        \item[(b)] Compute $\frac{\partial M}{\partial u}$ and $\frac{\partial M}{\partial v}$ at $u = 3$ and $v = -1$ if $M = x e^{y - z^2}$, $x = 2uv$, $y = u - v$, and $z = u + v$.
    \end{enumerate}
}

\sol{\\
	a) 
	\[ \frac{dz}{dt} = \frac{dz}{dx} \cdot \frac{dx}{dt} + \frac{dz}{dy} \cdot \frac{dy}{dt}  \]
	\[ \frac{dz}{dx} = z_{x} = \sec\left(\frac{y}{x}\right)^{2} \cdot \left( \frac{(0)x - (1)(y)}{x^{2}} \right) \Rightarrow \sec\left(\frac{y}{x}\right)^{2} \cdot \left(\frac{-y}{x^{2}} \right) \] 
	\[ \frac{dz}{dy} = z_{y} = \sec\left(\frac{y}{x}\right)^{2} \cdot \left( \frac{(1)x - (0)(y)}{x^{2}} \right) \Rightarrow \sec\left(\frac{y}{x}\right)^{2} \cdot \left(\frac{1}{x} \right)\] 
	\[ \frac{dx}{dt} = e^{t} \quad \frac{dy}{dt}= e^{-t} \]
	\[ \frac{dz}{dt} = \sec\left(\frac{1-e^{-t}}{e^{t}}\right)^{2} \cdot \left(\frac{-(1-e^{t})}{e^{2t}} \right) \cdot e^{t} + \sec\left(\frac{1-e^{-t}}{e^{t}}\right)^{2} \cdot \left(\frac{1}{e^{t}} \right) \cdot e^{-t} \]  
	b)
	\[ \frac{\partial M}{\partial u} = \frac{\partial M}{\partial x} \cdot \frac{dx}{dt} + \frac{\partial M}{\partial y} \cdot \frac{dy}{dt} + \frac{\partial M}{\partial z} \cdot \frac{dz}{dt}\] 
	\[ M_{x} = e^{y-z^{2}} \quad \frac{dx}{du} = 2v \quad \frac{dx}{dv} = 2u\] 
	\[ M_{y} = xe^{y-z^{2}} \quad \frac{dy}{du} = 1 \quad \frac{dy}{dv} = -1\]
	\[ M_{z} = -2zxe^{y-z^{2}} \quad \frac{dz}{du} = 1 \quad \frac{dz}{dt} = 1\]
	\[ \frac{\partial M}{\partial u} = e^{y-z^{2}} (2v) + \left(xe^{y-z^{2}}\right)(1) + \left(-2zxe^{y-z^{2}}\right)(1) \]
	\[ x = 2uv \Rightarrow 2(3)(-1) = - 6 \]     
	\[ y = u - v \Rightarrow (3) - -(1) = 4 \] 
	\[ z = u + v \Rightarrow 3 + (-1) = 2 \] 
	\[ \frac{\partial M}{\partial u} = e^{4-2^{2}} (2(-1)) + \left((-6) e^{4-2^{2}}\right)(1) + \left(-2(2)(-6)e^{4-2^{2}}\right)(1) = 1(-2) + (-6)(1) + (24)(1) = 16 \]
	\[ \frac{\partial M}{\partial v} = \frac{\partial M}{\partial x} \cdot \frac{dx}{dv} + \frac{\partial M}{\partial y} \cdot \frac{dy}{dv} + \frac{\partial M}{\partial z} \cdot \frac{dz}{dv}\] 
	\[ \frac{\partial M}{\partial v} = e^{y-z^{2}} (2u) + \left(xe^{y-z^{2}}\right)(-1) + \left(-2zxe^{y-z^{2}}\right)(1) \]
	\[ \frac{\partial M}{\partial v} = e^{4-2^{2}} (2(3)) + \left((-6) e^{4-2^{2}}\right)(-1) + \left(-2(2)(-6)e^{4-2^{2}}\right)(1) = 1(6) + (-6)(1) + (24)(1) = 36 \]

}

\qs{}{
	Use implicit differentiation to compute $\frac{\partial z}{\partial x}$ and $\frac{\partial z}{\partial y}$ for the surface given by $z^3 + 3xyz + x^2 + y^2 = 0$ at the point $(1, -2, 1)$.
}




\sol{
	\[ \frac{\partial z}{\partial x} \left(z^3 + 3xyz + x^2 + y^2 \right) = 3z^{2}\frac{\partial z}{\partial x} + 3yz + 3xy \frac{\partial z}{\partial x} + 2x = 0   \] 
	\[ 3z^{2}\frac{\partial z}{\partial x} + 3yz + 3xy \frac{\partial z}{\partial x} + 2x = 0  \Rightarrow 3z^{2} \frac{\partial z}{\partial x}  + 3xy \frac{\partial z}{\partial x}  = -3yz - 2x \] 
	\[ 3x^{2} \frac{\partial z}{\partial x}  + 3xy \frac{\partial z}{\partial x}  = -3yz - 2x \Rightarrow \frac{\partial z}{\partial x} \left( 3z^{2} + 3xy \right) = -3yz - 2x  \] 
	\[ \frac{\partial z}{\partial x} \left( 3z^{2} + 3xy \right) = -3yz - 2x  \Rightarrow \frac{\partial z}{\partial x} = \frac{-3yz - 2x }{3z^{2} + 3xy}\] 
	\[ \frac{\partial z}{\partial x} = \frac{-3(-2)(1) - 2(1) }{3(1)^{2} + 3(1)(-2)} = \frac{-4}{3}  \] 
	\[ \frac{\partial z}{\partial y} \left(z^3 + 3xyz + x^2 + y^2 \right) = 3z^{2}\frac{\partial z}{\partial y} + 3xz + 3xy \frac{\partial z}{\partial y} + 2y = 0   \] 
	\[ 3z^{2}\frac{\partial z}{\partial y} + 3xz + 3xy \frac{\partial z}{\partial y} + 2y  = 0  \Rightarrow 3z^{2} \frac{\partial z}{\partial y}  + 3xy \frac{\partial z}{\partial x}  = -3xz - 2y \] 
	\[ 3z^{2} \frac{\partial z}{\partial y}  + 3xy \frac{\partial z}{\partial y}  = -3xz - 2y \Rightarrow \frac{\partial z}{\partial y} \left( 3z^{2} + 3xy \right) = -3xz - 2y  \] 
	\[ \frac{\partial z}{\partial y} \left( 3z^{2} + 3xy \right) = -3yz - 2y  \Rightarrow \frac{\partial z}{\partial x} = \frac{-3xz - 2y }{3z^{2} + 3xy}\] 
	\[  \frac{\partial z}{\partial y} = \frac{-3(1)(1) - 2(-2) }{3(1)^{2} + 3(1)(-2)} = -\frac{1}{3}  \] 
}

\qs{}{
	[(5.) (Stewart problem 14.5.36)] Wheat production $W$ in a given year depends on the average temperature $T$ and annual rainfall $R$. Scientists estimate that the average temperature is rising at a rate of $0.15^\circ \text{C/year}$ and rainfall is decreasing at a rate of $0.1 \text{ cm/year}$. They also estimate that at current production levels, $\partial W/\partial T = -2$ and $\partial W/\partial R = 8$.
    \begin{enumerate}
        \item[(a)] What is the significance of the signs of these partial derivatives?
        
        \item[(b)] Estimate the current rate of change of wheat production $\frac{dW}{dt}$.
    \end{enumerate}
}
\end{document}
