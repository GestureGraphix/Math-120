\documentclass{report}

\input{preamble}
\input{macros}
\input{letterfonts}

\title{\Huge{Math 120}}
\author{\huge{PSet 6}}
\date{Oct 9 2024}

\begin{document}

\maketitle
\newpage% or \cleardoublepage
% \pdfbookmark[<level>]{<title>}{<dest>}
\pdfbookmark[section]{\contentsname}{toc}
\tableofcontents
\pagebreak

\chapter{}
\section{PSet 6}

\qs{}{Find all the (local) maximum and minimum values and saddle points of the function.
\begin{itemize}
    \item[(a)] \( f(x, y) = xy + \frac{1}{x} + \frac{1}{y} \)
    \item[(b)] \( f(x, y) = e^y(x^2 - y^2) \)
\end{itemize}}

\sol{
    \\
    a) 
    \[ f_{x} = y - \frac{1}{x^{2}} \quad f_{y} = x - \frac{1}{y^{2}} \]
    \[ y = \frac{1}{x^{2}} \quad x = \frac{1}{y^{2}} \] 
    \[ \frac{1}{x^{2}} = 0 \quad \frac{1}{y^{2}} = 0 \] 
    \[ x = \pm 1 \quad y = 1 \quad x \geq 0 \quad y = \frac{1}{x^{2}} \quad x = 1 \]
    \[ f_{xx} = \frac{2x}{x^{3}} \quad f_{yy} = \frac{2y}{y^{3}} \quad f_{xy} = 0 \]
    \[ f_{xx}(1,1) = \frac{2(1)}{2^{3}} = 2 \quad f_{yy}(1,1) = \frac{2(1)}{2^{3}} = 2\]
    \[ f_{xx}(1,1) f_{yy}(1,1) - 0^{2} = 4 \]
    (1,1) is a local min because $D > 0$ and $f_{xx} > 0$
    b) 
    \[ f_{x} = 2x^{ey} \quad f_{y} = x^{2}e^{y}-e^{y}y^{2} - 2ye^{y} \]
    \[ 2x^{ey} = 0  \Rightarrow x = 0 \] 
    \[ x^{2}e^{y}-e^{y}y^{2} - 2ye^{y} \Rightarrow 0^{2}e^{y}-e^{y}y^{2} - 2ye^{y}\Rightarrow -e^{y}y^{2} - 2ye^{y} \] 
    \[ -e^{y}y^{2} - 2ye^{y} \Rightarrow 2ye^{y} = y^{2} e^{y} \Rightarrow y = 0 \quad y = -2 \]  
    \[ f_{xx} = 2e^{y} \quad f_{yy} = x^{2}e^{y} - 2ye^{y} - e^{y}y^{2} - 2e^{y} - 2ye^{y} \] 
    \[ D = 2e^{0} \left(0^{2}e^{0} - 2(0)e^{0} - e^{0}(0)^{2} - 2e^{0} - 2(0)e^{0} \right)= (2)(-2) = - 4 \] 
    \[ D = 2e^{-} \left(0^{2}e^{-2} - 2(-2)e^{-2} - e^{-2}(-2)^{2} - 2e^{-2} - 2(-2)e^{-2} \right)=  \frac{16}{e^{4}} \]
    (0,0) is a saddle point becase $D < 0$ and (0,-2) is a lacal min because $D > 0$ and $f_{xx} > 0 $ 

}

\qs{}{Find the absolute maximum and minimum values of the function
\[
    f(x, y) = x + y - xy
\]
on the closed triangular region with vertices \((0,0)\), \((0,2)\), and \((4,0)\).
}

\sol{
    \[ f_{x} = 1 - y = 0 \Rightarrow y = 1 \]
    \[ f_{y} = 1 - x = 0 \Rightarrow x = 1 \]  
}

\qs{}{Find the absolute maximum and minimum values of the function
\[
    f(x, y) = xy^2
\]
on the region \(x^2 + y^2 \leq 3\), \(x \geq 0\), \(y \geq 0\).
}

\sol{
    \[ f_{x} = \frac{\partial}{\partial x} xy^2 = y^2 \]
    \[ f_{y} = \frac{\partial}{\partial y} xy^2 = 2xy \]
    \[ f_{x} = y^{2} = 0 \Rightarrow y = 0 \]
    \[ f_{y} = 2xy = 0 \Rightarrow x = 0, y = 0 \]
    \[ x^2 + y^2 \leq 3 \quad x \geq 0 \quad y \geq 0 \]
    \[ x = \sqrt{3} \cos \theta \]
    \[ y = \sqrt{3} \sin \theta \]
    \[ f(\theta) = \left(\sqrt{3} \cos \theta \right) \left(\sqrt{3} \sin \theta \right)^{2} = 3\sqrt{3} \cos \theta \sin^{2} \theta \]
    \[ \frac{d}{d \theta } = 3 \sqrt{3} \sin \theta \cos \theta (2 \cos \theta \sin \theta ) \]
    \[ 3 \sqrt{3} \sin \theta \cos \theta (2 \cos \theta \sin \theta ) = 0 \]
    \[ \sin \theta = 0 \Rightarrow \theta = 0 \]
    \[ \cos \theta = 0 \Rightarrow \theta = \frac{\pi}{2} \]
    \[ 2 \cos \theta = \sin \theta \Rightarrow \tan \theta = 2 \Rightarrow \theta = \arctan (2) \]
    \[ f(0) = 3\sqrt{3} \cos(0) \sin^{2}(0) = 0 \]              
    \[ f\left(\frac{\pi}{2}\right) = 3\sqrt{3} \cos\left(\frac{\pi}{2}\right) \sin^{2}\left(\frac{\pi}{2}\right) = 0 \] 
    \[ f(\arctan (2)) = 3\sqrt{3} \left(\frac{1}{\sqrt{5}} \right) \left(\frac{2}{\sqrt{5}} \right)^{2} = \frac{24\sqrt{3}}{25}\]       
    max o f$\frac{24\sqrt{3}}{25}$ and min of 0.
}

\qs{}{Find the maximum and minimum values of the function \(f(x, y) = x + 4y\) subject to the constraint
\[
    \sqrt{x} + \sqrt{y} = 3.
\]}

\sol{
    \[ \nabla f(x,y) = \lambda \nabla g(x,y) \]
    \[ \nabla f(x,y) = (1,4) \]
    \[ \nabla g = \left( \frac{\partial }{\partial x } \sqrt{x} + \sqrt{y} - 3, \frac{\partial }{\partial y } \sqrt{x} + \sqrt{y} - 3\right)\]
    \[ \nabla g = \left( \frac{1}{2\sqrt{x}}, \frac{1}{2 \sqrt{y}} \right)\]    
    \[ 1 = \lambda \frac{1}{2\sqrt{x}} \]
    \[ 4 = \lambda \frac{1}{2\sqrt{y}} \] 
    \[ \lambda = 2 \sqrt{x} \]
    \[ 4 = 2 \sqrt{x} \left( \frac{1}{2 \sqrt{y}} \right) = \frac{\sqrt{x}}{\sqrt{y}} \]
    \[ \sqrt{16 y} + \sqrt{y} = 3 \Rightarrow 4 \sqrt{y} + \sqrt{y} = 3 \Rightarrow 5 \sqrt{3} = 3 \]
    \[ \sqrt{y} = \frac{3}{5} \Rightarrow y = \frac{9}{25} \]
    \[ x = 16 \cdot \frac{9}{25} = \frac{144}{25} \]
    \[ f\left( \frac{144}{25}, \frac{9}{25} \right) = \frac{180}{25}\]       
}

\qs{}{Consider the function \( f(x, y) = e^{xy} \) and the constraint \( x^3 + y^3 = 16 \).
\begin{itemize}
    \item[(a)] Use Lagrange multipliers to find the coordinates \( (x, y) \) of any points on the constraint where the function \( f \) could attain a maximum or minimum.
    \item[(b)] For each point you found in part (a), is the point a maximum, a minimum, both or neither? Explain your answer carefully. What are the minimum and maximum values of \( f \) on the constraint? Please explain your answers carefully.
    \item[(c)] The Extreme Value Theorem, which we covered last week, guarantees that under the right circumstances, we are guaranteed to find absolute minima and maxima for a function \( f \) on a certain constraint. Please explain why parts (a) and (b) don’t violate the Extreme Value Theorem.
\end{itemize}}

\sol{
    \\ 
    a) \\
    \[ \nabla f (x,y) = \lambda \nabla g(x,y) \]
    \[ (ye^{yx} = \lambda (3x^{2}, 3y^{2})) \]
    \[ ye^{yx} = \lambda 3x^{2} \]
    \[ xe^{xy} = \lambda 3y^{2} \]
    \[ \frac{ye^{yx}}{xe^{xy}} = \frac{\lambda 3x^{2}}{\lambda 3y^{2}} \]
    \[ \frac{y}{x} = \frac{x^{2}}{y^{2}} \]
    \[ y^{3} = x^{3} \]
    \[ y = x \quad y = - x \]
    \[ x^{3} + x^{3} = 16 \Rightarrow 2x^{3} = 16 \Rightarrow x = 2 \]
    \[ x^{3} + (-x)^{3} = 16 \Rightarrow 0 = 16 \]
    point is (2,2) \\
    b) \\        
}

\qs{}{Use Lagrange multipliers to find the maximum and minimum values of the function \( f(x, y, z) = x^2 y^2 z^2 \) subject to the constraint \( x^2 + y^2 + z^2 = 1 \).
}

\sol{
    \[ \nabla f (x,y,z) = \lambda \nabla g(x,y,z) \]
    \[ \nabla f = \left(2xy^{2}z^{2}, 2yx^{2}z^{2}, 2zx^{2}y^{2}\right) \]
    \[ \nabla g = \left( 2x, 2y, 2z \right) \]
    \[ 2xy^{2}z^{2} = \lambda 2x \]
    \[ 2yx^{2}z^{2} = \lambda 2y \]
    \[ 2zx^{2}y^{2} = \lambda 2z \]   
    \[ y^{2}z^{2} = \lambda \]
    \[ z^{2}x^{2} = \lambda \]
    \[ x^{2}y^{2} = \lambda \]
    \[ x^{2}y^{2} = z^{2}x^{2} = y^{2}z^{2} \]
    \[ x = y = z \]
    \[ x^{2} + y^{2} + z^{2} = 1 \]
    \[ 3x^{2} = 1 \Rightarrow x^{2} = \frac{1}{3} \Rightarrow x = \pm \frac{1}{\sqrt{3}} \]
    \[ y = x = z = \frac{1}{\sqrt{3}} \]
    \[ f\left( \frac{1}{\sqrt{3}}, \frac{1}{\sqrt{3}}, \frac{1}{\sqrt{3}} \right) = \frac{1}{27} \]
    max: $\frac{1}{27}$  \\
    min: 0 
}

\qs{}{Use Lagrange multipliers to find the maximum and minimum values of \( f(x, y, z) = x^2 + y^2 + z^2 \) subject to the constraint \( x^4 + y^4 + z^4 = 1 \).
}

\sol{
    \[ \nabla f (x,y,z) = (2x,2y,2z) \]
    \[ \nabla g(x,y,z) = \left(4x^{3}, 4y^{3}, 4z^{3} \right) \]  
    \[ \nabla f = \lambda \nabla g \]
    \[ 2x = \lambda 4x^{3} \quad zy = \lambda 4y^{3} \quad xy = \lambda 4z^{3} \]
    \[ z = \lambda 4x^{2} \quad z = \lambda 4y^{3} \quad z = \lambda 4z^{3} \]
    \[ \lambda = \frac{1}{2x^{2}} \quad \lambda = \frac{1}{2y^{2}} \quad \lambda = \frac{1}{2z^{2}} \]   
    \[ x^{4} + y^{4} + z^{4} = 1 \quad x^{4} = y^{4} = z^{4} = t \]
    \[ 3t = 1 \quad t = \frac{1}{3} \]   
    \[ z^{4} = \frac{1}{3} \quad z^{2} = \frac{1}{\sqrt{3}} \]
    \[ f(x,y,z) = 3 \cdot \frac{1}{\sqrt{3}} = \sqrt{3} \]
    \[ \text{Case 1: One variabe is 0 } \]
    \[ y^{2} + z^{2} = 1 \quad 2y^{4} = 1 \]
    \[ y^{2} = \frac{1}{\sqrt{2}} \]
    \[ f(x,y,z) = 2 \cdot \frac{1}{\sqrt{2}} = \sqrt{2} \] 
    \[ \text{Case 1: Two variabes are 0} \]
    \[ z^{4} = 1 \]
    \[ z^{2} = 1 \]
    \[ f(x,y,z) = z^{2} = 1 \] 
    Min 1 \\
    Max $\sqrt{3}$   
}

\qs{}{ Find the absolute minimum and maximum values of the function \( f(x, y) = x^2 - (y - 2)^2 \) on the region
\[
D = \{x^2 + y^2 \leq 9 \text{ and } y \geq 0\},
\]
and the points at which those extrema occur.}

\sol{
    \[ \nabla f(x,y) = (2x - 2y + 4) \]
    \[ \nabla g(x,y) = 2x, 2y \]
    \[ \nabla f(x,y) = \lambda \nabla g(x,y) \]
    \[ 2x = \lambda 2x \]
    \[ -2y + 4 = \lambda 2y \]
    \[ x^{2} + y^{2} = 9 \]
    \[ 2x = 0 \quad -2y + 4 = 0 \Rightarrow y = 2 \]
    \[ f(0,2) = 0^{2} - (2-2)^{2} = 0 \]
    \[ \text{Case: } x = 0 \]
    \[ 0^{2} + y^{2} = 9 \Rightarrow y = \pm  \quad y \geq 0 \quad y = 3 \]
    \[ f(0,2) = 0^{2} - (3-2)^{2} = - 1 \]
    \[ \text{Case: } x \neq 0 \]
    \[ 1 = \lambda \]
    \[ -2y + 4 = 2y \Rightarrow 4 = 4y \Rightarrow y = 1 \]  
    \[ f(x,1) = x^{2} + 1^{2} = 9 \Rightarrow x = \pm 2\sqrt{2} \]
    \[ f(0,2) = 0 \quad f(0,3) = -1 \quad f(\pm 2\sqrt{2}, 1) = 7 \]
    Max: 7 at (0,2)  \\
    Min: -1 at (0,3)              

}

\end{document}
