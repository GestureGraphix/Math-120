\documentclass{report}

\input{preamble}
\input{macros}
\input{letterfonts}

\title{\Huge{Math 120}}
\author{\huge{PSet 6}}
\date{Oct 9 2024}

\begin{document}

\maketitle
\newpage% or \cleardoublepage
% \pdfbookmark[<level>]{<title>}{<dest>}
\pdfbookmark[section]{\contentsname}{toc}
\tableofcontents
\pagebreak

\chapter{}
\section{PSet 6}

\qs{}{Find all the (local) maximum and minimum values and saddle points of the function.
\begin{itemize}
    \item[(a)] \( f(x, y) = xy + \frac{1}{x} + \frac{1}{y} \)
    \item[(b)] \( f(x, y) = e^y(x^2 - y^2) \)
\end{itemize}}

\sol{}

\qs{}{Find the absolute maximum and minimum values of the function
\[
    f(x, y) = x + y - xy
\]
on the closed triangular region with vertices \((0,0)\), \((0,2)\), and \((4,0)\).
}

\sol{}

\qs{}{Find the absolute maximum and minimum values of the function
\[
    f(x, y) = xy^2
\]
on the region \(x^2 + y^2 \leq 3\), \(x \geq 0\), \(y \geq 0\).
}

\sol{
    \[ f_{x} = \frac{\partial}{\partial x} xy^2 = y^2 \]
    \[ f_{y} = \frac{\partial}{\partial y} xy^2 = 2xy \]
    \[ f_{x} = y^{2} = 0 \Rightarrow y = 0 \]
    \[ f_{y} = 2xy = 0 \Rightarrow x = 0, y = 0 \]
    \[ x^2 + y^2 \leq 3 \quad x \geq 0 \quad y \geq 0 \]
    \[ x = \sqrt{3} \cos \theta \]
    \[ y = \sqrt{3} \sin \theta \]
    \[ f(\theta) = \left(\sqrt{3} \cos \theta \right) \left(\sqrt{3} \sin \theta \right)^{2} = 3\sqrt{3} \cos \theta \sin^{2} \theta \]
    \[ \frac{d}{d \theta } = 3 \sqrt{3} \sin \theta \cos \theta (2 \cos \theta \sin \theta ) \]
    \[ 3 \sqrt{3} \sin \theta \cos \theta (2 \cos \theta \sin \theta ) = 0 \]
    \[ \sin \theta = 0 \Rightarrow \theta = 0 \]
    \[ \cos \theta = 0 \Rightarrow \theta = \frac{\pi}{2} \]
    \[ 2 \cos \theta = \sin \theta \Rightarrow \tan \theta = 2 \Rightarrow \theta = \arctan (2) \]
    \[ f(0) = 3\sqrt{3} \cos(0) \sin^{2}(0) = 0 \]              
    \[ f\left(\frac{\pi}{2}\right) = 3\sqrt{3} \cos\left(\frac{\pi}{2}\right) \sin^{2}\left(\frac{\pi}{2}\right) = 0 \] 
    \[ f(\arctan (2)) = 3\sqrt{3} \left(\frac{1}{\sqrt{5}} \right) \left(\frac{2}{\sqrt{5}} \right)^{2} = \frac{24\sqrt{3}}{25}\]       
    max o f$\frac{24\sqrt{3}}{25}$ and min of 0.
}

\qs{}{Find the maximum and minimum values of the function \(f(x, y) = x + 4y\) subject to the constraint
\[
    \sqrt{x} + \sqrt{y} = 3.
\]}

\qs{}{Consider the function \( f(x, y) = e^{xy} \) and the constraint \( x^3 + y^3 = 16 \).
\begin{itemize}
    \item[(a)] Use Lagrange multipliers to find the coordinates \( (x, y) \) of any points on the constraint where the function \( f \) could attain a maximum or minimum.
    \item[(b)] For each point you found in part (a), is the point a maximum, a minimum, both or neither? Explain your answer carefully. What are the minimum and maximum values of \( f \) on the constraint? Please explain your answers carefully.
    \item[(c)] The Extreme Value Theorem, which we covered last week, guarantees that under the right circumstances, we are guaranteed to find absolute minima and maxima for a function \( f \) on a certain constraint. Please explain why parts (a) and (b) don’t violate the Extreme Value Theorem.
\end{itemize}}

\sol{}

\qs{}{Use Lagrange multipliers to find the maximum and minimum values of the function \( f(x, y, z) = x^2 y^2 z^2 \) subject to the constraint \( x^2 + y^2 + z^2 = 1 \).
}

\sol{}

\qs{}{Use Lagrange multipliers to find the maximum and minimum values of \( f(x, y, z) = x^2 + y^2 + z^2 \) subject to the constraint \( x^4 + y^4 + z^4 = 1 \).
}

\sol{}

\qs{}{ Find the absolute minimum and maximum values of the function \( f(x, y) = x^2 - (y - 2)^2 \) on the region
\[
D = \{x^2 + y^2 \leq 9 \text{ and } y \geq 0\},
\]
and the points at which those extrema occur.}

\sol{} 

\end{document}
