\documentclass{report}

\input{preamble}
\input{macros}
\input{letterfonts}

\title{\Huge{Math 120}}
\author{\huge{PSet 2}}
\date{Sep 12 2024}

\begin{document}

\maketitle
\newpage% or \cleardoublepage
% \pdfbookmark[<level>]{<title>}{<dest>}
\pdfbookmark[section]{\contentsname}{toc}
\tableofcontents
\pagebreak

\chapter{}
\section{PSet 2}

\qs{}{Consider the line $L_1$ given by $x + 2y = 7$ and the line $L_2$ given by $5x - y = 2$.
\begin{enumerate}
	\item There are two unit vectors that are parallel to $L_1$. What are they?
	\item There are two unit vectors that are perpendicular to $L_1$. What are they?
	\item Find the acute angle between the lines $L_1$ and $L_2$. First find an exact expression and then approximate to the nearest degree.
\end{enumerate}}

\sol{
	\\
	a)
	\[ L_{1} = x + 2y = 7 \] 
	\[ y = - \frac{1}{2} x + \frac{7}{2} \] 
	\[ v_{1} = (1, m_{1}) = (1, - \frac{1}{2}) \]
	\[ |v_{1}| = \sqrt{1^{2} + \left( -\frac{1}{2}\right)^{2}} = \sqrt{1 + \frac{1}{4}} = \sqrt{\frac{5}{4}} = \frac{\sqrt{5}}{2} \]
	\[ u_{1} = \frac{1}{\frac{\sqrt{5}}{2}} \left( 1, - \frac{1}{2}\right) = \frac{2}{\sqrt{5}} \left( 1, -\frac{1}{2} \right) = \left( \frac{\sqrt{2}}{5},- \frac{1}{\sqrt{5}}\right) \] 
	\[ -u_{1} = \left( - \frac{\sqrt{2}}{5}, \frac{1}{\sqrt{5}}\right) \] 

	b) 
	\[ \text{slope of line perpendicular to }L_{1}: -\frac{1}{-\frac{1}{2}} = 2 \] 
	\[ \langle 1, 2 \rangle \] 
	\[ |n| = \sqrt{1^{2} + 2^{2}} = \sqrt{5} \]
	\[ u_{1} = \left\langle \frac{1}{\sqrt{5}}, \frac{2}{\sqrt{5}} \right\rangle\]  
	\[ u_{2} = - u_{1} = \left\langle -\frac{1}{\sqrt{5}}, -\frac{2}{\sqrt{5}} \right\rangle\] 
	
	c)
	\[ \vec{v_{1}} = \langle 1, -2 \rangle \]
	\[ \vec{v_{2}} = \langle 5, 1 \rangle \]
	\[ \cos(\theta) = \frac{v_{1} \cdot v_{2}}{|v_{1}| |v_{2}|}\] 
	\[ \cos(\theta) = \frac{-2 + 5}{\sqrt{5}\sqrt{26}} = \frac{3}{\sqrt{130}} \]
	\[ \theta = \arccos \left( \frac{3}{\sqrt{130}}\right)\]  	
}



\qs{}{
	Find all values of $x$ such that the angle between the vectors $\langle 1, -1, 0 \rangle$ and $\langle 2, x, 1 \rangle$ is $\frac{\pi}{3}$.
}

\sol{
	\[ v_{1} = \langle 1, -1, 0 \rangle \]
	\[ v_{2} = \langle 2, x, 1, \rangle \]
	\[ \cos(\frac{\pi}{2}) = \frac{1}{2} \]
	\[ \frac{1}{2} = \frac{v_{1} \cdot v_{2}}{|v_{1}||v_{2}|} = \frac{2 - x}{(\sqrt{2})\sqrt{5 + x^{2}}}\]    
	\[ 4 - 2x = \sqrt{10 + 2x^{2}}\]
	\[ 10 + 2x^{2} = 16 - 16x + 4x^{2} \]
	\[ -2x^{2} + 16 x - 6 = 0\]
	\[ x^{2} - 8x + 3 = 0 \] 
	\[ x = \frac{8 \pm \sqrt{(-8)^{2} - 4(1)(3)}}{2 \cdot 1}\]
	\[ x = \frac{8 \pm \sqrt{52}}{2} \]
	\[ x = 4 - \sqrt{13} \]      
}

\qs{}{
	Find the scalar and vector projections of $\vec{b} = \hat{i} + \hat{j}$ onto $\vec{a} = -\hat{i} + 3\hat{j}$, and illustrate your answers with a sketch.
}

\sol{
	\\
	Scalar Projection: 
	\[ \vec{b} = \hat{\imath} + \hat{\jmath} \] 
	\[ \vec{a} = \hat{\imath} + 3 \hat{\jmath} \]
	\[ \text{comp}_{a}\mathbf{b} = \frac{a \cdot b}{|a|} \]
	\[ \frac{a \cdot b}{|a|} = \frac{2}{\sqrt{10}} = \frac{2\sqrt{10}}{10} = \frac{\sqrt{10}}{5} \]

	\begin{center}
		\begin{tikzpicture}[scale=1]
			% Draw axes
			\draw[->] (-2,0) -- (2,0) node[anchor=north west] {$x$};
			\draw[->] (0,-1) -- (0,4) node[anchor=south east] {$y$};
			
			% Draw vector a
			\draw[->, thick, blue] (0,0) -- (-1,3) node[anchor=east] {$\vec{a}$};
			
			% Draw vector b
			\draw[->, thick, red] (0,0) -- (1,1) node[anchor=south west] {$\vec{b}$};
			
			% Draw projection of b onto a
			\draw[->, thick, green] (0,0) -- (-0.2,0.6) node[anchor=north] {$\text{proj}_{\vec{a}} \vec{b}$};
			
			% Dashed line for scalar projection
			\draw[dashed, gray] (1,1) -- (-0.2,0.6);
			
			% Labels for angles
			\draw (0.2, 0.2) node[anchor=north west] {$\theta$};
			
		\end{tikzpicture}
	\end{center}

	Vector Projection: 
	\[ \text{proj}_{a}\mathbf{b} = \left(\frac{a \cdot b}{|a|^{2}} \right) a \]
	\[ \left(\frac{a \cdot b}{|a|^{2}} \right) a  = \frac{2}{\sqrt{10}^{2}} a = \frac{2}{10} a = \frac{1}{5} a \]
	\[ \frac{1}{5} a = \frac{1}{5} ( - \hat{\imath}, 3 \hat{\jmath}) = \langle -\frac{1}{5} \hat{\imath}, \frac{3}{5} \hat{\jmath}\rangle \] 


}

\qs{}{
	Find two vectors of length 2 that are orthogonal to both $\vec{v} = \langle 2, 4, 4 \rangle$ and $\vec{w} = \langle 1, -1, -3 \rangle$.
}
\sol{
	\[ v \times w = \langle 4(-3) - 4(-1), 4(1) - 2(-3), 2(-1) - 4(1)\rangle  = \langle -8 , 10 , - 6 \rangle \] 
	\[ |u| = \sqrt{(-8)^{2} + 10^{2} + (-6)^{2}} = \sqrt{200}\]
	\[ 2 = x \cdot \sqrt{200} \]
	\[ x = \frac{2}{\sqrt{200}} = \frac{1}{5\sqrt{2}}\]
	\[ u_{1} = \frac{1}{5 \sqrt{2}} \langle -8, 10, -6 \rangle \]
	\[ u_{2} = -\frac{1}{5 \sqrt{2}} \langle -8, 10, -6 \rangle \]  
}


\qs{}{
	Let $\vec{a} = \langle 3, 1, 0 \rangle$. Find all vectors $\vec{b} = \langle b_1, b_2, b_3 \rangle$ such that $\vec{a} \times \vec{b}$ is parallel to the $z$-axis and pointing in the positive $z$ direction. Illustrate with a sketch, in which all vectors are drawn as position vectors, i.e., with the tail at the origin.
}

\sol{
	\[ a \times b = \langle 0, 0, c \rangle \]
	\[ \langle 3, 1, 0 \rangle \times \langle b_{1}, b_{2}, b_{3} \rangle = \langle 0, 0, c \rangle \]  
	\[ \langle 3, 1, 0 \rangle \times \langle b_{1}, b_{2}, b_{3} \rangle = \langle 1(b_{3}) - 0(b_{2}), 0(b_{1}) - 3(b_{3}), 3(b_{2}) - 1(b_{1}) \rangle \] 
	\[ \langle 3, 1, 0 \rangle \times \langle b_{1}, b_{2}, b_{3} \rangle = \langle b_{3}, - 3b_{3}, 3b_{2} - b_{1} \rangle \]
	\[ 3b_{2} - b_{1} > 0 \]
	\[ b_{3} = 0 \]
	\[ -3b_{3} = 0\] 	
	All vectors in the form of $\langle b_{1}, b_{2}, 0 \rangle$ where $3b_{2} - b_{1} > 0$
}

\qs{}{
	Consider the four points in $\mathbb{R}^3$, $K(1, 2, 3)$, $L(1, 3, 6)$, $M(3, 8, 6)$, and $N(3, 7, 3)$.
    \begin{enumerate}
        \item Show that the vectors $\overrightarrow{KL}$, $\overrightarrow{KM}$, and $\overrightarrow{KN}$ are coplanar. Explain why this means that $K$, $L$, $M$, and $N$ all lie in the same plane.
        \item From part (a), we know that $K$, $L$, $M$, and $N$ are the vertices of a quadrilateral. Explain how you can tell that this quadrilateral is actually a parallelogram.
        \item Find the area of the parallelogram with vertices $K$, $L$, $M$, and $N$.
    \end{enumerate}
}

\sol{
	a)
	\[ \ray{KL} = \langle 1 - 1, 3 - 2, 6 - 3 \rangle = \langle 0, 1, 3 \rangle \] 
	\[ \ray{KM} = \langle 3 - 1, 8 - 2, 6 - 3 \rangle = \langle 2, 6, 3 \rangle \] 
	\[ \ray{KN} = \langle 3 - 1, 7 - 2, 3 - 3 \rangle = \langle 2, 5, 0 \rangle \] 
	\[ \ray{KL} \cdot (\ray{KM} \times \ray{KN}) = \ray{KL} \times \langle 6(0) - 3(5), 3(2) - 2(0), 2(5) - 6(2) \rangle \] 
	\[ \ray{KL} \cdot (\ray{KM} \times \ray{KN}) = \ray{KL} \times \langle -15, 6, -2\rangle \] 
	\[ \ray{KL} \cdot (\ray{KM} \times \ray{KN}) = \ray{KL} \times \langle -15, 6, -2\rangle  = 0(15) + 6(1) + 3(-2) = 0 \]
	\[ |\ray{KL} \cdot (\ray{KM} \times \ray{KN})| = 0 \]
	They are coplanar because the volume determined by the vectors is 0, therefore they must lie on the same plane. \\ 
	b).   
	\[ \ray{KL} = \langle 0, 1, 3 \rangle \] 
	\[ \ray{MN} = \langle 3 - 3, 7 - 8, 3 - 6 \rangle = \langle 0, -1, -3 \rangle \]
	\[ \ray{KL} = -\ray{MN} \] 
	Since $ \ray{KL} = -\ray{MN}$ these two sides are parallel. 
	\[ \ray{KL} = \langle 2, 6, 3 \rangle \] 
	\[ \ray{LN} = \langle 3 - 1, 7 - 3, 3 - 6 \rangle = \langle 2, 4, -3 \rangle \]
	Although $ \ray{KM} = -\ray{LN}$ are not negatives of each other or equal in magnitude they form the other side of the parallelogram \\
	c) 
	\[ \ray{KL} \times \ray{KM} = \langle 1(3) - 3(6), 3(2) - 0(3), 0(6) - (1)(2) \rangle = \langle -15, 6, -2\rangle \] 
	\[ \sqrt{(-15)^{2} + 6^{2} + (-2)^{2}} = \sqrt{265}\] 
}

\qs{}{
	Find the vector equation and parametric equations for the line through the point $(1, 2, -2)$ parallel to the line $x = t - 2$, $y = -2t + 1$, $z = 3$.	
}

\sol{
	\[ x = t -2 \quad y = -2t + 1 \quad z = 3 \]
	\[ \vec{d} = \langle 1, -2, 0 \rangle \]
	\[ \vec{r}(t) = \vec{r}_{0} + t \vec{d} \] 
	\[ \vec{r}(t) = \langle 1, 2, -2 \rangle + t \langle 1, -2, 0 \rangle \]   
	\[ x(t) = 1 + t \quad y(t) = 2 - 2t \quad z(t) = - 2 \]
	Vector Equation: $\vec{r}(t) = \langle 1, 2, -2 \rangle + t \langle 1, -2, 0 \rangle $ \\
	Parametric Equation: $x(t) = 1 + t$,  $y(t) = 2 - 2t$, $z(t) = - 2$
}


\qs{}{
	Consider the lines $L_1 : x = t + 3$, $y = 2t - 1$, $z = -t$, and $L_2 : x = t - 1$, $y = t - 4$, $z = -t + 4$. Determine whether the $L_1$ and $L_2$ are parallel, skew, or intersecting. If they intersect, find the point of intersection.
}

\sol{ 
	\[ d_{1} = \langle 1, 2, -1 \rangle \quad  \langle d_{2} = 1, 1, -1 \rangle \] 
	\[ \frac{1}{1} \neq \frac{2}{1} \neq \frac{-1}{-1} \]
	\[ t_{1} + 3 = t_{2} - 1 \quad 2t_{1} - 1 = t_{2} - 4 \quad -t_{1} = -t_{2} + 4 \]
	\[ t_{1} + 3 = t_{2} - 1 \Rightarrow t_{1} - t_{2} = -4 \]
	\[ -t_{1} = -t_{2} + 4 \Rightarrow t_{1} = t_{2} - 4 \]
	\[ (t_{2} - 4) - t_{2} = - 4 \Rightarrow -4 = - 4 \]
	\[ 2t_{1} - 1= t_{2} - 4 \]
	\[ 2(t_{2} - 4) - 1 = t_{2} - 4 \]
	\[ 2t_{2} - 9 = t_{2} - 4 \] 
	\[ t_{2} = 5 \]
	\[ t_{1} = t_{2} - 4 = 5 - 4 = 1 \]
	\[ x_{1} = 1 + 3 = 4 \quad y_{1} = 2(1) - 1 = 1 \quad z_{1} = -1 \]
	Point of Intersection: (4, 1, - 1)         
}

\qs{}{
	Consider the planes $x + y + 2z = 4$ and $2x - y - 2z = 1$.
    \begin{enumerate}
        \item Find a vector equation for the line of intersection of the planes.
        \item Find the angle between the planes. First find an exact expression and then approximate to the nearest degree.
    \end{enumerate}
}

\sol{
	\\
	a) 
	\[ n_{1} = \langle 1, 1, 2\rangle \quad n_{2} = \langle 2, -1, -2 \rangle \]
	\[ d = n_{1} \times n_{2} = \langle 1(-2) - 2(-1), 2(2) - 1(-2), 1 (-1) - 1(2) \rangle = \langle 0, 6, -3 \rangle \]
	\[ (x + y + 2z) + (2x - y - 2z) = 4 + 1 \]
	\[ 3x = 5 \Rightarrow x = \frac{5}{3}\]
	\[ \frac{5}{3} + y + 2z = 4 = \frac{7}{3}\]
	\[ y + 2z = \frac{7}{3}\]
	\[ 2\left(\frac{5}{3}\right) - y - 2z = 1\]
	\[ -y - 2z = - \frac{7}{3}\]
	\[ y + 2z = \frac{7}{3}\]
	\[ y + 0 = \frac{7}{3}\]
	\[ \left( \frac{5}{3}, \frac{7}{3}, 0\right)\] 
	\[ \vec{r}(t) = \left( \frac{5}{3}, \frac{7}{3}, 0\right) + t \langle 0, 6, -3 \rangle \]           
	b) 
	\[ n_{1} = \langle 1, 1, 2\rangle \quad n_{2} = \langle 2, -1, -2 \rangle \]
	\[ \cos(\theta) = \frac{n_{1} \cdot n_{2}}{|n_{1}||n_{2}|} = \frac{1(2) + 1(-1) + 2(-2)}{\sqrt{1^{2}+1^{2} + 2^{2}}\sqrt{2^{2}+(-1)^{2} + (-2)^{2}}} = \frac{-3}{\sqrt{6}\sqrt{9}} = -\frac{1}{\sqrt{6}}\] 
	\[ \theta = 180 - \arccos\left(-\frac{1}{\sqrt{6}}\right) \approx 180 - 114 = 66\] 
}

\qs{}{
	Let $P$ be the plane $x + y + 2z = 1$ and let $A$ be the point $(1,1,1)$.
	\begin{enumerate}
		\item[(a)] Find an equation of the plane through point $A$ parallel to plane $P$.
		\item[(b)] Find a vector equation for the line through the point $A$ which is perpendicular to the plane $P$. Call this line $L$.
		\item[(c)] Find the point of intersection of the line $L$ (from part (b)) and the plane $P$.
		\item[(d)] Find the point on the plane $P$ closest to the point $A$, and then find the shortest distance from the point $A$ to the plane $P$.
	\end{enumerate}
}

\sol{
	\\
	a)
	\[ P: x + y + 2z = 1 \quad n_{1} = \langle 1, 1, 2 \rangle \]
	\[ n_{2} = t \langle 1, 1, 2 \rangle \]
	\[ P_{2}: tx + ty + 2tz = d\] 
	\[ t1 + t1 + 2t(1) = d \Rightarrow t + t + 2t = d \Rightarrow d = 4t\]   
	\[ tx + ty + 2tz = 4t \Rightarrow x + y + 2z = 4\] 
	b) 
	\[ n = (1,1,2)\]
	\[ r(t) = (1,1,1) + t(1,1,2)\]
	c)
	\[ P: x + y + 2z = 1 \] 
	\[ L: x = 1 + t \quad y = 1 + t \quad z = 1 + 2t \]   
	\[ (1 + t) + (1 + t) + 2(1 + 2t) = 1 \]
	\[ 2 + 2t + 2 + 4t = 1 \Rightarrow 6t + 4 = 1 \] 
	\[ 6t = -3 \Rightarrow t = -\frac{1}{2} \]
	\[ x: -\frac{1}{2} + 1 \quad y: -\frac{1}{2} + 1 \quad z: 1 + 2\left(-\frac{1}{2}\right)\]
	Point: $\left(\frac{1}{2}, \frac{1}{2}, 0\right)$ \\
	d) 
	Point: $\left(\frac{1}{2}, \frac{1}{2}, 0\right)$ because it is when the line through point A is perpendicular to the plane P\\
	\[ d = \sqrt{\left(1 - \frac{1}{2} \right) + \left(1 - \frac{1}{2}\right)^{2} + (1 - 0)^{2} } = \sqrt{\frac{1}{4} + \frac{1}{4} + 1} = \sqrt{\frac{3}{2}}\] 
}


\end{document}
