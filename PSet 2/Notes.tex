\documentclass{report}

\input{preamble}
\input{macros}
\input{letterfonts}

\title{\Huge{Math 120}}
\author{\huge{PSet 2}}
\date{Sep 12 2024}

\begin{document}

\maketitle
\newpage% or \cleardoublepage
% \pdfbookmark[<level>]{<title>}{<dest>}
\pdfbookmark[section]{\contentsname}{toc}
\tableofcontents
\pagebreak

\chapter{}
\section{PSet 2}

\qs{}{Consider the line $L_1$ given by $x + 2y = 7$ and the line $L_2$ given by $5x - y = 2$.
\begin{enumerate}
	\item There are two unit vectors that are parallel to $L_1$. What are they?
	\item There are two unit vectors that are perpendicular to $L_1$. What are they?
	\item Find the acute angle between the lines $L_1$ and $L_2$. First find an exact expression and then approximate to the nearest degree.
\end{enumerate}}



\qs{}{
	Find all values of $x$ such that the angle between the vectors $\langle 1, -1, 0 \rangle$ and $\langle 2, x, 1 \rangle$ is $\frac{\pi}{3}$.
}

\qs{}{
	Find the scalar and vector projections of $\vec{b} = \hat{i} + \hat{j}$ onto $\vec{a} = -\hat{i} + 3\hat{j}$, and illustrate your answers with a sketch.
}

\qs{}{
	Find two vectors of length 2 that are orthogonal to both $\vec{v} = \langle 2, 4, 4 \rangle$ and $\vec{w} = \langle 1, -1, -3 \rangle$.
}

\qs{}{
	Let $\vec{a} = \langle 3, 1, 0 \rangle$. Find all vectors $\vec{b} = \langle b_1, b_2, b_3 \rangle$ such that $\vec{a} \times \vec{b}$ is parallel to the $z$-axis and pointing in the positive $z$ direction. Illustrate with a sketch, in which all vectors are drawn as position vectors, i.e., with the tail at the origin.
}

\qs{}{
	Consider the four points in $\mathbb{R}^3$, $K(1, 2, 3)$, $L(1, 3, 6)$, $M(3, 8, 6)$, and $N(3, 7, 3)$.
    \begin{enumerate}
        \item Show that the vectors $\overrightarrow{KL}$, $\overrightarrow{KM}$, and $\overrightarrow{KN}$ are coplanar. Explain why this means that $K$, $L$, $M$, and $N$ all lie in the same plane.
        \item From part (a), we know that $K$, $L$, $M$, and $N$ are the vertices of a quadrilateral. Explain how you can tell that this quadrilateral is actually a parallelogram.
        \item Find the area of the parallelogram with vertices $K$, $L$, $M$, and $N$.
    \end{enumerate}
}

\qs{}{
	Find the vector equation and parametric equations for the line through the point $(1, 2, -2)$ parallel to the line $x = t - 2$, $y = -2t + 1$, $z = 3$.	
}



\qs{}{
	Consider the lines $L_1 : x = t + 3$, $y = 2t - 1$, $z = -t$, and $L_2 : x = t - 1$, $y = t - 4$, $z = -t + 4$. Determine whether the $L_1$ and $L_2$ are parallel, skew, or intersecting. If they intersect, find the point of intersection.
}

\qs{}{
	Consider the planes $x + y + 2z = 4$ and $2x - y - 2z = 1$.
    \begin{enumerate}
        \item Find a vector equation for the line of intersection of the planes.
        \item Find the angle between the planes. First find an exact expression and then approximate to the nearest degree.
    \end{enumerate}
}

\qs{}{
	Let $P$ be the plane $x + y + 2z = 1$ and let $A$ be the point $(1,1,1)$.
	\begin{enumerate}
		\item[(a)] Find an equation of the plane through point $A$ parallel to plane $P$.
		\item[(b)] Find a vector equation for the line through the point $A$ which is perpendicular to the plane $P$. Call this line $L$.
		\item[(c)] Find the point of intersection of the line $L$ (from part (b)) and the plane $P$.
		\item[(d)] Find the point on the plane $P$ closest to the point $A$, and then find the shortest distance from the point $A$ to the plane $P$.
	\end{enumerate}
}


\end{document}
