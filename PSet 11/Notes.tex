\documentclass{report}

\input{preamble}
\input{macros}
\input{letterfonts}

\title{\Huge{Math 120}}
\author{\huge{PSet 11}}
\date{Nov 21 2024}

\begin{document}

\maketitle
\newpage% or \cleardoublepage
% \pdfbookmark[<level>]{<title>}{<dest>}
\pdfbookmark[section]{\contentsname}{toc}
\tableofcontents
\pagebreak

\chapter{}
\section{PSet 10}

\qs{}{ 
    Find the surface area of the part of the paraboloid  $x = 3y^2 + 3z^2 $ satisfying  $x \leq 3$.
}

\sol{
    \[
    y = r \cos \theta, \quad z = r \sin \theta, \quad x = 3r^2
    \]

    \[
    A = \iint_{\text{Domain}} \left\| \frac{\partial \mathbf{S}}{\partial r} \times \frac{\partial \mathbf{S}}{\partial \theta} \right\| dr d\theta
    \]

    \[
    \mathbf{S}(r, \theta) = (3r^2, r \cos \theta, r \sin \theta)
    \]

    \[
    \frac{\partial \mathbf{S}}{\partial r} = (6r, \cos \theta, \sin \theta), \quad \frac{\partial \mathbf{S}}{\partial \theta} = (0, -r \sin \theta, r \cos \theta)
    \]

    \[
    \frac{\partial \mathbf{S}}{\partial r} \times \frac{\partial \mathbf{S}}{\partial \theta} = (r, -6r^2 \cos \theta, -6r^2 \sin \theta)
    \]

    \[
    \left\| \frac{\partial \mathbf{S}}{\partial r} \times \frac{\partial \mathbf{S}}{\partial \theta} \right\| = r \sqrt{1 + 36r^2}
    \]

    \[
    A = \int_{0}^{2\pi} \int_{0}^{1} r \sqrt{1 + 36r^2} dr d\theta
    \]

    \[
    u = 1 + 36r^2, \quad du = 72r dr, \quad r dr = \frac{du}{72}
    \]

    \[
    \int_{r=0}^{r=1} r \sqrt{1 + 36r^2} dr = \frac{1}{72} \int_{u=1}^{u=37} \sqrt{u} du = \frac{1}{72} \left[ \frac{2}{3} u^{3/2} \right]_{u=1}^{u=37}
    \]

    \[
    \frac{1}{72} \left[ \frac{2}{3} u^{3/2} \right]_{u=1}^{u=37} = \frac{1}{108} \left( 37^{3/2} - 1 \right)
    \]

    \[
    A = \frac{\pi}{54} \left( 37^{3/2} - 1 \right)
    \]

    \[
    37^{3/2} = \sqrt{37^3} = 37 \sqrt{37}
    \]

    \[
    A = \frac{\pi}{54} \left( 37 \sqrt{37} - 1 \right)
    \]

}

\qs{}{
    In this problem, we'll find the surface area of the part of the cylinder \(x^2 + y^2 = 9\) that lies between the planes \(x + y + z = -6\) and \(x + y + z = 5\).
    \begin{enumerate}
        \item[(a)] Find a parametrization \(\vec{r}_1(u)\) of the curve \(C_1\) of intersection of the plane \(x + y + z = 5\) with the cylinder \(x^2 + y^2 = 9\). Also find a parametrization \(\vec{r}_2(u)\) of the curve \(C_2\) of intersection of the plane \(x + y + z = -6\) with the cylinder.
        
        \item[(b)] Write down a parametrization \(\vec{s}(u, v)\) of the part of the cylinder that lies between the two planes. The curves \(C_1\) and \(C_2\) should be two grid curves of the parametrization, and the bounds on the parameters should be of the form \(a \leq v \leq b\) and \(c \leq u \leq d\) for constants \(a\), \(b\), \(c\), and \(d\).
        
        \item[(c)] Use the parametrization you found to calculate the surface area of the part of the cylinder that lies between the two planes.
        
        \item[(d)] Does your answer from (c) make sense? Give an intuitive geometric reason that the surface area you found is the same as the surface area of a cylinder of radius 3 and height 11.
    \end{enumerate}
}

\sol{
    \begin{enumerate}
        \item[(a)]
        \[
        x = 3 \cos u, \quad y = 3 \sin u
        \]
        \[
        z = 5 - x - y = 5 - (3 \cos u + 3 \sin u)
        \]
        \[
        \vec{r}_1(u) = \left( 3 \cos u, \ 3 \sin u, \ 5 - 3 \cos u - 3 \sin u \right)
        \]
        \[
        z = -6 - x - y = -6 - (3 \cos u + 3 \sin u)
        \]
        \[
        \vec{r}_2(u) = \left( 3 \cos u, \ 3 \sin u, \ -6 - 3 \cos u - 3 \sin u \right)
        \]
        \item[(b)]
        \[
        \vec{s}(u, v) = \left( 3 \cos u, \ 3 \sin u, \ v - 3 \cos u - 3 \sin u \right)
        \]
        \[
        0 \leq u \leq 2\pi, \quad -6 \leq v \leq 5
        \]
        \item[(c)]
        \[
        \frac{\partial \vec{s}}{\partial u} = \left( -3 \sin u, \ 3 \cos u, \ 3 \sin u - 3 \cos u \right)
        \]
        \[
        \frac{\partial \vec{s}}{\partial v} = \left( 0, \ 0, \ 1 \right)
        \]
        \[
        \frac{\partial \vec{s}}{\partial u} \times \frac{\partial \vec{s}}{\partial v} = \left( 3 \cos u, \ 3 \sin u, \ 0 \right)
        \]
        \[
        \left\| \frac{\partial \vec{s}}{\partial u} \times \frac{\partial \vec{s}}{\partial v} \right\| = \sqrt{(3 \cos u)^2 + (3 \sin u)^2} = 3
        \]
        \[
        A = \int_{v=-6}^{5} \int_{u=0}^{2\pi} 3 \, du \, dv = 3 \cdot \left( \int_{u=0}^{2\pi} du \right) \cdot \left( \int_{v=-6}^{5} dv \right) = 3 \cdot 2\pi \cdot 11 = 66\pi
        \]
        \item[(d)]
        \[
        A = 2\pi r h = 2\pi \times 3 \times 11 = 66\pi
        \]
    \end{enumerate}    
}

\newpage 

\qs{}{
    Evaluate the surface integral
    \[
    \iint_S (x^2 + y^2 + z^2) \, dS
    \]
    where $S$ is the surface of the solid cylinder defined by the inequalities $x^2 + z^2 \leq 1$ and $0 \leq y \leq 5$. Note that $S$ consists of a hollow cylinder and two disks. 
}

\sol{
    \[
    I = \iint_S (x^2 + y^2 + z^2) \, dS
    \]

    \[
    x^2 + z^2 \leq 1, \quad 0 \leq y \leq 5
    \]

    \[
    S = S_{\text{cyl}} \cup S_{\text{top}} \cup S_{\text{bot}}
    \]

    \[
    \text{For } S_{\text{cyl}}, \quad x = \cos\theta, \quad z = \sin\theta, \quad y = y, \quad \theta \in [0, 2\pi), \quad y \in [0, 5]
    \]

    \[
    \vec{r}(\theta, y) = (\cos\theta, y, \sin\theta)
    \]

    \[
    \vec{r}_\theta = (-\sin\theta, 0, \cos\theta), \quad \vec{r}_y = (0, 1, 0)
    \]

    \[
    \vec{r}_\theta \times \vec{r}_y = (-\cos\theta, 0, -\sin\theta), \quad |\vec{r}_\theta \times \vec{r}_y| = 1
    \]

    \[
    dS_{\text{cyl}} = d\theta \, dy
    \]

    \[
    x^2 + y^2 + z^2 = \cos^2\theta + y^2 + \sin^2\theta = 1 + y^2
    \]

    \[
    I_{\text{cyl}} = \int_0^{2\pi} \int_0^5 (1 + y^2) \, dy \, d\theta
    \]

    \[
    \int_0^5 (1 + y^2) \, dy = \int_0^5 1 \, dy + \int_0^5 y^2 \, dy = [y]_0^5 + \left[\frac{y^3}{3}\right]_0^5 = 5 + \frac{125}{3} = \frac{140}{3}
    \]

    \[
    I_{\text{cyl}} = \int_0^{2\pi} \frac{140}{3} \, d\theta = \frac{140}{3} \cdot 2\pi = \frac{280\pi}{3}
    \]

    \[
    \text{For } S_{\text{top}}, \quad x = r\cos\theta, \quad z = r\sin\theta, \quad y = 5, \quad r \in [0, 1], \quad \theta \in [0, 2\pi)
    \]

    \[
    dS_{\text{top}} = r \, dr \, d\theta
    \]

    \[
    x^2 + y^2 + z^2 = r^2 + 25
    \]

    \[
    I_{\text{top}} = \int_0^{2\pi} \int_0^1 (r^2 + 25) r \, dr \, d\theta
    \]

    \[
    \int_0^1 (r^2 + 25)r \, dr = \int_0^1 (r^3 + 25r) \, dr = \left[\frac{r^4}{4} + \frac{25r^2}{2}\right]_0^1 = \frac{1}{4} + \frac{25}{2} = \frac{51}{4}
    \]

    \[
    I_{\text{top}} = \int_0^{2\pi} \frac{51}{4} \, d\theta = \frac{51}{4} \cdot 2\pi = \frac{51\pi}{2}
    \]

    \[
    \text{For } S_{\text{bot}}, \quad x = r\cos\theta, \quad z = r\sin\theta, \quad y = 0
    \]

    \[
    x^2 + y^2 + z^2 = r^2
    \]

    \[
    dS_{\text{bot}} = r \, dr \, d\theta
    \]

    \[
    I_{\text{bot}} = \int_0^{2\pi} \int_0^1 r^3 \, dr \, d\theta
    \]

    \[
    \int_0^1 r^3 \, dr = \left[\frac{r^4}{4}\right]_0^1 = \frac{1}{4}
    \]

    \[
    I_{\text{bot}} = \int_0^{2\pi} \frac{1}{4} \, d\theta = \frac{1}{4} \cdot 2\pi = \frac{\pi}{2}
    \]

    \[
    I = I_{\text{cyl}} + I_{\text{top}} + I_{\text{bot}} = \frac{280\pi}{3} + \frac{51\pi}{2} + \frac{\pi}{2}
    \]

    \[
    I = \frac{560\pi}{6} + \frac{153\pi}{6} + \frac{3\pi}{6} = \frac{716\pi}{6} = \frac{358\pi}{3}
    \]

    \[
    \boxed{\frac{358\pi}{3}}
    \]

}

\qs{}{
    Find the mass of a thin funnel in the shape of a cone $z = \sqrt{x^2 + y^2}$, $1 \leq z \leq 4$, if its density function is $\rho(x, y, z) = z + 2$.
}

\sol{
    \[
    z = \sqrt{x^2 + y^2}, \quad 1 \leq z \leq 4
    \]

    \[
    \rho(x, y, z) = z + 2
    \]

    \[
    \begin{cases}
    x = z \cos \theta \\
    y = z \sin \theta \\
    z = z
    \end{cases}
    \]

    \[
    \mathbf{r}(\theta, z) = (z \cos \theta, z \sin \theta, z)
    \]

    \[
    \mathbf{r}_\theta = \left( -z \sin \theta, z \cos \theta, 0 \right)
    \]

    \[
    \mathbf{r}_z = \left( \cos \theta, \sin \theta, 1 \right)
    \]

    \[
    \mathbf{N} = \mathbf{r}_\theta \times \mathbf{r}_z = \left( z \cos \theta, z \sin \theta, -z \right)
    \]

    \[
    |\mathbf{N}| = \sqrt{(z \cos \theta)^2 + (z \sin \theta)^2 + (-z)^2} = z\sqrt{2}
    \]

    \[
    dS = z\sqrt{2} \, d\theta \, dz
    \]

    \[
    M = \int_{\theta=0}^{2\pi} \int_{z=1}^{4} \rho(z) \, dS
    \]

    \[
    M = \int_{z=1}^{4} \int_{\theta=0}^{2\pi} (z + 2) \, z\sqrt{2} \, d\theta \, dz
    \]

    \[
    \int_{\theta=0}^{2\pi} d\theta = 2\pi
    \]

    \[
    M = 2\pi \sqrt{2} \int_{z=1}^{4} (z + 2) z \, dz
    \]

    \[
    M = 2\pi \sqrt{2} \int_{z=1}^{4} (z^2 + 2z) \, dz
    \]

    \[
    \int (z^2 + 2z) \, dz = \frac{1}{3} z^3 + z^2
    \]

    \[
    \left[ \frac{1}{3} (4)^3 + (4)^2 \right] - \left[ \frac{1}{3} (1)^3 + (1)^2 \right]
    \]

    \[
    \left( \frac{64}{3} + 16 \right) - \left( \frac{1}{3} + 1 \right)
    \]

    \[
    36
    \]

    \[
    M = 2\pi \sqrt{2} \times 36
    \]

    \[
    M = 72\pi \sqrt{2}
    \]

}

\qs{}{
    Evaluate the surface integral
    \[
    \iint_S \vec{F} \cdot d\vec{S}, \quad \text{where } \vec{F} = \langle x, y, 2z \rangle
    \]
    and $S$ is the part of the paraboloid $z = 4 - x^2 - y^2$, oriented downwards, that lies above the unit square $[0, 1] \times [0, 1]$. 
}

\sol{
    \[
    \iint_S \vec{F} \cdot d\vec{S}, \quad \vec{F} = \langle x, y, 2z \rangle
    \]

    \[
    z = 4 - x^2 - y^2
    \]

    \[
    \vec{n} = \left( \frac{\partial f}{\partial x}, \frac{\partial f}{\partial y}, -1 \right)
    \]

    \[
    \frac{\partial f}{\partial x} = -2x, \quad \frac{\partial f}{\partial y} = -2y
    \]

    \[
    \vec{n} = (-2x, -2y, -1)
    \]

    \[
    d\vec{S} = \vec{n} \, dx \, dy = (-2x, -2y, -1) \, dx \, dy
    \]

    \[
    \vec{F} = (x, y, 2z) = \left( x, y, 2(4 - x^2 - y^2) \right)
    \]

    \[
    \vec{F} \cdot d\vec{S} = (x, y, 2z) \cdot (-2x, -2y, -1) \, dx \, dy
    \]

    \[
    \vec{F} \cdot d\vec{S} = [ -2x^2 - 2y^2 - 2(4 - x^2 - y^2) ] \, dx \, dy
    \]

    \[
    \vec{F} \cdot d\vec{S} = [ -2x^2 - 2y^2 - 8 + 2x^2 + 2y^2 ] \, dx \, dy = -8 \, dx \, dy
    \]

    \[
    \iint_S \vec{F} \cdot d\vec{S} = \iint_D (-8) \, dx \, dy = -8 \iint_D dx \, dy
    \]

    \[
    \iint_S \vec{F} \cdot d\vec{S} = -8 \times 1 = -8
    \]

    \[
    \iint_S \vec{F} \cdot d\vec{S} = -8
    \]

}

\qs{}{
    Evaluate the surface integral
    \[
    \iint_S \vec{F} \cdot d\vec{S}, \quad \text{where } \vec{F} = \langle -z, x, y \rangle
    \]
    and $S$ is the part of the unit sphere $x^2 + y^2 + z^2 = 1$ in the first octant, oriented upwards.
}

\sol{
    \[ \text{Divergence theorem: } \iint_{S_{\text{closed}}} \vec{F} \cdot d\vec{S} = \iiint_V \nabla \cdot \vec{F} \, dV.  \]
    \[ \nabla \cdot \vec{F} = \frac{\partial (-z)}{\partial x} + \frac{\partial x}{\partial y} + \frac{\partial y}{\partial z} = 0. \]
    \[ \iint_{S_{\text{closed}}} \vec{F} \cdot d\vec{S} = 0. \]
    \[ \iint_S \vec{F} \cdot d\vec{S} = -\left( \iint_{S_1} \vec{F} \cdot d\vec{S} + \iint_{S_2} \vec{F} \cdot d\vec{S} + \iint_{S_3} \vec{F} \cdot d\vec{S} \right). \]
    \[ \iint_{S_1} \vec{F} \cdot d\vec{S}, \quad \vec{n}_1 = \langle -1, 0, 0 \rangle, \quad \vec{F} = \langle -z, 0, y \rangle.  \]
    \[ \vec{F} \cdot \vec{n}_1 = (-z)(-1) = z.  \]
    \[ \iint_{S_1} z \, dy \, dz, \quad y^2 + z^2 \leq 1, \quad y, z \geq 0. \]
    \[ \text{Polar coordinates: } y = r \cos\theta, \, z = r \sin\theta, \, r \in [0, 1], \, \theta \in [0, \frac{\pi}{2}]. \]
    \[ \iint_{S_1} z \, dy \, dz = \int_0^{\frac{\pi}{2}} \int_0^1 r^2 \sin\theta \, dr \, d\theta = \int_0^{\frac{\pi}{2}} \sin\theta \, d\theta \int_0^1 r^2 \, dr. \]
    \[ \int_0^{\frac{\pi}{2}} \sin\theta \, d\theta = 1, \quad \int_0^1 r^2 \, dr = \frac{1}{3}. \]
    \[ \iint_{S_1} z \, dy \, dz = \frac{1}{3} \]
    \[ \iint_{S_2} \vec{F} \cdot d\vec{S}, \quad \vec{n}_2 = \langle 0, -1, 0 \rangle, \quad \vec{F} = \langle -z, x, 0 \rangle. \]
    \[ \vec{F} \cdot \vec{n}_2 = x(-1) = -x. \]
    \[ -\iint_{S_2} x \, dx \, dz, \quad x^2 + z^2 \leq 1, \quad x, z \geq 0. \]
    \[ \iint_{S_2} x \, dx \, dz = \int_0^{\frac{\pi}{2}} \int_0^1 r^2 \cos\theta \, dr \, d\theta = \int_0^{\frac{\pi}{2}} \cos\theta \, d\theta \int_0^1 r^2 \, dr. \]
    \[ \int_0^{\frac{\pi}{2}} \cos\theta \, d\theta = 1, \quad \int_0^1 r^2 \, dr = \frac{1}{3}. \]
    \[ \iint_{S_2} x \, dx \, dz = \frac{1}{3}, \quad -\iint_{S_2} x \, dx \, dz = -\frac{1}{3}. \]
    \[ \iint_{S_3} \vec{F} \cdot d\vec{S}, \quad \vec{n}_3 = \langle 0, 0, -1 \rangle, \quad \vec{F} = \langle -z, x, y \rangle. \]
    \[ \vec{F} \cdot \vec{n}_3 = y(-1) = -y. \]
    \[ -\iint_{S_3} y \, dx \, dy, \quad x^2 + y^2 \leq 1, \quad x, y \geq 0. \]
    \[ \iint_{S_3} y \, dx \, dy = \int_0^{\frac{\pi}{2}} \int_0^1 r^2 \sin\theta \, dr \, d\theta = \int_0^{\frac{\pi}{2}} \sin\theta \, d\theta \int_0^1 r^2 \, dr.  \]
    \[ \int_0^{\frac{\pi}{2}} \sin\theta \, d\theta = 1, \quad \int_0^1 r^2 \, dr = \frac{1}{3}. \]
    \[ \iint_{S_3} y \, dx \, dy = \frac{1}{3}, \quad -\iint_{S_3} y \, dx \, dy = -\frac{1}{3}. \]
    \[ \iint_S \vec{F} \cdot d\vec{S} = -\left( \frac{1}{3} - \frac{1}{3} - \frac{1}{3} \right) = \frac{1}{3}. \]
}

\qs{}{
    Find the flux of $\vec{F}(x, y, z) = z \hat{i} + y \hat{j} + x \hat{k}$ across the helicoid
    \[
    \vec{r}(u, v) = \langle u \cos v, u \sin v, v \rangle, \quad 0 \leq u \leq 1, \, 0 \leq v \leq 2\pi,
    \]
    oriented upward.
}

\sol{
    \[
    \vec{r}(u, v) = \langle u \cos v, u \sin v, v \rangle
    \]
    \[
    \vec{r}_u = \langle \cos v, \sin v, 0 \rangle, \quad \vec{r}_v = \langle -u \sin v, u \cos v, 1 \rangle
    \]
    \[
    \vec{n} = \vec{r}_u \times \vec{r}_v = \begin{vmatrix}
    \hat{i} & \hat{j} & \hat{k} \\
    \cos v & \sin v & 0 \\
    -u \sin v & u \cos v & 1
    \end{vmatrix} = \langle \sin v, -\cos v, u \rangle
    \]
    \[
    \vec{F}(x, y, z) = z\,\hat{i} + y\,\hat{j} + x\,\hat{k}, \quad \vec{F}(\vec{r}(u, v)) = \langle v, u \sin v, u \cos v \rangle
    \]
    \[
    \vec{F} \cdot \vec{n} = v \sin v - u \sin v \cos v + u^2 \cos v
    \]
    \[
    \Phi = \iint_D (\vec{F} \cdot \vec{n}) \, du \, dv = \int_0^{2\pi} \int_0^1 \left( v \sin v - u \sin v \cos v + u^2 \cos v \right) du \, dv
    \]
    \[
    \int_0^1 v \sin v \, du = v \sin v, \quad \int_0^1 u \sin v \cos v \, du = \frac{1}{2} \sin v \cos v, \quad \int_0^1 u^2 \cos v \, du = \frac{1}{3} \cos v
    \]
    \[
    \Phi = \int_0^{2\pi} \left( v \sin v - \frac{1}{2} \sin v \cos v + \frac{1}{3} \cos v \right) dv
    \]
    \[
    \sin v \cos v = \frac{1}{2} \sin 2v
    \]
    \[
    \Phi = \int_0^{2\pi} \left( v \sin v - \frac{1}{4} \sin 2v + \frac{1}{3} \cos v \right) dv
    \]
    \[
    \int_0^{2\pi} v \sin v \, dv = -2\pi, \quad \int_0^{2\pi} \sin 2v \, dv = 0, \quad \int_0^{2\pi} \cos v \, dv = 0
    \]
    \[
    \Phi = -2\pi - \frac{1}{4}(0) + \frac{1}{3}(0)
    \]
    \[
    \Phi = -2\pi
    \]

}

\qs{}{
    Let $S$ be the part of the elliptical cylinder $y^2 + 4z^2 = 4$ that lies above the $xy$-plane and between the planes $x = -2$ and $x = 2$. Let $S$ have the upward orientation; that is, let $S$ be oriented so that the normal vectors have positive $z$-component.
    \begin{enumerate}
        \item[(a)] Find a parameterization of $S$.
        \item[(b)] Does your parameterization match the given orientation of $S$? Explain.
        \item[(c)] Let $\vec{F}$ be the vector field
        \[
        \vec{F}(x, y, z) = e^{x^2 y^2 z^2} \hat{i} + x^2 y \hat{j} + z^2 e^{x/5} \hat{k}.
        \]
        Find the flux of $\vec{F}$ across the oriented surface $S$.
    \end{enumerate}
}


\sol{
    \begin{enumerate}
        \item[(a)] 
        \[ \vec{r}(x, \theta) = \langle x,\ 2\cos\theta,\ \sin\theta \rangle, \quad x \in [-2, 2], \quad \theta \in [0, \pi]. \]
    
        \item[(b)]
        \[ \vec{r}_x = \frac{\partial \vec{r}}{\partial x} = \langle 1, 0, 0 \rangle, \quad \vec{r}_\theta = \frac{\partial \vec{r}}{\partial \theta} = \langle 0, -2\sin\theta, \cos\theta \rangle. \]
        \[ \vec{n} = \vec{r}_\theta \times \vec{r}_x = \left| \begin{array}{ccc}
        \mathbf{i} & \mathbf{j} & \mathbf{k} \\
        0 & -2\sin\theta & \cos\theta \\
        1 & 0 & 0 \\
        \end{array} \right| = \langle 0, \cos\theta, 2\sin\theta \rangle. \]
        \[ n_z = 2\sin\theta \geq 0 \quad \text{for} \quad \theta \in [0, \pi]. \]
    
        \item[(c)]
        \[ \text{Flux} = \iint_S \vec{F} \cdot d\vec{S} = \iint_D \vec{F}(\vec{r}(x, \theta)) \cdot (\vec{r}_\theta \times \vec{r}_x) \, d\theta \, dx. \]
        \[ \vec{F}(x, y, z) = \langle e^{x^2y^2z^2}, x^2y, z^2e^{x/5} \rangle.  \]
        \[ \vec{F}(\vec{r}(x, \theta)) = \langle e^{4x^2\cos^2\theta\sin^2\theta},\ 2x^2\cos\theta,\ \sin^2\theta e^{x/5} \rangle. \]
        \[ \vec{F} \cdot \vec{n} = 2x^2\cos^2\theta + 2\sin^3\theta e^{x/5}. \]
        \[ \text{Flux} = \int_{x=-2}^{2} \int_{\theta=0}^\pi \left( 2x^2\cos^2\theta + 2\sin^3\theta e^{x/5} \right) d\theta \, dx. \]
        \[ \int_{0}^\pi \cos^2\theta \, d\theta = \frac{\pi}{2}, \quad \int_{0}^\pi \sin^3\theta \, d\theta = \frac{4}{3}. \]
        \[ \text{Flux} = \int_{x=-2}^2 \left[ 2x^2 \cdot \frac{\pi}{2} + 2 \cdot \frac{4}{3} \cdot e^{x/5} \right] dx. \]
        \[ \int_{x=-2}^2 x^2 dx = \left[ \frac{x^3}{3} \right]_{-2}^2 = \frac{8}{3} - \left(-\frac{8}{3}\right) = \frac{16}{3}. \]
        \[ \int_{x=-2}^2 e^{x/5} dx = 5\left[ e^{x/5} \right]_{-2}^2 = 5\left(e^{2/5} - e^{-2/5}\right). \]
        \[ \text{Flux} = \frac{16\pi}{3} + \frac{40}{3}\left(e^{2/5} - e^{-2/5}\right). \]
    \end{enumerate}
}

\qs{}{
    For each of the following parameterizations $\vec{r}(u, v)$, and vector fields $\vec{F}(x, y, z)$:

    \begin{enumerate}
        \item[i.] Describe the surface $S$ that is parameterized by $\vec{r}(u, v)$.
        \item[ii.] Describe in words the positive orientation of $S$ given by the family of unit normal vectors $\vec{n} = (\vec{r}_u \times \vec{r}_v)/|\vec{r}_u \times \vec{r}_v|$. That is, give a description of which way the normal vectors point for this orientation.
        \item[iii.] The flux of $\vec{F}$ through $S$ is
        \[
        \iint_S \vec{F} \cdot d\vec{S} = \iint_S \vec{F} \cdot \vec{n} \, dS = \iint_S \vec{F} \cdot (\vec{r}_u \times \vec{r}_v) \, du \, dv.
        \]
        Explain without any calculation whether the flux of $\vec{F}$ through $S$ is positive, negative, or zero; or explain why you don’t have enough information to do so.
        
        \begin{enumerate}
            \item[(a)] $\vec{r}(u, v) = \langle u, v, \sqrt{1 - u^2 - v^2} \rangle$ where $u^2 + v^2 \leq 1$. The vector field is $\vec{F}(x, y, z) = \langle -x, -y, -z \rangle$.
        
            \item[(b)] $\vec{r}(u, v) = \langle u \cos v, u \sin v, -u^2 \rangle$ where $0 \leq v \leq 2\pi$ and $0 \leq u$. The vector field is $\vec{F}(x, y, z) = \langle 0, 0, 1 \rangle$.
        
            \item[(c)] $\vec{r}(u, v) = \langle \cos u \sin v, \sin u \sin v, \cos v \rangle$ where $0 \leq v \leq \pi$ and $0 \leq u \leq \pi$. The vector field is $\vec{F}(x, y, z) = \langle 0, 1, z \rangle$.
        
            \item[(d)] $\vec{r}(u, v) = \langle \cos u, v, \sin u \rangle$ where $0 \leq u \leq 2\pi$ and $-\infty < v < \infty$. The vector field is $\vec{F}(x, y, z) = \langle 0, e^x, 0 \rangle$.
        \end{enumerate} 
    \end{enumerate}    
}

\sol{

    \subsection*{(a)}

    \textbf{Parameterization:} \(\vec{r}(u, v) = \langle u, v, \sqrt{1 - u^2 - v^2} \rangle\), where \(u^2 + v^2 \leq 1\). \\
    \textbf{Vector Field:} \(\vec{F}(x, y, z) = \langle -x, -y, -z \rangle\).

    \begin{enumerate}
        \item[i.] \textbf{Description of Surface \(S\):} The parameterization maps the unit disk \(u^2 + v^2 \leq 1\) in the \(uv\)-plane to the upper hemisphere of the unit sphere centered at the origin, as:
        \[
        x = u, \quad y = v, \quad z = \sqrt{1 - u^2 - v^2}.
        \]
        This satisfies \(x^2 + y^2 + z^2 = 1\) with \(z \geq 0\).

        \item[ii.] \textbf{Positive Orientation:} The normal vectors \(\vec{n}\) point outward from the sphere. The cross product \(\vec{r}_u \times \vec{r}_v\) aligns with the position vector, confirming the outward orientation.

        \item[iii.] \textbf{Flux through \(S\):} The vector field \(\vec{F}(x, y, z) = \langle -x, -y, -z \rangle\) points inward toward the origin. Since \(\vec{F}\) is opposite to the outward normals, the dot product \(\vec{F} \cdot \vec{n} < 0\), so the flux is \textbf{negative}.
    \end{enumerate}

    ---

    \subsection*{(b)}

    \textbf{Parameterization:} \(\vec{r}(u, v) = \langle u \cos v, u \sin v, -u^2 \rangle\), where \(0 \leq v \leq 2\pi\) and \(0 \leq u\). \\
    \textbf{Vector Field:} \(\vec{F}(x, y, z) = \langle 0, 0, 1 \rangle\).

    \begin{enumerate}
        \item[i.] \textbf{Description of Surface \(S\):} The parameterization describes the paraboloid:
        \[
        z = -(x^2 + y^2),
        \]
        which opens downward for \(u \geq 0\).

        \item[ii.] \textbf{Positive Orientation:} The normal vectors point upward in the positive \(z\)-direction. The computation of \(\vec{r}_u \times \vec{r}_v\) confirms that the \(z\)-component is positive.

        \item[iii.] \textbf{Flux through \(S\):} The vector field \(\vec{F} = \langle 0, 0, 1 \rangle\) points upward, aligning with the normals. Therefore, \(\vec{F} \cdot \vec{n} > 0\), and the flux is \textbf{positive}.
    \end{enumerate}

    ---

    \subsection*{(c)}

    \textbf{Parameterization:} \(\vec{r}(u, v) = \langle \cos u \sin v, \sin u \sin v, \cos v \rangle\), where \(0 \leq v \leq \pi\) and \(0 \leq u \leq \pi\). \\
    \textbf{Vector Field:} \(\vec{F}(x, y, z) = \langle 0, 1, z \rangle\).

    \begin{enumerate}
        \item[i.] \textbf{Description of Surface \(S\):} The parameterization covers the entire unit sphere centered at the origin, as:
        \[
        x^2 + y^2 + z^2 = 1.
        \]

        \item[ii.] \textbf{Positive Orientation:} The normal vectors \(\vec{n}\) point inward toward the center of the sphere, as shown by the cross product \(\vec{r}_u \times \vec{r}_v \propto -\vec{r}(u, v)\).

        \item[iii.] \textbf{Flux through \(S\):} The vector field \(\vec{F}\) has components that depend on \(z\). Without explicit calculation, it is unclear whether the contributions from the positive and negative flux cancel, so the sign of the flux cannot be determined.
    \end{enumerate}

    ---

    \subsection*{(d)}

    \textbf{Parameterization:} \(\vec{r}(u, v) = \langle \cos u, v, \sin u \rangle\), where \(0 \leq u \leq 2\pi\) and \(-\infty < v < \infty\). \\
    \textbf{Vector Field:} \(\vec{F}(x, y, z) = \langle 0, e^x, 0 \rangle\).

    \begin{enumerate}
        \item[i.] \textbf{Description of Surface \(S\):} The parameterization describes an infinite circular cylinder of radius 1 centered along the \(y\)-axis.

        \item[ii.] \textbf{Positive Orientation:} The normal vectors \(\vec{n}\) point radially inward toward the \(y\)-axis:
        \[
        \vec{n} \propto \langle -\cos u, 0, -\sin u \rangle.
        \]

        \item[iii.] \textbf{Flux through \(S\):} The vector field \(\vec{F} = \langle 0, e^x, 0 \rangle\) has only a \(y\)-component, while the normals have zero \(y\)-component. Therefore:
        \[
        \vec{F} \cdot \vec{n} = 0,
        \]
        and the flux is \textbf{zero}.
    \end{enumerate}
}

\newpage

\noindent
Fourier's Law of Heat Conduction states that if the temperature at a point is given by $u(x, y, z)$, then the heat flow is given by the vector field $\vec{F} = -K \nabla u$, where $K$ is the thermal conductivity of the material through which the heat is flowing. The rate of heat flow across a surface $S$ is then given by the vector surface integral
\[
\iint_S \vec{F} \cdot d\vec{S} = -K \iint_S \nabla u \cdot d\vec{S}.
\]

\qs{}{
    Suppose the temperature at a point in a ball with conductivity $K$ is inversely proportional to the distance from the center of the ball. Show that the rate of heat flow across a sphere $S$ of radius $a$ with center at the center of the ball and oriented outwards is independent of the radius $a$.
}

\sol{
    \[ u(x, y, z) = \frac{c}{r}, \quad r = \sqrt{x^2 + y^2 + z^2}. \]

    \[ \frac{\partial u}{\partial x} = -\frac{c x}{r^3}, \quad \frac{\partial u}{\partial y} = -\frac{c y}{r^3}, \quad \frac{\partial u}{\partial z} = -\frac{c z}{r^3}. \]

    \[ \nabla u = -\frac{c}{r^3} \langle x, y, z \rangle. \]

    \[ \vec{F} = -K \nabla u = \frac{K c}{r^3} \langle x, y, z \rangle. \]

    \[ \vec{n} = \frac{\langle x, y, z \rangle}{a}, \quad d\vec{S} = \vec{n} \, dS = \frac{\langle x, y, z \rangle}{a} dS. \]

    \[ \vec{F} \cdot d\vec{S} = \left( \frac{K c}{a^3} \langle x, y, z \rangle \right) \cdot \left( \frac{\langle x, y, z \rangle}{a} \right) dS = \frac{K c}{a^4} (x^2 + y^2 + z^2) dS. \]

    \[ x^2 + y^2 + z^2 = a^2 \implies \vec{F} \cdot d\vec{S} = \frac{K c}{a^4} a^2 dS = \frac{K c}{a^2} dS. \]

    \[ \iint_S \vec{F} \cdot d\vec{S} = \frac{K c}{a^2} \iint_S dS = \frac{K c}{a^2} (4\pi a^2) = 4\pi K c. \]
}

\newpage 

\qs{}{
    Use Stokesss Theorem to evaluate $\iint_S \text{curl} \, \vec{F} \cdot d\vec{S}$.

\begin{enumerate}
    \item[(a)] $\vec{F}(x, y, z) = x^2 y^3 z \, \hat{i} + \sin(xyz) \, \hat{j} + xyz \, \hat{k}$, \\
    $S$ is the part of the cone $y^2 = x^2 + z^2$ that lies between the planes $y = 0$ and $y = 3$, oriented in the direction of the positive $y$-axis.

    \item[(b)] $\vec{F}(x, y, z) = e^x \cos z \, \hat{i} + x^2 z \, \hat{j} + xy \, \hat{k}$, \\
    $S$ is the hemisphere $x = \sqrt{1 - y^2 - z^2}$, oriented in the direction of the positive $x$-axis.
\end{enumerate}
}

\sol{ \\
    \[ \text{(a)} \quad \vec{F}(x, y, z) = x^2 y^3 z \, \hat{i} + \sin(x y z) \, \hat{j} + x y z \, \hat{k}, \quad S: y^2 = x^2 + z^2, \, 0 \leq y \leq 3 \]

    \[ C: x^2 + z^2 = 9, \, y = 3 \]

    \[ \begin{aligned}
    x &= 3 \cos \theta, \quad z = 3 \sin \theta, \quad y = 3, \quad 0 \leq \theta \leq 2\pi, \\
    d\vec{r} &= \langle -3 \sin \theta, 0, 3 \cos \theta \rangle d\theta
    \end{aligned} \]

    \[ \vec{F} = \langle 729 \cos^2 \theta \sin \theta, \sin(27 \cos \theta \sin \theta), 27 \cos \theta \sin \theta \rangle \]

    \[ \vec{F} \cdot d\vec{r} = (-2187 \cos^2 \theta \sin^2 \theta + 81 \cos^2 \theta \sin \theta) d\theta \]

    \[ \oint_C \vec{F} \cdot d\vec{r} = \int_0^{2\pi} (-2187 \cos^2 \theta \sin^2 \theta + 81 \cos^2 \theta \sin \theta) d\theta \]

    \[ I_1 = -2187 \int_0^{2\pi} \cos^2 \theta \sin^2 \theta d\theta = -\frac{2187}{4} \pi, \quad I_2 = 81 \int_0^{2\pi} \cos^2 \theta \sin \theta d\theta = 0  \]

    \[ \oint_C \vec{F} \cdot d\vec{r} = I_1 + I_2 = -\frac{2187}{4} \pi \]

    \[ \text{(b)} \quad \vec{F}(x, y, z) = e^x \cos z \, \hat{i} + x^2 z \, \hat{j} + x y \, \hat{k}, \quad S: x = \sqrt{1 - y^2 - z^2} \]

    \[ C: y^2 + z^2 = 1, \, x = 0 \]

    \[ \begin{aligned}
    x &= 0, \quad y = \cos \theta, \quad z = \sin \theta, \quad 0 \leq \theta \leq 2\pi, \\
    d\vec{r} &= \langle 0, -\sin \theta, \cos \theta \rangle d\theta
    \end{aligned}  \]

    \[ \vec{F} = \langle \cos (\sin \theta), 0, 0 \rangle \]

    \[ \vec{F} \cdot d\vec{r} = 0  \]

    \[ \oint_C \vec{F} \cdot d\vec{r} = 0 \]
        
}
\end{document}