\documentclass{report}

\input{preamble}
\input{macros}
\input{letterfonts}

\title{\Huge{Math 120 QR}}
\author{\huge{Alex Hernandez Juarez}}
\date{Fall 2024}

\begin{document}

\maketitle
\newpage% or \cleardoublepage
% \pdfbookmark[<level>]{<title>}{<dest>}
\pdfbookmark[section]{\contentsname}{toc}
\tableofcontents
\pagebreak

\chapter{}
\section{Day 1 notes}

\defrose{Distance Formula}{DDefintion: 
	\begin{center}
		\[ |P_{1}P_{2}| = \sqrt{(x_{2} - x_{1})^{2} + (y_{2} - y_{1})^{2} + (z_{2} - z_{1})^{2} }\] 
	\end{center}
}

\defrose{Equation of a sphere}{
	DDefintion: An equation of a sphere with center $C(h,k,l)$, and radius $r$ is 
	\begin{center}
		\[ (x-h)^{2} + (y-k)^{2} + (z-l)^{2}\] 
	\end{center}
	In particular, if the center is the origin $O$, than an equation of the sphere is 
	\begin{center}
		\[ x^{2} + y^{2} + z^{2}\] 
	\end{center}
}
		
\section{Day 2 Notes}

\defrose{The lenght/magnitude of a vecotr}{IIn 2D, $\vec{v} = <a,b>$: $|\vec{v}| = \sqrt{a^{2} + b^{2}}$  \\
	In 3D, $\vec{v}$, $\vec{v} = <a,b,c>$: $|\vec{v}| = \sqrt{a^{2} + b^{2} + c^{2}}$\\
	A unit vector is a vector of length 1
}

\qs{}{If $\vec{v}$ is a a vector and $a$ is a scalar, then what is $|a \vec{v}|$}
\sol{
	\begin{center}
		\[ |a\vec{v}| = |a||\vec{v}|\] 
	\end{center}
}

\defrose{Vectors in $\mathbf{R}^{3}$}{TThe standard basis vectors in $\mathbf{R}^{3}$ are 
	\begin{center}
		\[i = <1,0,0>\]
		\[j = <0,1,0> \] 
		\[k = <0,0,1> \]  
	\end{center}	
}

\qs{}{What is special about i,j,k?}
\sol{
	\begin{itemize}
		\item Cannot make any of them as a linear combibnation of the other three. 
		\item Any vector $\vec{v} \epsilon \mathbf{R}^{3}$ can be written uniquily as a linear combibnation of i,j,k
	\end{itemize}
}

\exrose{}{$\vec{v} + \vec{u}$ = $\begin{pmatrix} u_{1}\\ u_{2}\\ u_{3} \end{pmatrix} + \begin{pmatrix} v_{1}\\ v_{2}\\ v_{3} \end{pmatrix}$ \\
$\vec{v} + \vec{u} = \begin{pmatrix} u_{1} + v_{1}\\ u_{2} + v_{2}\\ u_{3} + v_{3} \end{pmatrix}$ 

}

\defrose{Dot product}{AA dot product between $v$ and $w$ is: \\
In 2D: $v \cdot w = v_{1}w_{1} + v_{2}w_{2}$\\
In 3D: $v \cdot 2 = v_{1}w_{1} + v_{2}w_{2} + v_{3}w_{3}$ \\

Geometric defintion: \\
$|u \cdot w| = |u||w| \cos(\theta)$ where $\theta$ is the angle between $v$ and $w$
}

\exrose{Why the 2 defintions are the same}{
	\begin{center}
		\[ v_{1}w_{1} + v_{2}w_{2} = |v||w|\cos(\theta) = p|v|\] 
		\[ p = |w|\cos(\theta)\]
		\[ v_{1} = |v|\cos(\theta)\]
		\[ v_{2} = |v|\sin(\theta)\]  
		\[ w_{1} = |w|\cos(\theta)\]
		\[ w_{2} = |w|\sin(\theta)\]    
		\[ \text{LHS} = |v||w| \left(\cos(\alpha)\cos(\beta) + \sin(\alpha)\sin(\beta)\right)\]
		\[ \text{LHS} = |v||w| \sin(\alpha + \beta) \]
		\[ \text{LHS} = |v||w| \cos(\theta) = \text{RHS} \]    
	\end{center}
}

\exrose{what does the def mean}{
	How much effect of $\vec{w}$ act along $\vec{v}$ \\
	Work: $w = \vec{F} \cdot \vec{S}$
}

\qs{}{Find a relation between $|v|$ and $v \cdot v$}
\sol{$|v|^{2} = v \cdot v$}

\qs{}{$v$, $w$ of fixed lenghts when is $v \cdot w$ largest?} 
\sol{$v$ paralell: $\theta = 0$ \\
$\cos(\theta) = 1$}

\exrose{Projections}{Given $\vec{v} \neq \vec{o}$ \\
	The project of $w$ on $v$ is proj $w = \left(\frac{\vec{w} \cdot \vec{v}}{|v|}\right) \frac{\vec{v}}{|v|}$\\
	Directon of $v$ is $\frac{\vec{v}}{|v|}$ \\
	Dor project of $w$ with direction is: 
	\begin{center}
		\[ \vec{w} \cdot \frac{\vec{v}}{|v|}\] 
	\end{center}
	
}

\qs{}{TRUE or False: \\
	$u, v, w$: vectors 
	$(u \cdot v)w = u(v \cdot w)$
}
\sol{false}

\qs{}{TRUE or FALSE: \\
	$v - w| = |v| - |w|$ if $v \parallel w$
}
\sol{false}

\qs{}{When is this ideal square sum happening? \\
	\begin{center}
		\[ |v + w|^{2} = |v|^{2} + |w|^{2}\] 
	\end{center}	
}
\sol{when $v \perp w$}

\defrose{Cross Product}{TThe cross product of two vectors $v$ and $w$, $v * w$ is a vector u defined by 
	$u \perp v$ and $u \perp w$. \\
	Direction of $u$ is given by the right hand rule
	
	Magnitude: $|u|$ = Area of the parallelogram spanned by $v$ and $w$.
}

\section{Day 2 Reading notes} 

\defrose{Vector Addition}{IIf $\textbf{u}$ and $\textbf{v}$ are vectors positioned so the initial
point of $\textbf{v}$ is at the terminal point of $\textbf{u}$, then the $\textbf{sum u + v}$ is the vector from the
initial point of $\textbf{u}$ to the terminal point of $\textbf{v}$.}


\defrose{Scalar Multiplication}{IIf $c$ is a scalar and $\textbf{v}$ is a vector, then the
$\textbf{scalar multiple}$ $c \textbf{v}$ is the vector whose length is $|c|$ times the length of $\textbf{v}$ and whose direction is
the same as $\textbf{v}$ if $c > 0$ and is opposite to $\textbf{v}$ if $c=0$ or $\textbf{v}$ = 0, then c$\textbf{v} = 0$ }

\exrose{}{Given the points $A(x_{1}, y_{1},z_{1})$ and $B(x_{2}, y_{2}, z_{2})$, the vector $\textbf{a}$ with represenation 
	$\ray{AB}$ is: 
	\begin{center}
		\[ a = \langle x_{2} - x_{1}, y_{2}-y_{1}, z_{2} - z_{1} \rangle \] 
	\end{center}
}

\exrose{}{
	If $\textbf{a}  = \langle a_{1}, a_{2}\rangle$ and $\textbf{b} = \langle b_{1}, b_{2} \rangle$, then: 
	\begin{center}
		\[ \textbf{a} + \textbf{b} = \langle a_{1} + b_{1}, a_{2} + b_{2} \rangle \]
		\[ \textbf{a} - \textbf{b} = \langle a_{1} - b_{1}, a_{2} - b_{2} \rangle \]
		\[ c\textbf{a} = \langle ca_{1}, ca_{2} \rangle \]   
	\end{center} 
	
	Similarily, for three demensional vectors, 

	\begin{center}
		\[ \langle a_{1}, a_{2}, a_{3} \rangle + \langle b_{1}, b_{2}, b_{3} \rangle = \langle a_{1} + b_{1}, a_{2} + a_{3} + b_{3} \rangle \] 
		\[ \langle a_{1}, a_{2}, a_{3} \rangle - \langle b_{1}, b_{2}, b_{3} \rangle = \langle a_{1} - b_{1}, a_{2} - a_{3} - b_{3} \rangle \] 
		\[ c \langle a_{1}, a_{2}, a_{3} \rangle =  \langle ca_{1}, ca_{2}, ca_{3} \rangle \] 
	\end{center} 

}

\ntimg{Properties of vectors: If $\textbf{a}$, $\textbf{b}$, and $\textbf{c}$ are vectors in $V_{n}$ and $c$ and $d$ are scalars than 
	\begin{itemize}
		\item $\textbf{a} + \textbf{b} = \textbf{b} + \textbf{a}$
		\item $a + (\textbf{b} + \textbf{c}) = (\textbf{a} + \textbf{b}) + \textbf{c} $
		\item $\textbf{a} + 0 = \textbf{a}$
		\item $ \textbf{a} + \textbf{a} + \textbf{-a} = 0 $
		\item $c(\textbf{a} + \textbf{b}) = c\textbf{a} + c\textbf{b}$ 
		\item $(c + d)a = c \textbf{a} + d \textbf{a} $
		\item $(cd) \textbf{a} = c(d\textbf{a})$
		\item $l\textbf{a} = \textbf{a} $
	\end{itemize}
}

\section{Day 3 Reading notes}

\exrose{}{$\in$ }

\section{Day 3 Class notes}

\exrose{}{Use the geometric def of cross product to calculate:  \\
	$i \times ( i + j)$ \\ 
	$( i + j) \times (i - j)$	
}

\defrose{Second definition of dot product}{
	AArithmetic Definition: 
	\begin{center}
		\[ a \times b = \left[ \begin{matrix}
			i & j & k  \\
			a_{1} & a_{2} & a_{3} \\
			b_{1} & b_{2} & b_{3} \\
		\end{matrix}\right] = |a||b| \sin(\theta) \] 
		\[ \left[ \begin{matrix}
			a_{2} & a_{3} \\
			b_{2} & b_{3}  \\
		\end{matrix}\right] i - \left[ \begin{matrix}
			a_{1} & a_{3} \\
			b_{1} & b_{3}  \\
		\end{matrix}\right] j - \left[ \begin{matrix}
			a_{1} & a_{2} \\
			b_{1} & b_{2} k  \\
		\end{matrix}\right]\] 
		\[ = (a_{2}b_{3} - a_{3}b_{2}) i - (a_{1}b_{3} - a_{3}b_{1}) j + (a_{1} b_{2} - a_{2}b_{1}) k \]\
		\[ |a \times b |^{2} = (a_{2}b_{3} - a_{3}b_{2}) - (a_{1}b_{3} - a_{3}b_{1})  + (a_{1} b_{2} - a_{2}b_{1})  \] 
		\[ |a \times b |^{2} = (a_{2}b_{3} - a_{3}b_{2}) - (a_{1}b_{3} - a_{3}b_{1})  + (a_{1} b_{2} - a_{2}b_{1})  \] 

	\end{center}
}

\qs{}{Calculate the corss product of $v = \langle -1, 3, 4 \rangle$ and $w = \langle 2, 1, -2 \rangle$ }

\sol{ 
	\begin{center}
		\[ v \times w =  \left[ \begin{matrix}
			i & j & k  \\
			-1 & 4 & 4 \\
			2 & 1 & -2 \\
		\end{matrix}\right]  = 10 i + 6j - 7k \] 
	\end{center}
}

\qs{}{Calculate the area of the triangle with vertices P(1,0,1), Q (-3, 4, 0) and R(2,1,0)}

\sol{ 
	\begin{center}
		\[ \ray{RP} = \langle -1, -1, 1 \rangle \] 
		\[ \ray{RQ} = \langle 5, 0, 0\rangle \] 
		\[ \ray{RP} \times \ray{RQ} = \left[ \begin{matrix}
			i & j & k  \\
			-1 & -1 & 1 \\
			-5 & 0 & 0 \\
		\end{matrix}\right] = -5j - 5k \]
		\[ \text{area of } \delta PQR = \frac{1}{2} |\ray{RP \times \ray{RQ}} = \frac{1}{2}\sqrt{5^{2} + 5^{2}} = \frac{5 \sqrt{2}}{2}\]  
	\end{center}
}

\qs{}{What is special about: P(1,4,2), 1(3,3,5), R(9,1,11), and S(5, 11, 9)}

\sol{
	If ($\ray{AC} \times \ray{AB} \parallel (\ray{AD} \times \ray{AC}))$ then A, B, C, D are 
	on the same plane  
}

\qs{}{1. If $u \times w = u \times v$  then $ w = v$ \\
	2. If $u \cdot v = 0$ and $u \times v = \vec{0}$ then either $u = \vec{0}$ or $v = \vec{0}$  	
}

\sol{ \\
	1. False
		\begin{center}
			\[ u \times ( u + w) = u \times w\] 
		\end{center} 
	2. True \\ 
	The reason is because the dot product is $|u||v| \cos(\theta)$ and the cross product is 
	$|u||v| \sin(\theta)$ 
}

\qs{}{Find the distance of a point p to the line passing through A and B}

\sol{
	\begin{center}
		\[ = \frac{1}{2} \text{dist} (P, AB) \cdot |AB|\]
		\[ \text{dist} (P, AB) = \frac{|\ray{AB} \times \ray{AP}|}{|\ray{AB}}\]  
	\end{center}
}

\ntimg{Line in 2D equation: $ax + by = c$ \\
	Line in 3D: Equation of a line passing through $(x_{0}, y_{0}, z_{0})$
	and parallel to $\vec{v} = \langle a, b, c \rangle$ is 
	\begin{itemize}
		\item $x = x_{0} + at $
		\item $y = y_{0} + bt$
		\item $z = z_{0} + zt$ 
	\end{itemize}
}

\ntimg{
	Equation of a plane passing through $(x_{0}, y_{0}, z_{0})$ and orthogonal to $\vec{n} = \langle a, b, c \rangle $ is 
	$(r - r_{0}) \cdot n = 0 $ \\
	$\langle x - x_{0}, y - y_{0}, z - z_{0} \rangle \cdot \langle a, b, c \rangle = 0 $ \\
	$ax + by + cz + d = 0 $
}

\qs{}{What is the normal of the plane $3x + z + 2 = 0$? }

\sol{
	\begin{center}
		\[ \vec{n} = \langle 3, 0, 1 \rangle \] 
	\end{center}
}

\qs{}{Is the line $x = 2t$, $ y = 1 + 3t$, $z = 2 + 4t$ parallel to the plane $x - 2y + z = 7$}

\section{Integration Review}

\exrose{}{
	\begin{center}
		\[ \int \frac{x}{\sqrt{1- x^{2}}} dx \]
		\[ v = 1- x^{2} \]
		\[ dv = -2x dx \]
		\[ \int \frac{-\frac{1}{2}}{\sqrt{v}} dv \] 
		\[ = \frac{1}{2} \int  \frac{1}{\sqrt{v}} \]   
	\end{center}
}

\exrose{}{
	\begin{center}
		\[ (3x^{2} 5)e^{x^{3} + 5x} dx \] 
		\[ u  = x^{3} + 5x \]
		\[ du = 3x^{2} 5 dx \]
		\[ = \int e^{u} du \]
		\[ = e^{x^{3} + 5x} c\]    
	\end{center}
}

\exrose{}{
	\begin{center}
		\[ \cos^{5}(2t) dt \] 
		\[ u = 2t \]
		\[ du = 2 dt \]
		\[ = \frac{1}{2} \cos^{3}(u) du = \frac{1}{2} \cos^{4}(u) + \cos(u) du \]
		\[ = \frac{1}{2} \int (1 - \sin^{2}(u))^{2} \cdot \cos(u) du \]
		\[ v = \sin(u) \]
		\[ dv = \cos (u) du \]       
	\end{center}
}

\exrose{}{
	\begin{center}
		\[ xe^{2x} dx \] 
		\[ x' = 1\] 
		\[ \int e^{2x} dx = \frac{1}{2} e^{2x} + c \]
		\[ \int v dv = uv + \int v du \]
		\[ v = x \to dv = 1 dx\]
		\[ dv = e^{2x} dx \to v = \frac{1}{2}e^{2x}\]
		\[ \int = \frac{1}{2} x e^{2x} - \int \frac{1}{2} e^{2x} dx \]     
	\end{center}
}

\exrose{}{
	\begin{center}
		\[ \int t \sin(2t) dt \] 
		\[ t' = 1\] 
		\[ \int \sin(2t) dt = -\frac{1}{2}\cos(2t) \]
		\[ \int v dv = uv + \int v dv \]
		\[ v = t \to dv = 1 dt\]
		\[ dv = \sin(2t) dx \to v = -\frac{1}{2}\cos(2t)\]
		\[ \int = - \frac{1}{2}\cos(2t) - \int \frac{1}{2} e^{2x} dx \]
	\end{center}
}

\section{Day 4 Class Notes}

\exrose{Warm up 1}{True or False \\
	If $Q_{1}R$ are not on P and $\vec{n} \cdot \ray{QR} = 0$ 
	than $Q_{1}R$ are on the same side of (P). 
	True: Since the dot product of the normal and $\ray{QR}$ are 0 than it is parallel to the plane
}

\exrose{Warm up 2}{What is the shape (in 3D) of \\
	\[\textbf{a) } {x^{2} + y^{2}  = 1} \quad \textbf{b) } {x^{2} + y^{2} + z^{2} = 4} \] 
	a) Cylinder \\
	b) Sphere
}

\section{Day 5 Class Notes}

\defrose{Tangent Line}{
	Given a cruve $r(t)$, and a point $r(t_{0})$ Equation of the tangent line passing through 
	$r(t_{0})$ is 
	\[ l(t) = r(t_{0}) + t \cdot r'(t_{0}) \]
	\[ r(t) = \langle \cos(t), \sin(t), 0 \rangle \quad \text{for } t \in [0, 2\pi]\]
}

\ntimg{
	Differential Rules: 
	\begin{enumerate}
		\item $(u \pm v)' = u' \pm v'$
		\item $cu' = (cu)'$
		\item $(u \cdot v)' = u' \cdot v$ + $ u \cdot v'$ 
		\item $(u \times v)' = u' \times v + u \times v'$
		\item $ (u_{0} f)' = u'(f)f'$ 
	\end{enumerate}
}

\defrose{}{
	IIf a curve is parameterized by $r(t)$ for $ \leq t \leq b $ then its (arc)length is 
	\[ L = \int_{a}^{b} |r'(t)| dt \] 
}


\[ \langle -2 \cos(t), 0, -2\sin(t) \rangle \]
\[ \int_{- \pi}^{\frac{\pi}{2}} |\langle -2 \sin(t), -4, 2 \cos(t) \rangle \] 

\newpage 

\section{Day 7 Notes}

\ex{Warm Up}{
	Calculate first and second derivatives of $f(x,y) = 2x^{2} - 3xy^{2}$ and $f(x,y) = \sin(x - y) + \sin(x + y)$\\
	Chain rule  compute $\frac{d}{dt}$ in 2 ways (chain rule, and subtitution)

	\[f(x,y) = 2x^{2} - 3xy^{2} \] 
	\[f'(x) = 3x^{2} - 3y^{2} \]
	\[f'(y) = -6xy \]
	\[f''(x) = 6x \]
	\[f''(y) = - 6y \]    
}


\end{document}
