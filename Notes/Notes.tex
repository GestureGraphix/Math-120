\documentclass{report}

\input{preamble}
\input{macros}
\input{letterfonts}

\title{\Huge{Math 120 QR}}
\author{\huge{Alex Hernandez Juarez}}
\date{Fall 2024}

\begin{document}

\maketitle
\newpage% or \cleardoublepage
% \pdfbookmark[<level>]{<title>}{<dest>}
\pdfbookmark[section]{\contentsname}{toc}
\tableofcontents
\pagebreak

\chapter{}
\section{Day 1 notes}

\defrose{Distance Formula}{DDefintion: 
	\begin{center}
		\[ |P_{1}P_{2}| = \sqrt{(x_{2} - x_{1})^{2} + (y_{2} - y_{1})^{2} + (z_{2} - z_{1})^{2} }\] 
	\end{center}
}

\defrose{Equation of a sphere}{
	DDefintion: An equation of a sphere with center $C(h,k,l)$, and radius $r$ is 
	\begin{center}
		\[ (x-h)^{2} + (y-k)^{2} + (z-l)^{2}\] 
	\end{center}
	In particular, if the center is the origin $O$, than an equation of the sphere is 
	\begin{center}
		\[ x^{2} + y^{2} + z^{2}\] 
	\end{center}
}
		
\section{Day 2 Notes}

\defrose{The lenght/magnitude of a vecotr}{IIn 2D, $\vec{v} = <a,b>$: $|\vec{v}| = \sqrt{a^{2} + b^{2}}$  \\
	In 3D, $\vec{v}$, $\vec{v} = <a,b,c>$: $|\vec{v}| = \sqrt{a^{2} + b^{2} + c^{2}}$\\
	A unit vector is a vector of length 1
}

\qs{}{If $\vec{v}$ is a a vector and $a$ is a scalar, then what is $|a \vec{v}|$}
\sol{
	\begin{center}
		\[ |a\vec{v}| = |a||\vec{v}|\] 
	\end{center}
}

\defrose{Vectors in $\mathbf{R}^{3}$}{TThe standard basis vectors in $\mathbf{R}^{3}$ are 
	\begin{center}
		\[i = <1,0,0>\]
		\[j = <0,1,0> \] 
		\[k = <0,0,1> \]  
	\end{center}	
}

\qs{}{What is special about i,j,k?}
\sol{
	\begin{itemize}
		\item Cannot make any of them as a linear combibnation of the other three. 
		\item Any vector $\vec{v} \epsilon \mathbf{R}^{3}$ can be written uniquily as a linear combibnation of i,j,k
	\end{itemize}
}

\exrose{}{$\vec{v} + \vec{u}$ = $\begin{pmatrix} u_{1}\\ u_{2}\\ u_{3} \end{pmatrix} + \begin{pmatrix} v_{1}\\ v_{2}\\ v_{3} \end{pmatrix}$ \\
$\vec{v} + \vec{u} = \begin{pmatrix} u_{1} + v_{1}\\ u_{2} + v_{2}\\ u_{3} + v_{3} \end{pmatrix}$ 

}

\defrose{Dot product}{AA dot product between $v$ and $w$ is: \\
In 2D: $v \cdot w = v_{1}w_{1} + v_{2}w_{2}$\\
In 3D: $v \cdot 2 = v_{1}w_{1} + v_{2}w_{2} + v_{3}w_{3}$ \\

Geometric defintion: \\
$|u \cdot w| = |u||w| \cos(\theta)$ where $\theta$ is the angle between $v$ and $w$
}

\exrose{Why the 2 defintions are the same}{
	\begin{center}
		\[ v_{1}w_{1} + v_{2}w_{2} = |v||w|\cos(\theta) = p|v|\] 
		\[ p = |w|\cos(\theta)\]
		\[ v_{1} = |v|\cos(\theta)\]
		\[ v_{2} = |v|\sin(\theta)\]  
		\[ w_{1} = |w|\cos(\theta)\]
		\[ w_{2} = |w|\sin(\theta)\]    
		\[ \text{LHS} = |v||w| \left(\cos(\alpha)\cos(\beta) + \sin(\alpha)\sin(\beta)\right)\]
		\[ \text{LHS} = |v||w| \sin(\alpha + \beta) \]
		\[ \text{LHS} = |v||w| \cos(\theta) = \text{RHS} \]    
	\end{center}
}

\exrose{what does the def mean}{
	How much effect of $\vec{w}$ act along $\vec{v}$ \\
	Work: $w = \vec{F} \cdot \vec{S}$
}

\qs{}{Find a relation between $|v|$ and $v \cdot v$}
\sol{$|v|^{2} = v \cdot v$}

\qs{}{$v$, $w$ of fixed lenghts when is $v \cdot w$ largest?} 
\sol{$v$ paralell: $\theta = 0$ \\
$\cos(\theta) = 1$}

\exrose{Projections}{Given $\vec{v} \neq \vec{o}$ \\
	The project of $w$ on $v$ is proj $w = \left(\frac{\vec{w} \cdot \vec{v}}{|v|}\right) \frac{\vec{v}}{|v|}$\\
	Directon of $v$ is $\frac{\vec{v}}{|v|}$ \\
	Dor project of $w$ with direction is: 
	\begin{center}
		\[ \vec{w} \cdot \frac{\vec{v}}{|v|}\] 
	\end{center}
	
}

\qs{}{TRUE or False: \\
	$u, v, w$: vectors 
	$(u \cdot v)w = u(v \cdot w)$
}
\sol{false}

\qs{}{TRUE or FALSE: \\
	$v - w| = |v| - |w|$ if $v \parallel w$
}
\sol{false}

\qs{}{When is this ideal square sum happening? \\
	\begin{center}
		\[ |v + w|^{2} = |v|^{2} + |w|^{2}\] 
	\end{center}	
}
\sol{when $v \perp w$}

\defrose{Cross Product}{TThe cross product of two vectors $v$ and $w$, $v * w$ is a vector u defined by 
	$u \perp v$ and $u \perp w$. \\
	Direction of $u$ is given by the right hand rule
	
	Magnitude: $|u|$ = Area of the parallelogram spanned by $v$ and $w$.
}

\section{Day 2 Reading notes} 

\defrose{Vector Addition}{IIf $\textbf{u}$ and $\textbf{v}$ are vectors positioned so the initial
point of $\textbf{v}$ is at the terminal point of $\textbf{u}$, then the $\textbf{sum u + v}$ is the vector from the
initial point of $\textbf{u}$ to the terminal point of $\textbf{v}$.}


\defrose{Scalar Multiplication}{IIf $c$ is a scalar and $\textbf{v}$ is a vector, then the
$\textbf{scalar multiple}$ $c \textbf{v}$ is the vector whose length is $|c|$ times the length of $\textbf{v}$ and whose direction is
the same as $\textbf{v}$ if $c > 0$ and is opposite to $\textbf{v}$ if $c=0$ or $\textbf{v}$ = 0, then c$\textbf{v} = 0$ }

\exrose{}{Given the points $A(x_{1}, y_{1},z_{1})$ and $B(x_{2}, y_{2}, z_{2})$, the vector $\textbf{a}$ with represenation 
	$\ray{AB}$ is: 
	\begin{center}
		\[ a = \langle x_{2} - x_{1}, y_{2}-y_{1}, z_{2} - z_{1} \rangle \] 
	\end{center}
}

\exrose{}{
	If $\textbf{a}  = \langle a_{1}, a_{2}\rangle$ and $\textbf{b} = \langle b_{1}, b_{2} \rangle$, then: 
	\begin{center}
		\[ \textbf{a} + \textbf{b} = \langle a_{1} + b_{1}, a_{2} + b_{2} \rangle \]
		\[ \textbf{a} - \textbf{b} = \langle a_{1} - b_{1}, a_{2} - b_{2} \rangle \]
		\[ c\textbf{a} = \langle ca_{1}, ca_{2} \rangle \]   
	\end{center} 
	
	Similarily, for three demensional vectors, 

	\begin{center}
		\[ \langle a_{1}, a_{2}, a_{3} \rangle + \langle b_{1}, b_{2}, b_{3} \rangle = \langle a_{1} + b_{1}, a_{2} + a_{3} + b_{3} \rangle \] 
		\[ \langle a_{1}, a_{2}, a_{3} \rangle - \langle b_{1}, b_{2}, b_{3} \rangle = \langle a_{1} - b_{1}, a_{2} - a_{3} - b_{3} \rangle \] 
		\[ c \langle a_{1}, a_{2}, a_{3} \rangle =  \langle ca_{1}, ca_{2}, ca_{3} \rangle \] 
	\end{center} 

}

\ntimg{Properties of vectors: If $\textbf{a}$, $\textbf{b}$, and $\textbf{c}$ are vectors in $V_{n}$ and $c$ and $d$ are scalars than 
	\begin{itemize}
		\item $\textbf{a} + \textbf{b} = \textbf{b} + \textbf{a}$
		\item $a + (\textbf{b} + \textbf{c}) = (\textbf{a} + \textbf{b}) + \textbf{c} $
		\item $\textbf{a} + 0 = \textbf{a}$
		\item $ \textbf{a} + \textbf{a} + \textbf{-a} = 0 $
		\item $c(\textbf{a} + \textbf{b}) = c\textbf{a} + c\textbf{b}$ 
		\item $(c + d)a = c \textbf{a} + d \textbf{a} $
		\item $(cd) \textbf{a} = c(d\textbf{a})$
		\item $l\textbf{a} = \textbf{a} $
	\end{itemize}
}

\end{document}
