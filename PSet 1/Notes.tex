\documentclass{report}

\input{preamble}
\input{macros}
\input{letterfonts}

\title{\Huge{Math 120 QR}}
\author{\huge{Alex Hernandez Juarez}}
\date{Fall 2024}

\begin{document}

\maketitle
\newpage% or \cleardoublepage
% \pdfbookmark[<level>]{<title>}{<dest>}
\pdfbookmark[section]{\contentsname}{toc}
\tableofcontents
\pagebreak

\chapter{}
\section{PSet 1}

\qs{}{Find the lengths of the sides of the triangle with vertices P (2, 1, 3), Q(0, 2, 5) and R(4, -1, 4). Is the
triangle an acute triangle (all sides less than 90$^{\circ}$), a right triangle, or an obtuse triangle (one angle
greater than 90$^{\circ}$)}

\sol{
	Image: 
	\tdplotsetmaincoords{60}{120}
	\begin{tikzpicture}[tdplot_main_coords, scale=1]

		% Define the coordinates
		\coordinate (P) at (2, 1, 3);
		\coordinate (Q) at (0, 2, 5);
		\coordinate (R) at (4, -1, 4);
		
		% Draw the axes
		\draw[thick, -stealth] (0,0,0) -- (5,0,0) node[anchor=north east]{$x$};
		\draw[thick, -stealth] (0,0,0) -- (0,5,0) node[anchor=north west]{$y$};
		\draw[thick, -stealth] (0,0,0) -- (0,0,5) node[anchor=south]{$z$};
		
		% Draw the triangle
		\draw[thick] (P) -- (Q) -- (R) -- cycle;
		
		% Label the points
		\node at (2, 1, 3) [anchor=south west] {P(2, 1, 3)};
		\node at (0, 2, 5) [anchor=south east] {Q(0, 2, 5)};
		\node at (4, -1, 4) [anchor=north west] {R(4, -1, 4)};
		
	\end{tikzpicture}
	
	\begin{center}
		\[ \ray{PQ} = \langle 0 - 2, 2 - 1, 3 - 5 \rangle = \langle -2, 1, -2 \rangle \]
		\[ \ray{QR} = \langle 4 - 0, -1 - 2, 4 - 3 \rangle = \langle 4, -3, 1 \rangle \]
		\[ \ray{RP} = \langle 2 - 4, -1 - 1, 3 - 4 \rangle = \langle -2, -2, 1 \rangle \]
		\[ |\ray{PQ}| = \sqrt{(-2)^{2} + {1}^{2} + (-2)^{2}} = \sqrt{5} \]
		\[ |\ray{QR}| = \sqrt{(4)^{2} + (-3)^{2} + (1)^{2}} = \sqrt{26}\] 
		\[ |\ray{RP}| = \sqrt{(-2)^{2} + (-2)^{2} + (1)^{2}} = \sqrt{5} \] 
	\end{center}
	angles: 
	\begin{center}
		\[ |\ray{PQ}|^{2} = 26 = 5 + 5 - 2\sqrt{5}\sqrt{5} \cos{\theta}\]
		\[ 16 = - 2\sqrt{5}\sqrt{5} \cos{\theta}\]  
		\[ \frac{16}{2\sqrt{5}\sqrt{5}} = - \cos(\theta)\]
		\[ \frac{8}{5} = -\cos(\theta)\]  
	\end{center}
}

\exrose{Hi}{hi}

\end{document}
